\documentstyle{article}
\begin{document}
\begin{prop}
/////////////////////////////////////////////////////////////////////////////
//  This test implements a rewrite based simplifier for the abstract
//  syntax of a toy imperative language.  
//
//  The test is quite elobarate and requires polymorphic datatypes to work 
//  correctly with garbage collection, pretty printing and rewriting. 
/////////////////////////////////////////////////////////////////////////////
#include <iostream.h>

/////////////////////////////////////////////////////////////////////////////
//  The following recursive type equations define the abstract syntax
//  of our small language.
//  ( Note: we define our own boolean type because not all C++ compilers
//    have bool built-in yet. )
/////////////////////////////////////////////////////////////////////////////
datatype List<T> = #[]                             // a polymorphic list
                 | #[ T ... List<T> ]         

and      BOOL    = False | True                    // a boolean type

and      Exp     = integer ( int )                 // integers
                 | real    ( double )              // real numbers
                 | string  ( char * )              // strings
                 | boolean ( BOOL )                // booleans
                 | binop   ( BinOp, Exp, Exp )     // binary op expression
                 | unaryop ( UnaryOp, Exp )        // unary op expression 
                 | var     ( Id )                  // variables

and      BinOp   = add | sub | mul | divide | mod  // binary operators
                 | logical_and | logical_or     
                 | eq  | ge | le | lt | gt | ne

and      UnaryOp = uminus | logical_not            // unary operators

and      Stmt    = assign_stmt ( Id, Exp )                // assignments
                 | while_stmt  ( Exp, Stmt )              // while statements
                 | if_stmt     ( Exp, Stmt, Stmt )        // if statements
                 | print_stmt  ( Exp )                    // print statements
                 | block_stmt  ( List<Decl>, List<Stmt> ) // blocks

and      Type    = primitive_type ( Id )                  // type expressions
                 | pointer_type   ( Type )
                 | array_type     { element : Type, bound : Exp }
                 | function_type  { dom : Type, ran : Type }
                 | tuple_type     ( List<Type> ) 
                 | record_type    ( List<LabeledType> ) 

and      Decl    = decl { name : Id, typ : Type }        // declarations

and  LabeledType = labeled_type (Id, Type)               // labeled types

where type Id    = const char *
;   
\end{prop}

\begin{prop}
/////////////////////////////////////////////////////////////////////////////
//  Refine the implementation of the datatypes.
//  The qualifiers may be also declared in the datatype definition.
//  We qualify the datatypes here so that they won't clutter up
//  the equations above.
//
//  All types are declared to be garbage collectable, printable by
//  the printer method and rewritable.
/////////////////////////////////////////////////////////////////////////////
refine List<T>     :: collectable printable rewrite
and    BOOL        :: collectable printable rewrite
and    Exp         :: collectable printable rewrite
and    BinOp       :: collectable printable rewrite
and    UnaryOp     :: collectable printable rewrite
and    Stmt        :: collectable printable rewrite
and    Type        :: collectable printable rewrite
and    Decl        :: collectable printable rewrite
and    LabeledType :: collectable printable rewrite
\end{prop}

\begin{prop}
/////////////////////////////////////////////////////////////////////////////
//  Specify the pretty printing formats. 
/////////////////////////////////////////////////////////////////////////////
and printable 
    False          => "false"
  | True           => "true"
  | integer        => _
  | real           => _
  | string         => "\"" _ "\""
  | var            => _
  | boolean        => _
  | binop          => "(" #2 #1 #3 ")"  // reorder the arguments
  | unaryop        => "(" _ _")"
  | add            => " + "
  | sub            => " - "
  | mul            => " * "
  | divide         => " / "
  | mod            => " mod "
  | logical_and    => " and "
  | logical_or     => " or "
  | eq             => " = "
  | ne             => " <> "
  | gt             => " > "
  | lt             => " < "
  | ge             => " >= "
  | le             => " <= "
  | uminus         => "- "
  | logical_not    => "not "
  | assign_stmt    => _ " := " _ ";"
  | while_stmt     => "while " _ " do" { _ } "end while;"
  | if_stmt        => "if " _ " then" { _ } " else" { _ } "endif;"
  | print_stmt     => "print " _ ";" 
  | block_stmt     => "begin " { _ / _ } "end"
  | primitive_type => _
  | pointer_type   => _ "^"
  | array_type     => "array " bound " of " element 
  | function_type  => dom " -> " ran
  | decl           => "var " name " : " typ ";"
  | labeled_type   => _ " : " _
  | #[]            => ""
  | #[...]         => _ / _
  ;
\end{prop}

\begin{prop}
/////////////////////////////////////////////////////////////////////////////
//  Generate the interfaces to instantiated polymorphic datatypes.
//  These are not strictly necessary since the instantiation is in the
//  same file below.  However, in general the 'instantiate extern' declaration
//  must be placed in the .h files for each instance of a polymorphic
//  datatype.
/////////////////////////////////////////////////////////////////////////////
instantiate extern datatype 
   List<Type>, List<Stmt>, List<LabeledType>, List<Decl>;
\end{prop}

\begin{prop}
/////////////////////////////////////////////////////////////////////////////
//  Now instantiate all the datatypes.
//  As specified in the definition, garbage collector type info and
//  pretty printers will be generated automatically.
/////////////////////////////////////////////////////////////////////////////
instantiate datatype Exp, BOOL, BinOp, UnaryOp, Stmt, Type, Decl, LabeledType,
                     List<Type>, List<Stmt>, List<LabeledType>, List<Decl>;

/////////////////////////////////////////////////////////////////////////////
//  Defines the interface of a rewrite class Simplify.
//  All types that are referenced (directly or indirectly) should be
//  declared in the interface.
/////////////////////////////////////////////////////////////////////////////
rewrite class Simplify
   ( Exp, BinOp, BOOL, UnaryOp, Stmt, Type, Decl, LabeledType,
     List<Decl>, List<Stmt>, List<Type>, List<LabeledType>
   )
{
public:
   Simplify() {}
   // Method to print an error message 
   void error(const char message[]) { cerr << message << '\n'; }
};
\end{prop}

\begin{prop}
/////////////////////////////////////////////////////////////////////////////
//  Now defines the rewrite rules.  These rules will be compiled into 
//  efficient pattern matching code by the translator.  A real life 
//  system will probably have many more rules than presented here.
//  (A machine code generator usually needs a few hundred rules)
//  Currently there are about 60 rules in this class.
/////////////////////////////////////////////////////////////////////////////
rewrite Simplify
{  
   // Type coercion rules from integer -> real
   binop(some_op, integer x, real y): binop(some_op,real(x),real(y))
|  binop(some_op, real x, integer y): binop(some_op,real(x),real(y))

   // Constant folding rules for integers and reals.
|  binop(add,    integer x, integer y): integer(x + y)
|  binop(sub,    integer x, integer y): integer(x - y)
|  binop(mul,    integer x, integer y): integer(x * y)
|  binop(divide, integer x, integer 0): { error("division by zero"); }
|  binop(divide, integer x, integer y): integer(x / y)
|  binop(mod,    integer x, integer 0): { error("modulo by zero"); }
|  binop(mod,    integer x, integer y): integer(x % y)
|  binop(add,    real x,    real y):    real(x + y)
|  binop(sub,    real x,    real y):    real(x - y)
|  binop(mul,    real x,    real y):    real(x * y)
|  binop(divide, real x,    real 0.0):  { error("division by zero"); }
|  binop(divide, real x,    real y):    real(x / y)
|  unaryop(uminus, integer x):          integer(-x)
|  unaryop(uminus, real    x):          real(-x)

// Constant folding rules for relational operators
|  binop(eq,  integer x, integer y):    boolean((BOOL)(x == y))
|  binop(ne,  integer x, integer y):    boolean((BOOL)(x != y))
|  binop(gt,  integer x, integer y):    boolean((BOOL)(x > y))
|  binop(lt,  integer x, integer y):    boolean((BOOL)(x < y))
|  binop(ge,  integer x, integer y):    boolean((BOOL)(x >= y))
|  binop(le,  integer x, integer y):    boolean((BOOL)(x <= y))
|  binop(eq,  real x,    real y):       boolean((BOOL)(x == y))
|  binop(ne,  real x,    real y):       boolean((BOOL)(x != y))
|  binop(gt,  real x,    real y):       boolean((BOOL)(x > y))
|  binop(lt,  real x,    real y):       boolean((BOOL)(x < y))
|  binop(ge,  real x,    real y):       boolean((BOOL)(x >= y))
|  binop(le,  real x,    real y):       boolean((BOOL)(x <= y))

// Constant folding rules for boolean operators
|  binop(logical_and, boolean x, boolean y):  boolean((BOOL)(x && y))
|  binop(logical_or,  boolean x, boolean y):  boolean((BOOL)(x || y))
|  unaryop(logical_not, boolean x):           boolean((BOOL)(! x))

// Simple algebraic laws for integers
|  binop(add, integer 0, x        ):  x
|  binop(add, x,         integer 0):  x
|  binop(sub, x,         integer 0):  x
|  binop(mul, integer 0, x        ):  integer(0)
|  binop(mul, x,         integer 0):  integer(0)
|  binop(mul, integer 1, x        ):  x
|  binop(mul, x,         integer 1):  x
|  binop(divide, x,      integer 1):  x

// Simple algebraic laws for reals
|  binop(add, real 0.0, x        ):  x
|  binop(add, x,         real 0.0):  x
|  binop(sub, x,         real 0.0):  x
|  binop(mul, real 0.0, x        ):  real(0.0)
|  binop(mul, x,         real 0.0):  real(0.0)
|  binop(mul, real 1.0, x        ):  x
|  binop(mul, x,         real 1.0):  x
|  binop(divide, x,      real 1.0):  x
// more ...

// Simple strength reduction (assume CSE will be applied)
|  binop(mul, integer 2, x):  binop(add,x,x)
|  binop(mul, x, integer 2):  binop(add,x,x)
// more ...

// Simple boolean identities
|  binop(logical_and, boolean False, _): boolean(False)
|  binop(logical_and, boolean True, b):  b
|  binop(logical_and, _, boolean False): boolean(False)
|  binop(logical_and, b, boolean True):  b
|  binop(logical_or,  boolean True, _):  boolean(True)
|  binop(logical_or,  boolean False, b): b
|  binop(logical_or,  _, boolean True):  boolean(True)
|  binop(logical_or,  b, boolean False): b
// more ...

// The following rules eliminate unreachable statements. 
|  if_stmt(boolean True, a, _):          a
|  if_stmt(boolean False, _, b):         b
|  #[while_stmt(boolean False, _) ... continuation]: continuation
// more ...
};
\end{prop}

\begin{prop}
/////////////////////////////////////////////////////////////////////////////
//  The main program.
//  We'll test it with a simple program.
/////////////////////////////////////////////////////////////////////////////
int main()
{  
   // Instantiate a rewriting object.
   Simplify simplify;

   // The input is the following block:
   //
   //  var x : int;
   //  var y : int;
   //  var z : array [10 * 30] of int;
   //  begin
   //     x = 0 + y * 1;
   //     while (1 <> 1) y := y + 1;
   //     print (not false);
   //  end
   //
   Stmt prog = 
      block_stmt( #[ decl("x", primitive_type("integer")),
                     decl("y", primitive_type("integer")),
                     decl("z", array_type(primitive_type("integer"),
                                  binop(mul,integer(10), integer(30))
                                         )
                         )
                  ],
                  #[
                   assign_stmt("x", 
                      binop(add,integer(0),
                         binop(mul,var("y"),integer(1)))),
                   while_stmt(binop(ne, integer(1), integer(1)), 
                              assign_stmt("y", 
                                 binop(add,var("y"),integer(1)))),
                   print_stmt(unaryop(logical_not,boolean(False)))
                  ]
                );
   cout << "Before:\n" << prog << "\n\n";
   simplify(prog);
   cout << "After:\n"  << prog << "\n";
   return 0;
}
\end{prop}
\end{document}
