\def\ADLib{{\sf ADLib}}
\def\Prop{{\sf Prop}}
\def\CPP{C{\tt ++}}

\documentstyle{article}
   \title{Garbage Collection in the \ADLib{} Library(Draft)}
   \author{Allen Leung \\ {\tt leun7056@cs.nyu.edu}}

   %%%%%%%%%%%%%%%%%%%%%%%%%%%%%%%%%%%%%%%%%%%%%%%%%%%%%%%%%%%%%%%%%%%%%%%
   %  Readjust the margins
   %%%%%%%%%%%%%%%%%%%%%%%%%%%%%%%%%%%%%%%%%%%%%%%%%%%%%%%%%%%%%%%%%%%%%%%
   \setlength{\textheight}{8.3in}
   \setlength{\textwidth}{6.5in}
   \setlength{\oddsidemargin}{-.10in}
   \setlength{\topmargin}{-.5in}

\begin{document}
   \maketitle
   \begin{abstract} 
      We describe the design and implementation of a garbage collector
based on the Customisable Memory Management framework(CMM)\cite{CMM} in our
\ADLib{} \CPP{} class library.  Like the previous approach, we have 
implemented a mostly copying conservative collector based on the work
of Bartlett\cite{Mostly-copying}.  
 Similar to CMM, our architecture provides a protocol to allow 
multiple garbage collectors using different algorithms to coexist in the 
same memory space.  A few improvements are made to
improve the performance, the flexibility and the functionality of 
our collector: to reduce retention due to false roots identification,
blacklisting\cite{Boehm} is used to identify troublesome heap
addresses; the architecture of the system has been generalized so that it is 
now possible to have multiple instantiations of Bartlett-style heaps; 
finally, object finalization and weak pointer support
are added.  Our collector has been tested on gcc 2.5.8 under Linux 
and SunOS 4.1.3.
   \end{abstract}

   \bibliographystyle{alpha}
\section{Introduction}
   The \Prop{} project involves the design and implementation of 
an environment and an extension language based on \CPP{} 
for compiler construction and program transformation.  
An outgrowth of this project is the \ADLib{} 
\CPP{} class library, which contains an extensive set of support algorithms 
and data structures.  Since a typical compiler manipulates many complex tree,
dag and graph objects with extended lifetime, manual memory management using
\CPP's constructors and destructors, reference counting smart pointers, or
some other techniques is frequently complex and error prone.  
Furthermore, with the advent of new algorithms for garbage collection,
manual memory management techniques do not necessarily provide better 
performance or space utilization.  To make it possible to make use of garbage 
collection as a viable alternative for memory management in
\CPP\cite{Safe-C++-GC}, we have implemented a garbage collection class 
hierarchy as part of the \ADLib{} library.  The class library can be used
directly by the users who'd like to program in \CPP; it can also be
used as part of the runtime system of a highly level language implemented
in \CPP, as we have done for \Prop.

   We have several good reasons to prefer garbage collection over manual 
memory management.  The \Prop{} language contains many declarative constructs 
and extensions such as algebraic datatypes, pattern matching, rewriting, and
logical inference.  When a user programs in \Prop{} using these constructs, 
an applicative style of programming is the most natural paradigm.  

\section{The Framework}

   We base our design on the work on the Customisable Memory Management(CMM)
system\cite{CMM}.  In this framework, multiple independent heaps(including
the usually non-collectable heap) can coexist with each other.
Bartlett's mostly copying garbage collector is used as the primary collector.
CMM extends the work of Bartlett\cite{Mostly-copying} 
by allowing cross heap pointers and unrestricted interior pointers.

  However, all collectable objects are required to derive from a base class 
and reimplement a tracing method, which identifies the internal pointers of 
an object.  This burden is usually quite small and in fact can be
automated from the type definition\footnote{In \Prop, a special keyword
\verb|collectable| is used to identify garbage collected classes and types.
The \Prop{} translator uses this type annotation to derive the appropriate  
tracing methods.}

  One of the major advantages of the CMM framework is that multiple
coorperating collectors can coexist at the same time, which makes it
possible to customize the behavior of each collector with respect to
the lifetime behavior of the objects.  In \cite{CMM}, examples are
given in which the lifetime of certain class of objects exhibit
first-in/first-out behavior.  In this case a special collector can be
written to take full advantage of this property.

\subsection{Our Framework}
   Our framework retains all the benefits and flexibilities of CMM, 
while extending it in several minor but useful ways:  
\begin{itemize}
   \item Like CMM, interior pointers(i.e. pointers to the middle of an object)
and crossheap pointers (i.e. complex data structures linking nodes locating
in multiple logical heaps.) are supported.  Pointers to collectable
objects are non-intrusive: i.e. they are normal \CPP{} pointers 
and not encapsulated in templates.
   \item Also like CMM, we allow multiple garbage collectors using different
algorithms to coexist.  Unlike CMM, however, we allow multiple 
Bartlett-style collectors to be instantiated.  Most of the services 
involving low level page management and object marking have been relegated
to a separate heap manager.   
   \item We provide support for finalization and weakpointers.
   \item We have implemented blacklisting\cite{Boehm} to reduce the chance
of false roots identification.
\end{itemize}

\subsection{The Implementation}

   Our implementation has been completely written from scratch since
we do not desire to utilize copyrighted code and, more importantly,
we have to make a few important architectural changes:
 All low level memory management services, such as management of
the page table and the object bitmap, is now relegated to the class 
{\sf GCHeapManager}.  Collectors now act as clients as of the heap manager
and the Bartlett-style collector no longer has any special status. 

\subsection{Architecture}

   The architecture of the memory management system is partitioned into
a few classes, each responsible for providing distinct services:

\begin{itemize}
   \item {\sf GCHeapManager} --- The heap manager.  The heap manager
    manages the heap table, performs page level
    allocation and deallocation, and provides miscellaneous services
    like blacklisting.  It also manages the object bitmaps.
   \item {\sf GC} --- The base class for all garbage collectors.
    This base class describes the protocol used by all the collector
    classes\footnote{This base class is also inherited from class
    {\sf Mem}, so that it adheres to the \ADLib{} memory management
    protocol.}
   \item {\sf CGC} --- The base class for conservative collectors.
     This class is inherited from class {\sf GC} and implements some
     methods for locating the stack, heap, and static data areas.
   \item {\sf BGC} --- Bartlett-style mostly copying collector.
      This class is inherited from class {\sf CGC} and implements
      the Bartlett mostly copying algorithm.
   \item {\sf MarkSweepGC} --- Mark/sweep style conservative collector.
      This class is inherited from class {\sf CGC} and implements
      a mark/sweep collection algorithm.
   \item {\sf WeakPointerManager} --- The weakpointer manager.
      This class manages the weak pointer table and provides a few
      weak pointer scavenging services for the collectors.
\end{itemize}
  
\section{The Programmatic Interface}

   The programmatic interface to the garbage collector involves two
base classes, {\sf GC} and {\sf GCObject}.  The base class {\sf GC}
provides an uniform interface to all types of collectors and memory
managers while class {\sf GCObject} provides the interface to all 
collectable classes.  Table~\ref{GC-Classes} contains a listing
of the classes in our hierarchy.

\begin{table}
   \begin{center}
      \begin{tabular}{|l|l|} \hline
        Class           & Purpose \\ \hline \hline
        \sf GCObject    & Collectable object base class \\
        \sf GC          & Garbage collector base class \\
        \sf CGC         & Conservative garbage collector base class \\
        \sf BGC         & Bartlett style mostly copying collector \\
        \sf MarkSweepGC & A mark-sweep collector \\
        \sf GCVerifier  & A heap walker that verifies the integrity of a 
                          structure \\
      \hline 
      \end{tabular}
   \end{center}
   \caption{\label{GC-Classes} Garbage Collection Classes.}
\end{table}

Class {\sf GCObject} is extremely simple: it redefines the operators
\verb|new| and \verb|delete| to allocate memory from the default collector,
and declares a virtual method ``\verb|trace|'' to be defined by subclasses
(more on this later.)

Memory for a {\sf GCObject} is allocated and freed using the usually
\verb|new| and \verb|delete| operators.  Of course, freeing memory explicitly
using \verb|delete| is optional for many subclasses of {\sf GC}. 

\begin{verbatim}
   class GCObject {
   public:
      inline void * operator new(size_t n, GC& gc = *GC::default_gc) 
         { return gc.m_alloc(n); }
      inline void * operator new(size_t n, size_t N, GC& gc = *GC::default_gc) 
         { return gc.m_alloc(n > N ? n : N); }
      inline void   operator delete(void *);
      virtual void trace(GC *) = 0;
   };
\end{verbatim}

\subsection{Memory Allocation}

The base class {\sf GC} is slightly more complex, as it has to provide
a few different functionalities.  The first service that class {\sf GC} must 
provide is of course memory allocation and deallocation.  As a time saving
device we can specify what the default collector is using the methods
\verb|get_default_gc| and \verb|set_default_gc|.  Method \verb|GCObject::new| 
will use this collector by default, unless placement syntax is used.
Method \verb|GCObject::delete|, on the other hand, can correctly infer the 
proper collector to use by the address of the object.

The low level methods to allocate and deallocate memory are \verb|m_alloc|
and \verb|free| respectively.  The programmers usually do not have to
use these methods directly. 

The method to invoke a garbage collection of a specific level is 
\verb|collect(int level)|.  This forces an explicit collection.
Method \verb|grow_heap(size_t)| can also be used to explicitly increase
the heap size of a collector.  Depending of the actual behavior
of the subclasses, these methods may have different effects.

\begin{verbatim}
   class GC {
   protected:
      static GC * default_gc;
   public:
      static GC&  get_default_gc()       { return *default_gc; }
      static void set_default_gc(GC& gc) { default_gc = &gc; }
      virtual void * m_alloc   (size_t) = 0;
      virtual void   free      (void *);
      virtual void   collect   (int level = 0) = 0;
      virtual void   grow_heap (size_t) = 0;
      static  void garbage_collect() { default_gc->collect(); }
      virtual void set_gc_ratio(int);
      virtual void set_initial_heap_size (size_t);
      virtual void set_min_heap_growth   (size_t);
   };
\end{verbatim}

\subsection{The GC Protocol}
The collector and collectable objects must cooperate with
each other by abiding to a simple protocol:
\begin{itemize}
   \item All objects that are to be garbage collected must be derived
from {\sf GCObject}.  The application programmer must also supply a
``{\bf tracing}'' method.  The purpose of this method is to identify
all internal pointers to other {\sf GCObject}'s.   This method is not
used by the application programmer directly but only used internally
by the garbage collectors.
   \item The tracing method of each collectable must in turn call
    the tracing method in the class {\sf GC} with each pointer to
    a collectable object that the object owns:
\begin{verbatim}
   class GC {
   public:
      virtual GCObject * trace (GCObject *) = 0;
      inline  void trace (GCObject& obj);
   };
\end{verbatim}
    
     Briefly, the rules are as follows:
   \begin{enumerate}
      \item The tracing method of a collectable {\sf Foo} has the following
         general form:
          \begin{verbatim}
             void Foo::trace(GC * gc) 
             {
                ...
             }
          \end{verbatim} 
      \item If class {\sf Foo} has a member that is a pointer \verb|p|
            to a collectable object of type {\sf Bar}, then add:
           \begin{verbatim}
               p = (Bar)gc->trace(p);
           \end{verbatim}
            to the body of \verb|Foo::trace|. 
     \item  If class {\sf Foo} has a member object
            \verb|x| that is a subclass of {\sf GCObject}, also add:
           \begin{verbatim}
               gc->trace(x);
           \end{verbatim}
            to the body of \verb|Foo::trace|.
     \item  If class {\sf Foo} is derived from a class \verb|Bar| that
            is a subclass of {\sf GCObject}, add:
           \begin{verbatim}
               Bar::trace(gc);
           \end{verbatim}
            to the body of \verb|Foo::trace|.
   \end{enumerate} 
    Notice that these methods can be arranged in any order.
\end{itemize}

This protocol can be used by both copying and non-copying collectors.
In addition, the class {\sf GCVerifier} also uses this protocol to 
walk the heap in order to verify the integrity of a garbage collected 
data structure.

\subsection{Messages and Statistics}
    All garbage collectors use the following protocols for status
reporting and statistics gathering.  
  
\begin{verbatim}
   class GC {
   public:
      enum GCNotify {
         gc_no_notify               = 0
         gc_notify_minor_collection = 1,
         gc_notify_major_collection = 2,
         gc_print_collection_time   = 4,
         gc_print_debugging_info    = 8
      }
      virtual int      verbosity() const;
      virtual void     set_verbosity(int);
      virtual ostream& get_console() const;
      virtual void     set_console(ostream&);
   };
\end{verbatim}

\begin{verbatim}
   class GC {
   public:
      struct Statistics {
         const char *   algorithm;
         const char *   version;
         size_t         bytes_used;
         size_t         bytes_managed;
         size_t         bytes_free;
         struct timeval gc_user_time;
         struct timeval gc_system_time;
         struct timeval total_gc_user_time;
         struct timeval total_gc_system_time;
      }
      virtual Statistics statistics();
   };
\end{verbatim}

    Each collector has a current verbosity level, which can be set
and reset using the methods \verb|set_verbosity(int)| and
\verb|verbosity() const|.  The verbosity level is actually a bit
set containing flags of the type \verb|GC::GCNotify|.  A combination
of these options can be used.
\begin{itemize}
  \item \verb|gc_no_notify| --- no messages at all.
  \item \verb|gc_notify_minor_collection| --- all minor collections
    will be notified by printing a message on the current console.
  \item \verb|gc_notify_major_collection| --- similarly for all major
    collections.
  \item \verb|gc_print_collection_time| --- this option will let the
    collector to print the time spent during collection (and finalization).
  \item \verb|gc_print_debugging_info| --- this option will print additional
    information such as stack and heap addresses during garbage collection.
\end{itemize}

   The current console of a collector is defaulted to the \CPP{} stream
\verb|cerr|.  It can be set and inspected with the methods
\verb|set_console(ostream&)| and \verb|ostream& get_console()| respectively.
For example, the user can redirect all garbage collection messages
to a log file \verb|"gc.log"| by executing the following during initialization:

\begin{verbatim}
   ofstream gc_log("gc.log");
   GC::get_default_gc().set_console(gc_log); 
\end{verbatim}

\subsection{The Bartlett style mostly copying collector}

   Currently a Bartlett-style mostly copying collector has been implemented.
The current version is non-generational but we expect that a generational
version will be available in the near future.

   The Bartlett-style collector is implemented as the concrete class
{\sf BGC}.  Aside from all the services provided by class {\sf GC},
{\sf BGC} also provides the following method:

\begin{verbatim}
   class BGC : public CGC {
   public:
      virtual void set_gc_ratio(int);
      virtual void set_initial_heap_size (size_t);
      virtual void set_min_heap_growth   (size_t);
   }
\end{verbatim}

  The gc ratio refers to the ratio of used heap space versus
total heap space.  Class {\sf BGC} will invoke the garbage collection
if this is exceeded.  The default gc ratio for {\sf BGC} is 75\%.

  The initial size of the heap and the minimal number of bytes to 
request from the system during a heap expansion can be adjusted using 
the methods \verb|set_initial_heap_size| and \verb|set_min_heap_growth|
respectively.  These methods are declared in the {\sf GC}
protocol.   By default, the initial heap size for this collector
is 128K and each heap expansion will increase the heap by 512K. 

If an application uses a lot garbage collected storage it is a good
idea to set the heap size to larger capacity during initialization.
Otherwise, the collector will have to perform more garbage collection
and heap expansions to reach a stable state. 

\subsection{The Mark Sweep collector}

   An alternative to the copying collector is the the mark sweep collector.
Like the previous collector, this collector also uses conservative scanning 
to locate roots.  Unlikely the Boehm collector, type accurate marking
is used through the user supplied tracing method.

   Since the default collector is of the {\sf BGC} variety, the user must
create an instance of {\sf MarkSweepGC} if the mark sweep collector
is to be used.  

\begin{verbatim}
   class MarkSweepGC : public CGC {
   public:
      virtual void set_gc_ratio(int);
      virtual void set_initial_heap_size (size_t);
      virtual void set_min_heap_growth   (size_t);
   }
\end{verbatim}

   The gc ratio in class {\sf MarkSweepGC} determines whether heap
expansion is to be performed after a collection.  The default gc ratio
is 50\%, thus if after garbage collection the ratio of used versus total
heap space exceeds one half, the heap will be expanded. 

   For most purposes the two collectors are interchangeable.  Since
all types of garbage collectors under our framework use the same protocol, 
applications can be written without fixing a specific garbage collection
algorithm before hand.  

\subsection{Finalization} 

   One common \CPP{} idiom uses constructor and destructor for
resource allocation: resources (including memory but may include others,
such as file handles, graphical widgets, etc) that are acquired
in a class constructor are released(finalized) in the class'es destructor.

   There are, however, two opposing views in how finalization should be handled
in a garbage collector.  The first assumes that garbage collection
simulates an infinite amount of memory and so automatic finalization is
inapproriate.   The second takes a more pragmatic view, and assumes that
automatic finalization should be provided since otherwise explicit resource
tracking must be provided by the user for collectable datatypes, making garbage 
collection much less useful than it can be.

   We will refrain from participating in any religious and philosophical
wars on which dogma is the One True Way.  Instead, both types of collectors
are provided.

   By default, class {\sf BGC} does not perform finalization on garbage 
collected objects.  If object finalization is desired then the user
can do one of two things:
\begin{itemize}
   \item Enable the finalization of the default heap by putting
\begin{verbatim}
   GC::get_default_gc().set_finalization(true);
\end{verbatim}
   in the initialization code, or
   \item Instantiate a different instance of {\sf BGC} and
    allocate objects needing finalization from this collector.
    This may be a better method since not all objects need finalization
    and those that do not can still be allocated from the default collector.
    This way we only have to pay for the performance penalty (if any)
    proportional to the usage of finalization.
\end{itemize}

   For example, if a collectable class named {\sf Foo} should be properly
finalized we can declare it in the following manner:

\begin{verbatim}
   extern BGC bgc_f;  // somewhere an instance is defined

   class Foo : public GCObject {
      Resource r;
   public:
      Foo() { r.allocate(); /* allocate resource */ }
     ~Foo() { r.release(); /* release all resource */ }

      // Make object allocate from bgc_f instead
      // of the default non-collectable gc.
      void * operator new (size_t n) { return bgc_f.m_alloc(n); }
   };
\end{verbatim}

   When an instance of class \verb|Foo| is identified as garbage, its
destructor will be called.  [Aside: {\it currently issue such as 
finalization synchronization is not handled directly by the collector.
So if there is synchronization constraints during finalization it must
be handled by the object destructor.  Future versions of \ADLib{} will
provide subclasses with more sophisticated finalization mechanisms.} ]

   Notice that the default \verb|delete| operator of class {\sf BGC} will 
call the object explicitly destructor.  It is a good idea to use
explicit \verb|delete| if finalization is a rare need since this service
involves an explicit scan of the garbage memory, which may degrade performace
with excessive swapping: without finalization the garbage pages will not 
have to be referenced with a copying collector.

   It should also be noted that since the collector is conservative, there
is no guarantee that garbage will be identified as such:
there is no guarantee that all garbage resources will be released.
In general, however, the efficacy of the collector is quite good and
so non-critical or plentiful resources can be safely finalized with
this collector.

\subsection{Weak Pointers}

   It is frequently very useful to be able to keep track of garbage
collectable objects through the use of ``{\bf weak pointers}.''
Collectors ignore the presense of weak pointers to garbage collected
objects; objects with only referencing weak pointers will still be
collected.  Weak pointers to objects that become garbage will
be reset to null. 
 
   Weak pointers are provided thru a smart pointer template 
{\sf WeakPointer}, whose definition is shown below: 

\begin{verbatim}
template <class T> class WeakPointer {
   public:
      inline WeakPointer();
      inline WeakPointer(const WeakPointer<T>& wp); 
      inline WeakPointer(T * ptr); 
      inline WeakPointer<T>& operator = (const WeakPointer<T>& wp);
      inline WeakPointer<T>& operator = (T * ptr);
      inline bool is_null() const;
      inline operator const T * () const;
      inline operator       T * ();
      inline const T * operator -> () const;
      inline       T * operator -> ();
      inline const T&  operator *  () const;
      inline       T&  operator *  ();
   };
\end{verbatim}

\subsection{The Heap Walker}

   There is a simple class called {\sf GCVerify} that can be used
to verify the integrity of a complex allocated data structure.
This is frequently useful during the development of a new collector,
or a complex class derive class of {\sf GCObject.}  

   The interface of this class is shown below.

\begin{verbatim}
   class GCVerifier : protected GC {
   public:
      virtual bool is_valid_pointer          (GCObject *);
      virtual bool is_valid_interior_pointer (GCObject *);
      virtual bool is_valid_structure        (GCObject *);
      virtual bool is_valid_pointer          (GCObject *, GC *);
      virtual bool is_valid_interior_pointer (GCObject *, GC *);
      virtual bool is_valid_structure        (GCObject *, GC *);
      size_t number_of_nodes() const;
   };
\end{verbatim}

   Class {\sf GCVerify} is derived from class {\sf GC} so that the
same object tracing protocol can be used.  Briefly, three different methods 
are provided for verification:
\begin{itemize}
   \item Call to \verb|is_valid_pointer(p,gc)| verifies that
    \verb|p| is a valid pointer to an object within the collector \verb|gc|. 
   \item Call to \verb|is_valid_pointer(p,gc)| verifies that
    \verb|p| is a valid interior pointer to an object 
     within the collector \verb|gc|. 
   \item Call to \verb|is_valid_structure(p,gc)| verifies the entire
     structure reachable by \verb|p| is valid.  This method uses
     \verb|GCObject::trace| to locate the correct sub-structures.
\end{itemize}

   There are alternative versions of the same methods that assumes
the default collector is used.   Furthermore
\begin{itemize}
   \item Method \verb|number_of_nodes()| returns the node count of the
last structure traced using \verb|is_valid_structure|, and 
   \item If \verb|set_verbosity(GC::gc_print_debugging_info)| is used
then error messages will be printed during structure tracing.
\end{itemize}

   \bibliography{gc}
\end{document}
