%%%%%%%%%%%%%%%%%%%%%%%%%%%%%%%%%%%%%%%%%%%%%%%%%%%%%%%%%%%%%%%%%%%%%%%%%%%%%%%
%
%  Here are some abbreviations used within this document
%
%%%%%%%%%%%%%%%%%%%%%%%%%%%%%%%%%%%%%%%%%%%%%%%%%%%%%%%%%%%%%%%%%%%%%%%%%%%%%%%
\newfont{\bsf}{cmssbx10}
\newfont{\largebsf}{cmssbx12}
%\newfont{\nontermfont}{cmssbx10}
\newcommand{\nontermfont}{\it}
\newcommand{\codefont}{\tt}
\newcommand{\symfont}{\bf}

\newcommand{\Prop}{{\bsf Prop}}               % typeset in sans serif
\newcommand{\LargeProp}{{\largebsf Prop}}
\newcommand{\plusplus}{{\tt++}}             % ++
\newcommand{\Cpp}{{C{\tt++}}}               % C++ logo
\newcommand{\Gpp}{{G{\tt++}}}               % G++ logo
\newcommand{\email}{{\tt leunga@cs.nyu.edu}}  % my email address
\newcommand{\web}{{\tt http://valis.cs.nyu.edu:8888/\~{}leunga}} % my web page
\newcommand{\propweb}{\web{\tt/prop.html}}
\newcommand{\Version}{2.3.4}                  % Prop version
\newcommand{\comment}[1]{}

\newcommand{\sectionfont}{}
%\newcommand{\Section}[1]{\section{\sectionfont #1}}
%\newcommand{\Subsection}[1]{\subsection{\sectionfont #1}}
%\newcommand{\Subsubsection}[1]{\subsubsection{\sectionfont #1}}

\newcommand{\Section}[1]{\pagebreak\section{\sectionfont #1}%
   \markright{\usebox{\LINE}\large \thesection. \sc #1}
   }
\newcommand{\Subsection}[1]{\subsection{\sectionfont #1}}
\newcommand{\Subsubsection}[1]{\subsubsection{\sectionfont #1}}

\newcommand{\N}[1]{\mbox{{\nontermfont #1}}}
\newcommand{\T}[1]{{\codefont #1}}
\newcommand{\MORE}[2]{#1\T{#2} \mbox{$\ldots$} \T{#2} #1}
\newcommand{\ALT}[2]{#1\ \T{#2} \mbox{$\ldots$} \T{#2} #1}
\newcommand{\LIST}[1]{\MORE{#1}{,}}
\newcommand{\SEQ}[1]{\MORE{#1}{;}}
\newcommand{\OPT}[1]{{\it [ #1 ]}}
\newcommand{\C}[1]{\mbox{\rm #1}}
\newcommand{\IS}{{\symfont ::=}}
\newcommand{\OR}{\ \mbox{$\symfont \mid$}\ }
\newcommand{\Id}{\N{Id}\ }
\newcommand{\Pat}{\N{Pat}\ }
\newcommand{\Exp}{\N{Exp}\ }
\newcommand{\TypeExp}{\N{Type\_Exp}\ }
\newenvironment{syntax}{\begin{quotation}\noindent\begin{tabular}{lcll}} %
   {\end{tabular}\end{quotation}}

\newcommand{\CLASSDEF}[1]{\T{#1}\index{Classes!{\codefont #1}}}
\newcommand{\INDEX}[1]{\index{#1}}
\newcommand{\NDEF}[1]{\N{#1}\index{Syntax!Non--terminals!{\nontermfont #1}}}
\newcommand{\OPTIONDEF}[1]{{\tt #1}\index{Command line options!{\tt #1}}}
\newcommand{\TDEF}[1]{\T{#1}\index{Syntax!Keywords!{\codefont #1}}}
\newcommand{\INCOMPLETE}{{\em This section is incomplete.}}
\newenvironment{Error}{\begin{center}\begin{tabular}{p{3in}p{3in}}
  \bf Error & \bf Explanation}{\\\hline\end{tabular}\end{center}}
\newcommand{\errormsg}[1]{\\\hline {\tt #1}&}
\newenvironment{Tips}{\noindent {\bf Tips:} \quad}{}

\newcommand{\support}{
   {\small\noindent This work is supported in part by
    an award from Hewlett-Packard Corporation, from IBM corporation, and
    an NYU Research Challenge Grant.
   }}
\newcommand{\warning}{
   {\small\noindent The author will neither assume responsibility for any
   damages caused by the use of this product, nor accept warranty or update
   claims.  This product is distributed in the hope that it will be useful,
   but WITHOUT ANY WARRANTY; without even the implied warranty of
   MERCHANTABILITY or FITNESS FOR A PARTICULAR PURPOSE.

   \small\noindent This product is in the public domain,
   and may be freely distributed.

   \small\noindent {\sf Prop} is a research prototype. 
    The information contained in this document is subject
    to change without notice.
   }}

   \makeindex

%%%%%%%%%%%%%%%%%%%%%%%%%%%%%%%%%%%%%%%%%%%%%%%%%%%%%%%%%%%%%%%%%%%%%%%%%%%%%%
%
%   Beginning of the document
%
%%%%%%%%%%%%%%%%%%%%%%%%%%%%%%%%%%%%%%%%%%%%%%%%%%%%%%%%%%%%%%%%%%%%%%%%%%%%%%
\documentclass{article}
\usepackage{html}

   \pagestyle{myheadings}

   \title{{\Huge \bf Prop \\ 
           Language Reference Manual} \\ 
          {\large \bf (for version \Version)}
         } 
   \author{{\Large Allen Leung} \\ 
           E-mail: \email \\
           \htmladdnormallink{\web}{http://valis.cs.nyu.edu:8888/~leunga}
	   \\
           Courant Institute of Mathematical Sciences \\
           251 Mercer Street \\
           New York, NY 10012 \\
          }

   \setlength{\textwidth}{6.6in}
   \setlength{\evensidemargin}{0in}
   \setlength{\oddsidemargin}{0in}
   \setlength{\textheight}{8.75in}
   \setlength{\topmargin}{-0.5in}
   \setlength{\parskip}{1mm}
   \setlength{\parindent}{3ex}
   \setlength{\itemsep}{0.5ex}

   \newsavebox{\LINE}
   \savebox{\LINE}[0.0in][l]{\rule[-1ex]{\textwidth}{.01in}}
\begin{document}

   \maketitle
   \vspace{\fill}
   %\support

   \begin{abstract}
    This reference manual provides an introduction to
    \Prop, release \Version.  \Prop{} is a multiparadigm extension of \Cpp,  
    and is designed for building high performance compiler 
    and language transformation systems, using pattern matching and rewriting.  
    This guide describes the syntax and semantics of
    the \Prop{} language and describes how to develop programs
    using the \Prop{} to \Cpp{} translator.

   \end{abstract}
   \warning
   \titlepage
   \tableofcontents
   \bibliographystyle{alpha}

\Section{Introduction} \label{sec:intro}
 
   This is the language reference manual for \Prop, a compiler generator.
\Prop{} can be used to generate lexers, parsers, analysis and 
transformation tools using pattern matching and rewriting.  
The \Prop{} language is a {\em multi-paradigm} 
extension of \Cpp\cite{C++} and
\Cpp{} programs generated from \Prop{} specifications can be compiled
into efficient code.

   Version \Version{} of \Prop{} includes the following major features:
\begin{enumerate}
   \item Algebraic datatypes and pattern matching, and automatic
mapping of algebraic datatypes into \Cpp{} class hierarchies.
   \item A LALR(1) parser generator in the tradition of {\it yacc} and
{\it bison}.  Each grammar is encapsulated into a parser class and 
it is possible to have multiple instances of a parser.
   \item Regular expression string matching and lexer generation.
   \item Transformation of algebraic datatypes using conditional
   rewriting.
   \item In addition, using the view mechanism, 
it is possible to treat externally defined C or \Cpp{}
data structures as if they were \Prop{} defined datatypes.  
The full range of pattern matching and rewriting capability can
be used on these externally defined data..
\end{enumerate}

   In addition, we intend to implement/fix-up the following features in the 
near future:
\begin{enumerate}
   \item Persistence.
   \item Pretty printing/reading.
   \item Visualization of data structures using {\em vcg}\cite{vcg,vcg-manual}.
   \item Automatic mapping of graph structure specifications into
\Cpp{} classes.
   \item Graph rewriting system generation.
\end{enumerate}

\Subsection{Availability}

   The latest release of \Prop{} \Version{} is available from
\begin{quotation}
\htmladdnormallink{\propweb}{http://valis.cs.nyu.edu:8888/~leunga/prop.html}
\end{quotation}
New updates of the source code and documentation will be available
from this site.

\Subsection{Organization of this Reference Manual}
   This reference manual is organized as follows.  Section~\ref{sec:general}
overviews the general features of \Prop.
Section~\ref{sec:parser}
describes the lexer and parser specification formalisms.
Section~\ref{sec:datatype} describes the algebraic datatypes and 
pattern matching features.  Section~\ref{sec:tree-rewriting} describes
the tree rewriting mechanism.  Finally, section~\ref{sec:usage} describes
the process of running the \Prop{} translator.

   We'll use the following extended BNF notation to specify the syntax
of the language.  Terminals are typeset in {\tt typewriter} font,
while non-terminals are typeset in {\nontermfont itallics} font.
Given a syntactic class \N{S}, we use \LIST{\N{S}} to denote one
or more occurrences of \N{S} separated by commas.  Similarly,
\SEQ{\N{S}} denotes one or more occurrences of \N{S} separated by
semi-colons.  Slanted brackets are used to denote an optional occurrence
of a syntactic class.   We'll combined the two forms when we
want to denote zero or more occurrences of some term.
For example, \OPT{\LIST{\N{S}}} denotes
zero or more occurrences of \N{S}.

\Section{General Concepts} \label{sec:general}

\Subsection{Syntactic Conventions}

The syntax of \Prop{} is an extension of \Cpp.  Most new constructs
of \Prop{} starts with new keywords, listed below.  
\begin{center}
\tt
\begin{tabular}{lllllll}
applicative & arb & as & bagof & begin & bitfield & card \\
category & classof & collectable & constraint & dataflow & datatype & declare \\
dequeof & dom & domain & edges: & elsif & end & equiv: \\
exists & expect: & feature & finalizable & forall & fun & function \\
functor & graphrewrite & graphtype & implies: & inference & inherited & inline \\
instantiate & is & law & left: & less & lexeme & listof \\
loop & mapof & match & matchall & matchscan & module & multimapof \\
mutable & nodes: & of & persistent & prec: & printable & priqueueof \\
procedure & queueof & ran & refine & relation & rewrite & right: \\
setof & sharing & signature & space: & syntax & synthesized & then \\
time: & topdown & traced & treeparser & tupleof & unifiable & unique \\
view & where & with & xor: & bottomup: & topdown: & before: \\
preorder: & postorder: & cutrewrite & failrewrite & index: \\
\end{tabular}
\end{center}

   In addition, the following new multi-character symbols are added:
\begin{verbatim}
   <->  <=>  :=  :-  .(   .[
   #[   #(   #{  [|  |]   (|  |)  {|  |} 
\end{verbatim}

\Subsection{Basic syntactic classes}


We'll use \NDEF{Integer}, \NDEF{Char}, \NDEF{Real}, \NDEF{String} 
and \NDEF{Id} to denote the syntactic classes of 
integers, characters, real numbers, strings and identifiers, respectively.
As in \Cpp, characters are quoted with the single quotes \verb|'|, while
strings are quoted with double quotes \verb|"|.  An identifier is
any sequence of alphanumeric of characters that begins with a letter.
In addition, we'll use the syntactic class
\NDEF{Stmt} to denote any valid combination of \Cpp{} and \Prop{}
statements.

\Subsection{Literals}

  In addition to the usual \Cpp{} literals, \Prop{} introduces two new type
of literals: {\em quark literals} and {\em regular expression literals}.
Quark literals are of type \T{Quark} and are written as strings
prefixed by the \verb|#| symbol.
\begin{syntax}
  \NDEF{Quark} & \IS & \T{\#}\N{String} \\
\end{syntax}   

Quarks act like atoms in Lisp, in which name equality implies
pointer equality.  In contrast, in \Cpp{} two string literals that are
identical in value do not necessarily reside in the same address.
Given strings \verb|s1| and \verb|s2|, \verb|strcmp(s1,s2) == 0| does
not imply \verb|s1 == s2|.  Quarks are defined in the library include file
\verb|<AD/strings/quark.h>|.

Regular expression literals are similar to string literals except
that they are quoted by slashes \verb|/|.   We'll discuss
the regular expression in section~\ref{sec:regular-expressions}.

\Subsection{The \Prop{} Language}

The basic \Prop{} language is the same as \Cpp{} with the following
syntactic extensions:
\begin{quotation}
\begin{tabular}{lll}
         & \bf Keywords              & \bf Functions \\
  \bsf 1 & \T{datatype} \T{refine} \T{instantiate}
                                     & Algebraic datatype definitions \\
  \bsf 2 & \T{match} \T{matchall}    & Pattern matching \\
  \bsf 3 & \T{rewrite}               & Rewriting \\
  \bsf 4 & \T{lexeme}                & Lexical category definition \\
  \bsf 5 & \T{matchscan}             & Lexical scanning \\
  \bsf 6 & \T{syntax}                & Parser specification \\
  \bsf 7 & \T{graphtype}             & Graph data structure specification \\
  \bsf 8 & \T{graphrewrite}          & Graph rewriting \\
\end{tabular}
\end{quotation}

Like \Cpp, a \Prop{} is typically divided into the {\em specification},
which defines the data structure and its interface, and the
{\em implementation} parts.   \Prop{} specifications should
be placed in a file with a \verb|.ph| suffix, while an implementation
should be placed in a file with a \verb|.pC|\footnote{The suffixes {\tt .pcc},
{\tt .pCpp} etc. can also be used.}.  The translator will convert each
\verb|.p|$xxx$ into a file with the suffix \verb|.|$xxx$.  The translated
output can then be fed to the \Cpp{} compiler for compilation.
\vspace{\fill}

\Section{Lexer and Parser Generation} \label{sec:parser}

   In this section we'll describe the lexer and parser generation
mechanism of \Prop.  Note that the use of these features is not
mandatory: the user can substitute any other lexer/parser generation tools 
or use hand written lexers and parsers.  
On the other hand, higher level of integration is possible when using
\Prop's lexer and parser mechanisms.

\Subsection{Lexer Specification}
   Lexical analyzers are specified using the \verb|matchscan|
statement.  This construct is responsible for generating the actual string
matching DFA.  The actual buffering mechanisms are provided by the
classes \verb|LexerBuffer|, \verb|IOLexerBuffer| and \verb|IOLexerStack|. 
These classes are part of the support library distributed with \Prop.

   We assume that the user is familiar with lexer generator tools like
{\em lex}\cite{lex} or {\em flex}\cite{flex}.

\Subsubsection{Regular expressions} \label{sec:regular-expressions}
   Figure~\ref{fig:regexp} describes the set of current
supported regular expressions in \Prop.  The syntax is similar
to what is found in {\em lex}, or {\em egrep}.

\begin{figure}[htbp]
\begin{center}
  \begin{tabular}{|l|l|} \hline
    $c$                & matches character $c$ if it is not a meta character \\
    $e_1e_2$           & matches $e_1$ then $e_2$ \\
    \verb|.|           & matches any character except \verb|\n| \\
    $\verb|\|c$        & matches escape sequence $c$ \\
    \verb|^|$e$        & matches $e$ at the start of the line \\
    $e_1\verb.|.e_2$   & matches $e_1$ or $e_2$ \\
    $e\verb|*|$        & matches zero or more $e$ \\
    $e\verb|+|$        & matches one or more $e$ \\
    $e\verb|?|$        & matches zero or one $e$ \\
    \verb|(|$e$\verb|)| & grouping \\
    \verb|<<|$C_1, C_2 \ldots C_n$\verb|>>|$e$ & matches $e$ only
         if we are in a context from one of $C_1, C_2, \ldots C_n$ \\
    \verb|{|{\em lexeme}\verb|}| & matches lexeme \\
  \hline
  \end{tabular}
\end{center}
\caption{\label{fig:regexp} Regular expressions.}
\end{figure}

The symbols \verb%\ [ ] ( ) { } << >> * + . - ? |% are meta characters and are
interpreted non-literally.  The escape character \verb|\| 
can be prepended to the meta characters if they occur as literals in context.  

Precedence-wise, meta characters \verb|*|, \verb|+| and \verb|?| bind tighter
than juxtaposition.  Thus the regular expression
\verb|ab*| means \verb|a(b*)|.  Parenthesis can be used to override
the default precedence.

Character classes are of the form as found in {\em lex}: 
(i) $c_1$\verb|-|$c_2$ denotes the range of characters from
$c_1$ to $c_2$; (ii) single (non-meta) characters denote themselves; 
(iii) the meta character \verb|^| can be used to negate the set.
For example, the regular expression \verb|[a-zA-Z][a-zA-Z0-9]*| 
specifies an alphanumeric identifier that must starts with a letter.
Similarly, the regular expression \verb|[^ \t\n]| matches any character
except a space, a tab or a newline.

{\em Lexemes} are simply abbreviated names given to a regular expression
pattern.  They act like macros in {\em lex}.

While a lexer is scanning, it may be in one of many {\em contexts}.
Contexts can be used to group a set of related lexical rules; such rules
are only applicable when the contexts are active.  This makes the
lexer behave like a set of DFAs, with the ability to switch between
DFAs under programmer control.  


\Subsubsection{Lexeme}

Lexemes are defined using the \T{lexeme} declaration in \Prop. 
Its syntax is
\begin{syntax}
  \NDEF{Lexeme\_Decl} & \IS & \TDEF{lexeme} \ALT{\N{Lexeme\_Eq}}{|} \T{;} \\
  \NDEF{Lexeme\_Eq}   & \IS & \Id \T{=} \N{Regexp} \\
\end{syntax}

For example, the following lexeme definition is used in the \Prop{}
translator to define the lexical structure of common lexical items.

\begin{verbatim}
lexeme digits      = /[0-9]+/
     | sign        = /[\+\-]/
     | integer     = /{digits}/
     | exponent    = /[eE]{sign}?{digits}/
     | mantissa    = /({digits}\.{digits}?|\.{digits})/
     | real        = /{mantissa}{exponent}?/
     | string      = /"([^\\"\n]|\\.)*"/ 
     | character   = /'([^\\'\n]|\\.[0-9a-f]*)'/
     | regexp      = /\/([^\/\n*]|\\.)([^\/\n]|\\.)*\//
     ;
\end{verbatim}

Note that regular expression literals are written between
slashes: \verb|/|$re$\verb|/|.  

\Subsubsection{Lexeme class}

Often we wish to group a set of lexemes together into {\em lexeme classes} if 
they logically behave in some uniform manner; for example, if they
act uniformly in a lexical rule.  By grouping related lexemes together
into a class we can refer to them succinctly by their class name.

The syntax of a lexeme class declaration is

\begin{syntax}
  \NDEF{Lexeme\_Class\_Decl} & \IS & \TDEF{lexeme class} 
       \ALT{\N{Lexeme\_Class\_Eq}}{and} \T{;} \\
  \NDEF{Lexeme\_Class\_Eq} & \IS & \Id \T{=} \ALT{\N{Lexeme\_Spec}}{|} \\
  \NDEF{Lexeme\_Spec} & \IS & \N{String} \\
                   & \OR & \Id \N{Regexp} \\
\end{syntax}

The following example,
the lexeme classes \verb|MainKeywords|, \verb|Symbols|, and \verb|Literals| 
are defined.  
\begin{verbatim}
lexeme class MainKeywords =
    "rewrite" | "inference" | "match" | "matchall" | "matchscan"
  | "refine"  | "classof" | "type" | "datatype" | "instantiate"
  | "lexeme"  | "bitfield" | "begin" | "syntax"
  | "dataflow" | "module" | "signature" | "constraint" | "declare"
  | "procedure" | "fun" | "function" | "domain" 
  | "graphtype" | "graphrewrite"

and  Symbols =
    ".." | "..." | "<->" | "::" | "&&" | "||" | "++" | "--" | "->" 
  | "<<" | ">>" | ">=" | "<=" | "+=" | "-=" | "*=" | "/=" | "%=" | "==" 
  | "!=" | "<<=" | ">>=" | "&=" | "|=" | "^=" | "=>" | "<-" | "<=>"
  | ":=" | ":-" | LONG_BAR /\-\-\-\-\-+/

and Literals =
    INT_TOK     /{integer}/
  | REAL_TOK    /{real}/
  | CHAR_TOK    /{character}/
  | STRING_TOK  /{string}/
  ;
\end{verbatim}

\Subsubsection{Tokens}

Lexical tokens are defined using the \verb|datatype| declaration. 
The syntax is as follows.
\begin{syntax}
  \NDEF{Tokens\_Decl} & \IS & \TDEF{datatype} \Id \T{::} \TDEF{lexeme} \T{=} \\
      & & \quad \ALT{\N{Token\_Spec}}{|} \T{;} \\
  \NDEF{Token\_Spec} & \IS & \TDEF{lexeme class} \Id & 
       \C{Include a lexeme class} \\
                  & \OR & \Id \OPT{\N{Regexp}}  &
       \C{Single token spec} \\
\end{syntax}

A token datatype is defined by including one of more lexeme classes,
and by defining stand-alone tokens.  If a lexeme class is included,
all the tokens defined within the lexeme class are included.  A \Cpp{}
\T{enum} type of the same name as the token datatype is generated.
If a token is given an identifier name, then the same name is used
as the \T{enum} literal.  On the other hand, if a string is used
to denote a token, then it can be referred to by prefixing the
string with a dot \verb|.|  For example, the token \verb|"=>"| can
be referenced as \verb|."=>"| within a program.

As an example, the following token datatype definition is used within
the \Prop{} translator.  Here, the keywords are first partitioned into
6 different lexeme classes.  In additional, the tokens \T{ID\_TOK},
\T{REGEXP\_TOK}, etc. are defined.

\begin{verbatim}
datatype PropToken :: lexeme =
    lexeme class MainKeywords
  | lexeme class Keywords
  | lexeme class SepKeywords
  | lexeme class Symbols
  | lexeme class Special
  | lexeme class Literals
  | ID_TOK       /{patvar}/
  | REGEXP_TOK   /{regexp}/
  | QUARK_TOK    /#{string}/
  | BIGINT_TOK   /#{sign}{integer}/
  | PUNCTUATIONS /[\<\>\,\.\;\&\|\^\!\~\+\-\*\/\%\?\=\:\\]/
  ;
\end{verbatim}

\Subsubsection{The \T{matchscan} statement}\label{sec:matchscan}

The \T{matchscan} statement is used to perform tokenization.  The user can
specify a set of string pattern matching rules within a \T{matchscan}
construct. Given a object of class \verb|LexerBuffer|, the \T{matchscan}
statement looks for the rule that matches the longest prefix from
the input stream and executes the action associated with the rule.
Ties are broken by the lexical ordering of the rules.  

The general syntax is as follows:
\begin{syntax}
\NDEF{Matchscan} 
      & \IS & \N{Matchscan\_Mode} \OPT{\N{Context\_Spec}} \T{(} \N{Exp} \T{)} 
             & variant 1 \\
      & &  \T{\{} \ALT{\T{case} \N{Matchscan\_Rule}}{} \T{\}} \\
      & \OR & \N{Matchscan\_Mode} \OPT{\N{Context\_Spec}} 
            \T{(} \N{Exp} \T{)} & variant 2 \\
      & &   \T{\{} \ALT{\N{Matchscan\_Rule}}{|} \T{\}} \\
      & \OR & \N{Matchscan\_Mode} \OPT{\N{Context\_Spec}} \T{(} \N{Exp} \T{)} 
            \T{of} & variant 3 \\
      & & \quad \ALT{\N{Matchscan\_Rule}}{|} \\
      & & \T{end} \T{matchscan} \T{;}\\ 
\NDEF{Matchscan\_Mode} & \IS & 
           \TDEF{matchscan} \OPT{\T{while}} & case sensitive\\
   & \OR & \TDEF{matchscan*} \OPT{\T{while}}& case insensitive\\
\NDEF{Context\_Spec} & \IS & \T{[} \LIST{\N{Id}} \T{]} \\
\NDEF{Matchscan\_Rule} 
      & \IS & \OPT{\T{<<} \ALT{\N{Context}}{,} \T{>>}} \\
      &     & \quad \T{lexeme} \T{class} \N{Id} \T{:} \N{Matchscan\_Action} \\
      & \OR & \OPT{\T{<<} \ALT{\N{Context}}{,} \T{>>}} \\
      &    &  \quad \N{Regexp} \T{:} \N{Matchscan\_Action} \\
\NDEF{Matchscan\_Action} & \IS & \T{\{} \N{Code} \T{\}} \\
          & \OR & \N{Code} & for variant 1 only \\
\end{syntax}

The two different modes of operation are \T{matchscan} and \T{matchscan*},
which respectively match strings case sensitively and insensitively.
The modifier \T{where} may optionally specify that the matching process
should repeat until no rules apply, or the end of stream condition
is reached.  

By default, if no rules apply and if the input stream is non-empty,
then an error has occurred.  The \T{matchscan} statement will
invoke the method \verb|error()| of the \T{LexerBuffer} object by default.

For example, the following is part of the \Prop{} lexer specification.

\begin{verbatim}
   datatype LexicalContext = NONE | C | PROP | COMMENT | ...;

int PropParser::get_token()
{
   matchscan[LexicalContext] while (lexbuf)
   {  
      ...
   |  <<C>> /[ \t\\\014]/:            { emit(); }
   |  <<C>> /(\/\/.*)?\n/:            { emit(); line++; }
   |  <<C>> /^#.*/:                   { emit(); }
   |  <<PROP>> lexeme class MainKeywords: { return ?lexeme; }
   |  <<PROP>> lexeme class SepKeywords: { return ?lexeme; }
   |  <<PROP>> QUARK_TOK:             { return QUARK_TOK; }
   |  <<PROP>> BIGINT_TOK:            { return BIGINT_TOK; }
   |  <<PROP>> REGEXP_TOK:            { return REGEXP_TOK; }
   |  <<PROP>> PUNCTUATIONS:          { return lexbuf[0]; }
   |  <<PROP>> /[ \t\014]/:           { /* skip */ }
   |  <<PROP>> /(\/\/.*)?\n/:         { line++; }
   |  /\/\*/:                         { emit(); set_context(COMMENT); }
   |  <<COMMENT>> /\*\//:             { emit(); set_context(PROP); }
   |  <<COMMENT>> /\n/:               { emit(); line++; }
   |  <<COMMENT>> /./:                { emit(); }
   |  /./: { error("%Lillegal character %c\n", lexbuf[0]); }
   }
}
\end{verbatim}

Here, the lexer is partitioned in multiple lexical contexts: context
\verb|C| deals with \Cpp{} code while context \verb|PROP| deals with
\Prop{} extensions.  The special context \verb|COMMENT| is used
to parse \verb|/* */| delimited comments.  Contexts are changed
using the \verb|set_context| method defined in class \verb|LexerBuffer|.

The special variable \verb|?lexeme| can be used within a rule 
that matches a lexeme class.  For example, within the rule
\begin{verbatim}
   |  <<PROP>> lexeme class MainKeywords:    { return ?lexeme; }
\end{verbatim}
the variable \verb|?lexeme| is bound to 
the token \verb|."rewrite"| if the string ``rewrite'' is matched;
it is bound to the token \verb|."inference"| if the string
``inference'' is matched and so on.

\Subsubsection{Class \T{LexerBuffer}}
We'll next describe the class \T{LexerBuffer} and its subclasses.

Class \CLASSDEF{LexerBuffer} 
is the base class in the lexical buffer hierarchy.
It is defined in the library include file \verb|<AD/automata/lexerbuf.h>|.
This class is responsible for implementing a string buffer for use
during lexical analysis.   

As it stands, it can be used directly if the lexer input is directly
from a string.  Memory management of the buffer is assumed to be handled
by the user. 

The class \T{LexerBuffer} has three constructors.  The default
constructor initializes the string buffer to NULL.  The two
other constructors initialize the string buffer to a string given
by the user.  In the case when the length is not supplied, the buffer
is assumed to be \verb|'\0'|-terminated.  The two \verb|set_buffer| methods
can be used to set the current string buffer.  Notice that
all lexical analysis operations are done in place.  The user should
not alter the string buffer directly, but should use the interface
provided by this class instead.

\begin{verbatim}
class LexerBuffer {
public:
   LexerBuffer();    
   LexerBuffer(char *);    
   LexerBuffer(char *, size_t);    
   virtual ~LexerBuffer();
   virtual void set_buffer (char *, size_t);
           void set_buffer (char *);
};
\end{verbatim}

The following methods are used access the string buffer.
Method \T{capacity} returns the size of the buffer.  Method
\T{length} returns the length of the current matched token.
Methods \T{text} can be used to obtain a point to location
of the current matched token.  The string returned is guaranteed
to be \verb|'\0'|-terminated.  Methods \T{operator []} return
the $i$th character of the token.  Finally, method \T{lookahead}
returns the character code of the next character to be matched.

\begin{verbatim}
   int capacity () const;
   int length   () const;
   const char * text () const;
         char * text ();
   char  operator [] (int i) const;
   char& operator [] (int i);
   int lookahead () const;
   void push_back (int n) 
\end{verbatim}

In addition to the string buffer, the class \T{LexerBuffer} keeps
track of two additional types of information: the current context
of the DFA, and whether the next token starts at the
beginning of the line, or in our terminology, whether it is
{\em anchored}.  These are manipulated with the following methods:

\begin{verbatim}
   int context    () const;
   void set_context (int c = 0);
   Bool is_anchored() const;
   void set_anchored(Bool a = true);
\end{verbatim}

Finally, the following methods should be redefined by subclasses
to alter the behavior of this class.  By default, the class \T{LexerBuffer}
calls \T{fill\_buffer()} when it reaches the end of the string; subclasses
can use this method to refill the buffer and return the number of
characters read.  Currently, \T{fill\_buffer} is defined to do nothing 
and return 0.  When it reaches the end of the file 
(i.e. when \T{fill\_buffer()} fails to refill the buffer and the 
scanning process finishes), method \T{end\_of\_file} is called.
Currently, this is a no-op.  Finally, the error handling routine
\T{error()} is called with the position of the beginning and the end
of the buffer in which the error occurs.  By default, this routine
prints out the buffer. 

\begin{verbatim}
protected:           
   virtual size_t fill_buffer();
   virtual void   end_of_file();
   virtual void   error(const char * start, const char * stop);
\end{verbatim}

\Subsubsection{Class \T{IOLexerBuffer}}

  Class \CLASSDEF{IOLexerBuffer} 
is a derived class of \T{LexerBuffer}.  It 
automatically manages an internal buffer and receives input from
an \T{istream}.  It is defined in the file \verb|<AD/automata/iolexerbuf.h>|

This class provides the following additional features:
Constructor \verb|IOLexerBuffer(istream&)| ties the input to a
stream.  If the default constructor is used, then it automatically
ties the input to the stream \verb|cin|.  The method \verb|set_stream|
can be used to set the stream to some other input. 

\begin{verbatim}
class IOLexerBuffer : public LexerBuffer {
   size_t    buffer_size;  // size of the buffer
   istream * input;        // input stream
public:
   IOLexerBuffer();    
   IOLexerBuffer(istream&);    
   virtual ~IOLexerBuffer();
   virtual void set_stream (istream&);
};
\end{verbatim}

By default, class \verb|IOLexerBuffer| reads new data from the input
stream using a line buffering discipline.  This mechanism is suitable
for interactive, but for other applications it may be more efficient
to use block buffered input.  The protected method 
\verb|read_buffer| controls this reading behavior; it
is called whenever the data in the string buffer has been
consumed and new input is needed from the stream.  It is
passed the position of the buffer and its remaining capacity,
and it returns the number of new characters that are read.  Subclasses
can redefine this method. 

\begin{verbatim}
protected:
   virtual size_t read_buffer(char *, size_t);
\end{verbatim}

\Subsubsection{Class \T{IOLexerStack}}

Class \CLASSDEF{IOLexerStack}
is a derived class of \T{LexerStack}.  It provides
a mechanism of reading from a stack of \verb|istream|'s.  Streams
can be pushed and popped from the stack.  The next token is 
obtained from the stream from the top of the stack.  The class allows
allows easy implementation of 
constructs such as the \verb|#include| file mechanism of the C preprocessor.

The interface of this class is listed below:
\begin{verbatim}
class IOLexerStack : public IOLexerBuffer {
public:
   IOLexerStack();    
   IOLexerStack(istream&);    
   virtual ~IOLexerStack();
   
   virtual void     push_stream (istream&);
   virtual istream& pop_stream  ();
};
\end{verbatim}

\Subsection{Parser Specification}

Parsers are specified as a two phase process:
\begin{enumerate}
   \item First a {\em syntax class} is defined.  A syntax class
declaration is like a normal \Cpp{} class declaration and has a similar
syntax, except that \Prop{} will also generate the interface of the parser. 
All parsers that \Prop{} generates are encapsulated within a class.  This
makes it easy for the programmer to add additional data for parsing
use, and to have multiple instances of a parser. 
   \item Secondly, the grammar of the language is defined in a
{\em syntax} declaration as a set of productions, 
in a syntax similar to that of {\em yacc}.
\end{enumerate}

We'll describe these two phases.

\Subsubsection{Syntax class}

A syntax class definition specifies an object class that encapsulates
a parser.  Its syntax is the same as the usual \Cpp{} class definition,
except that the prefix \T{syntax} is used: 

\begin{syntax}
\NDEF{Syntax\_Class\_Decl} & \IS & \TDEF{syntax class} \Id 
   \OPT{\T{:} \N{Inherit\_List}} \T{\{} \N{Class\_Body} \T{\}} \T{;} \\
\end{syntax}

This specification automatically generate a class with the following interface
(assuming that the parser class in question is called \T{ParserClass})
\begin{verbatim}
public:  
   ParserClass();  // constructor
public:
   virtual void parse();
\end{verbatim}

The constructor and the method \verb|parse()| will be automatically
generated later. 

A parser class expects the following method to be defined:
\begin{verbatim}
   int get_token();
\end{verbatim}
\noindent The function of \verb|get_token| is to return the
next token from the lexer stream whenever it is called.
It should return \verb|EOF| if the stream is empty.

\Subsubsection{Syntax declaration}

The grammar of a language is specified in the \T{syntax} declaration.
Its syntax is listed as follows:

\begin{syntax}
\NDEF{Syntax\_Decl} & \IS & \TDEF{syntax} \Id \T{\{} \\
    & & \quad \OPT{\MORE{\N{Precedence\_Decl}}{}} \\
    & & \quad \OPT{\N{Expect\_Decl}} \\
    & & \quad \OPT{\MORE{\N{Production\_Rule}}{}} \\
    & & \T{\}} \\
\end{syntax}

The name of a \T{syntax} declaration should match that
of a \T{syntax} \T{class}.  A \T{syntax} declaration is divided
into three parts: (i) the operator precedence section, (ii) the
\T{expect:} $n$ section, and (iii) the grammar section.

\Subsubsection{Precedence}

Operator precedences are specified using precedence declarations.
The syntax is: 

\begin{syntax}
\NDEF{Precedence\_Decl} 
    & \IS & \TDEF{left:} \N{Integer} \ALT{\N{Operator}}{} \T{;} 
          & left associative operators \\
    & \OR & \TDEF{right:} \N{Integer} \ALT{\N{Operator}}{} \T{;} 
    & right associative operators \\
\NDEF{Operator} & \IS & \N{Char}   & single character operator \\
             & \OR & \N{String} & multi-character operator \\
             & \OR & \N{Cons}   & constructor \\
\end{syntax}

The integer associated with a precedence declaration is the precedence
level of an operator.  The higher the level of an operator, the {\em lower} 
its precedence.

For example, the following set of precedences are used in the \Prop{}
language parser:
\begin{verbatim}
left: 23 "as";
left: 22 "::";
left: 21 "||";
left: 20 "equiv:";
left: 19 "xor:";
left: 18 "implies:";
left: 17 "&&" "and";
right: 16 "|=" "&=" "^=" "<<=" ">>=";
right: 15 '=' ":=" "+=" "-=" "*=" "/=" "%=";
left: 14 '|';
left: 13 ':';
left: 12 ';';
left: 11 '^';
left: 10 '&';
left: 9 "==" "!=";
left: 8 '<' '>' ">=" "<=";
left: 7 "<<" ">>";
left: 6 '+' '-' "with" "less";
left: 5 '*' '/' '%';
left: 4 "++" "--";
left: 3 '!' '~' "arb" "card" "dom" "ran";
left: 2 '[' ']' '{' '}' '.' "->" ;
\end{verbatim}

\Subsubsection{{\tt expect:} $n$}

The declaration \T{expect} $n$ specifies that there should be $n$
shift/reduce errors in the grammar.  If the actual grammar deviates
from this number, a warning will be generated.  The alternative
form \T{expect:} \T{\_} suppresses all warnings.  The syntax is:

\begin{syntax}
\NDEF{Expect\_Decl}    & \IS & \TDEF{expect:} \N{Integer} \T{;}
    & expects $n$ shift/reduce errors \\
                     & \OR & \T{expect:} \T{\_} \T{;}
    & expects many shift/reduce errors \\
\end{syntax}

This declaration is optional.  If omitted, warnings will be printed
if parser conflicts are found during parser generation.

\Subsubsection{Production rules}


The syntax of the production rules is as follows: 

\begin{syntax}
\NDEF{Production\_Rule} & \IS & 
  \Id \OPT{\TypeExp} \T{:} \\
  & & \quad \OPT{\ALT{\N{One\_Alt}}{|}} \T{;} \\
\NDEF{One\_Alt}  & \IS & \OPT{\ALT{\N{Symbol}}{}} & one alternative \\
\NDEF{Symbol} 
    & \IS & \Id        & non-terminal \\
    & \OR & \N{Cons}   & terminal (a datatype constructor) \\
    & \OR & \N{String} & terminal (a datatype constructor) \\
    & \OR & \N{Char}   & terminal (from the ASCII character set) \\
    & \OR & \T{?}      & the error terminal \\
    & \OR & \T{\$}     & the end of file terminal \\
    & \OR & \T{\{} \N{Code} \T{\}} & embedded actions \\
\end{syntax}

Each rule specifies a set of productions of the form 
\[ \begin{array}{lcl}
    lhs &  :  & A_{11} A_{12} \ldots A_{1n_1} \\
        & \OR & A_{21} A_{22} \ldots A_{2n_2} \\
        & \vdots & \\
        & \OR & A_{m1} A_{m2} \ldots A_{mn_m} \\
   \end{array}
\]
where $lhs$ is the non-terminal and $A_{ij}$'s are a mixture of 
terminals, non-terminals and embedded actions.  If synthesized
attributes are used, then the type of the s-attribute should be
annotated next to the lhs non-terminal.

Terminals can be specified in a few ways: (i) a character literal
is a predefined terminal of the same name, (ii) a string literal
is matched with a \T{datatype} or a \T{lexeme} \T{class} constructor
of the same name, (iii) an identifier is a terminal if there is a 
\T{datatype} constructor of the same name.  Any other identifiers
are assumed to be non-terminals.

For example, the following are two sets of production rules used in the 
\Prop{} translator itself:
\begin{verbatim}
ty Ty:  simple_ty               { $$ = $1; }
|       ty '=' exp              { $$ = DEFVALty($1,$3); }
|       ty '*'                  { $$ = mkptrty($1); }
|       ty '&'                  { $$ = mkrefty($1); }
|       ty '[' ']'              { $$ = mkptrty($1); }
|       ty '[' exp ']'          { $$ = mkarrayty($1,$3); }
;
exp Exp:
        app_exp                 { $$ = $1; }
|       exp '+' exp             { $$ = BINOPexp("+",$1,$3); }
|       exp '-' exp             { $$ = BINOPexp("-",$1,$3); }
|       exp '*' exp             { $$ = BINOPexp("*",$1,$3); }
|       exp '/' exp             { $$ = BINOPexp("/",$1,$3); }
|       exp '%' exp             { $$ = BINOPexp("%",$1,$3); }
|       exp '^' exp             { $$ = BINOPexp("^",$1,$3); }
|       exp "+=" exp            { $$ = BINOPexp("+=",$1,$3); }
|       exp "-=" exp            { $$ = BINOPexp("-=",$1,$3); }
|       exp "*=" exp            { $$ = BINOPexp("*=",$1,$3); }
|       exp "/=" exp            { $$ = BINOPexp("/=",$1,$3); }
|       exp "%=" exp            { $$ = BINOPexp("%=",$1,$3); }
\end{verbatim}
\noindent {\em etc $\ldots$}

Here, \T{ty} is the non-terminal of this production.  The type of
the synthesized attribute of this rule is \T{Ty}, and it is
written next to the non-terminal.  Like {\em yacc}, the synthesized
attribute of a rule can be referenced using the \verb.$. variables: 
variable \TDEF{\$\$} denote the synthesized attribute of the current rule,
while \TDEF{\$$n$} where $n$ is 1, 2, 3, \ldots denote the synthesized
attribute of the $n$th rhs symbol of the current rule.

\Subsubsection{Parser report}

If the command line option \OPTIONDEF{-r} is used, then
in addition to generating the code for the grammar, the translator
will also generate a comprehensive report of the grammar and
the resulting LALR(1) automaton.  The amount of details in this
report can be varied using the the verbose option
\OPTIONDEF{-v}:

\begin{description}
   \item[{\tt -v1}]  Print full information for all LR(0) states. 
   \item[{\tt -v2}]  Also print out the lookahead set of a reduction rule.
This may be useful for tracking down shift/reduce and reduce/reduce conflicts.
   \item[{\tt -v3}]  Do both \verb|-v1| and \verb|-v2|.
\end{description}

\Subsubsection{Interfacing with the generated lexer}

The easiest way of interfacing with a lexer is to embed
an object of class \T{LexerBuffer}, \T{IOLexerBuffer}, or \T{IOLexerStack}
within a syntax class\footnote{Alternatively, inheritance may be used.}.
We can then define the tokenization method
\verb|get_token| using the \T{matchscan} statement described
in section~\ref{sec:matchscan}.

\Subsection{Debugging Parsers}

Currently, only the most rudimentary debugging facilities have been
implemented for parser.
The user can define the macro \verb|DEBUG_|$C$, where $C$ is the
name of a syntax class before the \T{syntax} definition.  This will activate
the parser debugging code.  During parsing, the reductions that are taken
will be printed to \verb|cerr|.

\Section{Algebraic Datatypes and Pattern Matching} \label{sec:datatype}

    \Prop{} implements algebraic datatypes and pattern matching in the
style of Standard ML\cite{SML}.  Tree, DAG and even graph structures can be
specified as a set of datatype equations in algebraic datatype specifications.
\Prop{} then proceeds to translate these into concrete \Cpp{} classes.  
This makes it very easy to implement complex data structures.

   Algebraic datatypes can be manipulated and transformed
using \Prop{} pattern matching constructs, which decompose a datatype
value into its constituents.  Complex patterns involving multiple
objects can be specified.  These are translated into efficient 
testing and selection code in \Cpp.

In the following we shall give a brief overview of the pattern matching
features of \Prop.  For most users of modern declarative languages many
of these features are already familiar constructs.
 
\Subsection{A quick tour of pattern matching}
 
   Algebraic datatypes are specified using \verb|datatype| definitions,
which define the inductive structure of one of more types using
a tree-grammar like syntax.  
%In addition, pretty printers,
%lexical scanners, parsers, persistence I/O methods and
%garbage collection inferences can also be specified with
%additional options in the same construct.  
When a datatype is declared,
the following operations are implicitly defined by the datatype compiler:
(1) the constructors for all the variants of a type;
(2) the identity test operator \verb|==|, and the assignment
    operator \verb|=| for this type; and
(3) the member functions needed to decompose a datatype value during
    pattern matching.
 
   We'll select the internals of a compiler for a simplified imperative
language as the running example in this section.
Suppose that in this language an expression is composed of
identifiers, integer constants and the four arithmetic operators.
Then the structure of the abstract syntax tree can be specified as follows:
 
\begin{verbatim}
   datatype Exp = INT (int)
                | ID  (const char *)
                | ADD (Exp, Exp)
                | SUB (Exp, Exp)
                | MUL (Exp, Exp)
                | DIV (Exp, Exp)
                ;
\end{verbatim}

Abstract syntax of an expression such as {\em a * b - 17}
can be constructed directly in a prefix syntax, directly mirroring
that of the definition.  The \Prop{} datatype
compiler will automatically generate a \Cpp{} class hierarchy to represent
the variants of type \verb|Exp|.  Datatype constructor
functions(not to be mistaken with \Cpp's class constructors) will
also be automatically generated using the same names as the variants.
 
\begin{verbatim}
   Exp formula = ADD(MUL(ID("a"),ID("b")),INT(17));
\end{verbatim}
 
Datatype values can be decomposed using the \verb|match| statement, which
can be seen as a generalization of \C's \verb|switch| construct.  Pattern
matching is a combination of conditional branching and value
binding.  For example, a typical evaluation function for the type \verb|Exp|
can be written as in the following example.  Notice that each arm of
a \verb|case| is in fact a pattern(with optional variables)
mirroring the syntax of a datatype.  The pattern variables(written
with the prefix \verb|?| in the sequel) of a matching
arm is {\em bound} to the value of the matching value, which can be
subsequently referenced and modified in the action of an arm.
 
\begin{verbatim}
int eval (Exp e, const map<const char *, int>& env)
{  match (e)
   {  case INT ?i:        return ?i;
      case ID  ?id:       return env[?id];
      case ADD (?e1,?e2): return eval(?e1,env) + eval(?e2,env);
      case SUB (?e1,?e2): return eval(?e1,env) - eval(?e2,env);
      case MUL (?e1,?e2): return eval(?e1,env) * eval(?e2,env);
      case DIV (?e1,?e2): return eval(?e1,env) / eval(?e2,env);
   }
}
\end{verbatim}
 
\Subsubsection{Pattern matching versus object-oriented style}
 
  Although a comparable evaluation function can be written
in object oriented style using late binding, as in below, in general pattern
matching is much more powerful than late binding in \Cpp, which only
allows dispatching based on the type of one receiver.
 
\begin{verbatim}
  // Class definitions
  class Exp {
  public:
     virtual int eval(const map<const char *, int>& env) const = 0;
  };
  class INT : Exp {
     int i;
  public:
     int eval(const map<const char *, int>& env);
  };
  class ID : Exp {
     const char * id
  public:
     int eval(const map<const char *, int>& env);
  };
  ...
 
  // Member functions
  int INT::eval(const map<const char *, int>& env) const { return i; }
  int ID ::eval(const map<const char *, int>& env) const { return id; }
  int ADD::eval(const map<const char *, int>& env) const
     { return e1->eval(env) + e2->eval(env); }
  int SUB::eval(const map<const char *, int>& env) const
     { return e1->eval(env) - e2->eval(env); }
  int MUL::eval(const map<const char *, int>& env) const
     { return e1->eval(env) * e2->eval(env); }
  int DIV::eval(const map<const char *, int>& env) const
     { return e1->eval(env) / e2->eval(env); }
\end{verbatim}

For example, in the following function we use nested patterns,
non-linear patterns (i.e. patterns with multiple occurrences of
a pattern variable), and guards to perform algebraic simplification
of an expression.  Although the patterns are relatively simple in
this example, in general arbitrarily complex patterns may be used.
 
\begin{verbatim}
Exp simplify (Exp redex)
{  // recursive traversal code omitted ...
 
   // match while repeats the matching process
   // until no more matches are found.
   match while (redex)
   {  ADD(INT 0,  ?x):     { redex = ?x; }
   |  ADD(INT ?x, INT ?y): { redex = INT(?x+?y); }
   |  ADD(?x,     INT 0)   { redex = ?x; }
   |  SUB(?x,     INT 0):  { redex = ?x; }
   |  SUB(?x,     ?x):     { redex = INT(0); }
   |  SUB(INT ?x, INT ?y): { redex = INT(?x-?y); }
   |  MUL(INT 0,  ?x):     { redex = INT(0); }
   |  MUL(?x,     INT 0):  { redex = INT(0); }
   |  DIV(?x,     ?x):     { redex = INT(1); }
        // don't divide by zero.
   |  DIV(INT ?x, INT ?y) | ?y != 0: { redex = INT(?x/?y); }
   |  ...
   }
   return redex;
}
\end{verbatim}
 
Pattern matching in \Prop{} is not restricted to one datatype at
a time.  In the following example, we use matching on multiple values to
define equality on expressions inductively.  For variety, we'll
use the \verb|fun| variant of \verb|match|, which defines a function
in rule form.  Notice that the last case
of the match set uses {\em wild cards} \verb|_| to catch all the other
non-equal combinations.  Since \Cpp{} does not provide multiple dispatching,
implementing binary (or $n$-ary) matching operations on variant datatypes are
in general cumbersome and verbose in object-oriented style. In contrast,
using an applicative pattern matching style many manipulations and
transformations on variant datatypes with tree-like or graph-like structure
can be expressed succinctly.
 
\begin{verbatim}
fun equal INT ?i,     INT ?j: bool: { return ?i == ?j; }
  | equal ID  ?a,     ID  ?b:       { return strcmp(a,b) == 0; }
  | equal ADD(?a,?b), ADD(?c,?d):   { return equal(?a,?c) && equal(?b,?d); }
  | equal SUB(?a,?b), SUB(?c,?d):   { return equal(?a,?c) && equal(?b,?d); }
  | equal MUL(?a,?b), MUL(?c,?d):   { return equal(?a,?c) && equal(?b,?d); }
  | equal DIV(?a,?b), DIV(?c,?d):   { return equal(?a,?c) && equal(?b,?d); }
  | equal _,          _:            { return false; }
  ;
\end{verbatim}

\Subsubsection{More examples} \label{sec:Wff}
 
  As another example, we can specify the term structure of
{\em well-formed formulas} in proposition calculus as follows.  Notice
that the constructors \verb|F| and \verb|T| are nullary.
 
\begin{verbatim}
   datatype Wff = F
                | T
                | Var     (const char *)
                | And     (Wff, Wff)
                | Or      (Wff, Wff)
                | Not     (Wff)
                | Implies (Wff, Wff)
                ;
\end{verbatim}
 
Datatypes that are parametrically polymorphic, such as lists and trees, can
be defined by parameterizing them with respect to one of more types.
For example, both lists and tree below are parametric on one type argument
\verb|T|.
 
\begin{verbatim}
   datatype List<T> = nil
                    | cons(T, List<T>);
   datatype Tree<T> = empty
                    | leaf(T)
                    | node(Tree<T>, T, Tree<T>);
 
   List<int> primes = cons(2,cons(3,cons(5,cons(7,nil))));
   List<int> more_primes = cons(11,cons(13,primes));
   Tree<char *> names = node(leaf("Church"),"Godel",empty);
\end{verbatim}
 
As a programming convenience, \Prop{} has a set of built-in list-like
syntactic forms.  Unlike languages such as ML, however,
these forms are not predefined to be any specific list types.  Instead, it is
possible for the user to use these forms on any datatypes with
a natural binary {\em cons} operator and a nullary {\em nil} constructor.
For instance, the previous list datatype can be redefined as follows:
 
\begin{verbatim}
   datatype List<T> = #[] | #[ T ... List<T> ];
   List<int> primes = #[ 2, 3, 5, 7 ];
   List<int> more_primes = #[ 11, 13 ... primes ];
   List<char *> names = #[ "Church", "Godel", "Turing", "Curry" ];
 
   template <class T>
      List<T> append (List<T> a, List<T> b)
      {  match (a)
         {  case #[]:          return b;
            case #[hd ... tl]: return #[hd ... append(tl,b)];
         }
      }
\end{verbatim}
The empty list is written as \verb|#[]|, while {\em cons(a,b)}
is written as \verb|#[ a ... b ]|.   An expression of the special form
\verb|#[a, b, c]|, for instance, is simple syntactic sugar for
repeated application of the {\em cons} operator, i.e.
 
\begin{verbatim}
    #[a, b, c] == #[a ... #[ b ... #[ c ... #[] ] ] ].
    #[a, b, c ... d ] == #[a ... #[ b ... #[ c ... d ] ] ].
\end{verbatim}
 
List-like special forms are not limited to datatypes with only two variants.
For example, we can define a datatype similar in structure to S-expressions
in {\em Lisp} or {\em scheme}.  Here's how such as datatype may be
defined\footnote{for simplicity, we'll use a string representation for atoms
instead of a more efficient method.}:
 
\begin{verbatim}
   datatype Sexpr = INT    (int)
                  | REAL   (double)
                  | STRING (char *)
                  | ATOM   (const char *)
                  | #()
                  | #( Sexpr ... Sexpr )
   where type Atom = Sexpr  // synonym for Sexpr
   ;
\end{verbatim}
 
   With this datatype specification in place, we can construct values
of type \verb|Sexpr| in a syntax close to that of {\em Lisp}.  For example,
we can define lambda expressions corresponding to the combinators $I$, $K$
and $S$ as follows:
 
\begin{verbatim}
   Atom LAMBDA = ATOM("LAMBDA");
   Atom f      = ATOM("f");
   Atom x      = ATOM("x");
   Atom y      = ATOM("y");
   Atom NIL    = #();
 
   Sexpr I = #(LAMBDA, #(x), x);
   Sepxr K = #(LAMBDA, #(x), #(LAMBDA, #(y), x));
   Sepxr S = #(LAMBDA, #(f),
                #(LAMBDA, #(x),
                   #(LAMBDA, #(y), #(#(f,x), #(g,x)))));
\end{verbatim}
 
Similar to list-like forms, vector-like forms are also available.
This addresses one of the flaws of the \Cpp{} language, which lacks
first class array constructors.  
Vectors are simply homogeneous arrays whose sizes
are fixed and are determined at creation time\footnote{For efficiency,
bounds checking is {\em not} performed.}.   Random access within
vectors can be done in constant time.  Unlike lists, however, the
prepending operation is not supported.   Vectors literals are delimited
with the composite brackets \verb!(| ... |)!, \verb![| ... |]!, or
\verb!{| ... |}!.  In the following example the datatype \verb|Exp|
uses vectors to represent the cooefficients of the polynomials:
 
\begin{verbatim}
   datatype Vector<T> = (| T |);
   datatype Exp = Polynomial (Var, Vector<int>)
                | Functor (Exp, Vector<Exp>)
                | Atom (Var)
                | ...
   where type Var = const char *;
   Exp formula = Polynomial("X", (| 1, 2, 3 |));
\end{verbatim}
 
Commonly used patterns can be given synonyms so that they can be readily
reused without undue repetition.  This can be accomplished by defining pseudo
datatype constructors to stand for common patterns using
{\em datatype law} definitions.  For example, the following set of
laws define some commonly used special forms for a {\em Lisp}-like language
using the previously defined \verb|Sexpr| datatype.
 
\begin{verbatim}
   datatype law inline Lambda(x,e)  = #(ATOM "LAMBDA", #(x), e)
              | inline Quote(x)     = #(ATOM "QUOTE", x)
              | inline If(a,b,c)    = #(ATOM "IF", a, b, c)
              | inline Nil          = #()
              | inline ProgN(exprs) = #(ATOM "PROGN" ... exprs)
              | SpecialForm  = #(ATOM ("LAMBDA" || "IF" ||
                                       "PROGN" || "QUOTE") ... _)
              | Call(f,args) = ! SpecialForm && #(f ... args)
              ;
\end{verbatim}
 
Note that the pattern \verb|SpecialForm| is meant to match all
special forms in our toy language: i.e. {\em lambdas}, {\em ifs}, {\em progn}'s
and {\em quotes}.  The pattern disjunction connective \verb.||. is used
to link these forms together.  Since we'd like the \verb|Call| pattern
to match only if the S-expression is not a special form, we use
the pattern negation and conjunction operators, \verb|!| and \verb|&&|
are used to screen out special forms.  


With these definitions in place,
an interpreter for our language can be written thus:
 
\begin{verbatim}
   Sexpr eval (Sexpr e)
   {  match (e)
      {  Call(?f,?args):     { /* perform function call */ }
      |  Lambda(?x,?e):      { /* construct closure */ }
      |  If(?e,?then,?else): { /* branching */ }
      |  Quote(?x):          { return ?x; }
      |  ...:                { /* others */ }
      }
   }
\end{verbatim}


As an interesting note, the special form pattern can also be rewritten
using regular expression string matching, as in the following:
 
\begin{verbatim}
 datatype law SpecialForm = #(ATOM /LAMBDA|IF|PROGN|QUOTE/ ... _)
\end{verbatim}
 
In addition, since the law constructors \verb|Lambda|, \verb|Quote|,
\verb|If|, \verb|Nil| and \verb|ProgN| are defined with the \verb|inline|
keyword, these constructors can be used to within expressions as abbreviates
for their rhs definitions.  For example, writing
\verb|Lambda(x,x)| is the same writing \verb|#(ATOM("Lambda"),#(x),x)|.

\Subsubsection{Variants of match}
 
Aside from the usual plain pattern matching constructs, a few variants of the
\verb|match| construct are offered.  We'll briefly enumerate a few of these:
\begin{itemize}
  \item The \verb|matchall| construct is a variant of \verb|match| that
   executes all matching rules(instead of just the first one) in sequence.
  \item Each rule of a \verb|match| statement can have associated cost
    expressions.  Instead of selecting the first matching rule to
    execute as in the default, all matching rules are considered
    and the rule with the least cost is executed.  Ties are broken by
    choosing the rule that comes first lexically.  For example:
 
   \begin{verbatim}
   match (ir)
   {  ADD(LOAD(?r0),?r1) \ e1: { ... }
   |  ADD(?r0,LOAD(?r0)) \ e2: { ... }
   |  ADD(?r0, ?r1)      \ e3: { ... }
   |  ...
   }
   \end{verbatim}
  \item A \verb|match| construct can be modified with the \verb|while|
     modifier, as in the following example.  A match modified thus is
     repeatedly matched until none of the patterns are applicable.
     For example, the following routine uses a \verb|match while| statement
     to traverse to the leftmost leaf.
 
\begin{verbatim}
   template <class T>
   Tree<T> left_most_leaf(Tree<T> tree)
   {  match while (tree)
      {  case node(t,_,_):  tree = t;
      }
      return tree;
   }
\end{verbatim}
  \item Finally, the \verb|matchscan| variant of match can be used to
     perform string matching on a stream.  For example, a simple lexical
     scanner can be written as follows.
 
     \begin{verbatim}
   int lexer (LexerBuffer& in)
   {  matchscan while (in)
      {  /if/:                     { return IF; }
      |  /else/:                   { return ELSE; }
      |  /while/:                  { return WHILE; }
      |  /for/:                    { return FOR; }
      |  /break/:                  { return BREAK; }
      |  /[0-9]+/:                 { return INTEGER; }
      |  /[a-zA-Z_][a-zA-Z_0-9]*/: { return IDENTIFIER; }
      |  /[ \t]+/:                 { /* skip spaces */ }
      |  ... // other rules
      }
   }
     \end{verbatim}
\end{itemize}

\Subsection{Algebraic Datatypes} \label{sec:algebraic-datatypes}
   Algebraic datatypes are defined using the \T{datatype}
construct.  Its syntax is as follows.

\begin{syntax}
\NDEF{Datatype\_Decl} 
   & \IS & \TDEF{datatype} \\
   &     & \quad \OPT{\ALT{\N{Datatype\_Spec}}{and}} \\
   &     & \quad \OPT{\TDEF{where type} \ALT{\N{Type\_Spec}}{and}} \\
   &     & \quad \OPT{\TDEF{law} \ALT{\N{Law\_Spec}}{and}} \\
   &     & \T{;} \\
\end{syntax}

Each \T{datatype} declaration specifies a set of 
(potentially mutually recursive) types, a set of type abbreviations,
and a set of pattern matching abbreviations called {\em laws}.

\begin{syntax}
\NDEF{Datatype\_Spec} 
   & \IS & \Id \OPT{\T{<} \LIST{\N{Id}} \T{>}} \\
   &     & \quad \OPT{\T{:} \N{Inherit\_List}} \\
   &     & \quad \OPT{\T{::} \N{Datatype\_Qualifiers}} \\
   &     & \quad \OPT{\T{=} \N{Cons\_Specs}} \\
\NDEF{Cons\_Specs} 
   & \IS & \ALT{\N{Cons\_Spec}}{|} \\
   &     & \quad \OPT{\T{public:} \N{Datatype\_Body}} \\
\NDEF{Cons\_Spec} 
   & \IS & \N{Simple\_Cons\_Spec}  \\
   &     &  \quad \OPT{\T{with} \T{\{} \N{Class\_Decl} \T{\}}} \\
\NDEF{Simple\_Cons\_Spec} 
   & \IS & \Id \OPT{\N{String}} & unit constructor \\
   & \OR & \N{String}              & unit constructor \\
   & \OR & \Id \OPT{\T{of}} \TypeExp  & constructor with arguments \\
   & \OR & \T{\#[} \T{]}                 & unit list constructor \\
   & \OR & \T{\#\{} \T{\}}               & unit list constructor \\
   & \OR & \T{\#(} \T{)}                 & unit list constructor \\
   & \OR & \T{\#[} \TypeExp \T{...} \TypeExp \T{]}   
      & list constructor \\
   & \OR & \T{\#\{} \TypeExp \T{...} \TypeExp \T{\}} 
      & list constructor \\
   & \OR & \T{\#(} \TypeExp \T{...} \TypeExp \T{)}   
      & list constructor \\
   & \OR & \T{[|} \TypeExp \T{|]} & vector constructor \\
   & \OR & \T{(|} \TypeExp \T{|)} & vector constructor \\
   & \OR & \T{\{|} \TypeExp \T{|\}} & vector constructor \\
\end{syntax} 


Datatypes can be annotated with {\em qualifiers}.  They tell the \Prop{}
translator to generate additional code for each datatype to provide
additional functionality. 

\begin{syntax}
\NDEF{Datatype\_Qualifiers} 
   & \IS & \MORE{\N{Datatype\_Qualifier}}{} \\
\NDEF{Datatype\_Qualifier} 
   & \IS & \TDEF{collectable} & garbage collectable \\
   & \OR & \TDEF{rewrite}     & optimized for tree rewriting \\
   & \OR & \TDEF{persistent}  & generate persistence interface \\
   & \OR & \TDEF{lexeme}      & used for parser/lexer tokens  \\
   & \OR & \TDEF{inline}      & use inlined implementation \\
   & \OR & \TDEF{extern}      & use non-inlined implementation \\
\end{syntax}

Type specifications assign abbreviations to commonly used type expressions. 
They act like \T{typedef}'s in \Cpp, except that \Prop{} can make use
of the type information provided by these specifications.

\begin{syntax}
\NDEF{Type\_Spec} 
   & \IS & \Id \T{=} \TypeExp \\
\end{syntax}

Law specifications are used to define abbreviations to datatype 
patterns\footnote{Patterns are discussed in 
section~\ref{sec:pattern-matching}.}.
They behave like macros in some sense, except that unlike macros, they are
properly type checked by \Prop.  In addition, if the keyword \T{inline} is 
used, then \Prop{} will treat the lhs as an expression and generate 
a function of the same name that can be used to construct the datatype term.

\begin{syntax}
\NDEF{Law\_Spec} 
   & \IS & \OPT{\T{inline}} \Id \OPT{\N{Law\_Arg}} \T{=} \Pat \\
\NDEF{Law\_Arg} 
   & \IS & \Id \\
   & \OR & \T{(} \LIST{\N{Id}} \T{)} \\
\end{syntax}

\Prop{} recognizes the following set of type expressions.  
\begin{syntax}
\NDEF{Type\_Exp}
   & \IS & \N{Type\_Id}                & type name \\
   & \OR & \TypeExp \T{*}              & pointer type \\
   & \OR & \TypeExp \T{\&}             & reference type \\
   & \OR & \N{Type\_Qualifier} \N{Type\_Id}      & qualified type \\
   & \OR & \N{Datatype\_Qualifier} \TypeExp       & annotation  \\
   & \OR & \T{(} \TypeExp \T{)}          & grouping \\
   & \OR & \T{(} \TypeExp \T{,} 
      \LIST{\N{TypeExp}} \T{)} & tuple type \\
   & \OR & \T{(} \TypeExp \T{,} 
      \LIST{\N{TypeExp}} \T{)} & tuple class \\
   & \OR & \T{\{} \LIST{\N{Lab\_Type\_Exp}} \T{\}} & record type \\
   & \OR & \TypeExp \T{=} \N{Exp} & default value \\
\NDEF{Lab\_Type\_Exp} & \IS &
   \Id \T{:} \TypeExp \\
\NDEF{Type\_Qualifier} 
   & \IS & \T{class}       & a class type \\
   & \OR & \T{const}       & constant type \\
   & \OR & \T{rewrite}     & rewritable \\
   & \OR & \T{collectable} & garbage collected \\
   & \OR & \T{persistent}  & persistent object  \\
\end{syntax}

In general, four different types of datatype constructors can be defined:
\begin{description}
 \item[nullary constructors]  These are constructors without an argument.
 \Prop{} maps these into small integers and no heap space will be consumed.
 \item[tuple argument constructors] These are constructors with one
 or more unamed arguments.  
 \item[record argument constructors] These are constructors with one
 or more named arguments written within braces.  For example, in the
 following datatype definition a record argument constructor called
 \verb|Add| is defined:
 \begin{verbatim}
  datatype Exp = Add { left : Exp, right : Exp }
	       | ...
 \end{verbatim}
  To construct an \verb|Add| expression, we can use any one of the following
  forms:
  \begin{verbatim}
     Exp e1 = Add(x,y);
     Exp e2 = Add'{ left = x, right = y };
     Exp e3 = Add'{ right = y, left = x };
  \end{verbatim}
  Note that we are allowed to rearrange the labels in any order if
  the record expression form is used.
 \item[empty constructors] These are constructors written with
 an empty argument \verb|()|.  For example, in the following definition
 an empty constructor called \verb|NONE| is defined:
 \begin{verbatim}
    datatype Exp : public AST
	= NONE ()
	| NONE2
        | ...
 \end{verbatim}
 Here, the constructor \verb|NONE| will inherit from the class \verb|AST|,
 while the nullay constructor \verb|NONE2| will not.
\end{description}

\Subsubsection{Instantiating a datatype}

Each declared datatype should be {\em instantiated} in some implementation
file using the \TDEF{instantiate datatype} statement.  The translator
will generate additional code to implement the datatype.  The general
syntax is:
\begin{syntax}
 \NDEF{Instantiate\_Decl} 
    & \IS &
       \TDEF{instantiate datatype} \LIST{\N{Type\_Spec}} \\
    & \OR & 
       \TDEF{instantiate} \T{extern} \T{datatype} \LIST{\N{Type\_Spec}} \\
 \NDEF{Type\_Spec} & \IS & \Id \OPT{ \T{<} \LIST{\N{TypeExp}} \T{>}} \\
\end{syntax}

\begin{Tips}
   When a datatype is defined with the \TDEF{inline} qualifier,
\Prop{} will try to generate inline methods whenever possible.
On the other hand, if the \TDEF{extern} qualifier is used, \Prop{}
will try to generate an non-inline implementation, and only one
copy are generated at where the \T{instantiate datatype} statement occurs.

Since most \Prop{} datatypes are light-weight objects, \T{inline} is taken
as the default.  The user can override this default using
\OPTIONDEF{-save\_space} option of the translator 
(see page~\pageref{option:save-space}).
\end{Tips}

\begin{Tips}
   When the option \OPTIONDEF{-fno-implicit-templates} is used
under \Gpp, the user should \verb|#define| the symbol
\INDEX{{\tt PROP\_EXPLICIT\_TEMPLATE\_INSTANTIATION}} before
an \T{instantiate datatype} statement.   
This will enable the code for explicit template instantiation.
\end{Tips}

\Subsection{Pattern Matching} \label{sec:pattern-matching}
   The basic pattern matching construct is the \T{match} statement.    
A match statement can be seen as a generalization of \Cpp's \T{switch}
statement, in which the expression that follows a \T{case} can be
a general complex pattern.  

   Its basic syntax is as follows:
\begin{syntax}
\NDEF{Match} & \IS & 
   \N{Match\_Mode} \ALT{\T{(} \N{Exp} \T{)}}{and}  \\
        & & \T{\{} \MORE{\T{case} \N{Match\_Rule}}{} \T{\}} & variant 1 \\
          & \OR & 
   \N{Match\_Mode} \ALT{\T{(} \N{Exp} \T{)}}{and} \\
        & & \T{\{} \ALT{\N{Match\_Rule}}{|} \T{\}} & variant 2 \\
          & \OR &
   \N{Match\_Mode} \ALT{\T{(} \N{Exp} \T{)}}{and} \T{of} & variant 3 \\
        & & \quad \ALT{\N{Match\_Rule}}{|} \\
        & & \T{end} \T{match} \T{;} \\
\NDEF{Match\_Mode} & \IS & \TDEF{match} \OPT{\T{while}} \\
                & \OR & \TDEF{matchall} \OPT{\T{while}} \\
\NDEF{Match\_Rule} & \IS & \Pat \OPT{\N{Guard}} \OPT{\N{Cost}}
        \T{:} \N{Match\_Action} \\
\NDEF{Guard}       & \IS & \T{if} \N{Exp} \\
                & \OR & \T{where} \N{Exp} \\
                & \OR & \T{|} \N{Exp} \\
\NDEF{Cost} & \IS & \verb|\| \Exp & cost expression \\
\NDEF{Match\_Action} & \IS & \T{\{} \N{Code} \T{\}} \\
                  & \OR & \N{Code} & for variant 1 only \\
\end{syntax}

The 3 syntactic variants of the match statement are equivalent in meaning.
A \T{match} statement takes a list of match objects and a
set of pattern matching rules.  The first rule that matches the pattern
completely will be executed.  Notice that guards can be placed on each rule,
and a rule is only applicable if the guards evaluate to true. 

\paragraph{Cost minimization}
In addition, if cost expressions are specified, then the matching process
will proceed as follows: 
\begin{enumerate}
   \item Locate all applicable rules.  If no such rule exists then exit.
   \item Compute the cost expressions of all applicable rules.
   \item Choose the action that has the lowest cost to execute.  If
a tie occurs, resolve by the textual order of the rules. 
\end{enumerate}

\paragraph{Match all rules}
The \T{matchall} statement is a variant of \T{match}.  In the matchall
mode, the actions of {\em all} applicable rules are executed in the
textual order.  Caution: since the matching variable binding 
process occurs {\em before} the rules are executed, the rhs actions
should not alter the matched value or the pattern variables may
be bound to the wrong value.

\paragraph{Repetition} 
In addition, both \T{match} and \T{matchall} can be annotated with the
\TDEF{while} modifier.  It this case the semantics of the matching
construct is to repeat until no more rule is applicable.  For example,
the following code fragment returns the last element of a list
\begin{verbatim}
datatype List = #[] | #[ int ... List ];

int last (List l)
{  match while (l)
   { case #[]:         assert(0); // an error
     case #[i]:        return i;
     case #[h ... t]:  l = t;     
   }
}
\end{verbatim}

\paragraph{Pattern syntax}
The basic form of a pattern is a literal, which
that matches only a single integer, real number, string, etc.  Complex
patterns can be constructed inductively
using datatype constructors, tuples, lists, and 
{\em logical pattern connectives}.

The syntax of the patterns is as follows:
\begin{syntax}
\NDEF{Pat} & 
   \IS & \N{Integer} & matches integer literal \\
 & \OR & \N{Real}    & matches real literal \\
 & \OR & \N{Char}    & matches character literal \\
 & \OR & \N{Quark}   & matches quark literal \\
 & \OR & \N{String}  & matches string literal \\
 & \OR & \N{Regexp}  & matches any string that matches {\em regexp}\\
 & \OR & \N{Pat\_Var}  & a pattern variable; matches anything \\
 & \OR & \T{\_}      & wild card; matches anything \\
 & \OR & \N{Cons}     & matches a datatype constructor  \\
 & \OR & \N{Cons} \N{PatArg}& a constructor application pattern \\
 & \OR & \Id \T{as} \Pat & binds a variable to the sub-pattern \\
 & \OR & \Pat \T{:} \TypeExp & typed pattern \\
 & \OR & \Pat \T{||} \Pat  & matches either pattern \\
 & \OR & \Pat \T{\&\&} \Pat  & matches both patterns \\
 & \OR & \T{!} \Pat             & matches anything but the pattern \\
 & \OR & \T{(} \Pat \T{)}       & Grouping \\
 & \OR & \T{\#[} \LIST{\N{Pat}} \T{]} & list pattern; exact length match \\
 & \OR & \T{\#[} \LIST{\N{Pat}} \T{...} \T{]} & list pattern;   
     non-exact length match \\
 & \OR & \T{[|} \LIST{\N{Pat}} \T{|]} & vector pattern; exact length match \\
 & \OR & \T{[|} \LIST{\N{Pat}} \T{...} \T{|]} 
    & vector pattern; matches to the left \\
 & \OR & \T{[|} \T{...} \LIST{\N{Pat}} \T{|]} 
    & vector pattern; matches to the right \\
\NDEF{PatArg} & \IS & \Pat \\
           & \OR & \T{(} \LIST{\N{Pat}} \T{)} \\
           & \OR & \T{\{} \LIST{\N{Lab\_Pat}} \OPT{\T{...}} \T{\}} \\ 
\NDEF{Lab\_Pat} & \IS & \Id \T{=} \Pat \\
\NDEF{Pat\_Var} & \IS & \Id       \\
            & \OR & \T{?}\Id  \\
\end{syntax}

\Subsection{Refining a datatype}

   A datatype definition can be refined in a few ways:

\begin{enumerate}
  \item{\bf inheritance} A datatype can be declared to be inherited from
one or more classes $C_1, C_2, \ldots, C_n$. 
The effect is that all variants of a datatype will be inherited from the 
classes $C_1, C_2, \ldots C_2$.  

  \item{\bf member functions}  Member functions and addition attributes
can be attached to the datatype variants using the keyword \TDEF{with}.
\end{enumerate}

For example, the following datatype \T{Exp} is inherited from some user
defined classes \verb|AST| and \verb|Object|.   Furthermore, a virtual
member function called print is defined.  

\begin{verbatim}
datatype Exp : public AST, virtual public Object
   = INT int        
         with { ostream& print(ostream&); }
   | PLUS  Exp, Exp 
         with { ostream& print(ostream&); }
   | MINUS Exp, Exp 
         with { ostream& print(ostream&); }
   | MULT  Exp, Exp 
         with { ostream& print(ostream&); }
   | DIV   Exp, Exp 
         with { ostream& print(ostream&); }
   | VAR { name : const char *, typ : Type }
         with { ostream& print(ostream&); }
   public:
   {  virtual ostream& print (ostream&) = 0;
      void * operator new (size_t);
   }
;
\end{verbatim}

The allocator and printing methods can be defined as follows:
\begin{verbatim}
void * classof Exp::operator new(size_t) { ... }
ostream& classof INT::print(ostream& s)   { return s << INT; }
ostream& classof PLUS::print(ostream& s)  
   { return s << '(' << this->#1 << '+' << this->#2 <<')'; }
ostream& classof MINUS::print(ostream& s) 
   { return s << '(' << this->#1 << '-' << this->#2 <<')'; }
ostream& classof MULT::print(ostream& s)  
   { return s << '(' << this->#1 << '*' << this->#2 <<')'; }
ostream& classof DIV::print(ostream& s)  
   { return s << '(' << this->#1 << '/' << this->#2 <<')'; }
ostream& classof VAR::print(ostream& s)   { return s << name; }
\end{verbatim}

The special type form \TDEF{classof} {\em con} is used to reference
the type of a constructor.  Arguments of a constructor uses the following
naming convention: (i) if the constructor $C$ takes only one argument,
then the name of the argument is $C$; (ii) if $C$ takes two or more
unnamed arguments, then these are named \verb|#1|, \verb|#2|, etc;
(iii) finally, labeled arguments are given the same name as the label.

For readability reasons, it is often useful to separate a datatype definition
from the member functions and inheritance definitions.   The \TDEF{refine}
declaration can be used in this situation.  For example, the above
example can be restated more succinctly as follows:

\begin{verbatim}
//
// Datatype definition section
//
datatype Exp = INT int        
             | PLUS  Exp, Exp 
             | MINUS Exp, Exp 
             | MULT  Exp, Exp 
             | DIV   Exp, Exp 
             | VAR { name : const char *, typ : Type }
;

//
// Refinement section
//
refine Exp : public AST, virtual public Object
       { virtual ostream& print (ostream&) = 0;
         void * operator new (size_t);
       }
and    INT, PLUS, MINUS, MULT, DIV, VAR
       { ostream& print(ostream&); 
       }
;
\end{verbatim}

The general syntax of the \TDEF{refine} declaration is as follows:
\begin{syntax}
\NDEF{Refine\_Decl} 
   & \IS & \T{refine} \ALT{\N{Refine\_Spec}}{\T{and}} \T{;} \\
\NDEF{Refine\_Spec} 
   & \IS & \LIST{\N{Id}} \\
   &     & \quad \OPT{\T{:} \N{Inherit\_List}} \\
   &     & \quad \OPT{\T{::} \N{Datatype\_Qualifiers}} \\
   &     & \quad \OPT{\T{\{} \N{Datatype\_Body} \T{\}}} \\
\end{syntax}
Here, \N{Id} refers to either a constructor name or a datatype name.

\Subsection{Memory management}

   By default, a datatype constructor calls the \verb|operator new| to
allocated memory.  There are a few ways to change this behavior: (i)
redefine the \verb|operator new| within the datatype using refinement; or
(ii) inherit from a class that redefines this method.

   If a {\em placement} \verb|operator new| is defined in datatype,
the placement constructor syntax can be used to invoke this
operator:
\begin{syntax}
   \NDEF{Placement\_Cons} 
   & \IS & \N{Cons} \T{'(} \LIST{\N{Exp}} \T{)} \T{(} 
   \OPT{\LIST{\N{Exp}}} \T{)} \\
   & \OR & \N{Cons} \T{'(} \LIST{\N{Exp}} \T{)} \T{\{} 
   \OPT{\LIST{{\N{Id} = \N{Exp}}}}
     \T{\}} \\
\end{syntax}

For example, in the following the extra parameter \verb|mem| will
be passed to placement \verb|new| operator.
\begin{verbatim}
  Exp e1 = INT'(mem)(1);
  Exp e2 = VAR'(mem){ name = "foo", typ = t };
\end{verbatim}

\Subsubsection{Garbage collection}

  The support library contains an implementation\footnote{The implementation
has only been tested on Linux, SunOS and Solaris.  Please contact the
author for details
if you'd like to port the collector to your particular platform.} of a 
conservative garbage collector using two algorithms: one using
a mostly copying scheme\cite{Mostly-copying,Gen-mostly-copying} 
and one using mark-sweep.

  A datatype can be defined to be garbage collectable with the
qualifier \TDEF{collectable}.  In the following example, the datatype
\T{Exp} is will be allocated from the a garbage collected heap instead
of the default heap.  Since it references another 
user defined garbage collectable object
\verb|SymbolTable|, the pointer to that symbol table is also marked
as \T{collectable}.
\begin{verbatim}
   // SymbolTable is collectable
   class SymbolTable: public GCObject
   {
   public:
      class Id : public GCObject { ... };
      ...
   };

   datatype Exp :: collectable
      = INT int
      | VAR (collectable SymbolTable::Id, collectable SymbolTable *)
      | ADD Exp, Exp
      | SUB Exp, Exp
      | MUL Exp, Exp
      | DIV Exp, Exp
      ;
\end{verbatim}

The corresponding \TDEF{instantiate datatype} statement for \T{Exp}
will generate the appropriate tracing methods for garbage collection.

Please see appendix~\ref{appendix:gc} for more details concerning
garbage collection.

\Subsubsection{Persistence}

  Datatypes can be interfaced with the persistent object library
supplied with the translator using the \TDEF{persistent} qualifier.
A persistent object defines the protocol for serializing its representation
into an architecture independent format on a persistence stream.  

For example, the following datatype declaration defines a persistent 
type \verb|Exp|.  Note that a pointer to the symbol table
is annotated with the \T{persisent} qualifier to signify that the object
associated with the pointer should be traversed during the serialization
phase. 
\begin{verbatim}
   // SymbolTable is persistent
   class SymbolTable: public PObject
   {
   public:
      class Id : public PObject { ... };
      ...
   };

   datatype Exp :: persistent
      = INT int
      | VAR (persistent SymbolTable::Id, persistent SymbolTable *)
      | ADD Exp, Exp
      | SUB Exp, Exp
      | MUL Exp, Exp
      | DIV Exp, Exp
      ;
\end{verbatim}

A {persistent object id} {\em must} be provided with each distinct
datatype.  Object ids are used to identify the classes
of persistent objects.  Thus they must be unique within an application.
Each persistent type tag is simply a string, and can be defined using
the \TDEF{refine persistent} statement.  

For example, the following statement declares the persistent object
id for type \verb|Exp| above to be \verb|"Simple expressions"|.
\begin{verbatim}
   refine persistent Exp => "Simple expressions";
\end{verbatim}

The corresponding \TDEF{instantiate datatype} statement will generate
the appropriate methods to communicate with a persistent stream.
\begin{verbatim}
  instantiate datatype Exp;
\end{verbatim}

To write an object $e$ to the datafile \verb|"foo"|, we can say:
\begin{verbatim}
   #include <AD/persist/postream.h>
   ofstream out("foo");
   Postream pout(out);
   pout << e;
   out.close();
\end{verbatim}

To read the object back from the file, we can say:
\begin{verbatim} 
   #include <AD/persist/pistream.h>
   EXP e;
   ifstream in("foo");
   Pistream pin(in);
   e = (Exp)read_object(pin);
   in.close();
\end{verbatim}

For more details concerning persistent streams and persist objects, 
please see directory \verb|<AD/persist>| for details.

\Section{Inference} \label{sec:inference}

   Semantic processing
in compilers and other language processors, such as data flow analysis,
can frequently be specified as in a rule-based,
logical deductive style.  In \Prop, deductive inference using forward
chaining\footnote{
The current
implementation of {\sf Prop} translates inference into very naive code.
This feature will be phased out and replaced by
the more general and much faster graph rewriting mechanism.  
}
is provided as a built-in mechanism, orthogonal
to pattern matching and rewriting, for manipulating user-defined
algebraic datatypes.
 
Similar to syntax classes, {\bf inference classes} may be used
for data encapsulation.  An inference class is a combination of a \Cpp{}
class, a database of inference relations, and a collection of inference rules
of the form {\em lhs \verb|->| rhs}.  The lhs of an inference rule is a set
of patterns in conjunctive form.  During the inference process, a rule is
fired when its lhs condition is satisfied.  A fired rule then executes
the corresponding rhs action, which may assert or retract additional relations
from the database.   Using multiple inheritance, it is possible to combine
a rewriting class with an inference class such that the rewriting process
generates new relations to drive the inference process, or vice versa.

\Subsection{An Example} 
Datatype relations are not a distinct kind of data structure but are in fact
simply algebraic datatypes declared to have a \TDEF{relation} 
attribute.  For example, in
the following definition three relation types \verb|Person|, \verb|Parent|
and \verb|Gen| are defined.
 
\begin{verbatim}
   datatype Person :: relation = person (const char *)
       and  Parent :: relation = parent (const char *, const char *)
       and  Gen    :: relation = same_generation (const char *, const char *);
 
   instantiate datatype Person, Parent, Gen;
\end{verbatim}
 
Using these relations we can define an inference class that computes whether
two persons are in the same generation.   Nine axioms are defined (i.e.
those whose lhs are empty) in the following.  The two inference rules
state that (1) a person is in the same generation as him/herself, 
and (2) two persons are in the same generation if 
their parents are in the same generation.
 
\begin{verbatim}
   inference class SameGeneration {};
 
   inference SameGeneration
   {  -> person("p1") and person("p2") and
         person("p3") and person("p4") and
         person("p5");
 
      -> parent("p1", "p2") and
         parent("p1", "p3") and
         parent("p2", "p4") and
         parent("p3", "P5");
 
      person ?p -> same_generation (?p, ?p);
 
      parent (?x, ?y) and parent (?z, ?w) and same_generation (?x, ?z)
      -> same_generation(?y, ?w);
   };
\end{verbatim}
 
In general, datatypes qualified as \verb|relation|s will inherit
from the base class \verb|Fact|, while a rewrite class definition
implicitly defines two member functions used to assert and retract facts
in the internal database.  For example, in the above example, the following
protocol will be automatically generated by the inference compiler.
 
\begin{verbatim}
   class SameGeneration : ...
   {
   public:
       virtual Rete&    infer       ();       // start the inference process
       virtual ReteNet& operator += (Fact *); // assert fact
       virtual ReteNet& operator -= (Fact *); // retract fact
   };
\end{verbatim}
 

 
Using these methods, an application can insert or remove relations
from an inference class.  This will in turn trigger any attached inference
rules of the class.
 
\Subsubsection{Another example}
 
Consider the following example, which is used to compute Pythagorean
triangles.  Only one axiom and two rules are used.  The axiom and the
first rule are used to assert the relations \verb|num(1)| to \verb|num(n)|
into the database, where \verb|n| is limited by the term \verb|limit(n)|.
The second inference rule is responsible for printing out only
the appropriate combinations of numbers.
 
\begin{verbatim}
   datatype Number :: relation = num int | limit int;
 
   inference class Triangle {};
 
   inference Triangle
   {  ->  num 1;
 
          num m
      and limit n | n > m
      ->  num (m+1);
 
          num a
      and num b
      and num c | a < b && b < c && a*a + b*b == c*c
      ->  { cout << a << " * " << a << " + "
                 << b << " * " << b << " = "
                 << c << " * " << c << "\n";
          };
   };
\end{verbatim}
 
Now, to print all the triangle identities lying in range of $1$ to $100$,
we only have to create an instance of the inference class, insert the
limit, and start the inference process, as in below:
 
\begin{verbatim}
   Triangle triangle;
   triangle += limit(100);
   triangle.infer();
\end{verbatim}

\Subsection{Inference Class}
\INCOMPLETE

\Subsection{Inference Rules}
\INCOMPLETE
 
\Section{Tree Rewriting} \label{sec:tree-rewriting} 

\Prop's tree rewriting mechanism let us transform a tree in
algebraic datatype form into another tree according to a set of 
{\em rewriting rules}.  Unlike plain pattern matching described
in section~\ref{sec:pattern-matching}, which only apply to the
root of a tree, rewriting rules are applicable to all parts of
the tree.  This allows the user to concentrate on developing the
set of transformations; the actual traversal of the tree object
is handled implicitly by \Prop.  When used properly, this
mechanism has an advantage over plain pattern matching since
rewriting rules remain valid even when a new variant to a datatype
is introduced.

In \Prop, a rewriting system is developed in a similar manner as a parser.
The first phase requires the user to defines a {\em rewrite class}
to encapsulate the rewriting rules.  

Frequently, the rewriting mechanism is used to collect information about
a tree structure; furthermore, more transformation rules are {\em conditional}:
i.e. we want them to be applicable only if certain conditions are satisfied. 
We can accomplish both of these tasks by defining data members and methods 
in the rewriting class.  These are accessible during the rewriting
process.  Information collected during rewriting can be stored within
the data members.

In the rewriting formalism, equational
rules of the form {\em lhs} \verb|->| {\em rhs} are specified by the user.
During processing, each instance of the lhs in a complex tree is replaced
by an instance of the rhs, until no such replacement is possible.
Equational rules can often be used to specify semantics based simplification
(e.g. constant folding and simplification based on simple algebraic
identities) or transformation(e.g. code selection in a compiler
backend\cite{codegen-with-trees}).
 
Unlike plain pattern matching, however, the structural traversal process
in rewriting is implicitly inferred from the type structure of an
algebraic datatype, as specified in its definition. 
 
There are two main forms of rewriting modes available:
\begin{itemize}
   \item The first is {\bf normalization} mode: a given tree is reduced
using the matching rules until no more redexes are available.  There
are two modes of operations available:
       \begin{itemize}
          \item in {\bf replacement} mode, the redex of a tree will be
physically overwritten.
          \item in {\bf applicative} mode, on the other hand, a new
tree corresponding to the replacement value will be constructed.
       \end{itemize}
     Replacement mode is used as the default since it is usually
the more efficient of the two.
   \item The second form is {\bf reduction} and {\bf transformation}.
In this mode a tree parse of the input term is computed.  If cost functions
are attached to the rules, then they will also be used to determine a
minimal cost reduction sequence.  During this process attached actions of
a rule may be invoked to synthesize new data.
\end{itemize}
 
   Each independent set of rewriting rules in \Prop{} is encapsulated
in its own {\bf rewrite class}.  A rewrite class is basically a normal \Cpp{}
class with a set of rewriting rules attached.   During rewriting, the
data members and the member functions are visible according to the normal
\Cpp{} scoping rules.  This makes it is easy to encapsulate additional data
computed as a side effect during the rewriting process.
 
\Subsection{A rewriting example}
 
   Consider an abbreviated simplifier for the well-formed formula datatype
\verb|Wff| defined in the section~\ref{sec:Wff}.  The rewrite class for this
can be defined as follows.  Since there is no encapsulated data in this
example, only the default constructor for the class needs to be defined.
A rewrite class definition requires the traversal list to be
specified.  This is simply a list of datatypes 
involved in the rewriting traversal process.  
In this instance only \verb|Wff| is needed.
 
\begin{verbatim}
   rewrite class Simplify (Wff)
   {
   public:
      Simplify() {}
   };
\end{verbatim}
 
   The rewrite rules for the simplifier can then be specified succinctly as
follows.  Like the \verb|match| statement, in general the rhs
of a rewrite rule can be any statement.  A special statement
\verb|rewrite(e)| can be used to rewrite the current redex into another form.
If the rhs is of the form \verb|rewrite(e)|, then it can be abbreviated
to \verb|e|, as in below:
\begin{verbatim}
   rewrite Simplify
   {  And(F,  _):      F
   |  And(_,  F):      F
   |  And(T,  ?X):     ?X
   |  And(?X, T):      ?X
   |  Or (T,  _):      T
   |  Or (_,  T):      T
   |  Or (F,  ?X):     ?X
   |  Or (?X, F):      ?X
   |  Not(Not(?X)):    ?X
   |  Not(And(?X,?Y)): Or(Not(?X), Not(?Y))
   |  Not(Or(?X,?Y)):  And(Not(?X), Not(?Y))
   |  Implies(?X,?Y):  Or(Not(?X), ?Y)
   |  And (?X, ?X):     ?X
   |  Or (?X, ?X):      ?X
   // etc ...
   };
\end{verbatim}
 
The rewrite class definition creates a new class of the same name.  This new
class defines an implicit \verb|operator ()| with the protocol below.  This
member function can be invoked to perform the rewriting in a functional
syntax.
 
\begin{verbatim}
   class Simplify : ... {
   {  ...
   public:
      void operator () (Wff);
      // Wff operator () (Wff); // if rewrite class is applicative
   };
 
   Wff wff = ...;
   Simplify simplify; // create a new instance of the rewrite class
   simplify(wff);    // rewrite the term wff
\end{verbatim}

 
\Subsubsection{Conditional rewriting and actions}
  
   Rewriting rules may be guarded with predicates to limit their
applicability.  In addition, the {\em rhs} of a rewrite rule is not limited
to only a replacement expression: in general, any arbitrarily complex sequence
of code may be used.  For example, in the following set of rewriting
rules we use guards to prevent undesirable replacements to be made during
expression constant folding:
 
\begin{verbatim}
   rewrite ConstantFolding
   {  ADD (INT a, INT b):  INT(a+b)
   |  SUB (INT a, INT b):  INT(a-b)
   |  MUL (INT a, INT b):
      {  int c = a * b;                      // silent overflow
         if (a == 0 || b == 0 || c / b == a) // no overflow?
         {  rewrite(INT(c)); }
         else
         {  cerr << "Overflow in multiply\n"; }
      }
   |  DIV (INT a, INT b) | b == 0:  { cerr << "Division by zero\n"; }
   |  DIV (INT a, INT b):  INT(a/b)
   |  // etc...
   };
\end{verbatim}

\Subsection{Rewrite class} \label{sec:rewrite-class}

The syntax of a rewrite class declaration is as follows:

\begin{syntax}
\NDEF{Rewrite\_Class\_Decl} & \IS & \TDEF{rewrite class} \Id 
            \T{(} \LIST{\N{TypeExp}} \T{)} \\
   & & \quad \OPT{\T{:} \N{Inherit\_List}} 
        \OPT{\T{::} \MORE{\N{Rewrite\_Mode}}{}} \\
   & & \quad \T{\{} \N{Class\_Body} \T{\}} \T{;} \\
\NDEF{Rewrite\_Mode} & \IS & \TDEF{treeparser} \\
                  & \OR & \TDEF{applicative} \\
                  & \OR & \TDEF{topdown} \\
\end{syntax}

This is simply the usual \Cpp{} class declaration syntax
extended to the following information:
\begin{description}
   \item[a traversal list] enclosed in parenthesis.  The traversal
list of a rewrite class defines the set of types that rewriting
should traverse.  If a datatype object contains an argument of a type
not listed in the traversal list, its value will not be altered.
   \item[rewrite mode] this defines the rewriting mode of the
rewrite class.
\end{description}

\Subsection{Rewriting rules}

Rewriting rules are specified in a rewrite declaration.
The syntax of a rewriting specification is as follows:

\begin{syntax}
\NDEF{Rewrite\_Decl} 
   & \IS & \TDEF{rewrite} \Id & variant 1 \\
   &    & \T{\{} \OPT{\N{Index\_Decl}} 
        \ALT{\T{case} \N{Rewrite\_Rules}}{} \T{\}} \\
   & \OR & \T{rewrite} \Id & variant 2 \\
   &    & \T{\{} \OPT{\N{Index\_Decl}} \ALT{\N{Rewrite\_Rules}}{|} \T{\}} \\
\NDEF{Rewrite\_Rule}
  & \IS & \OPT{\N{Rewrite\_Modifier}} \OPT{\N{Id} \T{->}} \\
  & & \quad \Pat \OPT{\N{Guard}} \OPT{\N{Cost}} \T{:} \N{Rewrite\_Action} \\
\NDEF{Rewrite\_Modifier}
  & \IS & \T{bottomup:} & bottom up mode \\
  & \OR & \T{topdown:} & top down mode \\
  & \OR & \T{before:} & before actions \\
  & \OR & \T{preorder:} & preorder actions \\
  & \OR & \T{postorder:} & postorder actions \\
\NDEF{Rewrite\_Action} & \IS & \T{\{} \N{Code} \T{\}} \\
                  & \OR & \Exp \\
\end{syntax}

The name of a rewrite declaration should match the name of
a rewrite class.  
The two syntactic forms of \T{rewrite} have equivalent semantics.

The special statement \T{rewrite}($e$) may be used
inside the rhs action to rewrite the current redex into $e$.
Note that \T{rewrite}($e$) is a branching statement; statements
after \T{rewrite} are not reachable.

\Subsection{Rewriting modifiers}
In version 2.3.3 onward, rewrite rules can be modified by the 
keywords: \T{bottomup:}, \T{topdown:}, \T{before:}, \T{preorder:}
and \T{postorder:}.   These modifiers 
alter the order in which a set of rewriting rules is applied. 

These modifiers act like delimiters, similar
to the way scoping keywords like \T{public:}, \T{protected:} and 
\T{private:} are used to delimit declarations in C++.
For instance, a set of rules prefixed by the modifier \T{topdown:}
will utilize the topdown strategy for reduction.  If no modifier are
specified, then bottom-up will be the default.

Note that all five modes of rules can be used together in a rewriting system 
under tree rewriting mode (see \ref{sec:rewriting-modes})
and their executions are intermixed together.  

The semantics of these modifiers are as follows:
\begin{description}
\item[\TDEF{bottomup:}] This specifies the modified rewriting rules should
be applied in the default bottom-up mode.  In this mode, the 
innermost/leftmost redex is chosen as the next redex.  So reduction
occurs in a bottomup manner.  This means that if a term $t$ is a redex
and if the term $t$ contains a subterm $t'$ which is also a redex, 
$t'$ will be reduced before $t$. 
\item[\TDEF{topdown:}]  This specifies the modified rewriting rules should be
applied in a topdown mode.  In this mode, the rewriter
will first try to locate the outermost/leftmost redex.  Reduction
will occur in a (mostly) topdown manner.
\item[\TDEF{before:}]  This specifies that the modified rewriting rules
should be tried before the topdown phase.   
In addition, unlike topdown mode, if state-caching if used (see
\ref{sec:state-caching}), the rules are only tried once. 
Otherwise, these act just like the preorder modifier, described below. 
\item[\TDEF{preorder:}]  This specifies that
the modified rewriting rules should be
tried only at the preorder traversal of a term.   
\item[\TDEF{postorder:}]  This specifies the rewriting rules should be
tried only at the postorder traversal of a term. 
\end{description}

\Subsubsection{Rewriting modifier example} \label{sec:rewriting-modifier}

We'll use the following example, extracted from a query optimizer for
a subset of the relational calculus, to demonstrate the use
of the rewriting modifiers.

The query optimizer transforms the following abstract syntax, defined
as a set of \Prop's datatypes.  
\begin{verbatim}
datatype List<T> = #[] | #[ T ... List<T> ];
datatype Literal = INT int
                 | STRING const char *
                 | BOOL Bool

and      Exp     = OP Id, Exps
                 | APP Id, Exps
                 | LIT Literal
                 | ID Id
                 | TUPLE Exps                  // [ E1, E2, ... En ]
                 | FORALL Id, Exp, Exp         // forall x in E1 . E2
                 | EXISTS Id, Exp, Exp         // exists x in E1 . E2
                 | GUARD (Exp, Exp)            //
                 | GENERATOR (Ids, Exps, Exp)  // [x_1,...,x_n] in X_1*...*X_n
                 | LET  (Id, Exp, Exp)

law inline Int  i    = LIT(INT i)
|   inline Boolean b = LIT(BOOL b)
|          True      = Boolean true
|          False     = Boolean false
|   inline And a,b   = OP("and",#[a,b])
|   inline Or  a,b   = OP("or", #[a,b])
|   inline Not a     = OP("not",#[a])
|   inline Eq a,b    = OP("=",  #[a,b])
|   inline Ne a,b    = OP("/=", #[a,b])
|   inline Gt a,b    = OP(">",  #[a,b])
|   inline Ge a,b    = OP(">=", #[a,b])
|   inline Lt a,b    = OP("<",  #[a,b])
|   inline Le a,b    = OP("<=", #[a,b])
|   inline In a,b    = OP("in", #[a,b])
|   inline Union a,b = OP("union", #[a,b])
|   inline Count a   = OP("#", #[a])

where type Id         = const char *
and        Ids        = List<Id>
and        Exps       = List<Exp>
;
\end{verbatim}

Note that existential and 
universal quantifiers are represented by the constructors \T{FORALL}
and \T{EXISTS} respectively.  For example, $\exists x \in A.p$ is
presented as the term $\T{EXISTS}(x,A,p)$.  Here $x$ is a {\em binding}
occurrence for $x$; note that the variable $x$ may occur within 
the predicate $p$.  In addition, a list comprehension
expression such as 
\[ \{ e : [x_1,\ldots,x_n] \in X_1 
\times \ldots \times X_n\ |\ p \} \] 
is represented as the compound term 
\[
  \T{GENERATOR}(\#[x_1,\ldots,x_n],\#[X_1,\ldots,X_n],\T{GUARD}(p,e))
\]

Suppose we'll like to rename all variables in a query $Q$ 
so that no two binding occurrences are given the same name.  
This can be easily accomplished using a combination
of \T{preorder:} and \T{postorder:} rules in the following manner: 
\begin{enumerate}
\item[(i)] Use preorder rules to check for new variable bindings,
for example in $\T{EXISTS}(x,A,p)$.
For each new binding occurrence for $x$, create
a new name for $x$.  
These will be performed before the subterms $A$ and $p$ are traversed.
\item[(ii)] Whenever a variable $x$ is found during the sub-traversal
of $A$ and $p$, rename the variable by looking up its new name.
\item[(iii)] Finally, use postorder rules to remove the current 
variable substitution whenever we're exiting a binding occurrence.
\end{enumerate}

For example, the following set of rewriting rules accomplish this renaming
task for our query language using exactly this method.  
\begin{verbatim}
rewrite RemoveDuplicateNames
{
   // We use before and after rules to remove duplicates from variable names.
   // As each new binding occurrence is found, we enter a new binding
   // into the current environment.  Identifiers found within
   // the binding occurrence are then renamed.

preorder: // insert new bindings
   EXISTS(x,_,_):     { new_binding(x); }
|  FORALL(x,_,_):     { new_binding(x); }
|  GENERATOR(xs,_,_): { new_binding(xs); }
|  LET(x,_,_):        { new_binding(x); }

         // rename variables
|  ID x:             { rename(x); }

postorder: // removes the binding
|  EXISTS(x,A,_):      { old_binding(x); }
|  FORALL(x,A,_):      { old_binding(x); }
|  GENERATOR(xs,As,_): { old_binding(xs); }
|  LET(x,_,_):         { old_binding(x); }
};
\end{verbatim}

Note that we have accomplished a depth first traversal using rewriting
without writing any traversal code!  As a side benefit, 
since the traversal is automatically determined by 
the structure of the datatypes, we do not have to rewrite this
renaming code even if the abstract syntax of the query language is
extended, as long as no additional binding operators are added.

\Subsection{The \T{rewrite} statement}
 
   While the rewrite class construct provides a very general abstraction
for rewriting, in general its full power is unneeded.  It is often
convenient to be able to perform rewriting on a term without having
to make a new name for a class just for the occasion, especially if member
functions and member data are unneeded.  To accommodate these situations,
the \verb|rewrite| statement is provided
to perform a set rewriting transformations on a term without having
to define a temporary rewrite class.   It is simply syntactic sugar
for the more general rewrite class and rewrite
rules specifications.
For example, a simplify routine for type \verb|Exp| defined above can be
specified as follows:
 
\begin{verbatim}
   Exp simplify (Exp e)
   {  // transformations on e before
      rewrite (e) type (Exp)
      {  ADD (INT a, INT b):  INT(a+b)
      |  SUB (INT a, INT b):  INT(a-b)
      |  MUL (INT a, INT b):  INT(a*b)
      |  ...
      }
      // transformations on e after
      return e;
   }
\end{verbatim}
 
The \verb|rewrite| normally performs the replacement in-place.
An applicative version of the same can be written as follows\footnote{The
variable {\tt e} is assigned the new copy.}:
 
\begin{verbatim}
   Exp simplify (Exp e)
   {  rewrite (e) => e type (Exp)
      {  ADD (INT a, INT b):  INT(a+b)
      |  SUB (INT a, INT b):  INT(a+b)
      |  MUL (INT a, INT b):  INT(a*b)
      |  ...
      }
      return e;
   }
\end{verbatim}

The syntax of the rewrite statement is as follows.  The
traversal list of the set of rewrite rule is listed next to
the keyword \T{type}.
 
\begin{syntax}
\NDEF{Rewrite\_Stmt} 
   & \IS & \TDEF{rewrite} \T{(} \Exp \T{)} \OPT{\T{=>} \N{Exp}} \\
   &     & \quad \T{type} \T{(} \LIST{\N{TypeExp}} \T{)} \\
   &     & \T{\{} \OPT{\N{Index\_Decl}} \N{Rewrite\_Rules} \T{\}} \\
   & \OR & \TDEF{rewrite} \T{(} \N{Exp} \T{)} \OPT{\T{=>} \N{Exp}} \\
   &    & \quad \TDEF{type} \T{(} \LIST{\N{TypeExp}} \T{)} \T{of} \\
   &    & \quad \OPT{\N{Index\_Decl}} \N{Rewrite\_Rules} \\
   &    & \T{end} \T{rewrite} \T{;} \\ 
\end{syntax}

\Subsubsection{Rewriting modes} \label{sec:rewriting-modes}
   There are two basic modes of operations in a rewrite class: (i) 
{\em tree rewriting} and (ii) {\em tree parsing}.   
Tree rewriting repeatedly looks for redexes in a tree and transforms it into
another tree.   There are two sub-modes of operations: (a) {\em applicative}
and (b) {\em in-place}.  Applicative mode is specified using the
\T{applicative} mode specifier when declaring a rewrite class. 
In this mode, a new tree is built from a 
bottom-up manner, and the subject tree is left unaltered.  The default
mode is ``in-place.''  In this mode, the subject tree is overwritten
with the transformed expression.  

   On the other hand, in tree parsing mode, the left hand side
of a rewrite rules specification is treated as a tree grammar.
The tree parser will look for a derivation of given tree object from
the start non-terminal.
A guarded rule will only be used if the guard evaluates to true.

   If rules are annotated with cost expressions, then the
tree parser will try to locate a derivation chain with a minimal total
cost.  After a derivation tree is found, the tree can be repeated
traversed.  The rhs actions will be invoked during this traversal process.  

\Subsubsection{State caching optimization} \label{sec:state-caching}

   The pattern matching process proceeds in a bottom up manner: 
for each subtree of the form $l(t_1,t_2,\ldots,t_n)$, $n \ge 0$, 
a state number is computed from the current tree label $l$ and the
state numbers of its subtrees $t_1, t_2, \ldots, t_n$.  The set of matched
rules can be directly computed from the current state number.
This means that in the absence of redex replacements, pattern matching
takes time proportional only $O(|t| + r)$, where $|t|$ is the size 
of the input tree $t$ and $r$ is time it takes to execute the rhs actions.  
That is, the runtime is insensitive to the number of rewrite rules or the 
complexity of the patterns.

   However, if the input is a DAG rather than a tree\footnote{Rewriting
of a generic graph structure is not recommended with \T{rewrite}.},
or if replacements are performed, then the above analysis may not hold.
Replacements during rewriting often require state information of the
replacement term to be recomputed to further the matching process.  
Since computation of state can involve a complete traversal of a term, 
replacement can become expensive if the replacement term is large.  
For instance, consider the following replacement rule, which replaces 
all expressions of the form {\em 2*x} into {\em x+x}:
 
\begin{verbatim}
   rewrite class StrengthReduction
   {
      MUL (INT 2, ?x):  ADD(?x, ?x)
      ...
   }
\end{verbatim}
 
\noindent
Since the subterm \verb|?x| could be arbitrarily large, recomputing the
state encoding for \verb|ADD(?x,?x)| naively takes time in proportion to the
size of \verb|?x|.  However, since the subtree \verb|?x| occurs as part
of the pattern, its state has already been computed, and it would be
wasteful to recompute its state from scratch.  

A similar observation can be made to the case when the input is a DAG,
where many subcomponents are shared between nodes.  In order to speed
up the rewriting process, the state of each shared copy of the nodes 
should be computed only once.

In order to speedup the state computation process in these situations,
{\em \INDEX{state caching}} can be enabled by
specifying a {\em rewriting index}.
This can be accomplished in two ways: (i) use the qualifier
\TDEF{rewrite} when defining a datatype.  This specifies the
{\em primary index}.  Or alternatively, (ii) use index declarations
to specify one or more {\em secondary indices}.

We'll first describe the first method. 
A \TDEF{rewrite} qualifier can be used 
in the definition of a datatype.  For instance:
 
\begin{verbatim}
   datatype Exp :: rewrite
      = INT (int)
      | ID  (const char *)
      | ADD (Exp, Exp)
      | SUB (Exp, Exp)
      | MUL (Exp, Exp)
      | DIV (Exp, Exp)
      ;
\end{verbatim}

The \T{rewrite} qualifier tells the translator to generate an extra
state field in the implementation of \T{Exp}.  This state field is implicitly
updated during the rewriting process.  When the rewriter encounters a
node with a cached state it can quickly short-circuit all unnecessary
state computation.

{\bf A word of caution:} since each rewriting system computes
its own encoding, rewriting systems should not be mixed if they shard the 
same index, i.e. actions of a 
rewriting system should not invoke another rewriting
system on a term that is participating in the current rewriting system,
if both systems use the same index.  This limitation can be overcome
by using secondary indices, which we'll discuss next.

\Subsubsection{Specifying secondary indices}

From release 2.3.0 onward, it is possible to specify secondary indices
within a rewriting system.  This makes it possible to invoke other
rewriting systems within the execution of another, while retaining the
state caching optimizations throughout.

There are two forms of secondary index: {\em internal index} and 
{\em external index}.  Internal indices are stored directly within
a rewrite term and require the user to pre-allocate space for each term.
In contrast, external indices are stored in a separate data structure
such as a hash table. 

The syntax of secondary index declarations is as follows:
\begin{syntax}
\NDEF{Index\_Decl} & \IS & \TDEF{index:} \LIST{\N{Index\_Spec}} \T{;} \\
\NDEF{Index\_Spec} & \IS & 
   \TypeExp \T{=} \T{none} & \C{Disable indexing} \\
   & \OR & \TypeExp \T{=} \N{Id} & \C{Internal index} \\
   & \OR & \TypeExp \T{=} \TDEF{extern} \N{Id} & \C{External index} \\
\end{syntax}


\Subsubsection{Using an internal index}

In order to make use of an index, certain member functions have to defined
by the user.  Given an internal index named {\em foo}, the rewriter
invokes two functions named
\begin{quotation}
\noindent
   {\tt int get\_{\em foo}\_state() const; } \\
   {\tt void set\_{\em foo}\_state(int); }
\end{quotation}
\noindent within the generated code.  The user should provide the 
implementations for these functions within the datatypes.  

For example, reusing the well-formed formulas example (see 
section~\ref{sec:Wff}),
an internal index on the datatype \T{Wff} can be implemented as follows:
\begin{verbatim}
#include <AD/rewrite/burs.h>
class WffIndex {
   int state;
public:
   WffIndex() : state(BURS::undefined_state) {}
   int get_wff_state() const { return state; }
   void set_wff_state(int s) { state = s; }
};

datatype Wff : public WffIndex
   = F
   | T
   | Var     (const char *)
   | And     (Wff, Wff)
   | Or      (Wff, Wff)
   | Not     (Wff)
   | Implies (Wff, Wff)
   ;

...

rewrite Simplify {
   index: Wff = wff;
   ...
};
\end{verbatim}

Here, the class \T{WffIndex} provides the internal index interface
functions \verb|get_wff_state| and \verb|set_wff_state| expected by
the rewriter.  Note that each term must be initialized with the
state \verb|BURS::undefined_state|.  This constant is defined within
the library include file \verb|<AD/rewrite/burs.h>|.

To enable the index, in the rewriter 
we specify that datatype \T{Wff} should make use of the index named \T{wff},
using the \T{index:} declaration.  

\Subsubsection{Using an external index}

External indices are specified in a similar manner.  Given a datatype
{\em Foo} and an external index named {\em bar}, the rewriter invokes
calls to the following functions:
\begin{quotation}
\noindent
   {\tt int get\_{\em bar}\_state({\em Foo}) const; } \\
   {\tt void set\_{\em bar}\_state({\em Foo}, int); }
\end{quotation}
Typically, these are implemented as member functions in the rewrite class.

The rewriter uses the function {\tt get\_{\em bar}\_state} to lookup
previously cached information set by the function {\tt set\_{\em bar}\_state}.
Typically, the implementation of the two functions can be implemented
as hash tables, using the addresses of the terms as hash functions and
pointer equality comparisons.   Note that caching can be {\em lossy}; i.e.
it is perfectly alright for the cache to eliminate cached information to
keep its size under control.  If no cache information is found,
the function {\tt get\_{\em bar}\_state} should return 
\verb|BURS::undefined_state|.

\paragraph{Class \CLASSDEF{RewriteCache}}
   To make it easy for the users to implement their own external indices,
two sample external index implementation are provided.  The first
is the class \verb|RewriteCache| defined in the library file 
\verb|<AD/rewrite/cache.h>|.  

   The class \verb|RewriteCache| implements a simple fixed capacity
   cache with the following member functions.
\begin{verbatim}
   RewriteCache();
   RewriteCache(int capacity);
  ~RewriteCache();
 
   void clear();                   // after rewriting
   void initialize();              // before rewriting
   void initialize(int capacity);  // before rewriting/also set capacity
   
   int capacity() const; 
               
   void set_state(const void * t, int s);
   int get_state(const void * t) const;

   void invalidate(const void * t); 
\end{verbatim}

The user should call the function \verb|initialize| before rewriting
in order to setup the cache and/or clear out previously cached result.
The functions \verb|set_state| and \verb|get_state| can be used to
implement the necessary state caching functions expected by the rewriter.

Returning to our \T{Wff} example, we have the following possible implementation
of the external index:
\begin{verbatim}
   rewrite class Simplify (Wff)
   {  RewriteCache cache;
   public:
      int get_wff_state(Wff t) const { return cache.get_state(t); }
      void set_wff_state(Wff t, int s) { cache.set_state(t,s); }  
   };

   rewrite Simplify
   {  index: Wff = extern wff;
      ...
   };
\end{verbatim}

\begin{Tips}  
If manual memory management is used, 
a term should not be deallocated from memory if a cache
contains a handle to it.  The method \verb|invalidate| can be used to
make sure the entry associated with a term is removed from the cache.
\end{Tips}

\paragraph{Class \CLASSDEF{GCRewriteCache}}

{\em This class has not be completely tested.}

Since a garbage collectable object is not reclaimed by the collector
whenever there exists a reference to it, using the class
\verb|RewriteCache| may create problems since objects are not reclaimed
if it is referenced by the cache.  The template class \verb|GCRewriteCache|
solves this problem by using {\em weak pointers} in its internal 
implementation.  

The template class \verb|GCRewriteCache<T>| expects \verb|T| to be
a class derived from \verb|GCObject|.   
This class implements the following functions:
\begin{verbatim}
   GCRewriteCache();
   GCRewriteCache(int capacity);
  ~GCRewriteCache();
 
   void clear();                   // after rewriting
   void initialize();              // before rewriting
   void initialize(int capacity);  // before rewriting/also set capacity
   
   int capacity() const;
               
   void set_state(T * t, int s);
   int get_state(T * t) const;

   void invalidate(T * t); 
\end{verbatim}

The usage of these functions are identical to that of class 
\verb|RewriteCache|.

\Subsection{Short circuiting rewrite paths with \TDEF{cutrewrite}}

   From version 2.3.0 onward, the \T{cutrewrite}($e$) 
statement may be used wherever a \T{rewrite}($e)$ statement is expected.
The \T{cutrewrite} statement can be used to selectively ignore certain
subterms during the rewriting process.

The semantics of \T{cutrewrite}($e$) is to replace the current redex
with $e$ but do not bother with looking for other redexes in the replacement
term.  This means that the replacement term will not be re-traversed
immediately.  Furthermore, the replaced term will not match any other
left hand side sub-patterns except pattern variables and wildcards.

   This feature is very useful for preventing the rewriter from
looking inside certain terms. 

   To demonstrate, consider the simple following example:
\begin{verbatim}
   datatype Exp = NUM of int 
                | ADD of Exp, Exp
                | SUB of Exp, Exp
                | MUL of Exp, Exp
                | DIV of Exp, Exp
                | FINAL of Exp
                ;
   Exp e = ADD(NUM(1),MUL(NUM(2),NUM(3)));
\end{verbatim}

Suppose we want to find all numbers within 
the expression \T{e} and increment their values by 1.  
The straight forward way of writing:
\begin{verbatim}
   rewrite (e) type (Exp) {
      NUM x: NUM(x+1) 
   }
\end{verbatim}
\noindent will not work, since the replacement term is a new redex,
and the rewriting system will not terminate.

We can reimplement the rewriting system as a two stage process.
First, we mark all terms that we do not want to be changed; whenever we
find a marked term, we execute a \T{cutrewrite} to prevent the term
from being changed.  Then we unmark the terms.
\begin{verbatim}
   rewrite (e) type (Exp) {
      NUM x:   FINAL(NUM(x+1));
   preorder:
      FINAL _: cutrewrite;
   }
   rewrite (e) type (Exp) {
      FINAL x: x
   }
\end{verbatim}

In the first rewrite statement, replacement terms that we 
want to stay fixed are encapsulated
within the auxiliary \T{FINAL} constructor.  
Recall that preorder rules are tried before the subterms of a redex are 
reduced.  Thus we can make sure all that
all terms that are arguments to a \T{FINAL} term is left unaltered
by adding a \T{preorder} rule that executes a \T{cutrewrite}\footnote{
If an expression argument is not given to a \T{cutrewrite} statement, then
the redex is left unchanged.}
Finally, after the first rewrite statement finishes, we use 
the second rewrite statement to remove all \T{FINAL} terms generated
in the first.

The semantics of \T{cutrewrite} needs further explanation.  Consider
the following rewrite system:
\begin{verbatim}
   rewrite (e) type (Exp) {
      NUM x:          cutrewrite(NUM(x+1))
   |  MUL(X, NUM 1):  X
   |  MUL(NUM 1,X):   X
   }
\end{verbatim}
Given the term \verb|MUL(NUM(0),NUM(2))|, the subterms \verb|NUM(0)|
and \verb|NUM(1)|
will be rewritten by the first rule
into \verb|NUM(1)| and \verb|NUM(3)| respectively.
Furthermore, the replacement term \verb|NUM(1)| will {\em not} match
the left subpattern in \verb|MUL(NUM 1,X)|.  Thus the result of the rewrite
will be \verb|MUL(NUM(1),NUM(3))|.

\begin{Tips}
Since the semantics of a rewriting system with \T{cutrewrite} can
be tricky to analyze, its use should be avoided as much as possible. 
\end{Tips}

\Subsection{Conditional failure with \TDEF{failrewrite}}

   From version 2.3.0 onward, the \T{failrewrite}
statement may be used wherever a \T{rewrite} statement is expected.
The \T{failrewrite} statement is an alternate way of writing
conditional rewriting rules, and it is often useful when the condition
to be checked is too complex to be written as a guard.

When a \T{failrewrite} statement is executed within the rhs action of a
rewrite rule, the current pattern is rejected and the rewriter
will proceed to try other matches.   

For instance, suppose the current redex is 
\verb|DIV(INT 0,INT 0)| and the following rules are specified within
a rewriting system:
\begin{verbatim}
   DIV(INT X,INT Y): { if (Y == 0) failrewrite; else rewrite(NUM(X/Y); }
|  DIV(INT 0,Y):     NUM(0)
\end{verbatim} 

The execution of these rules will proceed as follows:  both patterns
match the redex, but the first rule will be tried first.  Since
\verb|Y| is zero, the \T{failrewrite} statement is executed.  The rewriter
will proceed to try the
second rule and the redex will be replaced with \verb|NUM(0)|. 
Note that if we omit the \T{failrewrite} statement
in the first rule, as in:
\begin{verbatim}
   DIV(INT X,INT Y): { if (Y != 0) rewrite(NUM(X/Y); }
|  DIV(INT 0,Y):     NUM(0)
\end{verbatim}
the rewriter will commit on the first rule
and no action will be performed on the redex.

Like \T{rewrite} and \T{cutrwerite}, \T{failrewrite}
acts as a branching statement and all statements
immediatedly following \T{failrewrite} are dead code.

\Subsection{Confluence and termination}

   Currently no automatic checking is available to make sure that
a set of rewriting rules is confluent and will terminate.  This is currently
the user's responsibility.  Some verification features will be developed
in the future.

\Subsection{Debugging Tree Rewriting}

   By default, \Prop{} generates a macro \verb|DEBUG_|$C$ for each
rewrite class named $C$.  The user can redefine this macro in order to
trace the replacements that are performed during the execution of a 
set of rewriting rules.  The default definition of \verb|DEBUG_|$C$
is as follows:
\[
\verb|#define DEBUG_|C({\it R,redex,filename,linenumber,ruletext})\ {\it R}
\]
i.e. it simply returns the value of {\it R}.

The arguments to the macro are as follows: 
\begin{itemize} 
\item {\it R} is the new replacement term.
\item {\it redex} is the current redex, before the replacement has been
performed.
\item {\it filename} is the file name of the current rewriting rule.
\item {\it linenumber}  is the location of the current rewriting rule.
\item {\it ruletext} is a string that describes the current rule.
\end{itemize}

In order to activate tracing, the user can redefine the macro
\verb|DEBUG_FOO| (for rewrite class \verb|FOO|) as follows:
\begin{verbatim}
#define DEBUG_FOO(a,b,c,d,e) print_rule(a,b,c,d,e)

template <class T>
  T print_rule(T replacement, T redex, 
               const char * file_name, 
               int line, const char * rule_text)
  { cerr << file_name << ':' << line << ": " << rule_text 
         << " redex = " << redex << " replacement = " << replacement
         << endl;
    return replacement; 
  }
\end{verbatim}
Note that these definitions must appear
before the \verb|rewrite FOO| declaration.

\Subsection{Optimizing Tree Rewriting} \label{sec:optimize-rewriting}

   An experimental rewriting optimizer based on partial evaluation
techniques has been implemented in release 2.3.4 of \Prop.  The optimizer
can be enabled with the option \OPTIONDEF{-Orewriting}.  

Informally speaking, the rewriting optimizer will try to locate 
``longer'' reduction sequence and add these to your rewriting rules.  
The result is that during execution these bigger reduction steps are
taken whenever possible.

   For example, consider the \verb|Wff| datatype defined
in section~\ref{sec:rewriting-modifier}.  Suppose we have the following
two rewriting rules in our rule set:
\begin{verbatim}
   Not(Not X): X
|  FORALL(X,A,P): Not(EXISTS(X,A,Not(P)))
\end{verbatim}

Given an input term of the form \verb|FORALL(X,A,Not(P))|, the term will
first be reduced to \verb|Not(EXISTS(X,A,Not(Not(P))))| via the second rule,
then to \verb|Not(EXISTS(X,A,P))| via the first rule.  The optimizer
can discover that this two step reduction can be compressed into one
step if the rewriting system is augmented with the following rule:

\begin{verbatim}
   FORALL(X,A,Not P): Not(EXISTS(X,A,P))
\end{verbatim}

Note that this augmented rule is a {\em specialization} of the two rules 
given by the user.  In general, the rewriting optimizer will repeatedly
look for useful specializations and add them to the user rule set\footnote{
To ensure termination of the optimization process, 
augmented rules will not be further specialized.}.  
These specialized rules will perform multiple reductions in one single step.

To make use of the optimizer, the rewriting system must have the following
properties:
\begin{enumerate}
   \item  It must be strongly normalizing, i.e. it will terminate under all
   inputs.  
   \item  It must be confluent, or else confluence is not necessary. 
   Notice that the optimizer can and will alternate the reduction order
   of a rule set.  Thus if a rewriting system is non-confluent, an
   optimized rewriting system may give a different result.
\end{enumerate}

In addition, the following things should hold to maximize the effectiveness
of the optimizer.
\begin{enumerate}
   \item  Indices (i.e. state caching) should be enabled.   Otherwise
   the optimizer will not be able to optimize a rule.
   \item  Specializable rules must be simple, i.e. the right hand side
   must be written as an expression consist of only \Prop{} datatypes.
   If the right hand side is a complex statement or if it involves
   external function calls, the optimizer will fail to analyze it properly.
   \item  Only bottom-up rules may be present.
   \item  Avoid conditional rules whenever possible.  
\end{enumerate}

Finally, it should be warned that an optimized rewriting system  
may generate a lot more code than the un-optimized one.  The user should
view the generated report and check that excessive specializations have not 
been performed.

\Section{User defined datatypes: Views} \label{sec:views}

   For many applications it is necessary to interface \Prop{} to 
data structures predefined in libraries, or generated by other tools.
In order to make it possible to work with these external data structures,
\Prop{}, from version 2.3.3. onward, provides a {\em view} mechanism to
interface with these data structures.  Using views, an externally defined
type can be used within \Prop's pattern matching and rewriting
mechanisms transparently, just as if it were defined by \Prop. 
The \Prop{} translator will call the
appropriate interface functions specified by the user to manipulate
these external datatypes.

\Subsection{A first example} \label{sec:tagged-union}.
To illustrate the use of views, we'll begin with a simple example.

Suppose in our application we use the following tagged C union to
represent our expression datatype.  An expression is represented
simply as a pointer to this structure.  An null expression is
represented simply as the NULL pointer.  

\begin{verbatim}
enum exp_tag { Int = 1, Ident, Add, Sub, Mul, Div };
struct exp {
   enum exp_tag tag;
   union {
      int          number;
      const char * ident;
      struct { struct exp * l, * r } children;
   } u;
};
typedef struct exp * Exp;
/* User defined constructor functions */
Exp NONE = NULL;
extern Exp INT(int);
extern Exp IDENT(const char *);
extern Exp ADD(Exp, Exp);
extern Exp SUB(Exp, Exp);
extern Exp MUL(Exp, Exp);
extern Exp DIV(Exp, Exp);
\end{verbatim}

Suppose this data structure is already used extensively in other 
modules, and we'd like to \Prop{} pattern matching mechanisms while keeping 
the old code intact.  We can accomplish this by declaring a
view to this data structure, using a \TDEF{datatype view}
definition:
\begin{verbatim}
dataype Exp :: view =
   match (this ? this->tag : 0)
     view 0     => NONE
   | view Int   => INT(int = this->u.number)
   | view Ident => ID (const char * = this->u.ident)
   | view Add   => ADD(Exp = this->u.children.l, Exp = this->u.children.r)
   | view Sub   => SUB(Exp = this->u.children.l, Exp = this->u.children.r)
   | view Mul   => MUL(Exp = this->u.children.l, Exp = this->u.children.r)
   | view Div   => DIV(Exp = this->u.children.l, Exp = this->u.children.r)
   ;
\end{verbatim}

In the above, a view named \verb|Exp| is defined.   Let's describe
what the definition means: in general, 
a view definition has three main parts, the {\em view selector},
a set of {\em view transformation rules}, and a set {\em of view accessor
expressions}.  These parts perform the following tasks:

\begin{description}
\item[View selector]  The {\em view selector} is specified in the
the \TDEF{match} part of the definition.   In this example, 
the expression associated with \T{match} is 
\verb|this ? 0 : this->tag|.  This returns a variant discrimination integer
from \verb|0| to \verb|Div|.  
Note that the pseudo variable \TDEF{this} 
refers to an object of type \verb|Exp|.

\item[View transformation rules]
Each of the clauses 
\[ \TDEF{view} {\it exp} \verb|=>| {\it term} \]
specifies
a {\em view transformation rule}.   These rules are used to specify
how to transform a computed variant tag into a pattern.    In general,
expression {\it exp} must be a constant expression usable in a 
\verb|case| statement.

\item[View accessors]
In each of the constructor arguments, an {\em view accessor} expression
can be defined to access the logical components of the constructor.
In the rule:
\begin{verbatim}
   | view Add   => ADD(Exp = this->u.children.l, Exp = this->u.children.r)
\end{verbatim}

The expressions \verb|this->u.children.l| and \verb|this->u.children.r|
tell \Prop{} how to access the left and right child of the an addition
expression node.

Note that since the variant \verb|NONE| has no arguments, it is represented
as a nullary constructor, and no accessors expressions are defined.
\end{description}

A pretty printing function for \verb|struct exp| can be implemented
in \Prop{} using pattern matching as follows:
\begin{verbatim}
ostream& operator << (ostream& s, Exp e)
{  match (e)
   {  NONE:     { s << "none"; }
   |  INT i:    { s << i; }
   |  ID id:    { s << id; }
   |  ADD(a,b): { s << '(' << a << " + " << b << '); }
   |  SUB(a,b): { s << '(' << a << " - " << b << '); }
   |  MUL(a,b): { s << '(' << a << " * " << b << '); }
   |  DIV(a,b): { s << '(' << a << " / " << b << '); }
   }
   return s;
}
\end{verbatim}

Note that this is exactly the same code we would've written if
\verb|Exp| is defined as a datatype rather than a view.  This means
that the user can reimplement the expression data structure
as a \Prop's datatype at a latter time, without altering the \Prop{}
source code.  In effect, the {\em logical} view of a datatype, and its
{\em physical} implementation can separately.
With this {\em representation transparency} capability,
programs written in \Prop's pattern matching formalisms can evolve 
in an incrementally and pain-free manner.

The generated code for the above function will resemble the following
output.  Note that all the view selectors and view accessors are
inlined into the code, and in general, views incur no overhead to
their use.
\begin{verbatim}
ostream& operator << (ostream& s, Exp e)
{  switch (e ? e->tag : 0)
   {  case 0:      { s << "none"; }
      case Int:    { s << e->u.number; }
      case Ident:  { s << e->u.ident; }
      case Add:    { s << '(' << e->u.children.l 
                       << " + " << e->children.r << '); }
      case Sub:    { s << '(' << e->u.children.l 
                       << " - " << e->children.r << '); }
      case Mul:    { s << '(' << e->u.children.l 
                       << " * " << e->children.r << '); }
      case Div:    { s << '(' << e->u.children.l 
                       << " / " << e->children.r << '); }
   }
   return s;
}
\end{verbatim}

\Subsection{Another view example}

Suppose we use the following alternative implementation, a
\verb|Expression| class hierarchy to represent
an expression in our AST.  
Note that \verb|Exp| is an abstract base class inherited
by the classes \verb|Number|, \verb|Operator|, \verb|Identifer|, etc.

\begin{verbatim}
class Expression
{
public:

   // This class uses a ``wide'' interface
   virtual int  number() const         = 0;  // returns a number
   virtual const char * ident() const  = 0;  // returns an identifier

   // returns the ith subchild
   virtual const Expression * child(int) const = 0; 
   // mutable version of above
   virtual Expression *& child(int) = 0;     
};

typedef Expression * Exp;

class NullExp  : public Expression { ... };
class Number   : public Expression { ... };
class Operator : public Expression { ... };
class Identifier : public Expression { ... };
\end{verbatim}

In order to provide a suitable interface to \Prop's view mechanism,
we introduce an additional member function to return a variant tag:

\begin{verbatim}
class Expression
{
public:
   enum Variant { None, Int, Ident, Add, Sub, Mul, Div };
   virtual Variant get_variant() const = 0;
}
\end{verbatim}

In addition, we assume the following constructor functions have been
implemented:

\begin{verbatim}
extern Exp INT(int);
extern Exp IDENT(const char *);
extern Exp ADD(Exp, Exp);
extern Exp SUB(Exp, Exp);
extern Exp MUL(Exp, Exp);
extern Exp DIV(Exp, Exp);
\end{verbatim}

Now, with these definitions in place, we can define a datatype view of the
above \verb|Expression| hierarchy using the following definition:

\begin{verbatim}
datatype Exp :: view
   = match (this->get_variant())
     view Expression::None  => NONE
   | view Expression::Int   => INT (int = this->number()) 
   | view Expression::Ident => ID  (const char * = this->ident())
   | view Expression::Add   => ADD (Exp = this->child(0), 
                                    Exp = this->child(1))
   | view Expression::Sub   => SUB (Exp = this->child(0), 
                                    Exp = this->child(1))
   | view Expression::Mul   => MUL (Exp = this->child(0), 
                                    Exp = this->child(1))
   | view Expression::Div   => DIV (Exp = this->child(0), 
                                    Exp = this->child(1))
   ;
\end{verbatim}

An evaluation function can be written
using \Prop's pattern matching construct:

\begin{verbatim}
int eval(Exp e, const Env& env)
{  match (e) 
   {  NONE:          { return 0; }
   |  INT i:         { return i; }
   |  ID id:         { return env(id); }
   |  ADD (x,y):     { return eval(x,env) + eval(y,env); }
   |  SUB (x,y):     { return eval(x,env) - eval(y,env); }
   |  MUL (x,y):     { return eval(x,env) * eval(y,env); }
   |  DIV (x,INT 0): { cerr << "Division by zero"; return 0; }
   |  DIV (x,y):     { return eval(x,env) / eval(y,env); }
   }
}
\end{verbatim}
Note that this code works equally well with the tagged union representation
defined in section~\ref{sec:tagged-union}. 

We can also use rewriting to transform an expression tree as follows

\begin{verbatim}
void simplify(Exp& e)
{
   rewrite (e) type (Exp)
   {  ADD(INT i, INT j): INT(i+j)
   |  SUB(INT i, INT j): INT(i-j)
   |  MUL(INT i, INT j): INT(i*j)
   |  DIV(INT i, INT j): INT(i/j)
   }
}
\end{verbatim}

\Subsection{Syntax of view definitions}

The general syntax of a view definition resembles that of a
\T{datatype} definition:

\begin{syntax}
\NDEF{Datatype\_Spec} 
   & \IS & \N{Datatype\_View\_Spec} \\
\NDEF{Datatype\_View\_Spec} 
   & \IS & \Id \T{::} \T{view} \T{=} \\ 
   &     & \quad \T{match} \T{(} \N{Exp} \T{)} \\
   &     & \quad \N{View\_Cons\_Specs} \\
\NDEF{View\_Cons\_Specs} 
   & \IS & \ALT{\TDEF{view} \N{Exp} \T{=>} \N{Cons\_Spec}}{|} \\
\end{syntax}

Default arguments in a \N{Cons\_Spec} are interpreted as view accessors.

There are a few general rules to observe when defining a view:
\begin{enumerate}
 \item A \T{datatype view} definition only defines the interface
 to an external data structure and no code is generated.  In particular,
 the defined view name is not an actual \Cpp{} type.
 Thus the user is responsible for defining all appropriate type definitions.
 \item All other qualifiers other than \T{view} are ignored.   Thus 
 automatic generation of pretty printers, garbage collection interface
 routines, etc. are disabled for views.
 \item In general, the view accessors must be usable as lvalues
 if the user wants to modify a view datatype.  This includes the
 situation when in-place rewriting is used.
 \item Finally, the user are responsible for generating the
 constructor functions.  In the two examples given above, the user
 should implement the functions \T{INT}, \T{ID}, etc.  
\end{enumerate}

\Section{Graph Types and Graph Rewriting} \label{sec:graph}

A typical compiler uses many types of graphs to represent internal program 
information.  For instance, control flow graphs, 
data dependence graphs, call graphs, def/use chains are
some of the most common examples.  In order to automate the process
of implementing and manipulating these graph structures, \Prop{} provides
a generic formalism-- {\em graph types}--
for specifying multisorted, attributed, hierarchical directed graphs in 
very high level manner.  The data structure mapping process of translating
high level graph specifications into concrete \Cpp{} classes is completely
automated.  Thus the user can concentrate on specifying the semantic
relationships between the elements of the domain, instead of the
implementation details.

The graph structure mapping process uses real-time set machine
simulation\cite{real-time-SETL} and subtype analysis\cite{subtype-SETL} 
to convert associative accesses involving sets, maps, and multimaps
into hash free, worst-case constant time pointer manipulations.
This optimization is performed transparently by the translator.

Manipulation of graphs structures are done in three ways: 
\begin{description} 
\item[object-oriented] 
 Using a standard interface generated from \Prop{} the user can 
manipulate a graph in the usual procedural/object-oriented manner. 
\item[set formalism]  Using an embedded SETL\cite{SETL}-like sublanguage, 
the user can manipulate the graphs using high level set operations 
such as set comprehension.  
\item[graph rewriting]  At the highest level, analysis and transformation 
can be carried out using the graph rewriting formalism.  
\end{description}

In this section we'll describe each of these topics in detail.

\Subsection{Graph Types} \label{sec:graph-types}

A \INDEX{graph type} declaration is used to specify a
graph structure 
with multiple sorts of labeled nodes and labeled directed edges.  
Attributes can be attached to both nodes and edges.  
The syntax of \T{graphtype} declarations is as follows:
\begin{syntax}
\NDEF{Graph\_Type} 
      & \IS & \TDEF{graphtype} \Id \\
      & & \quad \OPT{\T{:} \N{Inherit\_List}} \\
      & & \quad \OPT{\T{::} \MORE{\N{Graph\_Mode}}{}} \\
      & & \T{declare} \\
      & & \quad \TDEF{node:} \ALT{\N{Node\_Def}}{|} \\
      & & \quad \TDEF{edge:} \ALT{\N{Edge\_Def}}{|} \\
      & & \T{begin} \\
      & & \quad \N{Code} \\
      & & \T{end} \T{graphtype} \T{;} \\
\NDEF{Graph\_Mode} & \IS & \\
\NDEF{Node\_Def} & \IS & \Id \OPT{\OPT{\T{of}} \N{TypeExp}} \\
\NDEF{Edge\_Def} 
    & \IS & \Id \OPT{\T{of}} \TypeExp \T{->} \TypeExp \\
    & \OR & \Id \OPT{\T{of}} \TypeExp \T{<->} \TypeExp \\
    & \OR & \Id \OPT{\T{of}} \TypeExp \T{<=>} \TypeExp \\
    & \OR & \Id \OPT{\T{of}} \TypeExp \T{<=>*} \TypeExp \\
\end{syntax}

\Subsection{The Graph Interface} \label{sec:graph-interface}
\INCOMPLETE

\Subsection{Set Formalisms} \label{sec:set-formalisms}
\INCOMPLETE

\Subsection{Graph Rewriting} \label{sec:graph-rewriting}
\INCOMPLETE

\Section{Running the Translator} \label{sec:usage} 

The \Prop{} translator is a program called \verb|prop|.  The translator
uses the following command line syntax:
\begin{syntax}
  \NDEF{Running\_Prop} & ::= &
   \T{prop} \OPT{{\em prop\_options}} \ALT{{\em file}}{} \\
\end{syntax}

Here, \N{File} is a file with suffix \verb|.p|$*$.  By default, the output
file will have the same name with the \verb|p| extension removed.
For example, a file named ``foo.pC'' will be translated into
the file ``foo.C''  Similarly, a file named ``bar.ph'' will be translated 
into ``bar.h''  

\Subsection{Options}
The available options to the translator are as follows:
\begin{description}
\item[\OPTIONDEF{-G -GNU}]      Use GNU style error messages.
\item[\OPTIONDEF{-I{\em path}}] Use the search path {\em path}.
\Prop{} will search {\em path} to look for \Prop{} files with 
the suffix \verb|.ph| inside a \verb|#include| directive.  Multiple
paths can be specified by using multiple \verb|-I| options.
\item[\OPTIONDEF{-l -no\_line\_directives}]   By default, \Prop{} will try
to generate \verb|#line| directives to correlate the source and
the translated output.  This option suppresses this feature.
\item[\OPTIONDEF{-M -make\_depends}]  Similarly to the \verb|-M| option
in \verb|cc| and \verb|cpp|.  This option generates a suitable dependency
list for Makefiles.
\item[\OPTIONDEF{-memory\_usage}]  Print memory use during translation.
\item[\OPTIONDEF{-n -no\_codegen}]  
Don't output any code; perform semantic checking 
only.
\item[\OPTIONDEF{-N -non\_linear}]  
Use non-linear patterns during pattern matching.
A non-linear pattern is one in which a variable occurs more than
once.   By default, \Prop{} consider this an error.
\item[\OPTIONDEF{-o{\em outfile}}]  Send output to {\em outfile} instead of 
the default.
\item[\OPTIONDEF{-Ofast\_string\_match}]  
Generate (potentially) faster string matching
code by expanding string literal tests into character tests.  By default, 
\Prop{} generates string tests using \verb|strcmp| and binary search.
\item[\OPTIONDEF{-Oadaptive\_matching}]  Use the adaptive pattern matching 
algorithm\footnote{The implementation may not be effective.}.
\item[\OPTIONDEF{-Oinline\_casts}]  
Generate inline type casts instead of function
calls that do the same.  This may be faster for compilation.
\item[\OPTIONDEF{-Orewriting}]  Enable the rewriting optimizer. 
See also section~\ref{sec:optimize-rewriting} for details.
\item[\OPTIONDEF{-Otagged\_pointer}]  
Use a tagged pointer representation instead
of representing variant tags explicitly in a structure field.  For
example, if there are two variants called \verb|Foo| and
\verb|Bar|  in a datatype.  Then a pointer to \verb|Foo| can be tagged
with a low order bit 0 while a pointer to \verb|Bar| can be tagged with 
a low order bit 1.  Accessing the tag thus becomes just a simple bit
test.  This should save space and may be faster\footnote{However, this feature
has not been fully tested.}
\item[\OPTIONDEF{-r -report}]  
Generate a report whenever necessary.  Parser, string
matching, and rewriting constructs produce reports.
\item[\OPTIONDEF{-s -strict}]  Use strict semantic checking mode.  
All warnings are considered to be errors.
\item[\OPTIONDEF{-S -save\_space}]\label{option:save-space} 
Use space saving mode.  
Try not to use inline functions if code space can be saved.  
Datatype constructors will not be inlined
in this mode.  Instead, datatype constructors will be generated at
where the corresponding \TDEF{instantiate datatype} declaration occurs.
\item[\OPTIONDEF{-t -stdout}]   Send translated program to the standard output.
\item[\OPTIONDEF{-use\_global\_pool}]  Use global memory pool for allocation.
\item[\OPTIONDEF{-v{\em num}}]  Use verbose mode in report generation.
This will provide more detailed information.
\end{description}

\Subsection{Error Messages}

The following is a canonical list of all error messages generated by
the \Prop{} translator.  Each message is prefixed by the
file name and line number in which the error occurs.    Most of these
errors messages are self explanatory; more detailed explanations are provided
below.

\begin{Error}
\errormsg{unknown option {\em option}}  The translator does not recognize 
the command line {\em option}.   Type {\em prop} to see a list of options.
\errormsg{Error in command: {\em command}}  The translator fails when trying
to execution {\em command}.  When multiple files are specified in a command
line, {\em prop} invokes itself on each of the file. 
\errormsg{{\em pat} with type {\em type} is not unifiable}  Pattern has not
been declared to be a unifiable type.   
\errormsg{Sorry: pattern not supported in rewriting: {\em pat}}  Pattern
{\em pat} is not supported.  Currently, logical pattern connectives
are not supported in rewriting.  You'll have to rewrite them into
non-logical form.
\errormsg{Unknown complexity operator: {\em op}}  
\errormsg{accessor is undefined for view pattern: {\em pat}} An accessor
function has not been declared for a constructor.  When using datatype views
\errormsg{arity mismatch (expecting $n$) in pattern: {\em pat}}  Pattern
{\em pat} is expected to have arity $n$.   The arity is the number of
expressions that you're matching concurrently.  If you are matching $n$
objects then there must be $n$ patterns for each rule.
\errormsg{arity mismatch in logical pattern: {\em pat}}  Logical patterns
do not match the arity of a pattern.
\errormsg{bad constructor type {\em type}}
\errormsg{bad view constructor pattern: {\em pat}}
\errormsg{can't find include file {\em file}}  Additional search paths
can be specified with the \verb|-I| option
\errormsg{component \#$i$ not found in constructor {\em con}}  
Datatype constructor {\em con} does not have a component named \#$i$.
This can happen when you write a record constructor expression and misspelt
one of the labels.
\errormsg{component $l$ not found in constructor {\em con}}  Datatype
constructor {\em con} does not have a labeled component $l$.
\errormsg{constructor {\em con} already has print formats}
\errormsg{constructor {\em con} is not a class}
\errormsg{constructor {\em con} is not a lexeme}  A constructor is used
as a terminal in the parser but it has not been declared to be a lexeme.
\errormsg{constructor {\em con} doesn't take labeled arguments}  You're trying
to use record constructor syntax on a constructor take does not take
record arguments.
\errormsg{constructor {\em con} is undefined}  
\errormsg{cyclic type definition in type {\em id} = {\em type}}
Type abbreviations cannot be cyclic.  For recursive types, use \verb|datatype|
definitions.
\errormsg{datatype {\em T} is not a lexeme type}  {\em T} is used as a
terminal within syntax class when it has not been declared as a lexeme type.
All non-terminals must be a defined using the \verb|lexeme| qualifier.
\errormsg{determinism {\em det} not recognized}
\end{Error}

\begin{Error}
\errormsg{duplicated definition of pattern constructor '{\em con}'}
  The constructor {\em con} has already been in another datatype.  {\em Prop}
  does not allow overloading of constructor names.
\errormsg{duplicated label '$l$' in expression: {\em exp}}
  A record type has one of its labels duplicated.
\errormsg{duplicated label '$l$' in type {\em type}}
\errormsg{duplicated pattern variable '{\em id}'.  Use option -N}
   By default,
   pattern variables may not be duplicated within a pattern.  Non-linear
   patterns are allowed only when the option \verb|-N| is invoked.
\errormsg{edge $e$ is not defined in graphtype {\em id} }
\errormsg{empty type list in rewrite (...) ... }
\errormsg{expecting $c_1$ ... $c_2$ (from line $l$) but found $c_3$ ... $c_4$ 
   instead}  Quotation symbols are not properly balanced. 
\errormsg{expecting non-terminal {\em nt} to have type $t_1$ but found $t_2$}
   Non-terminal {\em nt} has already been defined to have a
   synthesized attribute type of $t_1$ but $t_2$ is found.   This can happen
   if you have rules with the same lhs non-terminal previously defined.  
\errormsg{expecting type $t_1$ but found $t_2$ in pattern variable '{\em v}'}
{\em Prop} performs type inference on all the lhs patterns to determine
the types of all variables.  Type errors can occur if the patterns are
miswritten.
\errormsg{expecting type $t_1$ but found $t_2$}
\errormsg{flexible vector pattern currently not supported in 
rewriting: {\em pat}}
\errormsg{format \#$i$ used on constructor {\em con}}
\errormsg{format \#$l$ used on non-record constructor {\em con}}
\errormsg{function name mismatch: expecting $f$ ...}
\errormsg{illegal character $c$}
\errormsg{illegal context type: {\em type}}  A context type in
a \verb|matchscan| statement must be previously defined to be datatype.
The unit constructors of the datatype can be used as context values.
\errormsg{illegal context(s) in pattern {\em pat}}
\errormsg{illegal format '\_' on constructor {\em con}}
\errormsg{illegal incomplete record pattern: {\em pat}}
\errormsg{illegal print format '$c$' in constructor {\em con}}
\errormsg{illegal record label '$l$' in expression: {\em exp}}
\errormsg{illegal width in bitfield '{\em id} ($n$)'}
\errormsg{inherited attribute '{\em type}' can only be used in treeparser 
mode in rewrite class {\em id}}  
\end{Error}

\begin{Error}
\errormsg{law {\em id}(...) = {\em pat} is not invertible}   Pattern {\em pat}
cannot be treated as an expression.  It may involve logical patterns and
wild cards.
\errormsg{law '{\em id}': bound variable '$v$' is absent in body {\em exp}}
A parameter variable $v$ must be present within the body of a law.

\errormsg{law '{\em id}': type {\em type} cannot be used in parameter {\em id}}

\errormsg{lexeme {\em id} is undefined}

\errormsg{lexeme class {\em id} is undefined}

\errormsg{lexeme pattern is undefined for constructor {\em con}}

\errormsg{lexeme \{{\em id}\} already defined as {\em regexp}}

\errormsg{lexeme \{{\em id}\} is undefined in {\em regexp}}

\errormsg{missing '\}' in regular expression {\em regexp}}

\errormsg{missing label '$l$' in expression: {\em con} {\em exp}}

\errormsg{missing non-terminal in tree grammar rule: {\em nt}}
\errormsg{missing type {\em type} in the traversal list of rewrite class {\em id}} 
Within the rewriting rules, you have used a pattern that involve a constructor
of type {\em  type} directly but no such types are defined in the 
rewrite class definition.  You should add the type to the traversal list.

\errormsg{missing type info in expression {\em exp} : {\em type}}
   Sometimes {\em prop} fails to infer the type of an expression within
a {\em match} statement.  It such
cases it is necessary to help out the translator by adding additional
type information in the patterns.  For example, rewrite some pattern $p$
as $(p : \tau)$ where $\tau$ is the type of $p$.

\errormsg{missing type info in function {\em f} {\em type}}  Similar to above
\errormsg{missing type info in rule: {\em f} {\em pat} : {\em type}}
   Similar to above.
\errormsg{missing view selector for pattern: {\em pat}}
\errormsg{multiple mixed polarity pattern variable '{\em v}'}
\errormsg{multiple unit constructors in polymorphic type {\em id} {\em arg} 
is not supported}
\errormsg{negative cost {\em cost} is illegal}
\errormsg{node {\em id} is not defined in graphtype {\em id}}
\errormsg{non lexeme type {\em type} in pattern {\em pat}}
\errormsg{non-relation type {\em type} in pattern: {\em pat}}
\errormsg{pattern is undefined for lexeme {\em l}}
\errormsg{pattern scope stack overflows}
\errormsg{pattern scope stack underflows}
\end{Error}

\begin{Error}
\errormsg{pattern variable '$v$' has no binding at this point}
\errormsg{persist object id is undefined for {\em con}}
\errormsg{persistence pid {\em name} already allocated for type {\em type}}
\errormsg{persistence redefined for type {\em type}}
\errormsg{precedence symbol must be terminal: {\em term}}
\errormsg{predicate {\em p}: expecting type $t_1$ but found $t_2$}
\errormsg{redefinition of bitfield '{\em field}'.}
\errormsg{redefinition of constructor '{\em con}'}
\errormsg{redefinition of datatype {\em id}}
\errormsg{redefinition of lexeme class {\em id}}
\errormsg{replacement not in rewrite class: rewrite {\em exp}}
\errormsg{rewrite class {\em id} has already been defined}
\errormsg{rewrite class {\em id} is undefined}
\errormsg{rule $r$ has incomplete type {\em type}}
\errormsg{rule $r$ is of a non datatype: {\em type}}
\errormsg{syntax class {\em id} has already been defined}
\errormsg{syntax class {\em id} is undefined}
\errormsg{synthesized attribute '{\em type}' 
can only be used in treeparser mode in rewrite class {\em id}}
\errormsg{the class representation of constructor {\em id} has been elided}
\errormsg{this rule is never selected: $r$}
\errormsg{this rule is never used: $r$}
\errormsg{too few arguments {\em arg} in instantiation of type scheme {\em type}}
\errormsg{too many arguments {\em arg} in instantiation of type scheme {\em type}}
\errormsg{type {\em type} is not a datatype}
\errormsg{type {\em type} is undefined}
\end{Error}

\begin{Error}
\errormsg{type {\em id} = {\em type} is not a datatype}
\errormsg{type {\em id} has already been defined as {\em type}}
\errormsg{type {\em id} has unknown class: {\em C}}
\errormsg{type '{\em type}' is not rewritable in treeparser mode rewrite class 
{\em id}}
    All datatypes used within the treeparser mode of rewriting must be defined
    with the \verb|rewrite| qualifier.
\errormsg{type error in pattern {\em pat}: {\em type}}
\errormsg{type mismatch between nonterminal {\em nt}(type $t_1$) 
and rule $r$(type $t_2$)}
\errormsg{type mismatch in pattern: {\em pat}}
\errormsg{type mismatch in rule $r$}
\errormsg{type mismatch in rule $r$ {\em pat}}
\errormsg{type or constructor {\em con} is undefined}
\errormsg{unable to apply pattern scheme {\em pat}}
\errormsg{unbalanced $c_1$ ... $c_2$ at end of file}
\errormsg{undefined non-terminal {\em nt}}
\errormsg{unification fails occurs check with $t_1$ and $t_2$}
\errormsg{unmatched ending quote $c$}
\errormsg{unrecognized quoted expression `{\em exp}`}
\errormsg{unrecognized quoted pattern `{\em pat}`}
\end{Error}

\pagebreak
\appendix

\Section{Garbage Collection in the {\sf Prop} Library} \label{appendix:gc}
   In this appendix
we describe the design and implementation of a garbage collector
based on the Customisable Memory Management framework(CMM)\cite{CMM} in our
\Prop{} \Cpp{} class library.  Like the previous approach, we have 
implemented a mostly copying conservative collector based on the work
of Bartlett\cite{Mostly-copying}.  
Similar to CMM, our architecture provides a protocol to allow 
multiple garbage collectors using different algorithms to coexist in the 
same memory space.  A few improvements are made to
improve the performance, the flexibility and the functionality of 
our collector: to reduce retention due to false roots identification,
blacklisting\cite{Boehm} is used to identify troublesome heap
addresses; the architecture of the system has been generalized so that it is 
now possible to have multiple instantiations of Bartlett-style heaps; 
finally, object finalization and weak pointer support
are added.  Our collector has been tested on gcc 2.5.8 under Linux,
and SunOS 4.1.x\footnote{An alpha version also works under Solaris. Thanks
to Paul Dietz for the patch.}

\Subsection{Introduction}
   The \Prop{} project involves the design and implementation of 
an environment and an extension language based on \Cpp{} 
for compiler construction and program transformation.  
An outgrowth of this project is the \Prop{} 
\Cpp{} class library, which contains an extensive set of support algorithms 
and data structures.  Since a typical compiler manipulates many complex tree,
dag and graph objects with extended lifetime, manual memory management using
\Cpp's constructors and destructors, reference counting smart pointers, or
some other techniques is frequently complex and error prone.  
Furthermore, with the advent of new algorithms for garbage collection,
manual memory management techniques do not necessarily provide better 
performance or space utilization.  To make it possible to make use of garbage 
collection as a viable alternative for memory management in
\Cpp\cite{Safe-C++-GC}, we have implemented a garbage collection class 
hierarchy as part of the \Prop{} library.  The class library can be used
directly by the users who'd like to program in \Cpp; it can also be
used as part of the runtime system of a highly level language implemented
in \Cpp, as we have done for \Prop.

   We have several good reasons to prefer garbage collection over manual 
memory management.  The \Prop{} language contains many declarative constructs 
and extensions such as algebraic datatypes, pattern matching, rewriting, and
logical inference.  When a user programs in \Prop{} using these constructs, 
an applicative style of programming is the most natural paradigm.  

\Subsection{The Framework}

   We base our design on the work on the Customisable Memory Management(CMM)
system\cite{CMM}.  In this framework, multiple independent heaps(including
the usually non-collectable heap) can coexist with each other.
Bartlett's mostly copying garbage collector is used as the primary collector.
CMM extends the work of Bartlett\cite{Mostly-copying} 
by allowing cross heap pointers and unrestricted interior pointers.

  However, all collectable objects are required to derive from a base class 
and reimplement a tracing method, which identifies the internal pointers of 
an object.  This burden is usually quite small and in fact can be
automated from the type definition\footnote{In \Prop, a special keyword
\T{collectable} is used to identify garbage collected classes and types.
The \Prop{} translator uses this type annotation to derive the appropriate  
tracing methods.}

  One of the major advantages of the CMM framework is that multiple
coorperating collectors can coexist at the same time, which makes it
possible to customize the behavior of each collector with respect to
the lifetime behavior of the objects.  In \cite{CMM}, examples are
given in which the lifetime of certain class of objects exhibit
first-in/first-out behavior.  In this case a special collector can be
written to take full advantage of this property.

\subsection{Our Framework}
   Our framework retains all the benefits and flexibilities of CMM, 
while extending it in several minor but useful ways:  
\begin{itemize}
   \item Like CMM, interior pointers(i.e. pointers to the middle of an object)
and crossheap pointers (i.e. complex data structures linking nodes locating
in multiple logical heaps.) are supported.  Pointers to collectable
objects are non-intrusive: i.e. they are normal \Cpp{} pointers 
and not encapsulated in templates.
   \item Also like CMM, we allow multiple garbage collectors using different
algorithms to coexist.  Unlike CMM, however, we allow multiple 
Bartlett-style collectors to be instantiated.  Most of the services 
involving low level page management and object marking have been relegated
to a separate heap manager.   
   \item We provide support for finalization and weakpointers.
   \item We have implemented blacklisting\cite{Boehm} to reduce the chance
of false roots identification.
\end{itemize}

\subsection{The Implementation}

   Our implementation has been completely written from scratch since
we do not desire to utilize copyrighted code and, more importantly,
we have to make a few important architectural changes:
 All low level memory management services, such as management of
the page table and the object bitmap, is now relegated to the class 
\CLASSDEF{GCHeapManager}.  Collectors now act as clients as of the heap manager
and the Bartlett-style collector no longer has any special status. 

\subsection{Architecture}

   The architecture of the memory management system is partitioned into
a few classes, each responsible for providing distinct services:

\begin{itemize}
   \item \CLASSDEF{GCHeapManager} --- The heap manager.  The heap manager
    manages the heap table, performs page level
    allocation and deallocation, and provides miscellaneous services
    like blacklisting.  It also manages the object bitmaps.
   \item \CLASSDEF{GC} --- The base class for all garbage collectors.
    This base class describes the protocol used by all the collector
    classes\footnote{This base class is also inherited from class
    \CLASSDEF{Mem}, so that it adheres to the \Prop{} memory management
    protocol.}
   \item \CLASSDEF{CGC} --- The base class for conservative collectors.
     This class is inherited from class \CLASSDEF{GC} and implements some
     methods for locating the stack, heap, and static data areas.
   \item \CLASSDEF{BGC} --- Bartlett-style mostly copying collector.
      This class is inherited from class \CLASSDEF{CGC} and implements
      the Bartlett mostly copying algorithm.
   \item \CLASSDEF{MarkSweepGC} --- Mark/sweep style conservative collector.
      This class is inherited from class \CLASSDEF{CGC} and implements
      a mark/sweep collection algorithm.
   \item \CLASSDEF{WeakPointerManager} --- The weakpointer manager.
      This class manages the weak pointer table and provides a few
      weak pointer scavenging services for the collectors.
\end{itemize}
  
\Subsection{The Programmatic Interface}

   The programmatic interface to the garbage collector involves two
base classes, \CLASSDEF{GC} and \CLASSDEF{GCObject}.  The base class 
\CLASSDEF{GC} provides an uniform interface to all types of collectors 
and memory
managers while class {\sf GCObject} provides the interface to all 
collectable classes.  Table~\ref{GC-Classes} contains a listing
of the classes in our hierarchy.

\begin{table}
   \begin{center}
      \begin{tabular}{|l|l|} \hline
        Class           & Purpose \\ \hline \hline
        \sf GCObject    & Collectable object base class \\
        \sf GC          & Garbage collector base class \\
        \sf CGC         & Conservative garbage collector base class \\
        \sf BGC         & Bartlett style mostly copying collector \\
        \sf MarkSweepGC & A mark-sweep collector \\
        \sf UserHeap    & A heap for pointerless object \\
        \sf GCVerifier  & A heap walker that verifies the integrity of a 
                          structure \\
      \hline 
      \end{tabular}
   \end{center}
   \caption{\label{GC-Classes} Garbage Collection Classes.}
\end{table}

Class \CLASSDEF{GCObject} is extremely simple: it redefines the operators
\verb|new| and \verb|delete| to allocate memory from the default collector,
and declares a virtual method ``\verb|trace|'' to be defined by subclasses
(more on this later.)

Memory for a \CLASSDEF{GCObject} is allocated and freed using the usually
\verb|new| and \verb|delete| operators.  Of course, freeing memory explicitly
using \verb|delete| is optional for many subclasses of \CLASSDEF{GC}. 

\begin{verbatim}
   class GCObject {
   public:
      inline void * operator new(size_t n, GC& gc = *GC::default_gc) 
         { return gc.m_alloc(n); }
      inline void * operator new(size_t n, size_t N, GC& gc = *GC::default_gc) 
         { return gc.m_alloc(n > N ? n : N); }
      inline void   operator delete(void *);
      virtual void trace(GC *) = 0;
   };
\end{verbatim}

\subsection{Memory Allocation}

The base class \CLASSDEF{GC} is slightly more complex, as it has to provide
a few different functionalities.  The first service that class \CLASSDEF{GC} 
must 
provide is of course memory allocation and deallocation.  As a time saving
device we can specify what the default collector is using the methods
\verb|get_default_gc| and \verb|set_default_gc|.  Method \verb|GCObject::new| 
will use this collector by default, unless placement syntax is used.
Method \verb|GCObject::delete|, on the other hand, can correctly infer the 
proper collector to use by the address of the object.

The low level methods to allocate and deallocate memory are \verb|m_alloc|
and \verb|free| respectively.  The programmers usually do not have to
use these methods directly. 

The method to invoke a garbage collection of a specific level is 
\verb|collect(int level)|.  This forces an explicit collection.
Method \verb|grow_heap(size_t)| can also be used to explicitly increase
the heap size of a collector.  Depending of the actual behavior
of the subclasses, these methods may have different effects.

\begin{verbatim}
   class GC {
   protected:
      static GC * default_gc;
   public:
      static GC&  get_default_gc()       { return *default_gc; }
      static void set_default_gc(GC& gc) { default_gc = &gc; }
      virtual void * m_alloc   (size_t) = 0;
      virtual void   free      (void *);
      virtual void   collect   (int level = 0) = 0;
      virtual void   grow_heap (size_t) = 0;
      static  void garbage_collect() { default_gc->collect(); }
      virtual void set_gc_ratio(int);
      virtual void set_initial_heap_size (size_t);
      virtual void set_min_heap_growth   (size_t);
   };
\end{verbatim}

\subsection{The GC Protocol}
The collector and collectable objects must cooperate with
each other by abiding to a simple protocol:
\begin{itemize}
   \item All objects that are to be garbage collected must be derived
from \CLASSDEF{GCObject}.  The application programmer must also supply a
``{\bf tracing}'' method.  The purpose of this method is to identify
all internal pointers to other \CLASSDEF{GCObject}'s.   This method is not
used by the application programmer directly but only used internally
by the garbage collectors.
   \item The tracing method of each collectable must in turn call
    the tracing method in the class \CLASSDEF{GC} with each pointer to
    a collectable object that the object owns:
\begin{verbatim}
   class GC {
   public:
      virtual GCObject * trace (GCObject *) = 0;
      inline  void trace (GCObject& obj);
   };
\end{verbatim}
    
     Briefly, the rules are as follows:
   \begin{enumerate}
      \item The tracing method of a collectable \T{Foo} has the following
         general form:
          \begin{verbatim}
             void Foo::trace(GC * gc) 
             {
                ...
             }
          \end{verbatim} 
      \item If class \T{Foo} has a member that is a pointer \verb|p|
            to a collectable object of type \T{Bar}, then add:
           \begin{verbatim}
               p = (Bar)gc->trace(p);
           \end{verbatim}
            to the body of \verb|Foo::trace|. 
     \item  If class \T{Foo} has a member object
            \verb|x| that is a subclass of \T{GCObject}, also add:
           \begin{verbatim}
               gc->trace(x);
           \end{verbatim}
            to the body of \verb|Foo::trace|.
     \item  If class \T{Foo} is derived from a class \verb|Bar| that
            is a subclass of \T{GCObject}, add:
           \begin{verbatim}
               Bar::trace(gc);
           \end{verbatim}
            to the body of \verb|Foo::trace|.
   \end{enumerate} 
    Notice that these methods can be arranged in any order.
\end{itemize}

This protocol can be used by both copying and non-copying collectors.
In addition, the class \CLASSDEF{GCVerifier} also uses this protocol to 
walk the heap in order to verify the integrity of a garbage collected 
data structure.

\subsection{Messages and Statistics}
    All garbage collectors use the following protocols for status
reporting and statistics gathering.  
  
\begin{verbatim}
   class GC {
   public:
      enum GCNotify {
         gc_no_notify,             
         gc_notify_minor_collection,
         gc_notify_major_collection,
         gc_notify_weak_pointer_collection,
         gc_print_collection_time,
         gc_print_debugging_info
      }
      virtual int      verbosity() const;
      virtual void     set_verbosity(int);
      virtual ostream& get_console() const;
      virtual void     set_console(ostream&);
   };
\end{verbatim}

\begin{verbatim}
   class GC {
   public:
      struct Statistics {
         const char *   algorithm;
         const char *   version;
         size_t         bytes_used;
         size_t         bytes_managed;
         size_t         bytes_free;
         struct timeval gc_user_time;
         struct timeval gc_system_time;
         struct timeval total_gc_user_time;
         struct timeval total_gc_system_time;
      }
      virtual Statistics statistics();
   };
\end{verbatim}

    Each collector has a current verbosity level, which can be set
and reset using the methods \verb|set_verbosity(int)| and
\verb|verbosity() const|.  The verbosity level is actually a bit
set containing flags of the type \verb|GC::GCNotify|.  A combination
of these options can be used.
\begin{itemize}
  \item \verb|gc_no_notify| --- no messages at all.
  \item \verb|gc_notify_minor_collection| --- all minor collections
    will be notified by printing a message on the current console.
  \item \verb|gc_notify_major_collection| --- similarly for all major
    collections.
  \item \verb|gc_notify_weak_pointer_collection| --- notify the user
  when weak pointers are being collected.
  \item \verb|gc_print_collection_time| --- this option will let the
    collector to print the time spent during collection (and finalization).
  \item \verb|gc_print_debugging_info| --- this option will print additional
    information such as stack and heap addresses during garbage collection.
\end{itemize}

   The current console of a collector is defaulted to the \Cpp{} stream
\verb|cerr|.  It can be set and inspected with the methods
\verb|set_console(ostream&)| and \verb|ostream& get_console()| respectively.
For example, the user can redirect all garbage collection messages
to a log file \verb|"gc.log"| by executing the following during initialization:

\begin{verbatim}
   ofstream gc_log("gc.log");
   GC::get_default_gc().set_console(gc_log); 
\end{verbatim}

\subsection{The Bartlett style mostly copying collector}

   Currently a Bartlett-style mostly copying collector has been implemented.
The current version is non-generational but we expect that a generational
version will be available in the near future.

   The Bartlett-style collector is implemented as the concrete class
\CLASSDEF{BGC}.  Aside from all the services provided by class \CLASSDEF{GC},
\CLASSDEF{BGC} also provides the following method:

\begin{verbatim}
   class BGC : public CGC {
   public:
      virtual void set_gc_ratio(int);
      virtual void set_initial_heap_size (size_t);
      virtual void set_min_heap_growth   (size_t);
   }
\end{verbatim}

  The gc ratio refers to the ratio of used heap space versus
total heap space.  Class \CLASSDEF{BGC} will invoke the garbage collection
if this is exceeded.  The default gc ratio for \T{BGC} is 75\%.

  The initial size of the heap and the minimal number of bytes to 
request from the system during a heap expansion can be adjusted using 
the methods \verb|set_initial_heap_size| and \verb|set_min_heap_growth|
respectively.  These methods are declared in the \T{GC}
protocol.   By default, the initial heap size for this collector
is 128K and each heap expansion will increase the heap by 512K. 

If an application uses a lot garbage collected storage it is a good
idea to set the heap size to larger capacity during initialization.
Otherwise, the collector will have to perform more garbage collection
and heap expansions to reach a stable state. 

\subsection{The Mark Sweep collector}

   An alternative to the copying collector is the the mark sweep collector.
Like the previous collector, this collector also uses conservative scanning 
to locate roots.  Unlikely the Boehm collector, type accurate marking
is used through the user supplied tracing method.

   Since the default collector is of the \T{BGC} variety, the user must
create an instance of \T{MarkSweepGC} if the mark sweep collector
is to be used.  

\begin{verbatim}
   class MarkSweepGC : public CGC {
   public:
      virtual void set_gc_ratio(int);
      virtual void set_initial_heap_size (size_t);
      virtual void set_min_heap_growth   (size_t);
   }
\end{verbatim}

   The gc ratio in class \T{MarkSweepGC} determines whether heap
expansion is to be performed after a collection.  The default gc ratio
is 50\%, thus if after garbage collection the ratio of used versus total
heap space exceeds one half, the heap will be expanded. 

   For most purposes the two collectors are interchangeable.  Since
all types of garbage collectors under our framework use the same protocol, 
applications can be written without fixing a specific garbage collection
algorithm before hand.  

\subsection{Finalization} 

   One common \Cpp{} idiom uses constructor and destructor for
resource allocation: resources (including memory but may include others,
such as file handles, graphical widgets, etc) that are acquired
in a class constructor are released(finalized) in the class'es destructor.

   There are, however, two opposing views in how finalization should be handled
in a garbage collector.  The first assumes that garbage collection
simulates an infinite amount of memory and so automatic finalization is
inapproriate.   The second takes a more pragmatic view, and assumes that
automatic finalization should be provided since otherwise explicit resource
tracking must be provided by the user for collectable datatypes, making garbage 
collection much less useful than it can be.

   We will refrain from participating in any religious and philosophical
wars on which dogma is the ``One True Way''. 
nstead, both types of collectors are provided.

   By default, all garbage collection classes do 
not perform finalization on garbage 
collected objects.  If object finalization is desired then the user
can do one of two things:
\begin{itemize}
   \item Enable the finalization of the default heap by putting
\begin{verbatim}
   GC::get_default_gc().set_finalization(true);
\end{verbatim}
   in the initialization code, or
   \item Instantiate a different instance of the garbage collector
    and allocate objects needing finalization from this collector.
    This may be a better method since not all objects need finalization
    and those that do not can still be allocated from the default collector.
    This way we only have to pay for the performance penalty (if any)
    proportional to the usage of finalization.
\end{itemize}

   For example, if a collectable class named \T{Foo} should be properly
finalized we can declare it in the following manner:

\begin{verbatim}
   extern BGC bgc_f;  // somewhere an instance is defined

   class Foo : public GCObject {
      Resource r;
   public:
      Foo() { r.allocate(); /* allocate resource */ }
     ~Foo() { r.release(); /* release all resource */ }

      // Make object allocate from bgc_f instead
      // of the default non-collectable gc.
      void * operator new (size_t n) { return bgc_f.m_alloc(n); }
   };
\end{verbatim}

   When an instance of class \verb|Foo| is identified as garbage, its
destructor will be called.  [Aside: {\it currently issue such as 
finalization synchronization is not handled directly by the collector.
So if there is synchronization constraints during finalization it must
be handled by the object destructor.  Future versions of \Prop{} will
provide subclasses with more sophisticated finalization mechanisms.} ]

   Notice that the default \verb|delete| operator of all garbage collectable
classes will call the object destructor explicitly.  It is a good idea to use
explicit \verb|delete| if finalization is a rare need since this service
involves an explicit scan of the garbage memory, which may degrade performace
with excessive swapping: without finalization the garbage pages will not 
have to be referenced with a copying collector.

   It should also be noted that since the collector is conservative, there
is no guarantee that garbage will be identified as such:
there is no guarantee that all garbage resources will be released.
In general, however, the efficacy of the collector is quite good and
so non-critical or plentiful resources can be safely finalized with
this collector.

\Subsubsection{Weak Pointers}

   It is frequently very useful to be able to keep track of garbage
collectable objects through the use of ``{\bf weak pointers}.''
Collectors ignore the presence of weak pointers to garbage collected
objects; objects with only referencing weak pointers will still be
collected.  Weak pointers to objects that become garbage will
be automatically reset to null.
 
   Weak pointers are provided through a smart pointer template 
\CLASSDEF{WeakPointer}, whose definition is shown below: 

\begin{verbatim}
template <class T> class WeakPointer {
   public:
      inline WeakPointer();
      inline WeakPointer(const WeakPointer<T>& wp); 
      inline WeakPointer(T * ptr); 
      inline WeakPointer<T>& operator = (const WeakPointer<T>& wp);
      inline WeakPointer<T>& operator = (T * ptr);
      inline bool is_null() const;
      inline operator const T * () const;
      inline operator       T * ();
      inline const T * operator -> () const;
      inline       T * operator -> ();
      inline const T&  operator *  () const;
      inline       T&  operator *  ();
   };
\end{verbatim}

A weakpointer to a garbage collectable 
class \verb|A| can be defined as \verb|WeakPointer<A>|.
If a \Prop{} datatype \verb|T| has been defined, a weakpointer to type
\verb|T| can defined using the
\TDEF{classof} keyword.  For example:

\begin{verbatim}
   datatype Wff :: collectable = True | False | And(...)  | ...;

   WeakPointer<classof Wff> a = And(True,False);
\end{verbatim}

\Subsubsection{The Heap Walker}

   There is a simple class called \CLASSDEF{GCVerify} that can be used
to verify the integrity of a complex allocated data structure.
This is frequently useful during the development of a new collector,
or a complex class derive class of \T{GCObject}.

   The interface of this class is shown below.

\begin{verbatim}
   class GCVerifier : protected GC {
   public:
      virtual bool is_valid_pointer          (GCObject *);
      virtual bool is_valid_interior_pointer (GCObject *);
      virtual bool is_valid_structure        (GCObject *);
      virtual bool is_valid_pointer          (GCObject *, GC *);
      virtual bool is_valid_interior_pointer (GCObject *, GC *);
      virtual bool is_valid_structure        (GCObject *, GC *);
      size_t number_of_nodes() const;
   };
\end{verbatim}

   Class \CLASSDEF{GCVerify} is derived from class \CLASSDEF{GC} so that the
same object tracing protocol can be used.  Briefly, three different methods 
are provided for verification:
\begin{itemize}
   \item Call to \verb|is_valid_pointer(p,gc)| verifies that
    \verb|p| is a valid pointer to an object within the collector \verb|gc|. 
   \item Call to \verb|is_valid_pointer(p,gc)| verifies that
    \verb|p| is a valid interior pointer to an object 
     within the collector \verb|gc|. 
   \item Call to \verb|is_valid_structure(p,gc)| verifies the entire
     structure reachable by \verb|p| is valid.  This method uses
     \verb|GCObject::trace| to locate the correct sub-structures.
\end{itemize}

   There are alternative versions of the same methods that assumes
the default collector is used.   Furthermore
\begin{itemize}
   \item Method \verb|number_of_nodes()| returns the node count of the
last structure traced using \verb|is_valid_structure|, and 
   \item If \verb|set_verbosity(GC::gc_print_debugging_info)| is used
then error messages will be printed during structure tracing.
\end{itemize}

\bibliography{refman}

%%%%%%%%%%%%%%%%%%%%%%%%%%%%%%%%%%%%%%%%%%%%%%%%%%%%%%%%%%%%%%%%%%%%%%%%%%%%%%%
%
%  Here are some abbreviations used within this document
%
%%%%%%%%%%%%%%%%%%%%%%%%%%%%%%%%%%%%%%%%%%%%%%%%%%%%%%%%%%%%%%%%%%%%%%%%%%%%%%%
\newfont{\bsf}{cmssbx10}
\newfont{\largebsf}{cmssbx12}
%\newfont{\nontermfont}{cmssbx10}
\newcommand{\nontermfont}{\it}
\newcommand{\codefont}{\tt}
\newcommand{\symfont}{\bf}

\newcommand{\Prop}{{\bsf Prop}}               % typeset in sans serif
\newcommand{\LargeProp}{{\largebsf Prop}}
\newcommand{\plusplus}{{\tt++}}             % ++
\newcommand{\Cpp}{{C{\tt++}}}               % C++ logo
\newcommand{\Gpp}{{G{\tt++}}}               % G++ logo
\newcommand{\email}{{\tt leunga@cs.nyu.edu}}  % my email address
\newcommand{\web}{{\tt http://valis.cs.nyu.edu:8888/\~{}leunga}} % my web page
\newcommand{\propweb}{\web{\tt/prop.html}}
\newcommand{\Version}{2.3.4}                  % Prop version
\newcommand{\comment}[1]{}

\newcommand{\sectionfont}{}
%\newcommand{\Section}[1]{\section{\sectionfont #1}}
%\newcommand{\Subsection}[1]{\subsection{\sectionfont #1}}
%\newcommand{\Subsubsection}[1]{\subsubsection{\sectionfont #1}}

\newcommand{\Section}[1]{\pagebreak\section{\sectionfont #1}%
   \markright{\usebox{\LINE}\large \thesection. \sc #1}
   }
\newcommand{\Subsection}[1]{\subsection{\sectionfont #1}}
\newcommand{\Subsubsection}[1]{\subsubsection{\sectionfont #1}}

\newcommand{\N}[1]{\mbox{{\nontermfont #1}}}
\newcommand{\T}[1]{{\codefont #1}}
\newcommand{\MORE}[2]{#1\T{#2} \mbox{$\ldots$} \T{#2} #1}
\newcommand{\ALT}[2]{#1\ \T{#2} \mbox{$\ldots$} \T{#2} #1}
\newcommand{\LIST}[1]{\MORE{#1}{,}}
\newcommand{\SEQ}[1]{\MORE{#1}{;}}
\newcommand{\OPT}[1]{{\it [ #1 ]}}
\newcommand{\C}[1]{\mbox{\rm #1}}
\newcommand{\IS}{{\symfont ::=}}
\newcommand{\OR}{\ \mbox{$\symfont \mid$}\ }
\newcommand{\Id}{\N{Id}\ }
\newcommand{\Pat}{\N{Pat}\ }
\newcommand{\Exp}{\N{Exp}\ }
\newcommand{\TypeExp}{\N{Type\_Exp}\ }
\newenvironment{syntax}{\begin{quotation}\noindent\begin{tabular}{lcll}} %
   {\end{tabular}\end{quotation}}

\newcommand{\CLASSDEF}[1]{\T{#1}\index{Classes!{\codefont #1}}}
\newcommand{\INDEX}[1]{\index{#1}}
\newcommand{\NDEF}[1]{\N{#1}\index{Syntax!Non--terminals!{\nontermfont #1}}}
\newcommand{\OPTIONDEF}[1]{{\tt #1}\index{Command line options!{\tt #1}}}
\newcommand{\TDEF}[1]{\T{#1}\index{Syntax!Keywords!{\codefont #1}}}
\newcommand{\INCOMPLETE}{{\em This section is incomplete.}}
\newenvironment{Error}{\begin{center}\begin{tabular}{p{3in}p{3in}}
  \bf Error & \bf Explanation}{\\\hline\end{tabular}\end{center}}
\newcommand{\errormsg}[1]{\\\hline {\tt #1}&}
\newenvironment{Tips}{\noindent {\bf Tips:} \quad}{}

\newcommand{\support}{
   {\small\noindent This work is supported in part by
    an award from Hewlett-Packard Corporation, from IBM corporation, and
    an NYU Research Challenge Grant.
   }}
\newcommand{\warning}{
   {\small\noindent The author will neither assume responsibility for any
   damages caused by the use of this product, nor accept warranty or update
   claims.  This product is distributed in the hope that it will be useful,
   but WITHOUT ANY WARRANTY; without even the implied warranty of
   MERCHANTABILITY or FITNESS FOR A PARTICULAR PURPOSE.

   \small\noindent This product is in the public domain,
   and may be freely distributed.

   \small\noindent {\sf Prop} is a research prototype. 
    The information contained in this document is subject
    to change without notice.
   }}

   \makeindex

%%%%%%%%%%%%%%%%%%%%%%%%%%%%%%%%%%%%%%%%%%%%%%%%%%%%%%%%%%%%%%%%%%%%%%%%%%%%%%
%
%   Beginning of the document
%
%%%%%%%%%%%%%%%%%%%%%%%%%%%%%%%%%%%%%%%%%%%%%%%%%%%%%%%%%%%%%%%%%%%%%%%%%%%%%%
\documentclass{article}
\usepackage{html}

   \pagestyle{myheadings}

   \title{{\Huge \bf Prop \\ 
           Language Reference Manual} \\ 
          {\large \bf (for version \Version)}
         } 
   \author{{\Large Allen Leung} \\ 
           E-mail: \email \\
           \htmladdnormallink{\web}{http://valis.cs.nyu.edu:8888/~leunga}
	   \\
           Courant Institute of Mathematical Sciences \\
           251 Mercer Street \\
           New York, NY 10012 \\
          }

   \setlength{\textwidth}{6.6in}
   \setlength{\evensidemargin}{0in}
   \setlength{\oddsidemargin}{0in}
   \setlength{\textheight}{8.75in}
   \setlength{\topmargin}{-0.5in}
   \setlength{\parskip}{1mm}
   \setlength{\parindent}{3ex}
   \setlength{\itemsep}{0.5ex}

   \newsavebox{\LINE}
   \savebox{\LINE}[0.0in][l]{\rule[-1ex]{\textwidth}{.01in}}
\begin{document}

   \maketitle
   \vspace{\fill}
   %\support

   \begin{abstract}
    This reference manual provides an introduction to
    \Prop, release \Version.  \Prop{} is a multiparadigm extension of \Cpp,  
    and is designed for building high performance compiler 
    and language transformation systems, using pattern matching and rewriting.  
    This guide describes the syntax and semantics of
    the \Prop{} language and describes how to develop programs
    using the \Prop{} to \Cpp{} translator.

   \end{abstract}
   \warning
   \titlepage
   \tableofcontents
   \bibliographystyle{alpha}

\Section{Introduction} \label{sec:intro}
 
   This is the language reference manual for \Prop, a compiler generator.
\Prop{} can be used to generate lexers, parsers, analysis and 
transformation tools using pattern matching and rewriting.  
The \Prop{} language is a {\em multi-paradigm} 
extension of \Cpp\cite{C++} and
\Cpp{} programs generated from \Prop{} specifications can be compiled
into efficient code.

   Version \Version{} of \Prop{} includes the following major features:
\begin{enumerate}
   \item Algebraic datatypes and pattern matching, and automatic
mapping of algebraic datatypes into \Cpp{} class hierarchies.
   \item A LALR(1) parser generator in the tradition of {\it yacc} and
{\it bison}.  Each grammar is encapsulated into a parser class and 
it is possible to have multiple instances of a parser.
   \item Regular expression string matching and lexer generation.
   \item Transformation of algebraic datatypes using conditional
   rewriting.
   \item In addition, using the view mechanism, 
it is possible to treat externally defined C or \Cpp{}
data structures as if they were \Prop{} defined datatypes.  
The full range of pattern matching and rewriting capability can
be used on these externally defined data..
\end{enumerate}

   In addition, we intend to implement/fix-up the following features in the 
near future:
\begin{enumerate}
   \item Persistence.
   \item Pretty printing/reading.
   \item Visualization of data structures using {\em vcg}\cite{vcg,vcg-manual}.
   \item Automatic mapping of graph structure specifications into
\Cpp{} classes.
   \item Graph rewriting system generation.
\end{enumerate}

\Subsection{Availability}

   The latest release of \Prop{} \Version{} is available from
\begin{quotation}
\htmladdnormallink{\propweb}{http://valis.cs.nyu.edu:8888/~leunga/prop.html}
\end{quotation}
New updates of the source code and documentation will be available
from this site.

\Subsection{Organization of this Reference Manual}
   This reference manual is organized as follows.  Section~\ref{sec:general}
overviews the general features of \Prop.
Section~\ref{sec:parser}
describes the lexer and parser specification formalisms.
Section~\ref{sec:datatype} describes the algebraic datatypes and 
pattern matching features.  Section~\ref{sec:tree-rewriting} describes
the tree rewriting mechanism.  Finally, section~\ref{sec:usage} describes
the process of running the \Prop{} translator.

   We'll use the following extended BNF notation to specify the syntax
of the language.  Terminals are typeset in {\tt typewriter} font,
while non-terminals are typeset in {\nontermfont itallics} font.
Given a syntactic class \N{S}, we use \LIST{\N{S}} to denote one
or more occurrences of \N{S} separated by commas.  Similarly,
\SEQ{\N{S}} denotes one or more occurrences of \N{S} separated by
semi-colons.  Slanted brackets are used to denote an optional occurrence
of a syntactic class.   We'll combined the two forms when we
want to denote zero or more occurrences of some term.
For example, \OPT{\LIST{\N{S}}} denotes
zero or more occurrences of \N{S}.

\Section{General Concepts} \label{sec:general}

\Subsection{Syntactic Conventions}

The syntax of \Prop{} is an extension of \Cpp.  Most new constructs
of \Prop{} starts with new keywords, listed below.  
\begin{center}
\tt
\begin{tabular}{lllllll}
applicative & arb & as & bagof & begin & bitfield & card \\
category & classof & collectable & constraint & dataflow & datatype & declare \\
dequeof & dom & domain & edges: & elsif & end & equiv: \\
exists & expect: & feature & finalizable & forall & fun & function \\
functor & graphrewrite & graphtype & implies: & inference & inherited & inline \\
instantiate & is & law & left: & less & lexeme & listof \\
loop & mapof & match & matchall & matchscan & module & multimapof \\
mutable & nodes: & of & persistent & prec: & printable & priqueueof \\
procedure & queueof & ran & refine & relation & rewrite & right: \\
setof & sharing & signature & space: & syntax & synthesized & then \\
time: & topdown & traced & treeparser & tupleof & unifiable & unique \\
view & where & with & xor: & bottomup: & topdown: & before: \\
preorder: & postorder: & cutrewrite & failrewrite & index: \\
\end{tabular}
\end{center}

   In addition, the following new multi-character symbols are added:
\begin{verbatim}
   <->  <=>  :=  :-  .(   .[
   #[   #(   #{  [|  |]   (|  |)  {|  |} 
\end{verbatim}

\Subsection{Basic syntactic classes}


We'll use \NDEF{Integer}, \NDEF{Char}, \NDEF{Real}, \NDEF{String} 
and \NDEF{Id} to denote the syntactic classes of 
integers, characters, real numbers, strings and identifiers, respectively.
As in \Cpp, characters are quoted with the single quotes \verb|'|, while
strings are quoted with double quotes \verb|"|.  An identifier is
any sequence of alphanumeric of characters that begins with a letter.
In addition, we'll use the syntactic class
\NDEF{Stmt} to denote any valid combination of \Cpp{} and \Prop{}
statements.

\Subsection{Literals}

  In addition to the usual \Cpp{} literals, \Prop{} introduces two new type
of literals: {\em quark literals} and {\em regular expression literals}.
Quark literals are of type \T{Quark} and are written as strings
prefixed by the \verb|#| symbol.
\begin{syntax}
  \NDEF{Quark} & \IS & \T{\#}\N{String} \\
\end{syntax}   

Quarks act like atoms in Lisp, in which name equality implies
pointer equality.  In contrast, in \Cpp{} two string literals that are
identical in value do not necessarily reside in the same address.
Given strings \verb|s1| and \verb|s2|, \verb|strcmp(s1,s2) == 0| does
not imply \verb|s1 == s2|.  Quarks are defined in the library include file
\verb|<AD/strings/quark.h>|.

Regular expression literals are similar to string literals except
that they are quoted by slashes \verb|/|.   We'll discuss
the regular expression in section~\ref{sec:regular-expressions}.

\Subsection{The \Prop{} Language}

The basic \Prop{} language is the same as \Cpp{} with the following
syntactic extensions:
\begin{quotation}
\begin{tabular}{lll}
         & \bf Keywords              & \bf Functions \\
  \bsf 1 & \T{datatype} \T{refine} \T{instantiate}
                                     & Algebraic datatype definitions \\
  \bsf 2 & \T{match} \T{matchall}    & Pattern matching \\
  \bsf 3 & \T{rewrite}               & Rewriting \\
  \bsf 4 & \T{lexeme}                & Lexical category definition \\
  \bsf 5 & \T{matchscan}             & Lexical scanning \\
  \bsf 6 & \T{syntax}                & Parser specification \\
  \bsf 7 & \T{graphtype}             & Graph data structure specification \\
  \bsf 8 & \T{graphrewrite}          & Graph rewriting \\
\end{tabular}
\end{quotation}

Like \Cpp, a \Prop{} is typically divided into the {\em specification},
which defines the data structure and its interface, and the
{\em implementation} parts.   \Prop{} specifications should
be placed in a file with a \verb|.ph| suffix, while an implementation
should be placed in a file with a \verb|.pC|\footnote{The suffixes {\tt .pcc},
{\tt .pCpp} etc. can also be used.}.  The translator will convert each
\verb|.p|$xxx$ into a file with the suffix \verb|.|$xxx$.  The translated
output can then be fed to the \Cpp{} compiler for compilation.
\vspace{\fill}

\Section{Lexer and Parser Generation} \label{sec:parser}

   In this section we'll describe the lexer and parser generation
mechanism of \Prop.  Note that the use of these features is not
mandatory: the user can substitute any other lexer/parser generation tools 
or use hand written lexers and parsers.  
On the other hand, higher level of integration is possible when using
\Prop's lexer and parser mechanisms.

\Subsection{Lexer Specification}
   Lexical analyzers are specified using the \verb|matchscan|
statement.  This construct is responsible for generating the actual string
matching DFA.  The actual buffering mechanisms are provided by the
classes \verb|LexerBuffer|, \verb|IOLexerBuffer| and \verb|IOLexerStack|. 
These classes are part of the support library distributed with \Prop.

   We assume that the user is familiar with lexer generator tools like
{\em lex}\cite{lex} or {\em flex}\cite{flex}.

\Subsubsection{Regular expressions} \label{sec:regular-expressions}
   Figure~\ref{fig:regexp} describes the set of current
supported regular expressions in \Prop.  The syntax is similar
to what is found in {\em lex}, or {\em egrep}.

\begin{figure}[htbp]
\begin{center}
  \begin{tabular}{|l|l|} \hline
    $c$                & matches character $c$ if it is not a meta character \\
    $e_1e_2$           & matches $e_1$ then $e_2$ \\
    \verb|.|           & matches any character except \verb|\n| \\
    $\verb|\|c$        & matches escape sequence $c$ \\
    \verb|^|$e$        & matches $e$ at the start of the line \\
    $e_1\verb.|.e_2$   & matches $e_1$ or $e_2$ \\
    $e\verb|*|$        & matches zero or more $e$ \\
    $e\verb|+|$        & matches one or more $e$ \\
    $e\verb|?|$        & matches zero or one $e$ \\
    \verb|(|$e$\verb|)| & grouping \\
    \verb|<<|$C_1, C_2 \ldots C_n$\verb|>>|$e$ & matches $e$ only
         if we are in a context from one of $C_1, C_2, \ldots C_n$ \\
    \verb|{|{\em lexeme}\verb|}| & matches lexeme \\
  \hline
  \end{tabular}
\end{center}
\caption{\label{fig:regexp} Regular expressions.}
\end{figure}

The symbols \verb%\ [ ] ( ) { } << >> * + . - ? |% are meta characters and are
interpreted non-literally.  The escape character \verb|\| 
can be prepended to the meta characters if they occur as literals in context.  

Precedence-wise, meta characters \verb|*|, \verb|+| and \verb|?| bind tighter
than juxtaposition.  Thus the regular expression
\verb|ab*| means \verb|a(b*)|.  Parenthesis can be used to override
the default precedence.

Character classes are of the form as found in {\em lex}: 
(i) $c_1$\verb|-|$c_2$ denotes the range of characters from
$c_1$ to $c_2$; (ii) single (non-meta) characters denote themselves; 
(iii) the meta character \verb|^| can be used to negate the set.
For example, the regular expression \verb|[a-zA-Z][a-zA-Z0-9]*| 
specifies an alphanumeric identifier that must starts with a letter.
Similarly, the regular expression \verb|[^ \t\n]| matches any character
except a space, a tab or a newline.

{\em Lexemes} are simply abbreviated names given to a regular expression
pattern.  They act like macros in {\em lex}.

While a lexer is scanning, it may be in one of many {\em contexts}.
Contexts can be used to group a set of related lexical rules; such rules
are only applicable when the contexts are active.  This makes the
lexer behave like a set of DFAs, with the ability to switch between
DFAs under programmer control.  


\Subsubsection{Lexeme}

Lexemes are defined using the \T{lexeme} declaration in \Prop. 
Its syntax is
\begin{syntax}
  \NDEF{Lexeme\_Decl} & \IS & \TDEF{lexeme} \ALT{\N{Lexeme\_Eq}}{|} \T{;} \\
  \NDEF{Lexeme\_Eq}   & \IS & \Id \T{=} \N{Regexp} \\
\end{syntax}

For example, the following lexeme definition is used in the \Prop{}
translator to define the lexical structure of common lexical items.

\begin{verbatim}
lexeme digits      = /[0-9]+/
     | sign        = /[\+\-]/
     | integer     = /{digits}/
     | exponent    = /[eE]{sign}?{digits}/
     | mantissa    = /({digits}\.{digits}?|\.{digits})/
     | real        = /{mantissa}{exponent}?/
     | string      = /"([^\\"\n]|\\.)*"/ 
     | character   = /'([^\\'\n]|\\.[0-9a-f]*)'/
     | regexp      = /\/([^\/\n*]|\\.)([^\/\n]|\\.)*\//
     ;
\end{verbatim}

Note that regular expression literals are written between
slashes: \verb|/|$re$\verb|/|.  

\Subsubsection{Lexeme class}

Often we wish to group a set of lexemes together into {\em lexeme classes} if 
they logically behave in some uniform manner; for example, if they
act uniformly in a lexical rule.  By grouping related lexemes together
into a class we can refer to them succinctly by their class name.

The syntax of a lexeme class declaration is

\begin{syntax}
  \NDEF{Lexeme\_Class\_Decl} & \IS & \TDEF{lexeme class} 
       \ALT{\N{Lexeme\_Class\_Eq}}{and} \T{;} \\
  \NDEF{Lexeme\_Class\_Eq} & \IS & \Id \T{=} \ALT{\N{Lexeme\_Spec}}{|} \\
  \NDEF{Lexeme\_Spec} & \IS & \N{String} \\
                   & \OR & \Id \N{Regexp} \\
\end{syntax}

The following example,
the lexeme classes \verb|MainKeywords|, \verb|Symbols|, and \verb|Literals| 
are defined.  
\begin{verbatim}
lexeme class MainKeywords =
    "rewrite" | "inference" | "match" | "matchall" | "matchscan"
  | "refine"  | "classof" | "type" | "datatype" | "instantiate"
  | "lexeme"  | "bitfield" | "begin" | "syntax"
  | "dataflow" | "module" | "signature" | "constraint" | "declare"
  | "procedure" | "fun" | "function" | "domain" 
  | "graphtype" | "graphrewrite"

and  Symbols =
    ".." | "..." | "<->" | "::" | "&&" | "||" | "++" | "--" | "->" 
  | "<<" | ">>" | ">=" | "<=" | "+=" | "-=" | "*=" | "/=" | "%=" | "==" 
  | "!=" | "<<=" | ">>=" | "&=" | "|=" | "^=" | "=>" | "<-" | "<=>"
  | ":=" | ":-" | LONG_BAR /\-\-\-\-\-+/

and Literals =
    INT_TOK     /{integer}/
  | REAL_TOK    /{real}/
  | CHAR_TOK    /{character}/
  | STRING_TOK  /{string}/
  ;
\end{verbatim}

\Subsubsection{Tokens}

Lexical tokens are defined using the \verb|datatype| declaration. 
The syntax is as follows.
\begin{syntax}
  \NDEF{Tokens\_Decl} & \IS & \TDEF{datatype} \Id \T{::} \TDEF{lexeme} \T{=} \\
      & & \quad \ALT{\N{Token\_Spec}}{|} \T{;} \\
  \NDEF{Token\_Spec} & \IS & \TDEF{lexeme class} \Id & 
       \C{Include a lexeme class} \\
                  & \OR & \Id \OPT{\N{Regexp}}  &
       \C{Single token spec} \\
\end{syntax}

A token datatype is defined by including one of more lexeme classes,
and by defining stand-alone tokens.  If a lexeme class is included,
all the tokens defined within the lexeme class are included.  A \Cpp{}
\T{enum} type of the same name as the token datatype is generated.
If a token is given an identifier name, then the same name is used
as the \T{enum} literal.  On the other hand, if a string is used
to denote a token, then it can be referred to by prefixing the
string with a dot \verb|.|  For example, the token \verb|"=>"| can
be referenced as \verb|."=>"| within a program.

As an example, the following token datatype definition is used within
the \Prop{} translator.  Here, the keywords are first partitioned into
6 different lexeme classes.  In additional, the tokens \T{ID\_TOK},
\T{REGEXP\_TOK}, etc. are defined.

\begin{verbatim}
datatype PropToken :: lexeme =
    lexeme class MainKeywords
  | lexeme class Keywords
  | lexeme class SepKeywords
  | lexeme class Symbols
  | lexeme class Special
  | lexeme class Literals
  | ID_TOK       /{patvar}/
  | REGEXP_TOK   /{regexp}/
  | QUARK_TOK    /#{string}/
  | BIGINT_TOK   /#{sign}{integer}/
  | PUNCTUATIONS /[\<\>\,\.\;\&\|\^\!\~\+\-\*\/\%\?\=\:\\]/
  ;
\end{verbatim}

\Subsubsection{The \T{matchscan} statement}\label{sec:matchscan}

The \T{matchscan} statement is used to perform tokenization.  The user can
specify a set of string pattern matching rules within a \T{matchscan}
construct. Given a object of class \verb|LexerBuffer|, the \T{matchscan}
statement looks for the rule that matches the longest prefix from
the input stream and executes the action associated with the rule.
Ties are broken by the lexical ordering of the rules.  

The general syntax is as follows:
\begin{syntax}
\NDEF{Matchscan} 
      & \IS & \N{Matchscan\_Mode} \OPT{\N{Context\_Spec}} \T{(} \N{Exp} \T{)} 
             & variant 1 \\
      & &  \T{\{} \ALT{\T{case} \N{Matchscan\_Rule}}{} \T{\}} \\
      & \OR & \N{Matchscan\_Mode} \OPT{\N{Context\_Spec}} 
            \T{(} \N{Exp} \T{)} & variant 2 \\
      & &   \T{\{} \ALT{\N{Matchscan\_Rule}}{|} \T{\}} \\
      & \OR & \N{Matchscan\_Mode} \OPT{\N{Context\_Spec}} \T{(} \N{Exp} \T{)} 
            \T{of} & variant 3 \\
      & & \quad \ALT{\N{Matchscan\_Rule}}{|} \\
      & & \T{end} \T{matchscan} \T{;}\\ 
\NDEF{Matchscan\_Mode} & \IS & 
           \TDEF{matchscan} \OPT{\T{while}} & case sensitive\\
   & \OR & \TDEF{matchscan*} \OPT{\T{while}}& case insensitive\\
\NDEF{Context\_Spec} & \IS & \T{[} \LIST{\N{Id}} \T{]} \\
\NDEF{Matchscan\_Rule} 
      & \IS & \OPT{\T{<<} \ALT{\N{Context}}{,} \T{>>}} \\
      &     & \quad \T{lexeme} \T{class} \N{Id} \T{:} \N{Matchscan\_Action} \\
      & \OR & \OPT{\T{<<} \ALT{\N{Context}}{,} \T{>>}} \\
      &    &  \quad \N{Regexp} \T{:} \N{Matchscan\_Action} \\
\NDEF{Matchscan\_Action} & \IS & \T{\{} \N{Code} \T{\}} \\
          & \OR & \N{Code} & for variant 1 only \\
\end{syntax}

The two different modes of operation are \T{matchscan} and \T{matchscan*},
which respectively match strings case sensitively and insensitively.
The modifier \T{where} may optionally specify that the matching process
should repeat until no rules apply, or the end of stream condition
is reached.  

By default, if no rules apply and if the input stream is non-empty,
then an error has occurred.  The \T{matchscan} statement will
invoke the method \verb|error()| of the \T{LexerBuffer} object by default.

For example, the following is part of the \Prop{} lexer specification.

\begin{verbatim}
   datatype LexicalContext = NONE | C | PROP | COMMENT | ...;

int PropParser::get_token()
{
   matchscan[LexicalContext] while (lexbuf)
   {  
      ...
   |  <<C>> /[ \t\\\014]/:            { emit(); }
   |  <<C>> /(\/\/.*)?\n/:            { emit(); line++; }
   |  <<C>> /^#.*/:                   { emit(); }
   |  <<PROP>> lexeme class MainKeywords: { return ?lexeme; }
   |  <<PROP>> lexeme class SepKeywords: { return ?lexeme; }
   |  <<PROP>> QUARK_TOK:             { return QUARK_TOK; }
   |  <<PROP>> BIGINT_TOK:            { return BIGINT_TOK; }
   |  <<PROP>> REGEXP_TOK:            { return REGEXP_TOK; }
   |  <<PROP>> PUNCTUATIONS:          { return lexbuf[0]; }
   |  <<PROP>> /[ \t\014]/:           { /* skip */ }
   |  <<PROP>> /(\/\/.*)?\n/:         { line++; }
   |  /\/\*/:                         { emit(); set_context(COMMENT); }
   |  <<COMMENT>> /\*\//:             { emit(); set_context(PROP); }
   |  <<COMMENT>> /\n/:               { emit(); line++; }
   |  <<COMMENT>> /./:                { emit(); }
   |  /./: { error("%Lillegal character %c\n", lexbuf[0]); }
   }
}
\end{verbatim}

Here, the lexer is partitioned in multiple lexical contexts: context
\verb|C| deals with \Cpp{} code while context \verb|PROP| deals with
\Prop{} extensions.  The special context \verb|COMMENT| is used
to parse \verb|/* */| delimited comments.  Contexts are changed
using the \verb|set_context| method defined in class \verb|LexerBuffer|.

The special variable \verb|?lexeme| can be used within a rule 
that matches a lexeme class.  For example, within the rule
\begin{verbatim}
   |  <<PROP>> lexeme class MainKeywords:    { return ?lexeme; }
\end{verbatim}
the variable \verb|?lexeme| is bound to 
the token \verb|."rewrite"| if the string ``rewrite'' is matched;
it is bound to the token \verb|."inference"| if the string
``inference'' is matched and so on.

\Subsubsection{Class \T{LexerBuffer}}
We'll next describe the class \T{LexerBuffer} and its subclasses.

Class \CLASSDEF{LexerBuffer} 
is the base class in the lexical buffer hierarchy.
It is defined in the library include file \verb|<AD/automata/lexerbuf.h>|.
This class is responsible for implementing a string buffer for use
during lexical analysis.   

As it stands, it can be used directly if the lexer input is directly
from a string.  Memory management of the buffer is assumed to be handled
by the user. 

The class \T{LexerBuffer} has three constructors.  The default
constructor initializes the string buffer to NULL.  The two
other constructors initialize the string buffer to a string given
by the user.  In the case when the length is not supplied, the buffer
is assumed to be \verb|'\0'|-terminated.  The two \verb|set_buffer| methods
can be used to set the current string buffer.  Notice that
all lexical analysis operations are done in place.  The user should
not alter the string buffer directly, but should use the interface
provided by this class instead.

\begin{verbatim}
class LexerBuffer {
public:
   LexerBuffer();    
   LexerBuffer(char *);    
   LexerBuffer(char *, size_t);    
   virtual ~LexerBuffer();
   virtual void set_buffer (char *, size_t);
           void set_buffer (char *);
};
\end{verbatim}

The following methods are used access the string buffer.
Method \T{capacity} returns the size of the buffer.  Method
\T{length} returns the length of the current matched token.
Methods \T{text} can be used to obtain a point to location
of the current matched token.  The string returned is guaranteed
to be \verb|'\0'|-terminated.  Methods \T{operator []} return
the $i$th character of the token.  Finally, method \T{lookahead}
returns the character code of the next character to be matched.

\begin{verbatim}
   int capacity () const;
   int length   () const;
   const char * text () const;
         char * text ();
   char  operator [] (int i) const;
   char& operator [] (int i);
   int lookahead () const;
   void push_back (int n) 
\end{verbatim}

In addition to the string buffer, the class \T{LexerBuffer} keeps
track of two additional types of information: the current context
of the DFA, and whether the next token starts at the
beginning of the line, or in our terminology, whether it is
{\em anchored}.  These are manipulated with the following methods:

\begin{verbatim}
   int context    () const;
   void set_context (int c = 0);
   Bool is_anchored() const;
   void set_anchored(Bool a = true);
\end{verbatim}

Finally, the following methods should be redefined by subclasses
to alter the behavior of this class.  By default, the class \T{LexerBuffer}
calls \T{fill\_buffer()} when it reaches the end of the string; subclasses
can use this method to refill the buffer and return the number of
characters read.  Currently, \T{fill\_buffer} is defined to do nothing 
and return 0.  When it reaches the end of the file 
(i.e. when \T{fill\_buffer()} fails to refill the buffer and the 
scanning process finishes), method \T{end\_of\_file} is called.
Currently, this is a no-op.  Finally, the error handling routine
\T{error()} is called with the position of the beginning and the end
of the buffer in which the error occurs.  By default, this routine
prints out the buffer. 

\begin{verbatim}
protected:           
   virtual size_t fill_buffer();
   virtual void   end_of_file();
   virtual void   error(const char * start, const char * stop);
\end{verbatim}

\Subsubsection{Class \T{IOLexerBuffer}}

  Class \CLASSDEF{IOLexerBuffer} 
is a derived class of \T{LexerBuffer}.  It 
automatically manages an internal buffer and receives input from
an \T{istream}.  It is defined in the file \verb|<AD/automata/iolexerbuf.h>|

This class provides the following additional features:
Constructor \verb|IOLexerBuffer(istream&)| ties the input to a
stream.  If the default constructor is used, then it automatically
ties the input to the stream \verb|cin|.  The method \verb|set_stream|
can be used to set the stream to some other input. 

\begin{verbatim}
class IOLexerBuffer : public LexerBuffer {
   size_t    buffer_size;  // size of the buffer
   istream * input;        // input stream
public:
   IOLexerBuffer();    
   IOLexerBuffer(istream&);    
   virtual ~IOLexerBuffer();
   virtual void set_stream (istream&);
};
\end{verbatim}

By default, class \verb|IOLexerBuffer| reads new data from the input
stream using a line buffering discipline.  This mechanism is suitable
for interactive, but for other applications it may be more efficient
to use block buffered input.  The protected method 
\verb|read_buffer| controls this reading behavior; it
is called whenever the data in the string buffer has been
consumed and new input is needed from the stream.  It is
passed the position of the buffer and its remaining capacity,
and it returns the number of new characters that are read.  Subclasses
can redefine this method. 

\begin{verbatim}
protected:
   virtual size_t read_buffer(char *, size_t);
\end{verbatim}

\Subsubsection{Class \T{IOLexerStack}}

Class \CLASSDEF{IOLexerStack}
is a derived class of \T{LexerStack}.  It provides
a mechanism of reading from a stack of \verb|istream|'s.  Streams
can be pushed and popped from the stack.  The next token is 
obtained from the stream from the top of the stack.  The class allows
allows easy implementation of 
constructs such as the \verb|#include| file mechanism of the C preprocessor.

The interface of this class is listed below:
\begin{verbatim}
class IOLexerStack : public IOLexerBuffer {
public:
   IOLexerStack();    
   IOLexerStack(istream&);    
   virtual ~IOLexerStack();
   
   virtual void     push_stream (istream&);
   virtual istream& pop_stream  ();
};
\end{verbatim}

\Subsection{Parser Specification}

Parsers are specified as a two phase process:
\begin{enumerate}
   \item First a {\em syntax class} is defined.  A syntax class
declaration is like a normal \Cpp{} class declaration and has a similar
syntax, except that \Prop{} will also generate the interface of the parser. 
All parsers that \Prop{} generates are encapsulated within a class.  This
makes it easy for the programmer to add additional data for parsing
use, and to have multiple instances of a parser. 
   \item Secondly, the grammar of the language is defined in a
{\em syntax} declaration as a set of productions, 
in a syntax similar to that of {\em yacc}.
\end{enumerate}

We'll describe these two phases.

\Subsubsection{Syntax class}

A syntax class definition specifies an object class that encapsulates
a parser.  Its syntax is the same as the usual \Cpp{} class definition,
except that the prefix \T{syntax} is used: 

\begin{syntax}
\NDEF{Syntax\_Class\_Decl} & \IS & \TDEF{syntax class} \Id 
   \OPT{\T{:} \N{Inherit\_List}} \T{\{} \N{Class\_Body} \T{\}} \T{;} \\
\end{syntax}

This specification automatically generate a class with the following interface
(assuming that the parser class in question is called \T{ParserClass})
\begin{verbatim}
public:  
   ParserClass();  // constructor
public:
   virtual void parse();
\end{verbatim}

The constructor and the method \verb|parse()| will be automatically
generated later. 

A parser class expects the following method to be defined:
\begin{verbatim}
   int get_token();
\end{verbatim}
\noindent The function of \verb|get_token| is to return the
next token from the lexer stream whenever it is called.
It should return \verb|EOF| if the stream is empty.

\Subsubsection{Syntax declaration}

The grammar of a language is specified in the \T{syntax} declaration.
Its syntax is listed as follows:

\begin{syntax}
\NDEF{Syntax\_Decl} & \IS & \TDEF{syntax} \Id \T{\{} \\
    & & \quad \OPT{\MORE{\N{Precedence\_Decl}}{}} \\
    & & \quad \OPT{\N{Expect\_Decl}} \\
    & & \quad \OPT{\MORE{\N{Production\_Rule}}{}} \\
    & & \T{\}} \\
\end{syntax}

The name of a \T{syntax} declaration should match that
of a \T{syntax} \T{class}.  A \T{syntax} declaration is divided
into three parts: (i) the operator precedence section, (ii) the
\T{expect:} $n$ section, and (iii) the grammar section.

\Subsubsection{Precedence}

Operator precedences are specified using precedence declarations.
The syntax is: 

\begin{syntax}
\NDEF{Precedence\_Decl} 
    & \IS & \TDEF{left:} \N{Integer} \ALT{\N{Operator}}{} \T{;} 
          & left associative operators \\
    & \OR & \TDEF{right:} \N{Integer} \ALT{\N{Operator}}{} \T{;} 
    & right associative operators \\
\NDEF{Operator} & \IS & \N{Char}   & single character operator \\
             & \OR & \N{String} & multi-character operator \\
             & \OR & \N{Cons}   & constructor \\
\end{syntax}

The integer associated with a precedence declaration is the precedence
level of an operator.  The higher the level of an operator, the {\em lower} 
its precedence.

For example, the following set of precedences are used in the \Prop{}
language parser:
\begin{verbatim}
left: 23 "as";
left: 22 "::";
left: 21 "||";
left: 20 "equiv:";
left: 19 "xor:";
left: 18 "implies:";
left: 17 "&&" "and";
right: 16 "|=" "&=" "^=" "<<=" ">>=";
right: 15 '=' ":=" "+=" "-=" "*=" "/=" "%=";
left: 14 '|';
left: 13 ':';
left: 12 ';';
left: 11 '^';
left: 10 '&';
left: 9 "==" "!=";
left: 8 '<' '>' ">=" "<=";
left: 7 "<<" ">>";
left: 6 '+' '-' "with" "less";
left: 5 '*' '/' '%';
left: 4 "++" "--";
left: 3 '!' '~' "arb" "card" "dom" "ran";
left: 2 '[' ']' '{' '}' '.' "->" ;
\end{verbatim}

\Subsubsection{{\tt expect:} $n$}

The declaration \T{expect} $n$ specifies that there should be $n$
shift/reduce errors in the grammar.  If the actual grammar deviates
from this number, a warning will be generated.  The alternative
form \T{expect:} \T{\_} suppresses all warnings.  The syntax is:

\begin{syntax}
\NDEF{Expect\_Decl}    & \IS & \TDEF{expect:} \N{Integer} \T{;}
    & expects $n$ shift/reduce errors \\
                     & \OR & \T{expect:} \T{\_} \T{;}
    & expects many shift/reduce errors \\
\end{syntax}

This declaration is optional.  If omitted, warnings will be printed
if parser conflicts are found during parser generation.

\Subsubsection{Production rules}


The syntax of the production rules is as follows: 

\begin{syntax}
\NDEF{Production\_Rule} & \IS & 
  \Id \OPT{\TypeExp} \T{:} \\
  & & \quad \OPT{\ALT{\N{One\_Alt}}{|}} \T{;} \\
\NDEF{One\_Alt}  & \IS & \OPT{\ALT{\N{Symbol}}{}} & one alternative \\
\NDEF{Symbol} 
    & \IS & \Id        & non-terminal \\
    & \OR & \N{Cons}   & terminal (a datatype constructor) \\
    & \OR & \N{String} & terminal (a datatype constructor) \\
    & \OR & \N{Char}   & terminal (from the ASCII character set) \\
    & \OR & \T{?}      & the error terminal \\
    & \OR & \T{\$}     & the end of file terminal \\
    & \OR & \T{\{} \N{Code} \T{\}} & embedded actions \\
\end{syntax}

Each rule specifies a set of productions of the form 
\[ \begin{array}{lcl}
    lhs &  :  & A_{11} A_{12} \ldots A_{1n_1} \\
        & \OR & A_{21} A_{22} \ldots A_{2n_2} \\
        & \vdots & \\
        & \OR & A_{m1} A_{m2} \ldots A_{mn_m} \\
   \end{array}
\]
where $lhs$ is the non-terminal and $A_{ij}$'s are a mixture of 
terminals, non-terminals and embedded actions.  If synthesized
attributes are used, then the type of the s-attribute should be
annotated next to the lhs non-terminal.

Terminals can be specified in a few ways: (i) a character literal
is a predefined terminal of the same name, (ii) a string literal
is matched with a \T{datatype} or a \T{lexeme} \T{class} constructor
of the same name, (iii) an identifier is a terminal if there is a 
\T{datatype} constructor of the same name.  Any other identifiers
are assumed to be non-terminals.

For example, the following are two sets of production rules used in the 
\Prop{} translator itself:
\begin{verbatim}
ty Ty:  simple_ty               { $$ = $1; }
|       ty '=' exp              { $$ = DEFVALty($1,$3); }
|       ty '*'                  { $$ = mkptrty($1); }
|       ty '&'                  { $$ = mkrefty($1); }
|       ty '[' ']'              { $$ = mkptrty($1); }
|       ty '[' exp ']'          { $$ = mkarrayty($1,$3); }
;
exp Exp:
        app_exp                 { $$ = $1; }
|       exp '+' exp             { $$ = BINOPexp("+",$1,$3); }
|       exp '-' exp             { $$ = BINOPexp("-",$1,$3); }
|       exp '*' exp             { $$ = BINOPexp("*",$1,$3); }
|       exp '/' exp             { $$ = BINOPexp("/",$1,$3); }
|       exp '%' exp             { $$ = BINOPexp("%",$1,$3); }
|       exp '^' exp             { $$ = BINOPexp("^",$1,$3); }
|       exp "+=" exp            { $$ = BINOPexp("+=",$1,$3); }
|       exp "-=" exp            { $$ = BINOPexp("-=",$1,$3); }
|       exp "*=" exp            { $$ = BINOPexp("*=",$1,$3); }
|       exp "/=" exp            { $$ = BINOPexp("/=",$1,$3); }
|       exp "%=" exp            { $$ = BINOPexp("%=",$1,$3); }
\end{verbatim}
\noindent {\em etc $\ldots$}

Here, \T{ty} is the non-terminal of this production.  The type of
the synthesized attribute of this rule is \T{Ty}, and it is
written next to the non-terminal.  Like {\em yacc}, the synthesized
attribute of a rule can be referenced using the \verb.$. variables: 
variable \TDEF{\$\$} denote the synthesized attribute of the current rule,
while \TDEF{\$$n$} where $n$ is 1, 2, 3, \ldots denote the synthesized
attribute of the $n$th rhs symbol of the current rule.

\Subsubsection{Parser report}

If the command line option \OPTIONDEF{-r} is used, then
in addition to generating the code for the grammar, the translator
will also generate a comprehensive report of the grammar and
the resulting LALR(1) automaton.  The amount of details in this
report can be varied using the the verbose option
\OPTIONDEF{-v}:

\begin{description}
   \item[{\tt -v1}]  Print full information for all LR(0) states. 
   \item[{\tt -v2}]  Also print out the lookahead set of a reduction rule.
This may be useful for tracking down shift/reduce and reduce/reduce conflicts.
   \item[{\tt -v3}]  Do both \verb|-v1| and \verb|-v2|.
\end{description}

\Subsubsection{Interfacing with the generated lexer}

The easiest way of interfacing with a lexer is to embed
an object of class \T{LexerBuffer}, \T{IOLexerBuffer}, or \T{IOLexerStack}
within a syntax class\footnote{Alternatively, inheritance may be used.}.
We can then define the tokenization method
\verb|get_token| using the \T{matchscan} statement described
in section~\ref{sec:matchscan}.

\Subsection{Debugging Parsers}

Currently, only the most rudimentary debugging facilities have been
implemented for parser.
The user can define the macro \verb|DEBUG_|$C$, where $C$ is the
name of a syntax class before the \T{syntax} definition.  This will activate
the parser debugging code.  During parsing, the reductions that are taken
will be printed to \verb|cerr|.

\Section{Algebraic Datatypes and Pattern Matching} \label{sec:datatype}

    \Prop{} implements algebraic datatypes and pattern matching in the
style of Standard ML\cite{SML}.  Tree, DAG and even graph structures can be
specified as a set of datatype equations in algebraic datatype specifications.
\Prop{} then proceeds to translate these into concrete \Cpp{} classes.  
This makes it very easy to implement complex data structures.

   Algebraic datatypes can be manipulated and transformed
using \Prop{} pattern matching constructs, which decompose a datatype
value into its constituents.  Complex patterns involving multiple
objects can be specified.  These are translated into efficient 
testing and selection code in \Cpp.

In the following we shall give a brief overview of the pattern matching
features of \Prop.  For most users of modern declarative languages many
of these features are already familiar constructs.
 
\Subsection{A quick tour of pattern matching}
 
   Algebraic datatypes are specified using \verb|datatype| definitions,
which define the inductive structure of one of more types using
a tree-grammar like syntax.  
%In addition, pretty printers,
%lexical scanners, parsers, persistence I/O methods and
%garbage collection inferences can also be specified with
%additional options in the same construct.  
When a datatype is declared,
the following operations are implicitly defined by the datatype compiler:
(1) the constructors for all the variants of a type;
(2) the identity test operator \verb|==|, and the assignment
    operator \verb|=| for this type; and
(3) the member functions needed to decompose a datatype value during
    pattern matching.
 
   We'll select the internals of a compiler for a simplified imperative
language as the running example in this section.
Suppose that in this language an expression is composed of
identifiers, integer constants and the four arithmetic operators.
Then the structure of the abstract syntax tree can be specified as follows:
 
\begin{verbatim}
   datatype Exp = INT (int)
                | ID  (const char *)
                | ADD (Exp, Exp)
                | SUB (Exp, Exp)
                | MUL (Exp, Exp)
                | DIV (Exp, Exp)
                ;
\end{verbatim}

Abstract syntax of an expression such as {\em a * b - 17}
can be constructed directly in a prefix syntax, directly mirroring
that of the definition.  The \Prop{} datatype
compiler will automatically generate a \Cpp{} class hierarchy to represent
the variants of type \verb|Exp|.  Datatype constructor
functions(not to be mistaken with \Cpp's class constructors) will
also be automatically generated using the same names as the variants.
 
\begin{verbatim}
   Exp formula = ADD(MUL(ID("a"),ID("b")),INT(17));
\end{verbatim}
 
Datatype values can be decomposed using the \verb|match| statement, which
can be seen as a generalization of \C's \verb|switch| construct.  Pattern
matching is a combination of conditional branching and value
binding.  For example, a typical evaluation function for the type \verb|Exp|
can be written as in the following example.  Notice that each arm of
a \verb|case| is in fact a pattern(with optional variables)
mirroring the syntax of a datatype.  The pattern variables(written
with the prefix \verb|?| in the sequel) of a matching
arm is {\em bound} to the value of the matching value, which can be
subsequently referenced and modified in the action of an arm.
 
\begin{verbatim}
int eval (Exp e, const map<const char *, int>& env)
{  match (e)
   {  case INT ?i:        return ?i;
      case ID  ?id:       return env[?id];
      case ADD (?e1,?e2): return eval(?e1,env) + eval(?e2,env);
      case SUB (?e1,?e2): return eval(?e1,env) - eval(?e2,env);
      case MUL (?e1,?e2): return eval(?e1,env) * eval(?e2,env);
      case DIV (?e1,?e2): return eval(?e1,env) / eval(?e2,env);
   }
}
\end{verbatim}
 
\Subsubsection{Pattern matching versus object-oriented style}
 
  Although a comparable evaluation function can be written
in object oriented style using late binding, as in below, in general pattern
matching is much more powerful than late binding in \Cpp, which only
allows dispatching based on the type of one receiver.
 
\begin{verbatim}
  // Class definitions
  class Exp {
  public:
     virtual int eval(const map<const char *, int>& env) const = 0;
  };
  class INT : Exp {
     int i;
  public:
     int eval(const map<const char *, int>& env);
  };
  class ID : Exp {
     const char * id
  public:
     int eval(const map<const char *, int>& env);
  };
  ...
 
  // Member functions
  int INT::eval(const map<const char *, int>& env) const { return i; }
  int ID ::eval(const map<const char *, int>& env) const { return id; }
  int ADD::eval(const map<const char *, int>& env) const
     { return e1->eval(env) + e2->eval(env); }
  int SUB::eval(const map<const char *, int>& env) const
     { return e1->eval(env) - e2->eval(env); }
  int MUL::eval(const map<const char *, int>& env) const
     { return e1->eval(env) * e2->eval(env); }
  int DIV::eval(const map<const char *, int>& env) const
     { return e1->eval(env) / e2->eval(env); }
\end{verbatim}

For example, in the following function we use nested patterns,
non-linear patterns (i.e. patterns with multiple occurrences of
a pattern variable), and guards to perform algebraic simplification
of an expression.  Although the patterns are relatively simple in
this example, in general arbitrarily complex patterns may be used.
 
\begin{verbatim}
Exp simplify (Exp redex)
{  // recursive traversal code omitted ...
 
   // match while repeats the matching process
   // until no more matches are found.
   match while (redex)
   {  ADD(INT 0,  ?x):     { redex = ?x; }
   |  ADD(INT ?x, INT ?y): { redex = INT(?x+?y); }
   |  ADD(?x,     INT 0)   { redex = ?x; }
   |  SUB(?x,     INT 0):  { redex = ?x; }
   |  SUB(?x,     ?x):     { redex = INT(0); }
   |  SUB(INT ?x, INT ?y): { redex = INT(?x-?y); }
   |  MUL(INT 0,  ?x):     { redex = INT(0); }
   |  MUL(?x,     INT 0):  { redex = INT(0); }
   |  DIV(?x,     ?x):     { redex = INT(1); }
        // don't divide by zero.
   |  DIV(INT ?x, INT ?y) | ?y != 0: { redex = INT(?x/?y); }
   |  ...
   }
   return redex;
}
\end{verbatim}
 
Pattern matching in \Prop{} is not restricted to one datatype at
a time.  In the following example, we use matching on multiple values to
define equality on expressions inductively.  For variety, we'll
use the \verb|fun| variant of \verb|match|, which defines a function
in rule form.  Notice that the last case
of the match set uses {\em wild cards} \verb|_| to catch all the other
non-equal combinations.  Since \Cpp{} does not provide multiple dispatching,
implementing binary (or $n$-ary) matching operations on variant datatypes are
in general cumbersome and verbose in object-oriented style. In contrast,
using an applicative pattern matching style many manipulations and
transformations on variant datatypes with tree-like or graph-like structure
can be expressed succinctly.
 
\begin{verbatim}
fun equal INT ?i,     INT ?j: bool: { return ?i == ?j; }
  | equal ID  ?a,     ID  ?b:       { return strcmp(a,b) == 0; }
  | equal ADD(?a,?b), ADD(?c,?d):   { return equal(?a,?c) && equal(?b,?d); }
  | equal SUB(?a,?b), SUB(?c,?d):   { return equal(?a,?c) && equal(?b,?d); }
  | equal MUL(?a,?b), MUL(?c,?d):   { return equal(?a,?c) && equal(?b,?d); }
  | equal DIV(?a,?b), DIV(?c,?d):   { return equal(?a,?c) && equal(?b,?d); }
  | equal _,          _:            { return false; }
  ;
\end{verbatim}

\Subsubsection{More examples} \label{sec:Wff}
 
  As another example, we can specify the term structure of
{\em well-formed formulas} in proposition calculus as follows.  Notice
that the constructors \verb|F| and \verb|T| are nullary.
 
\begin{verbatim}
   datatype Wff = F
                | T
                | Var     (const char *)
                | And     (Wff, Wff)
                | Or      (Wff, Wff)
                | Not     (Wff)
                | Implies (Wff, Wff)
                ;
\end{verbatim}
 
Datatypes that are parametrically polymorphic, such as lists and trees, can
be defined by parameterizing them with respect to one of more types.
For example, both lists and tree below are parametric on one type argument
\verb|T|.
 
\begin{verbatim}
   datatype List<T> = nil
                    | cons(T, List<T>);
   datatype Tree<T> = empty
                    | leaf(T)
                    | node(Tree<T>, T, Tree<T>);
 
   List<int> primes = cons(2,cons(3,cons(5,cons(7,nil))));
   List<int> more_primes = cons(11,cons(13,primes));
   Tree<char *> names = node(leaf("Church"),"Godel",empty);
\end{verbatim}
 
As a programming convenience, \Prop{} has a set of built-in list-like
syntactic forms.  Unlike languages such as ML, however,
these forms are not predefined to be any specific list types.  Instead, it is
possible for the user to use these forms on any datatypes with
a natural binary {\em cons} operator and a nullary {\em nil} constructor.
For instance, the previous list datatype can be redefined as follows:
 
\begin{verbatim}
   datatype List<T> = #[] | #[ T ... List<T> ];
   List<int> primes = #[ 2, 3, 5, 7 ];
   List<int> more_primes = #[ 11, 13 ... primes ];
   List<char *> names = #[ "Church", "Godel", "Turing", "Curry" ];
 
   template <class T>
      List<T> append (List<T> a, List<T> b)
      {  match (a)
         {  case #[]:          return b;
            case #[hd ... tl]: return #[hd ... append(tl,b)];
         }
      }
\end{verbatim}
The empty list is written as \verb|#[]|, while {\em cons(a,b)}
is written as \verb|#[ a ... b ]|.   An expression of the special form
\verb|#[a, b, c]|, for instance, is simple syntactic sugar for
repeated application of the {\em cons} operator, i.e.
 
\begin{verbatim}
    #[a, b, c] == #[a ... #[ b ... #[ c ... #[] ] ] ].
    #[a, b, c ... d ] == #[a ... #[ b ... #[ c ... d ] ] ].
\end{verbatim}
 
List-like special forms are not limited to datatypes with only two variants.
For example, we can define a datatype similar in structure to S-expressions
in {\em Lisp} or {\em scheme}.  Here's how such as datatype may be
defined\footnote{for simplicity, we'll use a string representation for atoms
instead of a more efficient method.}:
 
\begin{verbatim}
   datatype Sexpr = INT    (int)
                  | REAL   (double)
                  | STRING (char *)
                  | ATOM   (const char *)
                  | #()
                  | #( Sexpr ... Sexpr )
   where type Atom = Sexpr  // synonym for Sexpr
   ;
\end{verbatim}
 
   With this datatype specification in place, we can construct values
of type \verb|Sexpr| in a syntax close to that of {\em Lisp}.  For example,
we can define lambda expressions corresponding to the combinators $I$, $K$
and $S$ as follows:
 
\begin{verbatim}
   Atom LAMBDA = ATOM("LAMBDA");
   Atom f      = ATOM("f");
   Atom x      = ATOM("x");
   Atom y      = ATOM("y");
   Atom NIL    = #();
 
   Sexpr I = #(LAMBDA, #(x), x);
   Sepxr K = #(LAMBDA, #(x), #(LAMBDA, #(y), x));
   Sepxr S = #(LAMBDA, #(f),
                #(LAMBDA, #(x),
                   #(LAMBDA, #(y), #(#(f,x), #(g,x)))));
\end{verbatim}
 
Similar to list-like forms, vector-like forms are also available.
This addresses one of the flaws of the \Cpp{} language, which lacks
first class array constructors.  
Vectors are simply homogeneous arrays whose sizes
are fixed and are determined at creation time\footnote{For efficiency,
bounds checking is {\em not} performed.}.   Random access within
vectors can be done in constant time.  Unlike lists, however, the
prepending operation is not supported.   Vectors literals are delimited
with the composite brackets \verb!(| ... |)!, \verb![| ... |]!, or
\verb!{| ... |}!.  In the following example the datatype \verb|Exp|
uses vectors to represent the cooefficients of the polynomials:
 
\begin{verbatim}
   datatype Vector<T> = (| T |);
   datatype Exp = Polynomial (Var, Vector<int>)
                | Functor (Exp, Vector<Exp>)
                | Atom (Var)
                | ...
   where type Var = const char *;
   Exp formula = Polynomial("X", (| 1, 2, 3 |));
\end{verbatim}
 
Commonly used patterns can be given synonyms so that they can be readily
reused without undue repetition.  This can be accomplished by defining pseudo
datatype constructors to stand for common patterns using
{\em datatype law} definitions.  For example, the following set of
laws define some commonly used special forms for a {\em Lisp}-like language
using the previously defined \verb|Sexpr| datatype.
 
\begin{verbatim}
   datatype law inline Lambda(x,e)  = #(ATOM "LAMBDA", #(x), e)
              | inline Quote(x)     = #(ATOM "QUOTE", x)
              | inline If(a,b,c)    = #(ATOM "IF", a, b, c)
              | inline Nil          = #()
              | inline ProgN(exprs) = #(ATOM "PROGN" ... exprs)
              | SpecialForm  = #(ATOM ("LAMBDA" || "IF" ||
                                       "PROGN" || "QUOTE") ... _)
              | Call(f,args) = ! SpecialForm && #(f ... args)
              ;
\end{verbatim}
 
Note that the pattern \verb|SpecialForm| is meant to match all
special forms in our toy language: i.e. {\em lambdas}, {\em ifs}, {\em progn}'s
and {\em quotes}.  The pattern disjunction connective \verb.||. is used
to link these forms together.  Since we'd like the \verb|Call| pattern
to match only if the S-expression is not a special form, we use
the pattern negation and conjunction operators, \verb|!| and \verb|&&|
are used to screen out special forms.  


With these definitions in place,
an interpreter for our language can be written thus:
 
\begin{verbatim}
   Sexpr eval (Sexpr e)
   {  match (e)
      {  Call(?f,?args):     { /* perform function call */ }
      |  Lambda(?x,?e):      { /* construct closure */ }
      |  If(?e,?then,?else): { /* branching */ }
      |  Quote(?x):          { return ?x; }
      |  ...:                { /* others */ }
      }
   }
\end{verbatim}


As an interesting note, the special form pattern can also be rewritten
using regular expression string matching, as in the following:
 
\begin{verbatim}
 datatype law SpecialForm = #(ATOM /LAMBDA|IF|PROGN|QUOTE/ ... _)
\end{verbatim}
 
In addition, since the law constructors \verb|Lambda|, \verb|Quote|,
\verb|If|, \verb|Nil| and \verb|ProgN| are defined with the \verb|inline|
keyword, these constructors can be used to within expressions as abbreviates
for their rhs definitions.  For example, writing
\verb|Lambda(x,x)| is the same writing \verb|#(ATOM("Lambda"),#(x),x)|.

\Subsubsection{Variants of match}
 
Aside from the usual plain pattern matching constructs, a few variants of the
\verb|match| construct are offered.  We'll briefly enumerate a few of these:
\begin{itemize}
  \item The \verb|matchall| construct is a variant of \verb|match| that
   executes all matching rules(instead of just the first one) in sequence.
  \item Each rule of a \verb|match| statement can have associated cost
    expressions.  Instead of selecting the first matching rule to
    execute as in the default, all matching rules are considered
    and the rule with the least cost is executed.  Ties are broken by
    choosing the rule that comes first lexically.  For example:
 
   \begin{verbatim}
   match (ir)
   {  ADD(LOAD(?r0),?r1) \ e1: { ... }
   |  ADD(?r0,LOAD(?r0)) \ e2: { ... }
   |  ADD(?r0, ?r1)      \ e3: { ... }
   |  ...
   }
   \end{verbatim}
  \item A \verb|match| construct can be modified with the \verb|while|
     modifier, as in the following example.  A match modified thus is
     repeatedly matched until none of the patterns are applicable.
     For example, the following routine uses a \verb|match while| statement
     to traverse to the leftmost leaf.
 
\begin{verbatim}
   template <class T>
   Tree<T> left_most_leaf(Tree<T> tree)
   {  match while (tree)
      {  case node(t,_,_):  tree = t;
      }
      return tree;
   }
\end{verbatim}
  \item Finally, the \verb|matchscan| variant of match can be used to
     perform string matching on a stream.  For example, a simple lexical
     scanner can be written as follows.
 
     \begin{verbatim}
   int lexer (LexerBuffer& in)
   {  matchscan while (in)
      {  /if/:                     { return IF; }
      |  /else/:                   { return ELSE; }
      |  /while/:                  { return WHILE; }
      |  /for/:                    { return FOR; }
      |  /break/:                  { return BREAK; }
      |  /[0-9]+/:                 { return INTEGER; }
      |  /[a-zA-Z_][a-zA-Z_0-9]*/: { return IDENTIFIER; }
      |  /[ \t]+/:                 { /* skip spaces */ }
      |  ... // other rules
      }
   }
     \end{verbatim}
\end{itemize}

\Subsection{Algebraic Datatypes} \label{sec:algebraic-datatypes}
   Algebraic datatypes are defined using the \T{datatype}
construct.  Its syntax is as follows.

\begin{syntax}
\NDEF{Datatype\_Decl} 
   & \IS & \TDEF{datatype} \\
   &     & \quad \OPT{\ALT{\N{Datatype\_Spec}}{and}} \\
   &     & \quad \OPT{\TDEF{where type} \ALT{\N{Type\_Spec}}{and}} \\
   &     & \quad \OPT{\TDEF{law} \ALT{\N{Law\_Spec}}{and}} \\
   &     & \T{;} \\
\end{syntax}

Each \T{datatype} declaration specifies a set of 
(potentially mutually recursive) types, a set of type abbreviations,
and a set of pattern matching abbreviations called {\em laws}.

\begin{syntax}
\NDEF{Datatype\_Spec} 
   & \IS & \Id \OPT{\T{<} \LIST{\N{Id}} \T{>}} \\
   &     & \quad \OPT{\T{:} \N{Inherit\_List}} \\
   &     & \quad \OPT{\T{::} \N{Datatype\_Qualifiers}} \\
   &     & \quad \OPT{\T{=} \N{Cons\_Specs}} \\
\NDEF{Cons\_Specs} 
   & \IS & \ALT{\N{Cons\_Spec}}{|} \\
   &     & \quad \OPT{\T{public:} \N{Datatype\_Body}} \\
\NDEF{Cons\_Spec} 
   & \IS & \N{Simple\_Cons\_Spec}  \\
   &     &  \quad \OPT{\T{with} \T{\{} \N{Class\_Decl} \T{\}}} \\
\NDEF{Simple\_Cons\_Spec} 
   & \IS & \Id \OPT{\N{String}} & unit constructor \\
   & \OR & \N{String}              & unit constructor \\
   & \OR & \Id \OPT{\T{of}} \TypeExp  & constructor with arguments \\
   & \OR & \T{\#[} \T{]}                 & unit list constructor \\
   & \OR & \T{\#\{} \T{\}}               & unit list constructor \\
   & \OR & \T{\#(} \T{)}                 & unit list constructor \\
   & \OR & \T{\#[} \TypeExp \T{...} \TypeExp \T{]}   
      & list constructor \\
   & \OR & \T{\#\{} \TypeExp \T{...} \TypeExp \T{\}} 
      & list constructor \\
   & \OR & \T{\#(} \TypeExp \T{...} \TypeExp \T{)}   
      & list constructor \\
   & \OR & \T{[|} \TypeExp \T{|]} & vector constructor \\
   & \OR & \T{(|} \TypeExp \T{|)} & vector constructor \\
   & \OR & \T{\{|} \TypeExp \T{|\}} & vector constructor \\
\end{syntax} 


Datatypes can be annotated with {\em qualifiers}.  They tell the \Prop{}
translator to generate additional code for each datatype to provide
additional functionality. 

\begin{syntax}
\NDEF{Datatype\_Qualifiers} 
   & \IS & \MORE{\N{Datatype\_Qualifier}}{} \\
\NDEF{Datatype\_Qualifier} 
   & \IS & \TDEF{collectable} & garbage collectable \\
   & \OR & \TDEF{rewrite}     & optimized for tree rewriting \\
   & \OR & \TDEF{persistent}  & generate persistence interface \\
   & \OR & \TDEF{lexeme}      & used for parser/lexer tokens  \\
   & \OR & \TDEF{inline}      & use inlined implementation \\
   & \OR & \TDEF{extern}      & use non-inlined implementation \\
\end{syntax}

Type specifications assign abbreviations to commonly used type expressions. 
They act like \T{typedef}'s in \Cpp, except that \Prop{} can make use
of the type information provided by these specifications.

\begin{syntax}
\NDEF{Type\_Spec} 
   & \IS & \Id \T{=} \TypeExp \\
\end{syntax}

Law specifications are used to define abbreviations to datatype 
patterns\footnote{Patterns are discussed in 
section~\ref{sec:pattern-matching}.}.
They behave like macros in some sense, except that unlike macros, they are
properly type checked by \Prop.  In addition, if the keyword \T{inline} is 
used, then \Prop{} will treat the lhs as an expression and generate 
a function of the same name that can be used to construct the datatype term.

\begin{syntax}
\NDEF{Law\_Spec} 
   & \IS & \OPT{\T{inline}} \Id \OPT{\N{Law\_Arg}} \T{=} \Pat \\
\NDEF{Law\_Arg} 
   & \IS & \Id \\
   & \OR & \T{(} \LIST{\N{Id}} \T{)} \\
\end{syntax}

\Prop{} recognizes the following set of type expressions.  
\begin{syntax}
\NDEF{Type\_Exp}
   & \IS & \N{Type\_Id}                & type name \\
   & \OR & \TypeExp \T{*}              & pointer type \\
   & \OR & \TypeExp \T{\&}             & reference type \\
   & \OR & \N{Type\_Qualifier} \N{Type\_Id}      & qualified type \\
   & \OR & \N{Datatype\_Qualifier} \TypeExp       & annotation  \\
   & \OR & \T{(} \TypeExp \T{)}          & grouping \\
   & \OR & \T{(} \TypeExp \T{,} 
      \LIST{\N{TypeExp}} \T{)} & tuple type \\
   & \OR & \T{(} \TypeExp \T{,} 
      \LIST{\N{TypeExp}} \T{)} & tuple class \\
   & \OR & \T{\{} \LIST{\N{Lab\_Type\_Exp}} \T{\}} & record type \\
   & \OR & \TypeExp \T{=} \N{Exp} & default value \\
\NDEF{Lab\_Type\_Exp} & \IS &
   \Id \T{:} \TypeExp \\
\NDEF{Type\_Qualifier} 
   & \IS & \T{class}       & a class type \\
   & \OR & \T{const}       & constant type \\
   & \OR & \T{rewrite}     & rewritable \\
   & \OR & \T{collectable} & garbage collected \\
   & \OR & \T{persistent}  & persistent object  \\
\end{syntax}

In general, four different types of datatype constructors can be defined:
\begin{description}
 \item[nullary constructors]  These are constructors without an argument.
 \Prop{} maps these into small integers and no heap space will be consumed.
 \item[tuple argument constructors] These are constructors with one
 or more unamed arguments.  
 \item[record argument constructors] These are constructors with one
 or more named arguments written within braces.  For example, in the
 following datatype definition a record argument constructor called
 \verb|Add| is defined:
 \begin{verbatim}
  datatype Exp = Add { left : Exp, right : Exp }
	       | ...
 \end{verbatim}
  To construct an \verb|Add| expression, we can use any one of the following
  forms:
  \begin{verbatim}
     Exp e1 = Add(x,y);
     Exp e2 = Add'{ left = x, right = y };
     Exp e3 = Add'{ right = y, left = x };
  \end{verbatim}
  Note that we are allowed to rearrange the labels in any order if
  the record expression form is used.
 \item[empty constructors] These are constructors written with
 an empty argument \verb|()|.  For example, in the following definition
 an empty constructor called \verb|NONE| is defined:
 \begin{verbatim}
    datatype Exp : public AST
	= NONE ()
	| NONE2
        | ...
 \end{verbatim}
 Here, the constructor \verb|NONE| will inherit from the class \verb|AST|,
 while the nullay constructor \verb|NONE2| will not.
\end{description}

\Subsubsection{Instantiating a datatype}

Each declared datatype should be {\em instantiated} in some implementation
file using the \TDEF{instantiate datatype} statement.  The translator
will generate additional code to implement the datatype.  The general
syntax is:
\begin{syntax}
 \NDEF{Instantiate\_Decl} 
    & \IS &
       \TDEF{instantiate datatype} \LIST{\N{Type\_Spec}} \\
    & \OR & 
       \TDEF{instantiate} \T{extern} \T{datatype} \LIST{\N{Type\_Spec}} \\
 \NDEF{Type\_Spec} & \IS & \Id \OPT{ \T{<} \LIST{\N{TypeExp}} \T{>}} \\
\end{syntax}

\begin{Tips}
   When a datatype is defined with the \TDEF{inline} qualifier,
\Prop{} will try to generate inline methods whenever possible.
On the other hand, if the \TDEF{extern} qualifier is used, \Prop{}
will try to generate an non-inline implementation, and only one
copy are generated at where the \T{instantiate datatype} statement occurs.

Since most \Prop{} datatypes are light-weight objects, \T{inline} is taken
as the default.  The user can override this default using
\OPTIONDEF{-save\_space} option of the translator 
(see page~\pageref{option:save-space}).
\end{Tips}

\begin{Tips}
   When the option \OPTIONDEF{-fno-implicit-templates} is used
under \Gpp, the user should \verb|#define| the symbol
\INDEX{{\tt PROP\_EXPLICIT\_TEMPLATE\_INSTANTIATION}} before
an \T{instantiate datatype} statement.   
This will enable the code for explicit template instantiation.
\end{Tips}

\Subsection{Pattern Matching} \label{sec:pattern-matching}
   The basic pattern matching construct is the \T{match} statement.    
A match statement can be seen as a generalization of \Cpp's \T{switch}
statement, in which the expression that follows a \T{case} can be
a general complex pattern.  

   Its basic syntax is as follows:
\begin{syntax}
\NDEF{Match} & \IS & 
   \N{Match\_Mode} \ALT{\T{(} \N{Exp} \T{)}}{and}  \\
        & & \T{\{} \MORE{\T{case} \N{Match\_Rule}}{} \T{\}} & variant 1 \\
          & \OR & 
   \N{Match\_Mode} \ALT{\T{(} \N{Exp} \T{)}}{and} \\
        & & \T{\{} \ALT{\N{Match\_Rule}}{|} \T{\}} & variant 2 \\
          & \OR &
   \N{Match\_Mode} \ALT{\T{(} \N{Exp} \T{)}}{and} \T{of} & variant 3 \\
        & & \quad \ALT{\N{Match\_Rule}}{|} \\
        & & \T{end} \T{match} \T{;} \\
\NDEF{Match\_Mode} & \IS & \TDEF{match} \OPT{\T{while}} \\
                & \OR & \TDEF{matchall} \OPT{\T{while}} \\
\NDEF{Match\_Rule} & \IS & \Pat \OPT{\N{Guard}} \OPT{\N{Cost}}
        \T{:} \N{Match\_Action} \\
\NDEF{Guard}       & \IS & \T{if} \N{Exp} \\
                & \OR & \T{where} \N{Exp} \\
                & \OR & \T{|} \N{Exp} \\
\NDEF{Cost} & \IS & \verb|\| \Exp & cost expression \\
\NDEF{Match\_Action} & \IS & \T{\{} \N{Code} \T{\}} \\
                  & \OR & \N{Code} & for variant 1 only \\
\end{syntax}

The 3 syntactic variants of the match statement are equivalent in meaning.
A \T{match} statement takes a list of match objects and a
set of pattern matching rules.  The first rule that matches the pattern
completely will be executed.  Notice that guards can be placed on each rule,
and a rule is only applicable if the guards evaluate to true. 

\paragraph{Cost minimization}
In addition, if cost expressions are specified, then the matching process
will proceed as follows: 
\begin{enumerate}
   \item Locate all applicable rules.  If no such rule exists then exit.
   \item Compute the cost expressions of all applicable rules.
   \item Choose the action that has the lowest cost to execute.  If
a tie occurs, resolve by the textual order of the rules. 
\end{enumerate}

\paragraph{Match all rules}
The \T{matchall} statement is a variant of \T{match}.  In the matchall
mode, the actions of {\em all} applicable rules are executed in the
textual order.  Caution: since the matching variable binding 
process occurs {\em before} the rules are executed, the rhs actions
should not alter the matched value or the pattern variables may
be bound to the wrong value.

\paragraph{Repetition} 
In addition, both \T{match} and \T{matchall} can be annotated with the
\TDEF{while} modifier.  It this case the semantics of the matching
construct is to repeat until no more rule is applicable.  For example,
the following code fragment returns the last element of a list
\begin{verbatim}
datatype List = #[] | #[ int ... List ];

int last (List l)
{  match while (l)
   { case #[]:         assert(0); // an error
     case #[i]:        return i;
     case #[h ... t]:  l = t;     
   }
}
\end{verbatim}

\paragraph{Pattern syntax}
The basic form of a pattern is a literal, which
that matches only a single integer, real number, string, etc.  Complex
patterns can be constructed inductively
using datatype constructors, tuples, lists, and 
{\em logical pattern connectives}.

The syntax of the patterns is as follows:
\begin{syntax}
\NDEF{Pat} & 
   \IS & \N{Integer} & matches integer literal \\
 & \OR & \N{Real}    & matches real literal \\
 & \OR & \N{Char}    & matches character literal \\
 & \OR & \N{Quark}   & matches quark literal \\
 & \OR & \N{String}  & matches string literal \\
 & \OR & \N{Regexp}  & matches any string that matches {\em regexp}\\
 & \OR & \N{Pat\_Var}  & a pattern variable; matches anything \\
 & \OR & \T{\_}      & wild card; matches anything \\
 & \OR & \N{Cons}     & matches a datatype constructor  \\
 & \OR & \N{Cons} \N{PatArg}& a constructor application pattern \\
 & \OR & \Id \T{as} \Pat & binds a variable to the sub-pattern \\
 & \OR & \Pat \T{:} \TypeExp & typed pattern \\
 & \OR & \Pat \T{||} \Pat  & matches either pattern \\
 & \OR & \Pat \T{\&\&} \Pat  & matches both patterns \\
 & \OR & \T{!} \Pat             & matches anything but the pattern \\
 & \OR & \T{(} \Pat \T{)}       & Grouping \\
 & \OR & \T{\#[} \LIST{\N{Pat}} \T{]} & list pattern; exact length match \\
 & \OR & \T{\#[} \LIST{\N{Pat}} \T{...} \T{]} & list pattern;   
     non-exact length match \\
 & \OR & \T{[|} \LIST{\N{Pat}} \T{|]} & vector pattern; exact length match \\
 & \OR & \T{[|} \LIST{\N{Pat}} \T{...} \T{|]} 
    & vector pattern; matches to the left \\
 & \OR & \T{[|} \T{...} \LIST{\N{Pat}} \T{|]} 
    & vector pattern; matches to the right \\
\NDEF{PatArg} & \IS & \Pat \\
           & \OR & \T{(} \LIST{\N{Pat}} \T{)} \\
           & \OR & \T{\{} \LIST{\N{Lab\_Pat}} \OPT{\T{...}} \T{\}} \\ 
\NDEF{Lab\_Pat} & \IS & \Id \T{=} \Pat \\
\NDEF{Pat\_Var} & \IS & \Id       \\
            & \OR & \T{?}\Id  \\
\end{syntax}

\Subsection{Refining a datatype}

   A datatype definition can be refined in a few ways:

\begin{enumerate}
  \item{\bf inheritance} A datatype can be declared to be inherited from
one or more classes $C_1, C_2, \ldots, C_n$. 
The effect is that all variants of a datatype will be inherited from the 
classes $C_1, C_2, \ldots C_2$.  

  \item{\bf member functions}  Member functions and addition attributes
can be attached to the datatype variants using the keyword \TDEF{with}.
\end{enumerate}

For example, the following datatype \T{Exp} is inherited from some user
defined classes \verb|AST| and \verb|Object|.   Furthermore, a virtual
member function called print is defined.  

\begin{verbatim}
datatype Exp : public AST, virtual public Object
   = INT int        
         with { ostream& print(ostream&); }
   | PLUS  Exp, Exp 
         with { ostream& print(ostream&); }
   | MINUS Exp, Exp 
         with { ostream& print(ostream&); }
   | MULT  Exp, Exp 
         with { ostream& print(ostream&); }
   | DIV   Exp, Exp 
         with { ostream& print(ostream&); }
   | VAR { name : const char *, typ : Type }
         with { ostream& print(ostream&); }
   public:
   {  virtual ostream& print (ostream&) = 0;
      void * operator new (size_t);
   }
;
\end{verbatim}

The allocator and printing methods can be defined as follows:
\begin{verbatim}
void * classof Exp::operator new(size_t) { ... }
ostream& classof INT::print(ostream& s)   { return s << INT; }
ostream& classof PLUS::print(ostream& s)  
   { return s << '(' << this->#1 << '+' << this->#2 <<')'; }
ostream& classof MINUS::print(ostream& s) 
   { return s << '(' << this->#1 << '-' << this->#2 <<')'; }
ostream& classof MULT::print(ostream& s)  
   { return s << '(' << this->#1 << '*' << this->#2 <<')'; }
ostream& classof DIV::print(ostream& s)  
   { return s << '(' << this->#1 << '/' << this->#2 <<')'; }
ostream& classof VAR::print(ostream& s)   { return s << name; }
\end{verbatim}

The special type form \TDEF{classof} {\em con} is used to reference
the type of a constructor.  Arguments of a constructor uses the following
naming convention: (i) if the constructor $C$ takes only one argument,
then the name of the argument is $C$; (ii) if $C$ takes two or more
unnamed arguments, then these are named \verb|#1|, \verb|#2|, etc;
(iii) finally, labeled arguments are given the same name as the label.

For readability reasons, it is often useful to separate a datatype definition
from the member functions and inheritance definitions.   The \TDEF{refine}
declaration can be used in this situation.  For example, the above
example can be restated more succinctly as follows:

\begin{verbatim}
//
// Datatype definition section
//
datatype Exp = INT int        
             | PLUS  Exp, Exp 
             | MINUS Exp, Exp 
             | MULT  Exp, Exp 
             | DIV   Exp, Exp 
             | VAR { name : const char *, typ : Type }
;

//
// Refinement section
//
refine Exp : public AST, virtual public Object
       { virtual ostream& print (ostream&) = 0;
         void * operator new (size_t);
       }
and    INT, PLUS, MINUS, MULT, DIV, VAR
       { ostream& print(ostream&); 
       }
;
\end{verbatim}

The general syntax of the \TDEF{refine} declaration is as follows:
\begin{syntax}
\NDEF{Refine\_Decl} 
   & \IS & \T{refine} \ALT{\N{Refine\_Spec}}{\T{and}} \T{;} \\
\NDEF{Refine\_Spec} 
   & \IS & \LIST{\N{Id}} \\
   &     & \quad \OPT{\T{:} \N{Inherit\_List}} \\
   &     & \quad \OPT{\T{::} \N{Datatype\_Qualifiers}} \\
   &     & \quad \OPT{\T{\{} \N{Datatype\_Body} \T{\}}} \\
\end{syntax}
Here, \N{Id} refers to either a constructor name or a datatype name.

\Subsection{Memory management}

   By default, a datatype constructor calls the \verb|operator new| to
allocated memory.  There are a few ways to change this behavior: (i)
redefine the \verb|operator new| within the datatype using refinement; or
(ii) inherit from a class that redefines this method.

   If a {\em placement} \verb|operator new| is defined in datatype,
the placement constructor syntax can be used to invoke this
operator:
\begin{syntax}
   \NDEF{Placement\_Cons} 
   & \IS & \N{Cons} \T{'(} \LIST{\N{Exp}} \T{)} \T{(} 
   \OPT{\LIST{\N{Exp}}} \T{)} \\
   & \OR & \N{Cons} \T{'(} \LIST{\N{Exp}} \T{)} \T{\{} 
   \OPT{\LIST{{\N{Id} = \N{Exp}}}}
     \T{\}} \\
\end{syntax}

For example, in the following the extra parameter \verb|mem| will
be passed to placement \verb|new| operator.
\begin{verbatim}
  Exp e1 = INT'(mem)(1);
  Exp e2 = VAR'(mem){ name = "foo", typ = t };
\end{verbatim}

\Subsubsection{Garbage collection}

  The support library contains an implementation\footnote{The implementation
has only been tested on Linux, SunOS and Solaris.  Please contact the
author for details
if you'd like to port the collector to your particular platform.} of a 
conservative garbage collector using two algorithms: one using
a mostly copying scheme\cite{Mostly-copying,Gen-mostly-copying} 
and one using mark-sweep.

  A datatype can be defined to be garbage collectable with the
qualifier \TDEF{collectable}.  In the following example, the datatype
\T{Exp} is will be allocated from the a garbage collected heap instead
of the default heap.  Since it references another 
user defined garbage collectable object
\verb|SymbolTable|, the pointer to that symbol table is also marked
as \T{collectable}.
\begin{verbatim}
   // SymbolTable is collectable
   class SymbolTable: public GCObject
   {
   public:
      class Id : public GCObject { ... };
      ...
   };

   datatype Exp :: collectable
      = INT int
      | VAR (collectable SymbolTable::Id, collectable SymbolTable *)
      | ADD Exp, Exp
      | SUB Exp, Exp
      | MUL Exp, Exp
      | DIV Exp, Exp
      ;
\end{verbatim}

The corresponding \TDEF{instantiate datatype} statement for \T{Exp}
will generate the appropriate tracing methods for garbage collection.

Please see appendix~\ref{appendix:gc} for more details concerning
garbage collection.

\Subsubsection{Persistence}

  Datatypes can be interfaced with the persistent object library
supplied with the translator using the \TDEF{persistent} qualifier.
A persistent object defines the protocol for serializing its representation
into an architecture independent format on a persistence stream.  

For example, the following datatype declaration defines a persistent 
type \verb|Exp|.  Note that a pointer to the symbol table
is annotated with the \T{persisent} qualifier to signify that the object
associated with the pointer should be traversed during the serialization
phase. 
\begin{verbatim}
   // SymbolTable is persistent
   class SymbolTable: public PObject
   {
   public:
      class Id : public PObject { ... };
      ...
   };

   datatype Exp :: persistent
      = INT int
      | VAR (persistent SymbolTable::Id, persistent SymbolTable *)
      | ADD Exp, Exp
      | SUB Exp, Exp
      | MUL Exp, Exp
      | DIV Exp, Exp
      ;
\end{verbatim}

A {persistent object id} {\em must} be provided with each distinct
datatype.  Object ids are used to identify the classes
of persistent objects.  Thus they must be unique within an application.
Each persistent type tag is simply a string, and can be defined using
the \TDEF{refine persistent} statement.  

For example, the following statement declares the persistent object
id for type \verb|Exp| above to be \verb|"Simple expressions"|.
\begin{verbatim}
   refine persistent Exp => "Simple expressions";
\end{verbatim}

The corresponding \TDEF{instantiate datatype} statement will generate
the appropriate methods to communicate with a persistent stream.
\begin{verbatim}
  instantiate datatype Exp;
\end{verbatim}

To write an object $e$ to the datafile \verb|"foo"|, we can say:
\begin{verbatim}
   #include <AD/persist/postream.h>
   ofstream out("foo");
   Postream pout(out);
   pout << e;
   out.close();
\end{verbatim}

To read the object back from the file, we can say:
\begin{verbatim} 
   #include <AD/persist/pistream.h>
   EXP e;
   ifstream in("foo");
   Pistream pin(in);
   e = (Exp)read_object(pin);
   in.close();
\end{verbatim}

For more details concerning persistent streams and persist objects, 
please see directory \verb|<AD/persist>| for details.

\Section{Inference} \label{sec:inference}

   Semantic processing
in compilers and other language processors, such as data flow analysis,
can frequently be specified as in a rule-based,
logical deductive style.  In \Prop, deductive inference using forward
chaining\footnote{
The current
implementation of {\sf Prop} translates inference into very naive code.
This feature will be phased out and replaced by
the more general and much faster graph rewriting mechanism.  
}
is provided as a built-in mechanism, orthogonal
to pattern matching and rewriting, for manipulating user-defined
algebraic datatypes.
 
Similar to syntax classes, {\bf inference classes} may be used
for data encapsulation.  An inference class is a combination of a \Cpp{}
class, a database of inference relations, and a collection of inference rules
of the form {\em lhs \verb|->| rhs}.  The lhs of an inference rule is a set
of patterns in conjunctive form.  During the inference process, a rule is
fired when its lhs condition is satisfied.  A fired rule then executes
the corresponding rhs action, which may assert or retract additional relations
from the database.   Using multiple inheritance, it is possible to combine
a rewriting class with an inference class such that the rewriting process
generates new relations to drive the inference process, or vice versa.

\Subsection{An Example} 
Datatype relations are not a distinct kind of data structure but are in fact
simply algebraic datatypes declared to have a \TDEF{relation} 
attribute.  For example, in
the following definition three relation types \verb|Person|, \verb|Parent|
and \verb|Gen| are defined.
 
\begin{verbatim}
   datatype Person :: relation = person (const char *)
       and  Parent :: relation = parent (const char *, const char *)
       and  Gen    :: relation = same_generation (const char *, const char *);
 
   instantiate datatype Person, Parent, Gen;
\end{verbatim}
 
Using these relations we can define an inference class that computes whether
two persons are in the same generation.   Nine axioms are defined (i.e.
those whose lhs are empty) in the following.  The two inference rules
state that (1) a person is in the same generation as him/herself, 
and (2) two persons are in the same generation if 
their parents are in the same generation.
 
\begin{verbatim}
   inference class SameGeneration {};
 
   inference SameGeneration
   {  -> person("p1") and person("p2") and
         person("p3") and person("p4") and
         person("p5");
 
      -> parent("p1", "p2") and
         parent("p1", "p3") and
         parent("p2", "p4") and
         parent("p3", "P5");
 
      person ?p -> same_generation (?p, ?p);
 
      parent (?x, ?y) and parent (?z, ?w) and same_generation (?x, ?z)
      -> same_generation(?y, ?w);
   };
\end{verbatim}
 
In general, datatypes qualified as \verb|relation|s will inherit
from the base class \verb|Fact|, while a rewrite class definition
implicitly defines two member functions used to assert and retract facts
in the internal database.  For example, in the above example, the following
protocol will be automatically generated by the inference compiler.
 
\begin{verbatim}
   class SameGeneration : ...
   {
   public:
       virtual Rete&    infer       ();       // start the inference process
       virtual ReteNet& operator += (Fact *); // assert fact
       virtual ReteNet& operator -= (Fact *); // retract fact
   };
\end{verbatim}
 

 
Using these methods, an application can insert or remove relations
from an inference class.  This will in turn trigger any attached inference
rules of the class.
 
\Subsubsection{Another example}
 
Consider the following example, which is used to compute Pythagorean
triangles.  Only one axiom and two rules are used.  The axiom and the
first rule are used to assert the relations \verb|num(1)| to \verb|num(n)|
into the database, where \verb|n| is limited by the term \verb|limit(n)|.
The second inference rule is responsible for printing out only
the appropriate combinations of numbers.
 
\begin{verbatim}
   datatype Number :: relation = num int | limit int;
 
   inference class Triangle {};
 
   inference Triangle
   {  ->  num 1;
 
          num m
      and limit n | n > m
      ->  num (m+1);
 
          num a
      and num b
      and num c | a < b && b < c && a*a + b*b == c*c
      ->  { cout << a << " * " << a << " + "
                 << b << " * " << b << " = "
                 << c << " * " << c << "\n";
          };
   };
\end{verbatim}
 
Now, to print all the triangle identities lying in range of $1$ to $100$,
we only have to create an instance of the inference class, insert the
limit, and start the inference process, as in below:
 
\begin{verbatim}
   Triangle triangle;
   triangle += limit(100);
   triangle.infer();
\end{verbatim}

\Subsection{Inference Class}
\INCOMPLETE

\Subsection{Inference Rules}
\INCOMPLETE
 
\Section{Tree Rewriting} \label{sec:tree-rewriting} 

\Prop's tree rewriting mechanism let us transform a tree in
algebraic datatype form into another tree according to a set of 
{\em rewriting rules}.  Unlike plain pattern matching described
in section~\ref{sec:pattern-matching}, which only apply to the
root of a tree, rewriting rules are applicable to all parts of
the tree.  This allows the user to concentrate on developing the
set of transformations; the actual traversal of the tree object
is handled implicitly by \Prop.  When used properly, this
mechanism has an advantage over plain pattern matching since
rewriting rules remain valid even when a new variant to a datatype
is introduced.

In \Prop, a rewriting system is developed in a similar manner as a parser.
The first phase requires the user to defines a {\em rewrite class}
to encapsulate the rewriting rules.  

Frequently, the rewriting mechanism is used to collect information about
a tree structure; furthermore, more transformation rules are {\em conditional}:
i.e. we want them to be applicable only if certain conditions are satisfied. 
We can accomplish both of these tasks by defining data members and methods 
in the rewriting class.  These are accessible during the rewriting
process.  Information collected during rewriting can be stored within
the data members.

In the rewriting formalism, equational
rules of the form {\em lhs} \verb|->| {\em rhs} are specified by the user.
During processing, each instance of the lhs in a complex tree is replaced
by an instance of the rhs, until no such replacement is possible.
Equational rules can often be used to specify semantics based simplification
(e.g. constant folding and simplification based on simple algebraic
identities) or transformation(e.g. code selection in a compiler
backend\cite{codegen-with-trees}).
 
Unlike plain pattern matching, however, the structural traversal process
in rewriting is implicitly inferred from the type structure of an
algebraic datatype, as specified in its definition. 
 
There are two main forms of rewriting modes available:
\begin{itemize}
   \item The first is {\bf normalization} mode: a given tree is reduced
using the matching rules until no more redexes are available.  There
are two modes of operations available:
       \begin{itemize}
          \item in {\bf replacement} mode, the redex of a tree will be
physically overwritten.
          \item in {\bf applicative} mode, on the other hand, a new
tree corresponding to the replacement value will be constructed.
       \end{itemize}
     Replacement mode is used as the default since it is usually
the more efficient of the two.
   \item The second form is {\bf reduction} and {\bf transformation}.
In this mode a tree parse of the input term is computed.  If cost functions
are attached to the rules, then they will also be used to determine a
minimal cost reduction sequence.  During this process attached actions of
a rule may be invoked to synthesize new data.
\end{itemize}
 
   Each independent set of rewriting rules in \Prop{} is encapsulated
in its own {\bf rewrite class}.  A rewrite class is basically a normal \Cpp{}
class with a set of rewriting rules attached.   During rewriting, the
data members and the member functions are visible according to the normal
\Cpp{} scoping rules.  This makes it is easy to encapsulate additional data
computed as a side effect during the rewriting process.
 
\Subsection{A rewriting example}
 
   Consider an abbreviated simplifier for the well-formed formula datatype
\verb|Wff| defined in the section~\ref{sec:Wff}.  The rewrite class for this
can be defined as follows.  Since there is no encapsulated data in this
example, only the default constructor for the class needs to be defined.
A rewrite class definition requires the traversal list to be
specified.  This is simply a list of datatypes 
involved in the rewriting traversal process.  
In this instance only \verb|Wff| is needed.
 
\begin{verbatim}
   rewrite class Simplify (Wff)
   {
   public:
      Simplify() {}
   };
\end{verbatim}
 
   The rewrite rules for the simplifier can then be specified succinctly as
follows.  Like the \verb|match| statement, in general the rhs
of a rewrite rule can be any statement.  A special statement
\verb|rewrite(e)| can be used to rewrite the current redex into another form.
If the rhs is of the form \verb|rewrite(e)|, then it can be abbreviated
to \verb|e|, as in below:
\begin{verbatim}
   rewrite Simplify
   {  And(F,  _):      F
   |  And(_,  F):      F
   |  And(T,  ?X):     ?X
   |  And(?X, T):      ?X
   |  Or (T,  _):      T
   |  Or (_,  T):      T
   |  Or (F,  ?X):     ?X
   |  Or (?X, F):      ?X
   |  Not(Not(?X)):    ?X
   |  Not(And(?X,?Y)): Or(Not(?X), Not(?Y))
   |  Not(Or(?X,?Y)):  And(Not(?X), Not(?Y))
   |  Implies(?X,?Y):  Or(Not(?X), ?Y)
   |  And (?X, ?X):     ?X
   |  Or (?X, ?X):      ?X
   // etc ...
   };
\end{verbatim}
 
The rewrite class definition creates a new class of the same name.  This new
class defines an implicit \verb|operator ()| with the protocol below.  This
member function can be invoked to perform the rewriting in a functional
syntax.
 
\begin{verbatim}
   class Simplify : ... {
   {  ...
   public:
      void operator () (Wff);
      // Wff operator () (Wff); // if rewrite class is applicative
   };
 
   Wff wff = ...;
   Simplify simplify; // create a new instance of the rewrite class
   simplify(wff);    // rewrite the term wff
\end{verbatim}

 
\Subsubsection{Conditional rewriting and actions}
  
   Rewriting rules may be guarded with predicates to limit their
applicability.  In addition, the {\em rhs} of a rewrite rule is not limited
to only a replacement expression: in general, any arbitrarily complex sequence
of code may be used.  For example, in the following set of rewriting
rules we use guards to prevent undesirable replacements to be made during
expression constant folding:
 
\begin{verbatim}
   rewrite ConstantFolding
   {  ADD (INT a, INT b):  INT(a+b)
   |  SUB (INT a, INT b):  INT(a-b)
   |  MUL (INT a, INT b):
      {  int c = a * b;                      // silent overflow
         if (a == 0 || b == 0 || c / b == a) // no overflow?
         {  rewrite(INT(c)); }
         else
         {  cerr << "Overflow in multiply\n"; }
      }
   |  DIV (INT a, INT b) | b == 0:  { cerr << "Division by zero\n"; }
   |  DIV (INT a, INT b):  INT(a/b)
   |  // etc...
   };
\end{verbatim}

\Subsection{Rewrite class} \label{sec:rewrite-class}

The syntax of a rewrite class declaration is as follows:

\begin{syntax}
\NDEF{Rewrite\_Class\_Decl} & \IS & \TDEF{rewrite class} \Id 
            \T{(} \LIST{\N{TypeExp}} \T{)} \\
   & & \quad \OPT{\T{:} \N{Inherit\_List}} 
        \OPT{\T{::} \MORE{\N{Rewrite\_Mode}}{}} \\
   & & \quad \T{\{} \N{Class\_Body} \T{\}} \T{;} \\
\NDEF{Rewrite\_Mode} & \IS & \TDEF{treeparser} \\
                  & \OR & \TDEF{applicative} \\
                  & \OR & \TDEF{topdown} \\
\end{syntax}

This is simply the usual \Cpp{} class declaration syntax
extended to the following information:
\begin{description}
   \item[a traversal list] enclosed in parenthesis.  The traversal
list of a rewrite class defines the set of types that rewriting
should traverse.  If a datatype object contains an argument of a type
not listed in the traversal list, its value will not be altered.
   \item[rewrite mode] this defines the rewriting mode of the
rewrite class.
\end{description}

\Subsection{Rewriting rules}

Rewriting rules are specified in a rewrite declaration.
The syntax of a rewriting specification is as follows:

\begin{syntax}
\NDEF{Rewrite\_Decl} 
   & \IS & \TDEF{rewrite} \Id & variant 1 \\
   &    & \T{\{} \OPT{\N{Index\_Decl}} 
        \ALT{\T{case} \N{Rewrite\_Rules}}{} \T{\}} \\
   & \OR & \T{rewrite} \Id & variant 2 \\
   &    & \T{\{} \OPT{\N{Index\_Decl}} \ALT{\N{Rewrite\_Rules}}{|} \T{\}} \\
\NDEF{Rewrite\_Rule}
  & \IS & \OPT{\N{Rewrite\_Modifier}} \OPT{\N{Id} \T{->}} \\
  & & \quad \Pat \OPT{\N{Guard}} \OPT{\N{Cost}} \T{:} \N{Rewrite\_Action} \\
\NDEF{Rewrite\_Modifier}
  & \IS & \T{bottomup:} & bottom up mode \\
  & \OR & \T{topdown:} & top down mode \\
  & \OR & \T{before:} & before actions \\
  & \OR & \T{preorder:} & preorder actions \\
  & \OR & \T{postorder:} & postorder actions \\
\NDEF{Rewrite\_Action} & \IS & \T{\{} \N{Code} \T{\}} \\
                  & \OR & \Exp \\
\end{syntax}

The name of a rewrite declaration should match the name of
a rewrite class.  
The two syntactic forms of \T{rewrite} have equivalent semantics.

The special statement \T{rewrite}($e$) may be used
inside the rhs action to rewrite the current redex into $e$.
Note that \T{rewrite}($e$) is a branching statement; statements
after \T{rewrite} are not reachable.

\Subsection{Rewriting modifiers}
In version 2.3.3 onward, rewrite rules can be modified by the 
keywords: \T{bottomup:}, \T{topdown:}, \T{before:}, \T{preorder:}
and \T{postorder:}.   These modifiers 
alter the order in which a set of rewriting rules is applied. 

These modifiers act like delimiters, similar
to the way scoping keywords like \T{public:}, \T{protected:} and 
\T{private:} are used to delimit declarations in C++.
For instance, a set of rules prefixed by the modifier \T{topdown:}
will utilize the topdown strategy for reduction.  If no modifier are
specified, then bottom-up will be the default.

Note that all five modes of rules can be used together in a rewriting system 
under tree rewriting mode (see \ref{sec:rewriting-modes})
and their executions are intermixed together.  

The semantics of these modifiers are as follows:
\begin{description}
\item[\TDEF{bottomup:}] This specifies the modified rewriting rules should
be applied in the default bottom-up mode.  In this mode, the 
innermost/leftmost redex is chosen as the next redex.  So reduction
occurs in a bottomup manner.  This means that if a term $t$ is a redex
and if the term $t$ contains a subterm $t'$ which is also a redex, 
$t'$ will be reduced before $t$. 
\item[\TDEF{topdown:}]  This specifies the modified rewriting rules should be
applied in a topdown mode.  In this mode, the rewriter
will first try to locate the outermost/leftmost redex.  Reduction
will occur in a (mostly) topdown manner.
\item[\TDEF{before:}]  This specifies that the modified rewriting rules
should be tried before the topdown phase.   
In addition, unlike topdown mode, if state-caching if used (see
\ref{sec:state-caching}), the rules are only tried once. 
Otherwise, these act just like the preorder modifier, described below. 
\item[\TDEF{preorder:}]  This specifies that
the modified rewriting rules should be
tried only at the preorder traversal of a term.   
\item[\TDEF{postorder:}]  This specifies the rewriting rules should be
tried only at the postorder traversal of a term. 
\end{description}

\Subsubsection{Rewriting modifier example} \label{sec:rewriting-modifier}

We'll use the following example, extracted from a query optimizer for
a subset of the relational calculus, to demonstrate the use
of the rewriting modifiers.

The query optimizer transforms the following abstract syntax, defined
as a set of \Prop's datatypes.  
\begin{verbatim}
datatype List<T> = #[] | #[ T ... List<T> ];
datatype Literal = INT int
                 | STRING const char *
                 | BOOL Bool

and      Exp     = OP Id, Exps
                 | APP Id, Exps
                 | LIT Literal
                 | ID Id
                 | TUPLE Exps                  // [ E1, E2, ... En ]
                 | FORALL Id, Exp, Exp         // forall x in E1 . E2
                 | EXISTS Id, Exp, Exp         // exists x in E1 . E2
                 | GUARD (Exp, Exp)            //
                 | GENERATOR (Ids, Exps, Exp)  // [x_1,...,x_n] in X_1*...*X_n
                 | LET  (Id, Exp, Exp)

law inline Int  i    = LIT(INT i)
|   inline Boolean b = LIT(BOOL b)
|          True      = Boolean true
|          False     = Boolean false
|   inline And a,b   = OP("and",#[a,b])
|   inline Or  a,b   = OP("or", #[a,b])
|   inline Not a     = OP("not",#[a])
|   inline Eq a,b    = OP("=",  #[a,b])
|   inline Ne a,b    = OP("/=", #[a,b])
|   inline Gt a,b    = OP(">",  #[a,b])
|   inline Ge a,b    = OP(">=", #[a,b])
|   inline Lt a,b    = OP("<",  #[a,b])
|   inline Le a,b    = OP("<=", #[a,b])
|   inline In a,b    = OP("in", #[a,b])
|   inline Union a,b = OP("union", #[a,b])
|   inline Count a   = OP("#", #[a])

where type Id         = const char *
and        Ids        = List<Id>
and        Exps       = List<Exp>
;
\end{verbatim}

Note that existential and 
universal quantifiers are represented by the constructors \T{FORALL}
and \T{EXISTS} respectively.  For example, $\exists x \in A.p$ is
presented as the term $\T{EXISTS}(x,A,p)$.  Here $x$ is a {\em binding}
occurrence for $x$; note that the variable $x$ may occur within 
the predicate $p$.  In addition, a list comprehension
expression such as 
\[ \{ e : [x_1,\ldots,x_n] \in X_1 
\times \ldots \times X_n\ |\ p \} \] 
is represented as the compound term 
\[
  \T{GENERATOR}(\#[x_1,\ldots,x_n],\#[X_1,\ldots,X_n],\T{GUARD}(p,e))
\]

Suppose we'll like to rename all variables in a query $Q$ 
so that no two binding occurrences are given the same name.  
This can be easily accomplished using a combination
of \T{preorder:} and \T{postorder:} rules in the following manner: 
\begin{enumerate}
\item[(i)] Use preorder rules to check for new variable bindings,
for example in $\T{EXISTS}(x,A,p)$.
For each new binding occurrence for $x$, create
a new name for $x$.  
These will be performed before the subterms $A$ and $p$ are traversed.
\item[(ii)] Whenever a variable $x$ is found during the sub-traversal
of $A$ and $p$, rename the variable by looking up its new name.
\item[(iii)] Finally, use postorder rules to remove the current 
variable substitution whenever we're exiting a binding occurrence.
\end{enumerate}

For example, the following set of rewriting rules accomplish this renaming
task for our query language using exactly this method.  
\begin{verbatim}
rewrite RemoveDuplicateNames
{
   // We use before and after rules to remove duplicates from variable names.
   // As each new binding occurrence is found, we enter a new binding
   // into the current environment.  Identifiers found within
   // the binding occurrence are then renamed.

preorder: // insert new bindings
   EXISTS(x,_,_):     { new_binding(x); }
|  FORALL(x,_,_):     { new_binding(x); }
|  GENERATOR(xs,_,_): { new_binding(xs); }
|  LET(x,_,_):        { new_binding(x); }

         // rename variables
|  ID x:             { rename(x); }

postorder: // removes the binding
|  EXISTS(x,A,_):      { old_binding(x); }
|  FORALL(x,A,_):      { old_binding(x); }
|  GENERATOR(xs,As,_): { old_binding(xs); }
|  LET(x,_,_):         { old_binding(x); }
};
\end{verbatim}

Note that we have accomplished a depth first traversal using rewriting
without writing any traversal code!  As a side benefit, 
since the traversal is automatically determined by 
the structure of the datatypes, we do not have to rewrite this
renaming code even if the abstract syntax of the query language is
extended, as long as no additional binding operators are added.

\Subsection{The \T{rewrite} statement}
 
   While the rewrite class construct provides a very general abstraction
for rewriting, in general its full power is unneeded.  It is often
convenient to be able to perform rewriting on a term without having
to make a new name for a class just for the occasion, especially if member
functions and member data are unneeded.  To accommodate these situations,
the \verb|rewrite| statement is provided
to perform a set rewriting transformations on a term without having
to define a temporary rewrite class.   It is simply syntactic sugar
for the more general rewrite class and rewrite
rules specifications.
For example, a simplify routine for type \verb|Exp| defined above can be
specified as follows:
 
\begin{verbatim}
   Exp simplify (Exp e)
   {  // transformations on e before
      rewrite (e) type (Exp)
      {  ADD (INT a, INT b):  INT(a+b)
      |  SUB (INT a, INT b):  INT(a-b)
      |  MUL (INT a, INT b):  INT(a*b)
      |  ...
      }
      // transformations on e after
      return e;
   }
\end{verbatim}
 
The \verb|rewrite| normally performs the replacement in-place.
An applicative version of the same can be written as follows\footnote{The
variable {\tt e} is assigned the new copy.}:
 
\begin{verbatim}
   Exp simplify (Exp e)
   {  rewrite (e) => e type (Exp)
      {  ADD (INT a, INT b):  INT(a+b)
      |  SUB (INT a, INT b):  INT(a+b)
      |  MUL (INT a, INT b):  INT(a*b)
      |  ...
      }
      return e;
   }
\end{verbatim}

The syntax of the rewrite statement is as follows.  The
traversal list of the set of rewrite rule is listed next to
the keyword \T{type}.
 
\begin{syntax}
\NDEF{Rewrite\_Stmt} 
   & \IS & \TDEF{rewrite} \T{(} \Exp \T{)} \OPT{\T{=>} \N{Exp}} \\
   &     & \quad \T{type} \T{(} \LIST{\N{TypeExp}} \T{)} \\
   &     & \T{\{} \OPT{\N{Index\_Decl}} \N{Rewrite\_Rules} \T{\}} \\
   & \OR & \TDEF{rewrite} \T{(} \N{Exp} \T{)} \OPT{\T{=>} \N{Exp}} \\
   &    & \quad \TDEF{type} \T{(} \LIST{\N{TypeExp}} \T{)} \T{of} \\
   &    & \quad \OPT{\N{Index\_Decl}} \N{Rewrite\_Rules} \\
   &    & \T{end} \T{rewrite} \T{;} \\ 
\end{syntax}

\Subsubsection{Rewriting modes} \label{sec:rewriting-modes}
   There are two basic modes of operations in a rewrite class: (i) 
{\em tree rewriting} and (ii) {\em tree parsing}.   
Tree rewriting repeatedly looks for redexes in a tree and transforms it into
another tree.   There are two sub-modes of operations: (a) {\em applicative}
and (b) {\em in-place}.  Applicative mode is specified using the
\T{applicative} mode specifier when declaring a rewrite class. 
In this mode, a new tree is built from a 
bottom-up manner, and the subject tree is left unaltered.  The default
mode is ``in-place.''  In this mode, the subject tree is overwritten
with the transformed expression.  

   On the other hand, in tree parsing mode, the left hand side
of a rewrite rules specification is treated as a tree grammar.
The tree parser will look for a derivation of given tree object from
the start non-terminal.
A guarded rule will only be used if the guard evaluates to true.

   If rules are annotated with cost expressions, then the
tree parser will try to locate a derivation chain with a minimal total
cost.  After a derivation tree is found, the tree can be repeated
traversed.  The rhs actions will be invoked during this traversal process.  

\Subsubsection{State caching optimization} \label{sec:state-caching}

   The pattern matching process proceeds in a bottom up manner: 
for each subtree of the form $l(t_1,t_2,\ldots,t_n)$, $n \ge 0$, 
a state number is computed from the current tree label $l$ and the
state numbers of its subtrees $t_1, t_2, \ldots, t_n$.  The set of matched
rules can be directly computed from the current state number.
This means that in the absence of redex replacements, pattern matching
takes time proportional only $O(|t| + r)$, where $|t|$ is the size 
of the input tree $t$ and $r$ is time it takes to execute the rhs actions.  
That is, the runtime is insensitive to the number of rewrite rules or the 
complexity of the patterns.

   However, if the input is a DAG rather than a tree\footnote{Rewriting
of a generic graph structure is not recommended with \T{rewrite}.},
or if replacements are performed, then the above analysis may not hold.
Replacements during rewriting often require state information of the
replacement term to be recomputed to further the matching process.  
Since computation of state can involve a complete traversal of a term, 
replacement can become expensive if the replacement term is large.  
For instance, consider the following replacement rule, which replaces 
all expressions of the form {\em 2*x} into {\em x+x}:
 
\begin{verbatim}
   rewrite class StrengthReduction
   {
      MUL (INT 2, ?x):  ADD(?x, ?x)
      ...
   }
\end{verbatim}
 
\noindent
Since the subterm \verb|?x| could be arbitrarily large, recomputing the
state encoding for \verb|ADD(?x,?x)| naively takes time in proportion to the
size of \verb|?x|.  However, since the subtree \verb|?x| occurs as part
of the pattern, its state has already been computed, and it would be
wasteful to recompute its state from scratch.  

A similar observation can be made to the case when the input is a DAG,
where many subcomponents are shared between nodes.  In order to speed
up the rewriting process, the state of each shared copy of the nodes 
should be computed only once.

In order to speedup the state computation process in these situations,
{\em \INDEX{state caching}} can be enabled by
specifying a {\em rewriting index}.
This can be accomplished in two ways: (i) use the qualifier
\TDEF{rewrite} when defining a datatype.  This specifies the
{\em primary index}.  Or alternatively, (ii) use index declarations
to specify one or more {\em secondary indices}.

We'll first describe the first method. 
A \TDEF{rewrite} qualifier can be used 
in the definition of a datatype.  For instance:
 
\begin{verbatim}
   datatype Exp :: rewrite
      = INT (int)
      | ID  (const char *)
      | ADD (Exp, Exp)
      | SUB (Exp, Exp)
      | MUL (Exp, Exp)
      | DIV (Exp, Exp)
      ;
\end{verbatim}

The \T{rewrite} qualifier tells the translator to generate an extra
state field in the implementation of \T{Exp}.  This state field is implicitly
updated during the rewriting process.  When the rewriter encounters a
node with a cached state it can quickly short-circuit all unnecessary
state computation.

{\bf A word of caution:} since each rewriting system computes
its own encoding, rewriting systems should not be mixed if they shard the 
same index, i.e. actions of a 
rewriting system should not invoke another rewriting
system on a term that is participating in the current rewriting system,
if both systems use the same index.  This limitation can be overcome
by using secondary indices, which we'll discuss next.

\Subsubsection{Specifying secondary indices}

From release 2.3.0 onward, it is possible to specify secondary indices
within a rewriting system.  This makes it possible to invoke other
rewriting systems within the execution of another, while retaining the
state caching optimizations throughout.

There are two forms of secondary index: {\em internal index} and 
{\em external index}.  Internal indices are stored directly within
a rewrite term and require the user to pre-allocate space for each term.
In contrast, external indices are stored in a separate data structure
such as a hash table. 

The syntax of secondary index declarations is as follows:
\begin{syntax}
\NDEF{Index\_Decl} & \IS & \TDEF{index:} \LIST{\N{Index\_Spec}} \T{;} \\
\NDEF{Index\_Spec} & \IS & 
   \TypeExp \T{=} \T{none} & \C{Disable indexing} \\
   & \OR & \TypeExp \T{=} \N{Id} & \C{Internal index} \\
   & \OR & \TypeExp \T{=} \TDEF{extern} \N{Id} & \C{External index} \\
\end{syntax}


\Subsubsection{Using an internal index}

In order to make use of an index, certain member functions have to defined
by the user.  Given an internal index named {\em foo}, the rewriter
invokes two functions named
\begin{quotation}
\noindent
   {\tt int get\_{\em foo}\_state() const; } \\
   {\tt void set\_{\em foo}\_state(int); }
\end{quotation}
\noindent within the generated code.  The user should provide the 
implementations for these functions within the datatypes.  

For example, reusing the well-formed formulas example (see 
section~\ref{sec:Wff}),
an internal index on the datatype \T{Wff} can be implemented as follows:
\begin{verbatim}
#include <AD/rewrite/burs.h>
class WffIndex {
   int state;
public:
   WffIndex() : state(BURS::undefined_state) {}
   int get_wff_state() const { return state; }
   void set_wff_state(int s) { state = s; }
};

datatype Wff : public WffIndex
   = F
   | T
   | Var     (const char *)
   | And     (Wff, Wff)
   | Or      (Wff, Wff)
   | Not     (Wff)
   | Implies (Wff, Wff)
   ;

...

rewrite Simplify {
   index: Wff = wff;
   ...
};
\end{verbatim}

Here, the class \T{WffIndex} provides the internal index interface
functions \verb|get_wff_state| and \verb|set_wff_state| expected by
the rewriter.  Note that each term must be initialized with the
state \verb|BURS::undefined_state|.  This constant is defined within
the library include file \verb|<AD/rewrite/burs.h>|.

To enable the index, in the rewriter 
we specify that datatype \T{Wff} should make use of the index named \T{wff},
using the \T{index:} declaration.  

\Subsubsection{Using an external index}

External indices are specified in a similar manner.  Given a datatype
{\em Foo} and an external index named {\em bar}, the rewriter invokes
calls to the following functions:
\begin{quotation}
\noindent
   {\tt int get\_{\em bar}\_state({\em Foo}) const; } \\
   {\tt void set\_{\em bar}\_state({\em Foo}, int); }
\end{quotation}
Typically, these are implemented as member functions in the rewrite class.

The rewriter uses the function {\tt get\_{\em bar}\_state} to lookup
previously cached information set by the function {\tt set\_{\em bar}\_state}.
Typically, the implementation of the two functions can be implemented
as hash tables, using the addresses of the terms as hash functions and
pointer equality comparisons.   Note that caching can be {\em lossy}; i.e.
it is perfectly alright for the cache to eliminate cached information to
keep its size under control.  If no cache information is found,
the function {\tt get\_{\em bar}\_state} should return 
\verb|BURS::undefined_state|.

\paragraph{Class \CLASSDEF{RewriteCache}}
   To make it easy for the users to implement their own external indices,
two sample external index implementation are provided.  The first
is the class \verb|RewriteCache| defined in the library file 
\verb|<AD/rewrite/cache.h>|.  

   The class \verb|RewriteCache| implements a simple fixed capacity
   cache with the following member functions.
\begin{verbatim}
   RewriteCache();
   RewriteCache(int capacity);
  ~RewriteCache();
 
   void clear();                   // after rewriting
   void initialize();              // before rewriting
   void initialize(int capacity);  // before rewriting/also set capacity
   
   int capacity() const; 
               
   void set_state(const void * t, int s);
   int get_state(const void * t) const;

   void invalidate(const void * t); 
\end{verbatim}

The user should call the function \verb|initialize| before rewriting
in order to setup the cache and/or clear out previously cached result.
The functions \verb|set_state| and \verb|get_state| can be used to
implement the necessary state caching functions expected by the rewriter.

Returning to our \T{Wff} example, we have the following possible implementation
of the external index:
\begin{verbatim}
   rewrite class Simplify (Wff)
   {  RewriteCache cache;
   public:
      int get_wff_state(Wff t) const { return cache.get_state(t); }
      void set_wff_state(Wff t, int s) { cache.set_state(t,s); }  
   };

   rewrite Simplify
   {  index: Wff = extern wff;
      ...
   };
\end{verbatim}

\begin{Tips}  
If manual memory management is used, 
a term should not be deallocated from memory if a cache
contains a handle to it.  The method \verb|invalidate| can be used to
make sure the entry associated with a term is removed from the cache.
\end{Tips}

\paragraph{Class \CLASSDEF{GCRewriteCache}}

{\em This class has not be completely tested.}

Since a garbage collectable object is not reclaimed by the collector
whenever there exists a reference to it, using the class
\verb|RewriteCache| may create problems since objects are not reclaimed
if it is referenced by the cache.  The template class \verb|GCRewriteCache|
solves this problem by using {\em weak pointers} in its internal 
implementation.  

The template class \verb|GCRewriteCache<T>| expects \verb|T| to be
a class derived from \verb|GCObject|.   
This class implements the following functions:
\begin{verbatim}
   GCRewriteCache();
   GCRewriteCache(int capacity);
  ~GCRewriteCache();
 
   void clear();                   // after rewriting
   void initialize();              // before rewriting
   void initialize(int capacity);  // before rewriting/also set capacity
   
   int capacity() const;
               
   void set_state(T * t, int s);
   int get_state(T * t) const;

   void invalidate(T * t); 
\end{verbatim}

The usage of these functions are identical to that of class 
\verb|RewriteCache|.

\Subsection{Short circuiting rewrite paths with \TDEF{cutrewrite}}

   From version 2.3.0 onward, the \T{cutrewrite}($e$) 
statement may be used wherever a \T{rewrite}($e)$ statement is expected.
The \T{cutrewrite} statement can be used to selectively ignore certain
subterms during the rewriting process.

The semantics of \T{cutrewrite}($e$) is to replace the current redex
with $e$ but do not bother with looking for other redexes in the replacement
term.  This means that the replacement term will not be re-traversed
immediately.  Furthermore, the replaced term will not match any other
left hand side sub-patterns except pattern variables and wildcards.

   This feature is very useful for preventing the rewriter from
looking inside certain terms. 

   To demonstrate, consider the simple following example:
\begin{verbatim}
   datatype Exp = NUM of int 
                | ADD of Exp, Exp
                | SUB of Exp, Exp
                | MUL of Exp, Exp
                | DIV of Exp, Exp
                | FINAL of Exp
                ;
   Exp e = ADD(NUM(1),MUL(NUM(2),NUM(3)));
\end{verbatim}

Suppose we want to find all numbers within 
the expression \T{e} and increment their values by 1.  
The straight forward way of writing:
\begin{verbatim}
   rewrite (e) type (Exp) {
      NUM x: NUM(x+1) 
   }
\end{verbatim}
\noindent will not work, since the replacement term is a new redex,
and the rewriting system will not terminate.

We can reimplement the rewriting system as a two stage process.
First, we mark all terms that we do not want to be changed; whenever we
find a marked term, we execute a \T{cutrewrite} to prevent the term
from being changed.  Then we unmark the terms.
\begin{verbatim}
   rewrite (e) type (Exp) {
      NUM x:   FINAL(NUM(x+1));
   preorder:
      FINAL _: cutrewrite;
   }
   rewrite (e) type (Exp) {
      FINAL x: x
   }
\end{verbatim}

In the first rewrite statement, replacement terms that we 
want to stay fixed are encapsulated
within the auxiliary \T{FINAL} constructor.  
Recall that preorder rules are tried before the subterms of a redex are 
reduced.  Thus we can make sure all that
all terms that are arguments to a \T{FINAL} term is left unaltered
by adding a \T{preorder} rule that executes a \T{cutrewrite}\footnote{
If an expression argument is not given to a \T{cutrewrite} statement, then
the redex is left unchanged.}
Finally, after the first rewrite statement finishes, we use 
the second rewrite statement to remove all \T{FINAL} terms generated
in the first.

The semantics of \T{cutrewrite} needs further explanation.  Consider
the following rewrite system:
\begin{verbatim}
   rewrite (e) type (Exp) {
      NUM x:          cutrewrite(NUM(x+1))
   |  MUL(X, NUM 1):  X
   |  MUL(NUM 1,X):   X
   }
\end{verbatim}
Given the term \verb|MUL(NUM(0),NUM(2))|, the subterms \verb|NUM(0)|
and \verb|NUM(1)|
will be rewritten by the first rule
into \verb|NUM(1)| and \verb|NUM(3)| respectively.
Furthermore, the replacement term \verb|NUM(1)| will {\em not} match
the left subpattern in \verb|MUL(NUM 1,X)|.  Thus the result of the rewrite
will be \verb|MUL(NUM(1),NUM(3))|.

\begin{Tips}
Since the semantics of a rewriting system with \T{cutrewrite} can
be tricky to analyze, its use should be avoided as much as possible. 
\end{Tips}

\Subsection{Conditional failure with \TDEF{failrewrite}}

   From version 2.3.0 onward, the \T{failrewrite}
statement may be used wherever a \T{rewrite} statement is expected.
The \T{failrewrite} statement is an alternate way of writing
conditional rewriting rules, and it is often useful when the condition
to be checked is too complex to be written as a guard.

When a \T{failrewrite} statement is executed within the rhs action of a
rewrite rule, the current pattern is rejected and the rewriter
will proceed to try other matches.   

For instance, suppose the current redex is 
\verb|DIV(INT 0,INT 0)| and the following rules are specified within
a rewriting system:
\begin{verbatim}
   DIV(INT X,INT Y): { if (Y == 0) failrewrite; else rewrite(NUM(X/Y); }
|  DIV(INT 0,Y):     NUM(0)
\end{verbatim} 

The execution of these rules will proceed as follows:  both patterns
match the redex, but the first rule will be tried first.  Since
\verb|Y| is zero, the \T{failrewrite} statement is executed.  The rewriter
will proceed to try the
second rule and the redex will be replaced with \verb|NUM(0)|. 
Note that if we omit the \T{failrewrite} statement
in the first rule, as in:
\begin{verbatim}
   DIV(INT X,INT Y): { if (Y != 0) rewrite(NUM(X/Y); }
|  DIV(INT 0,Y):     NUM(0)
\end{verbatim}
the rewriter will commit on the first rule
and no action will be performed on the redex.

Like \T{rewrite} and \T{cutrwerite}, \T{failrewrite}
acts as a branching statement and all statements
immediatedly following \T{failrewrite} are dead code.

\Subsection{Confluence and termination}

   Currently no automatic checking is available to make sure that
a set of rewriting rules is confluent and will terminate.  This is currently
the user's responsibility.  Some verification features will be developed
in the future.

\Subsection{Debugging Tree Rewriting}

   By default, \Prop{} generates a macro \verb|DEBUG_|$C$ for each
rewrite class named $C$.  The user can redefine this macro in order to
trace the replacements that are performed during the execution of a 
set of rewriting rules.  The default definition of \verb|DEBUG_|$C$
is as follows:
\[
\verb|#define DEBUG_|C({\it R,redex,filename,linenumber,ruletext})\ {\it R}
\]
i.e. it simply returns the value of {\it R}.

The arguments to the macro are as follows: 
\begin{itemize} 
\item {\it R} is the new replacement term.
\item {\it redex} is the current redex, before the replacement has been
performed.
\item {\it filename} is the file name of the current rewriting rule.
\item {\it linenumber}  is the location of the current rewriting rule.
\item {\it ruletext} is a string that describes the current rule.
\end{itemize}

In order to activate tracing, the user can redefine the macro
\verb|DEBUG_FOO| (for rewrite class \verb|FOO|) as follows:
\begin{verbatim}
#define DEBUG_FOO(a,b,c,d,e) print_rule(a,b,c,d,e)

template <class T>
  T print_rule(T replacement, T redex, 
               const char * file_name, 
               int line, const char * rule_text)
  { cerr << file_name << ':' << line << ": " << rule_text 
         << " redex = " << redex << " replacement = " << replacement
         << endl;
    return replacement; 
  }
\end{verbatim}
Note that these definitions must appear
before the \verb|rewrite FOO| declaration.

\Subsection{Optimizing Tree Rewriting} \label{sec:optimize-rewriting}

   An experimental rewriting optimizer based on partial evaluation
techniques has been implemented in release 2.3.4 of \Prop.  The optimizer
can be enabled with the option \OPTIONDEF{-Orewriting}.  

Informally speaking, the rewriting optimizer will try to locate 
``longer'' reduction sequence and add these to your rewriting rules.  
The result is that during execution these bigger reduction steps are
taken whenever possible.

   For example, consider the \verb|Wff| datatype defined
in section~\ref{sec:rewriting-modifier}.  Suppose we have the following
two rewriting rules in our rule set:
\begin{verbatim}
   Not(Not X): X
|  FORALL(X,A,P): Not(EXISTS(X,A,Not(P)))
\end{verbatim}

Given an input term of the form \verb|FORALL(X,A,Not(P))|, the term will
first be reduced to \verb|Not(EXISTS(X,A,Not(Not(P))))| via the second rule,
then to \verb|Not(EXISTS(X,A,P))| via the first rule.  The optimizer
can discover that this two step reduction can be compressed into one
step if the rewriting system is augmented with the following rule:

\begin{verbatim}
   FORALL(X,A,Not P): Not(EXISTS(X,A,P))
\end{verbatim}

Note that this augmented rule is a {\em specialization} of the two rules 
given by the user.  In general, the rewriting optimizer will repeatedly
look for useful specializations and add them to the user rule set\footnote{
To ensure termination of the optimization process, 
augmented rules will not be further specialized.}.  
These specialized rules will perform multiple reductions in one single step.

To make use of the optimizer, the rewriting system must have the following
properties:
\begin{enumerate}
   \item  It must be strongly normalizing, i.e. it will terminate under all
   inputs.  
   \item  It must be confluent, or else confluence is not necessary. 
   Notice that the optimizer can and will alternate the reduction order
   of a rule set.  Thus if a rewriting system is non-confluent, an
   optimized rewriting system may give a different result.
\end{enumerate}

In addition, the following things should hold to maximize the effectiveness
of the optimizer.
\begin{enumerate}
   \item  Indices (i.e. state caching) should be enabled.   Otherwise
   the optimizer will not be able to optimize a rule.
   \item  Specializable rules must be simple, i.e. the right hand side
   must be written as an expression consist of only \Prop{} datatypes.
   If the right hand side is a complex statement or if it involves
   external function calls, the optimizer will fail to analyze it properly.
   \item  Only bottom-up rules may be present.
   \item  Avoid conditional rules whenever possible.  
\end{enumerate}

Finally, it should be warned that an optimized rewriting system  
may generate a lot more code than the un-optimized one.  The user should
view the generated report and check that excessive specializations have not 
been performed.

\Section{User defined datatypes: Views} \label{sec:views}

   For many applications it is necessary to interface \Prop{} to 
data structures predefined in libraries, or generated by other tools.
In order to make it possible to work with these external data structures,
\Prop{}, from version 2.3.3. onward, provides a {\em view} mechanism to
interface with these data structures.  Using views, an externally defined
type can be used within \Prop's pattern matching and rewriting
mechanisms transparently, just as if it were defined by \Prop. 
The \Prop{} translator will call the
appropriate interface functions specified by the user to manipulate
these external datatypes.

\Subsection{A first example} \label{sec:tagged-union}.
To illustrate the use of views, we'll begin with a simple example.

Suppose in our application we use the following tagged C union to
represent our expression datatype.  An expression is represented
simply as a pointer to this structure.  An null expression is
represented simply as the NULL pointer.  

\begin{verbatim}
enum exp_tag { Int = 1, Ident, Add, Sub, Mul, Div };
struct exp {
   enum exp_tag tag;
   union {
      int          number;
      const char * ident;
      struct { struct exp * l, * r } children;
   } u;
};
typedef struct exp * Exp;
/* User defined constructor functions */
Exp NONE = NULL;
extern Exp INT(int);
extern Exp IDENT(const char *);
extern Exp ADD(Exp, Exp);
extern Exp SUB(Exp, Exp);
extern Exp MUL(Exp, Exp);
extern Exp DIV(Exp, Exp);
\end{verbatim}

Suppose this data structure is already used extensively in other 
modules, and we'd like to \Prop{} pattern matching mechanisms while keeping 
the old code intact.  We can accomplish this by declaring a
view to this data structure, using a \TDEF{datatype view}
definition:
\begin{verbatim}
dataype Exp :: view =
   match (this ? this->tag : 0)
     view 0     => NONE
   | view Int   => INT(int = this->u.number)
   | view Ident => ID (const char * = this->u.ident)
   | view Add   => ADD(Exp = this->u.children.l, Exp = this->u.children.r)
   | view Sub   => SUB(Exp = this->u.children.l, Exp = this->u.children.r)
   | view Mul   => MUL(Exp = this->u.children.l, Exp = this->u.children.r)
   | view Div   => DIV(Exp = this->u.children.l, Exp = this->u.children.r)
   ;
\end{verbatim}

In the above, a view named \verb|Exp| is defined.   Let's describe
what the definition means: in general, 
a view definition has three main parts, the {\em view selector},
a set of {\em view transformation rules}, and a set {\em of view accessor
expressions}.  These parts perform the following tasks:

\begin{description}
\item[View selector]  The {\em view selector} is specified in the
the \TDEF{match} part of the definition.   In this example, 
the expression associated with \T{match} is 
\verb|this ? 0 : this->tag|.  This returns a variant discrimination integer
from \verb|0| to \verb|Div|.  
Note that the pseudo variable \TDEF{this} 
refers to an object of type \verb|Exp|.

\item[View transformation rules]
Each of the clauses 
\[ \TDEF{view} {\it exp} \verb|=>| {\it term} \]
specifies
a {\em view transformation rule}.   These rules are used to specify
how to transform a computed variant tag into a pattern.    In general,
expression {\it exp} must be a constant expression usable in a 
\verb|case| statement.

\item[View accessors]
In each of the constructor arguments, an {\em view accessor} expression
can be defined to access the logical components of the constructor.
In the rule:
\begin{verbatim}
   | view Add   => ADD(Exp = this->u.children.l, Exp = this->u.children.r)
\end{verbatim}

The expressions \verb|this->u.children.l| and \verb|this->u.children.r|
tell \Prop{} how to access the left and right child of the an addition
expression node.

Note that since the variant \verb|NONE| has no arguments, it is represented
as a nullary constructor, and no accessors expressions are defined.
\end{description}

A pretty printing function for \verb|struct exp| can be implemented
in \Prop{} using pattern matching as follows:
\begin{verbatim}
ostream& operator << (ostream& s, Exp e)
{  match (e)
   {  NONE:     { s << "none"; }
   |  INT i:    { s << i; }
   |  ID id:    { s << id; }
   |  ADD(a,b): { s << '(' << a << " + " << b << '); }
   |  SUB(a,b): { s << '(' << a << " - " << b << '); }
   |  MUL(a,b): { s << '(' << a << " * " << b << '); }
   |  DIV(a,b): { s << '(' << a << " / " << b << '); }
   }
   return s;
}
\end{verbatim}

Note that this is exactly the same code we would've written if
\verb|Exp| is defined as a datatype rather than a view.  This means
that the user can reimplement the expression data structure
as a \Prop's datatype at a latter time, without altering the \Prop{}
source code.  In effect, the {\em logical} view of a datatype, and its
{\em physical} implementation can separately.
With this {\em representation transparency} capability,
programs written in \Prop's pattern matching formalisms can evolve 
in an incrementally and pain-free manner.

The generated code for the above function will resemble the following
output.  Note that all the view selectors and view accessors are
inlined into the code, and in general, views incur no overhead to
their use.
\begin{verbatim}
ostream& operator << (ostream& s, Exp e)
{  switch (e ? e->tag : 0)
   {  case 0:      { s << "none"; }
      case Int:    { s << e->u.number; }
      case Ident:  { s << e->u.ident; }
      case Add:    { s << '(' << e->u.children.l 
                       << " + " << e->children.r << '); }
      case Sub:    { s << '(' << e->u.children.l 
                       << " - " << e->children.r << '); }
      case Mul:    { s << '(' << e->u.children.l 
                       << " * " << e->children.r << '); }
      case Div:    { s << '(' << e->u.children.l 
                       << " / " << e->children.r << '); }
   }
   return s;
}
\end{verbatim}

\Subsection{Another view example}

Suppose we use the following alternative implementation, a
\verb|Expression| class hierarchy to represent
an expression in our AST.  
Note that \verb|Exp| is an abstract base class inherited
by the classes \verb|Number|, \verb|Operator|, \verb|Identifer|, etc.

\begin{verbatim}
class Expression
{
public:

   // This class uses a ``wide'' interface
   virtual int  number() const         = 0;  // returns a number
   virtual const char * ident() const  = 0;  // returns an identifier

   // returns the ith subchild
   virtual const Expression * child(int) const = 0; 
   // mutable version of above
   virtual Expression *& child(int) = 0;     
};

typedef Expression * Exp;

class NullExp  : public Expression { ... };
class Number   : public Expression { ... };
class Operator : public Expression { ... };
class Identifier : public Expression { ... };
\end{verbatim}

In order to provide a suitable interface to \Prop's view mechanism,
we introduce an additional member function to return a variant tag:

\begin{verbatim}
class Expression
{
public:
   enum Variant { None, Int, Ident, Add, Sub, Mul, Div };
   virtual Variant get_variant() const = 0;
}
\end{verbatim}

In addition, we assume the following constructor functions have been
implemented:

\begin{verbatim}
extern Exp INT(int);
extern Exp IDENT(const char *);
extern Exp ADD(Exp, Exp);
extern Exp SUB(Exp, Exp);
extern Exp MUL(Exp, Exp);
extern Exp DIV(Exp, Exp);
\end{verbatim}

Now, with these definitions in place, we can define a datatype view of the
above \verb|Expression| hierarchy using the following definition:

\begin{verbatim}
datatype Exp :: view
   = match (this->get_variant())
     view Expression::None  => NONE
   | view Expression::Int   => INT (int = this->number()) 
   | view Expression::Ident => ID  (const char * = this->ident())
   | view Expression::Add   => ADD (Exp = this->child(0), 
                                    Exp = this->child(1))
   | view Expression::Sub   => SUB (Exp = this->child(0), 
                                    Exp = this->child(1))
   | view Expression::Mul   => MUL (Exp = this->child(0), 
                                    Exp = this->child(1))
   | view Expression::Div   => DIV (Exp = this->child(0), 
                                    Exp = this->child(1))
   ;
\end{verbatim}

An evaluation function can be written
using \Prop's pattern matching construct:

\begin{verbatim}
int eval(Exp e, const Env& env)
{  match (e) 
   {  NONE:          { return 0; }
   |  INT i:         { return i; }
   |  ID id:         { return env(id); }
   |  ADD (x,y):     { return eval(x,env) + eval(y,env); }
   |  SUB (x,y):     { return eval(x,env) - eval(y,env); }
   |  MUL (x,y):     { return eval(x,env) * eval(y,env); }
   |  DIV (x,INT 0): { cerr << "Division by zero"; return 0; }
   |  DIV (x,y):     { return eval(x,env) / eval(y,env); }
   }
}
\end{verbatim}
Note that this code works equally well with the tagged union representation
defined in section~\ref{sec:tagged-union}. 

We can also use rewriting to transform an expression tree as follows

\begin{verbatim}
void simplify(Exp& e)
{
   rewrite (e) type (Exp)
   {  ADD(INT i, INT j): INT(i+j)
   |  SUB(INT i, INT j): INT(i-j)
   |  MUL(INT i, INT j): INT(i*j)
   |  DIV(INT i, INT j): INT(i/j)
   }
}
\end{verbatim}

\Subsection{Syntax of view definitions}

The general syntax of a view definition resembles that of a
\T{datatype} definition:

\begin{syntax}
\NDEF{Datatype\_Spec} 
   & \IS & \N{Datatype\_View\_Spec} \\
\NDEF{Datatype\_View\_Spec} 
   & \IS & \Id \T{::} \T{view} \T{=} \\ 
   &     & \quad \T{match} \T{(} \N{Exp} \T{)} \\
   &     & \quad \N{View\_Cons\_Specs} \\
\NDEF{View\_Cons\_Specs} 
   & \IS & \ALT{\TDEF{view} \N{Exp} \T{=>} \N{Cons\_Spec}}{|} \\
\end{syntax}

Default arguments in a \N{Cons\_Spec} are interpreted as view accessors.

There are a few general rules to observe when defining a view:
\begin{enumerate}
 \item A \T{datatype view} definition only defines the interface
 to an external data structure and no code is generated.  In particular,
 the defined view name is not an actual \Cpp{} type.
 Thus the user is responsible for defining all appropriate type definitions.
 \item All other qualifiers other than \T{view} are ignored.   Thus 
 automatic generation of pretty printers, garbage collection interface
 routines, etc. are disabled for views.
 \item In general, the view accessors must be usable as lvalues
 if the user wants to modify a view datatype.  This includes the
 situation when in-place rewriting is used.
 \item Finally, the user are responsible for generating the
 constructor functions.  In the two examples given above, the user
 should implement the functions \T{INT}, \T{ID}, etc.  
\end{enumerate}

\Section{Graph Types and Graph Rewriting} \label{sec:graph}

A typical compiler uses many types of graphs to represent internal program 
information.  For instance, control flow graphs, 
data dependence graphs, call graphs, def/use chains are
some of the most common examples.  In order to automate the process
of implementing and manipulating these graph structures, \Prop{} provides
a generic formalism-- {\em graph types}--
for specifying multisorted, attributed, hierarchical directed graphs in 
very high level manner.  The data structure mapping process of translating
high level graph specifications into concrete \Cpp{} classes is completely
automated.  Thus the user can concentrate on specifying the semantic
relationships between the elements of the domain, instead of the
implementation details.

The graph structure mapping process uses real-time set machine
simulation\cite{real-time-SETL} and subtype analysis\cite{subtype-SETL} 
to convert associative accesses involving sets, maps, and multimaps
into hash free, worst-case constant time pointer manipulations.
This optimization is performed transparently by the translator.

Manipulation of graphs structures are done in three ways: 
\begin{description} 
\item[object-oriented] 
 Using a standard interface generated from \Prop{} the user can 
manipulate a graph in the usual procedural/object-oriented manner. 
\item[set formalism]  Using an embedded SETL\cite{SETL}-like sublanguage, 
the user can manipulate the graphs using high level set operations 
such as set comprehension.  
\item[graph rewriting]  At the highest level, analysis and transformation 
can be carried out using the graph rewriting formalism.  
\end{description}

In this section we'll describe each of these topics in detail.

\Subsection{Graph Types} \label{sec:graph-types}

A \INDEX{graph type} declaration is used to specify a
graph structure 
with multiple sorts of labeled nodes and labeled directed edges.  
Attributes can be attached to both nodes and edges.  
The syntax of \T{graphtype} declarations is as follows:
\begin{syntax}
\NDEF{Graph\_Type} 
      & \IS & \TDEF{graphtype} \Id \\
      & & \quad \OPT{\T{:} \N{Inherit\_List}} \\
      & & \quad \OPT{\T{::} \MORE{\N{Graph\_Mode}}{}} \\
      & & \T{declare} \\
      & & \quad \TDEF{node:} \ALT{\N{Node\_Def}}{|} \\
      & & \quad \TDEF{edge:} \ALT{\N{Edge\_Def}}{|} \\
      & & \T{begin} \\
      & & \quad \N{Code} \\
      & & \T{end} \T{graphtype} \T{;} \\
\NDEF{Graph\_Mode} & \IS & \\
\NDEF{Node\_Def} & \IS & \Id \OPT{\OPT{\T{of}} \N{TypeExp}} \\
\NDEF{Edge\_Def} 
    & \IS & \Id \OPT{\T{of}} \TypeExp \T{->} \TypeExp \\
    & \OR & \Id \OPT{\T{of}} \TypeExp \T{<->} \TypeExp \\
    & \OR & \Id \OPT{\T{of}} \TypeExp \T{<=>} \TypeExp \\
    & \OR & \Id \OPT{\T{of}} \TypeExp \T{<=>*} \TypeExp \\
\end{syntax}

\Subsection{The Graph Interface} \label{sec:graph-interface}
\INCOMPLETE

\Subsection{Set Formalisms} \label{sec:set-formalisms}
\INCOMPLETE

\Subsection{Graph Rewriting} \label{sec:graph-rewriting}
\INCOMPLETE

\Section{Running the Translator} \label{sec:usage} 

The \Prop{} translator is a program called \verb|prop|.  The translator
uses the following command line syntax:
\begin{syntax}
  \NDEF{Running\_Prop} & ::= &
   \T{prop} \OPT{{\em prop\_options}} \ALT{{\em file}}{} \\
\end{syntax}

Here, \N{File} is a file with suffix \verb|.p|$*$.  By default, the output
file will have the same name with the \verb|p| extension removed.
For example, a file named ``foo.pC'' will be translated into
the file ``foo.C''  Similarly, a file named ``bar.ph'' will be translated 
into ``bar.h''  

\Subsection{Options}
The available options to the translator are as follows:
\begin{description}
\item[\OPTIONDEF{-G -GNU}]      Use GNU style error messages.
\item[\OPTIONDEF{-I{\em path}}] Use the search path {\em path}.
\Prop{} will search {\em path} to look for \Prop{} files with 
the suffix \verb|.ph| inside a \verb|#include| directive.  Multiple
paths can be specified by using multiple \verb|-I| options.
\item[\OPTIONDEF{-l -no\_line\_directives}]   By default, \Prop{} will try
to generate \verb|#line| directives to correlate the source and
the translated output.  This option suppresses this feature.
\item[\OPTIONDEF{-M -make\_depends}]  Similarly to the \verb|-M| option
in \verb|cc| and \verb|cpp|.  This option generates a suitable dependency
list for Makefiles.
\item[\OPTIONDEF{-memory\_usage}]  Print memory use during translation.
\item[\OPTIONDEF{-n -no\_codegen}]  
Don't output any code; perform semantic checking 
only.
\item[\OPTIONDEF{-N -non\_linear}]  
Use non-linear patterns during pattern matching.
A non-linear pattern is one in which a variable occurs more than
once.   By default, \Prop{} consider this an error.
\item[\OPTIONDEF{-o{\em outfile}}]  Send output to {\em outfile} instead of 
the default.
\item[\OPTIONDEF{-Ofast\_string\_match}]  
Generate (potentially) faster string matching
code by expanding string literal tests into character tests.  By default, 
\Prop{} generates string tests using \verb|strcmp| and binary search.
\item[\OPTIONDEF{-Oadaptive\_matching}]  Use the adaptive pattern matching 
algorithm\footnote{The implementation may not be effective.}.
\item[\OPTIONDEF{-Oinline\_casts}]  
Generate inline type casts instead of function
calls that do the same.  This may be faster for compilation.
\item[\OPTIONDEF{-Orewriting}]  Enable the rewriting optimizer. 
See also section~\ref{sec:optimize-rewriting} for details.
\item[\OPTIONDEF{-Otagged\_pointer}]  
Use a tagged pointer representation instead
of representing variant tags explicitly in a structure field.  For
example, if there are two variants called \verb|Foo| and
\verb|Bar|  in a datatype.  Then a pointer to \verb|Foo| can be tagged
with a low order bit 0 while a pointer to \verb|Bar| can be tagged with 
a low order bit 1.  Accessing the tag thus becomes just a simple bit
test.  This should save space and may be faster\footnote{However, this feature
has not been fully tested.}
\item[\OPTIONDEF{-r -report}]  
Generate a report whenever necessary.  Parser, string
matching, and rewriting constructs produce reports.
\item[\OPTIONDEF{-s -strict}]  Use strict semantic checking mode.  
All warnings are considered to be errors.
\item[\OPTIONDEF{-S -save\_space}]\label{option:save-space} 
Use space saving mode.  
Try not to use inline functions if code space can be saved.  
Datatype constructors will not be inlined
in this mode.  Instead, datatype constructors will be generated at
where the corresponding \TDEF{instantiate datatype} declaration occurs.
\item[\OPTIONDEF{-t -stdout}]   Send translated program to the standard output.
\item[\OPTIONDEF{-use\_global\_pool}]  Use global memory pool for allocation.
\item[\OPTIONDEF{-v{\em num}}]  Use verbose mode in report generation.
This will provide more detailed information.
\end{description}

\Subsection{Error Messages}

The following is a canonical list of all error messages generated by
the \Prop{} translator.  Each message is prefixed by the
file name and line number in which the error occurs.    Most of these
errors messages are self explanatory; more detailed explanations are provided
below.

\begin{Error}
\errormsg{unknown option {\em option}}  The translator does not recognize 
the command line {\em option}.   Type {\em prop} to see a list of options.
\errormsg{Error in command: {\em command}}  The translator fails when trying
to execution {\em command}.  When multiple files are specified in a command
line, {\em prop} invokes itself on each of the file. 
\errormsg{{\em pat} with type {\em type} is not unifiable}  Pattern has not
been declared to be a unifiable type.   
\errormsg{Sorry: pattern not supported in rewriting: {\em pat}}  Pattern
{\em pat} is not supported.  Currently, logical pattern connectives
are not supported in rewriting.  You'll have to rewrite them into
non-logical form.
\errormsg{Unknown complexity operator: {\em op}}  
\errormsg{accessor is undefined for view pattern: {\em pat}} An accessor
function has not been declared for a constructor.  When using datatype views
\errormsg{arity mismatch (expecting $n$) in pattern: {\em pat}}  Pattern
{\em pat} is expected to have arity $n$.   The arity is the number of
expressions that you're matching concurrently.  If you are matching $n$
objects then there must be $n$ patterns for each rule.
\errormsg{arity mismatch in logical pattern: {\em pat}}  Logical patterns
do not match the arity of a pattern.
\errormsg{bad constructor type {\em type}}
\errormsg{bad view constructor pattern: {\em pat}}
\errormsg{can't find include file {\em file}}  Additional search paths
can be specified with the \verb|-I| option
\errormsg{component \#$i$ not found in constructor {\em con}}  
Datatype constructor {\em con} does not have a component named \#$i$.
This can happen when you write a record constructor expression and misspelt
one of the labels.
\errormsg{component $l$ not found in constructor {\em con}}  Datatype
constructor {\em con} does not have a labeled component $l$.
\errormsg{constructor {\em con} already has print formats}
\errormsg{constructor {\em con} is not a class}
\errormsg{constructor {\em con} is not a lexeme}  A constructor is used
as a terminal in the parser but it has not been declared to be a lexeme.
\errormsg{constructor {\em con} doesn't take labeled arguments}  You're trying
to use record constructor syntax on a constructor take does not take
record arguments.
\errormsg{constructor {\em con} is undefined}  
\errormsg{cyclic type definition in type {\em id} = {\em type}}
Type abbreviations cannot be cyclic.  For recursive types, use \verb|datatype|
definitions.
\errormsg{datatype {\em T} is not a lexeme type}  {\em T} is used as a
terminal within syntax class when it has not been declared as a lexeme type.
All non-terminals must be a defined using the \verb|lexeme| qualifier.
\errormsg{determinism {\em det} not recognized}
\end{Error}

\begin{Error}
\errormsg{duplicated definition of pattern constructor '{\em con}'}
  The constructor {\em con} has already been in another datatype.  {\em Prop}
  does not allow overloading of constructor names.
\errormsg{duplicated label '$l$' in expression: {\em exp}}
  A record type has one of its labels duplicated.
\errormsg{duplicated label '$l$' in type {\em type}}
\errormsg{duplicated pattern variable '{\em id}'.  Use option -N}
   By default,
   pattern variables may not be duplicated within a pattern.  Non-linear
   patterns are allowed only when the option \verb|-N| is invoked.
\errormsg{edge $e$ is not defined in graphtype {\em id} }
\errormsg{empty type list in rewrite (...) ... }
\errormsg{expecting $c_1$ ... $c_2$ (from line $l$) but found $c_3$ ... $c_4$ 
   instead}  Quotation symbols are not properly balanced. 
\errormsg{expecting non-terminal {\em nt} to have type $t_1$ but found $t_2$}
   Non-terminal {\em nt} has already been defined to have a
   synthesized attribute type of $t_1$ but $t_2$ is found.   This can happen
   if you have rules with the same lhs non-terminal previously defined.  
\errormsg{expecting type $t_1$ but found $t_2$ in pattern variable '{\em v}'}
{\em Prop} performs type inference on all the lhs patterns to determine
the types of all variables.  Type errors can occur if the patterns are
miswritten.
\errormsg{expecting type $t_1$ but found $t_2$}
\errormsg{flexible vector pattern currently not supported in 
rewriting: {\em pat}}
\errormsg{format \#$i$ used on constructor {\em con}}
\errormsg{format \#$l$ used on non-record constructor {\em con}}
\errormsg{function name mismatch: expecting $f$ ...}
\errormsg{illegal character $c$}
\errormsg{illegal context type: {\em type}}  A context type in
a \verb|matchscan| statement must be previously defined to be datatype.
The unit constructors of the datatype can be used as context values.
\errormsg{illegal context(s) in pattern {\em pat}}
\errormsg{illegal format '\_' on constructor {\em con}}
\errormsg{illegal incomplete record pattern: {\em pat}}
\errormsg{illegal print format '$c$' in constructor {\em con}}
\errormsg{illegal record label '$l$' in expression: {\em exp}}
\errormsg{illegal width in bitfield '{\em id} ($n$)'}
\errormsg{inherited attribute '{\em type}' can only be used in treeparser 
mode in rewrite class {\em id}}  
\end{Error}

\begin{Error}
\errormsg{law {\em id}(...) = {\em pat} is not invertible}   Pattern {\em pat}
cannot be treated as an expression.  It may involve logical patterns and
wild cards.
\errormsg{law '{\em id}': bound variable '$v$' is absent in body {\em exp}}
A parameter variable $v$ must be present within the body of a law.

\errormsg{law '{\em id}': type {\em type} cannot be used in parameter {\em id}}

\errormsg{lexeme {\em id} is undefined}

\errormsg{lexeme class {\em id} is undefined}

\errormsg{lexeme pattern is undefined for constructor {\em con}}

\errormsg{lexeme \{{\em id}\} already defined as {\em regexp}}

\errormsg{lexeme \{{\em id}\} is undefined in {\em regexp}}

\errormsg{missing '\}' in regular expression {\em regexp}}

\errormsg{missing label '$l$' in expression: {\em con} {\em exp}}

\errormsg{missing non-terminal in tree grammar rule: {\em nt}}
\errormsg{missing type {\em type} in the traversal list of rewrite class {\em id}} 
Within the rewriting rules, you have used a pattern that involve a constructor
of type {\em  type} directly but no such types are defined in the 
rewrite class definition.  You should add the type to the traversal list.

\errormsg{missing type info in expression {\em exp} : {\em type}}
   Sometimes {\em prop} fails to infer the type of an expression within
a {\em match} statement.  It such
cases it is necessary to help out the translator by adding additional
type information in the patterns.  For example, rewrite some pattern $p$
as $(p : \tau)$ where $\tau$ is the type of $p$.

\errormsg{missing type info in function {\em f} {\em type}}  Similar to above
\errormsg{missing type info in rule: {\em f} {\em pat} : {\em type}}
   Similar to above.
\errormsg{missing view selector for pattern: {\em pat}}
\errormsg{multiple mixed polarity pattern variable '{\em v}'}
\errormsg{multiple unit constructors in polymorphic type {\em id} {\em arg} 
is not supported}
\errormsg{negative cost {\em cost} is illegal}
\errormsg{node {\em id} is not defined in graphtype {\em id}}
\errormsg{non lexeme type {\em type} in pattern {\em pat}}
\errormsg{non-relation type {\em type} in pattern: {\em pat}}
\errormsg{pattern is undefined for lexeme {\em l}}
\errormsg{pattern scope stack overflows}
\errormsg{pattern scope stack underflows}
\end{Error}

\begin{Error}
\errormsg{pattern variable '$v$' has no binding at this point}
\errormsg{persist object id is undefined for {\em con}}
\errormsg{persistence pid {\em name} already allocated for type {\em type}}
\errormsg{persistence redefined for type {\em type}}
\errormsg{precedence symbol must be terminal: {\em term}}
\errormsg{predicate {\em p}: expecting type $t_1$ but found $t_2$}
\errormsg{redefinition of bitfield '{\em field}'.}
\errormsg{redefinition of constructor '{\em con}'}
\errormsg{redefinition of datatype {\em id}}
\errormsg{redefinition of lexeme class {\em id}}
\errormsg{replacement not in rewrite class: rewrite {\em exp}}
\errormsg{rewrite class {\em id} has already been defined}
\errormsg{rewrite class {\em id} is undefined}
\errormsg{rule $r$ has incomplete type {\em type}}
\errormsg{rule $r$ is of a non datatype: {\em type}}
\errormsg{syntax class {\em id} has already been defined}
\errormsg{syntax class {\em id} is undefined}
\errormsg{synthesized attribute '{\em type}' 
can only be used in treeparser mode in rewrite class {\em id}}
\errormsg{the class representation of constructor {\em id} has been elided}
\errormsg{this rule is never selected: $r$}
\errormsg{this rule is never used: $r$}
\errormsg{too few arguments {\em arg} in instantiation of type scheme {\em type}}
\errormsg{too many arguments {\em arg} in instantiation of type scheme {\em type}}
\errormsg{type {\em type} is not a datatype}
\errormsg{type {\em type} is undefined}
\end{Error}

\begin{Error}
\errormsg{type {\em id} = {\em type} is not a datatype}
\errormsg{type {\em id} has already been defined as {\em type}}
\errormsg{type {\em id} has unknown class: {\em C}}
\errormsg{type '{\em type}' is not rewritable in treeparser mode rewrite class 
{\em id}}
    All datatypes used within the treeparser mode of rewriting must be defined
    with the \verb|rewrite| qualifier.
\errormsg{type error in pattern {\em pat}: {\em type}}
\errormsg{type mismatch between nonterminal {\em nt}(type $t_1$) 
and rule $r$(type $t_2$)}
\errormsg{type mismatch in pattern: {\em pat}}
\errormsg{type mismatch in rule $r$}
\errormsg{type mismatch in rule $r$ {\em pat}}
\errormsg{type or constructor {\em con} is undefined}
\errormsg{unable to apply pattern scheme {\em pat}}
\errormsg{unbalanced $c_1$ ... $c_2$ at end of file}
\errormsg{undefined non-terminal {\em nt}}
\errormsg{unification fails occurs check with $t_1$ and $t_2$}
\errormsg{unmatched ending quote $c$}
\errormsg{unrecognized quoted expression `{\em exp}`}
\errormsg{unrecognized quoted pattern `{\em pat}`}
\end{Error}

\pagebreak
\appendix

\Section{Garbage Collection in the {\sf Prop} Library} \label{appendix:gc}
   In this appendix
we describe the design and implementation of a garbage collector
based on the Customisable Memory Management framework(CMM)\cite{CMM} in our
\Prop{} \Cpp{} class library.  Like the previous approach, we have 
implemented a mostly copying conservative collector based on the work
of Bartlett\cite{Mostly-copying}.  
Similar to CMM, our architecture provides a protocol to allow 
multiple garbage collectors using different algorithms to coexist in the 
same memory space.  A few improvements are made to
improve the performance, the flexibility and the functionality of 
our collector: to reduce retention due to false roots identification,
blacklisting\cite{Boehm} is used to identify troublesome heap
addresses; the architecture of the system has been generalized so that it is 
now possible to have multiple instantiations of Bartlett-style heaps; 
finally, object finalization and weak pointer support
are added.  Our collector has been tested on gcc 2.5.8 under Linux,
and SunOS 4.1.x\footnote{An alpha version also works under Solaris. Thanks
to Paul Dietz for the patch.}

\Subsection{Introduction}
   The \Prop{} project involves the design and implementation of 
an environment and an extension language based on \Cpp{} 
for compiler construction and program transformation.  
An outgrowth of this project is the \Prop{} 
\Cpp{} class library, which contains an extensive set of support algorithms 
and data structures.  Since a typical compiler manipulates many complex tree,
dag and graph objects with extended lifetime, manual memory management using
\Cpp's constructors and destructors, reference counting smart pointers, or
some other techniques is frequently complex and error prone.  
Furthermore, with the advent of new algorithms for garbage collection,
manual memory management techniques do not necessarily provide better 
performance or space utilization.  To make it possible to make use of garbage 
collection as a viable alternative for memory management in
\Cpp\cite{Safe-C++-GC}, we have implemented a garbage collection class 
hierarchy as part of the \Prop{} library.  The class library can be used
directly by the users who'd like to program in \Cpp; it can also be
used as part of the runtime system of a highly level language implemented
in \Cpp, as we have done for \Prop.

   We have several good reasons to prefer garbage collection over manual 
memory management.  The \Prop{} language contains many declarative constructs 
and extensions such as algebraic datatypes, pattern matching, rewriting, and
logical inference.  When a user programs in \Prop{} using these constructs, 
an applicative style of programming is the most natural paradigm.  

\Subsection{The Framework}

   We base our design on the work on the Customisable Memory Management(CMM)
system\cite{CMM}.  In this framework, multiple independent heaps(including
the usually non-collectable heap) can coexist with each other.
Bartlett's mostly copying garbage collector is used as the primary collector.
CMM extends the work of Bartlett\cite{Mostly-copying} 
by allowing cross heap pointers and unrestricted interior pointers.

  However, all collectable objects are required to derive from a base class 
and reimplement a tracing method, which identifies the internal pointers of 
an object.  This burden is usually quite small and in fact can be
automated from the type definition\footnote{In \Prop, a special keyword
\T{collectable} is used to identify garbage collected classes and types.
The \Prop{} translator uses this type annotation to derive the appropriate  
tracing methods.}

  One of the major advantages of the CMM framework is that multiple
coorperating collectors can coexist at the same time, which makes it
possible to customize the behavior of each collector with respect to
the lifetime behavior of the objects.  In \cite{CMM}, examples are
given in which the lifetime of certain class of objects exhibit
first-in/first-out behavior.  In this case a special collector can be
written to take full advantage of this property.

\subsection{Our Framework}
   Our framework retains all the benefits and flexibilities of CMM, 
while extending it in several minor but useful ways:  
\begin{itemize}
   \item Like CMM, interior pointers(i.e. pointers to the middle of an object)
and crossheap pointers (i.e. complex data structures linking nodes locating
in multiple logical heaps.) are supported.  Pointers to collectable
objects are non-intrusive: i.e. they are normal \Cpp{} pointers 
and not encapsulated in templates.
   \item Also like CMM, we allow multiple garbage collectors using different
algorithms to coexist.  Unlike CMM, however, we allow multiple 
Bartlett-style collectors to be instantiated.  Most of the services 
involving low level page management and object marking have been relegated
to a separate heap manager.   
   \item We provide support for finalization and weakpointers.
   \item We have implemented blacklisting\cite{Boehm} to reduce the chance
of false roots identification.
\end{itemize}

\subsection{The Implementation}

   Our implementation has been completely written from scratch since
we do not desire to utilize copyrighted code and, more importantly,
we have to make a few important architectural changes:
 All low level memory management services, such as management of
the page table and the object bitmap, is now relegated to the class 
\CLASSDEF{GCHeapManager}.  Collectors now act as clients as of the heap manager
and the Bartlett-style collector no longer has any special status. 

\subsection{Architecture}

   The architecture of the memory management system is partitioned into
a few classes, each responsible for providing distinct services:

\begin{itemize}
   \item \CLASSDEF{GCHeapManager} --- The heap manager.  The heap manager
    manages the heap table, performs page level
    allocation and deallocation, and provides miscellaneous services
    like blacklisting.  It also manages the object bitmaps.
   \item \CLASSDEF{GC} --- The base class for all garbage collectors.
    This base class describes the protocol used by all the collector
    classes\footnote{This base class is also inherited from class
    \CLASSDEF{Mem}, so that it adheres to the \Prop{} memory management
    protocol.}
   \item \CLASSDEF{CGC} --- The base class for conservative collectors.
     This class is inherited from class \CLASSDEF{GC} and implements some
     methods for locating the stack, heap, and static data areas.
   \item \CLASSDEF{BGC} --- Bartlett-style mostly copying collector.
      This class is inherited from class \CLASSDEF{CGC} and implements
      the Bartlett mostly copying algorithm.
   \item \CLASSDEF{MarkSweepGC} --- Mark/sweep style conservative collector.
      This class is inherited from class \CLASSDEF{CGC} and implements
      a mark/sweep collection algorithm.
   \item \CLASSDEF{WeakPointerManager} --- The weakpointer manager.
      This class manages the weak pointer table and provides a few
      weak pointer scavenging services for the collectors.
\end{itemize}
  
\Subsection{The Programmatic Interface}

   The programmatic interface to the garbage collector involves two
base classes, \CLASSDEF{GC} and \CLASSDEF{GCObject}.  The base class 
\CLASSDEF{GC} provides an uniform interface to all types of collectors 
and memory
managers while class {\sf GCObject} provides the interface to all 
collectable classes.  Table~\ref{GC-Classes} contains a listing
of the classes in our hierarchy.

\begin{table}
   \begin{center}
      \begin{tabular}{|l|l|} \hline
        Class           & Purpose \\ \hline \hline
        \sf GCObject    & Collectable object base class \\
        \sf GC          & Garbage collector base class \\
        \sf CGC         & Conservative garbage collector base class \\
        \sf BGC         & Bartlett style mostly copying collector \\
        \sf MarkSweepGC & A mark-sweep collector \\
        \sf UserHeap    & A heap for pointerless object \\
        \sf GCVerifier  & A heap walker that verifies the integrity of a 
                          structure \\
      \hline 
      \end{tabular}
   \end{center}
   \caption{\label{GC-Classes} Garbage Collection Classes.}
\end{table}

Class \CLASSDEF{GCObject} is extremely simple: it redefines the operators
\verb|new| and \verb|delete| to allocate memory from the default collector,
and declares a virtual method ``\verb|trace|'' to be defined by subclasses
(more on this later.)

Memory for a \CLASSDEF{GCObject} is allocated and freed using the usually
\verb|new| and \verb|delete| operators.  Of course, freeing memory explicitly
using \verb|delete| is optional for many subclasses of \CLASSDEF{GC}. 

\begin{verbatim}
   class GCObject {
   public:
      inline void * operator new(size_t n, GC& gc = *GC::default_gc) 
         { return gc.m_alloc(n); }
      inline void * operator new(size_t n, size_t N, GC& gc = *GC::default_gc) 
         { return gc.m_alloc(n > N ? n : N); }
      inline void   operator delete(void *);
      virtual void trace(GC *) = 0;
   };
\end{verbatim}

\subsection{Memory Allocation}

The base class \CLASSDEF{GC} is slightly more complex, as it has to provide
a few different functionalities.  The first service that class \CLASSDEF{GC} 
must 
provide is of course memory allocation and deallocation.  As a time saving
device we can specify what the default collector is using the methods
\verb|get_default_gc| and \verb|set_default_gc|.  Method \verb|GCObject::new| 
will use this collector by default, unless placement syntax is used.
Method \verb|GCObject::delete|, on the other hand, can correctly infer the 
proper collector to use by the address of the object.

The low level methods to allocate and deallocate memory are \verb|m_alloc|
and \verb|free| respectively.  The programmers usually do not have to
use these methods directly. 

The method to invoke a garbage collection of a specific level is 
\verb|collect(int level)|.  This forces an explicit collection.
Method \verb|grow_heap(size_t)| can also be used to explicitly increase
the heap size of a collector.  Depending of the actual behavior
of the subclasses, these methods may have different effects.

\begin{verbatim}
   class GC {
   protected:
      static GC * default_gc;
   public:
      static GC&  get_default_gc()       { return *default_gc; }
      static void set_default_gc(GC& gc) { default_gc = &gc; }
      virtual void * m_alloc   (size_t) = 0;
      virtual void   free      (void *);
      virtual void   collect   (int level = 0) = 0;
      virtual void   grow_heap (size_t) = 0;
      static  void garbage_collect() { default_gc->collect(); }
      virtual void set_gc_ratio(int);
      virtual void set_initial_heap_size (size_t);
      virtual void set_min_heap_growth   (size_t);
   };
\end{verbatim}

\subsection{The GC Protocol}
The collector and collectable objects must cooperate with
each other by abiding to a simple protocol:
\begin{itemize}
   \item All objects that are to be garbage collected must be derived
from \CLASSDEF{GCObject}.  The application programmer must also supply a
``{\bf tracing}'' method.  The purpose of this method is to identify
all internal pointers to other \CLASSDEF{GCObject}'s.   This method is not
used by the application programmer directly but only used internally
by the garbage collectors.
   \item The tracing method of each collectable must in turn call
    the tracing method in the class \CLASSDEF{GC} with each pointer to
    a collectable object that the object owns:
\begin{verbatim}
   class GC {
   public:
      virtual GCObject * trace (GCObject *) = 0;
      inline  void trace (GCObject& obj);
   };
\end{verbatim}
    
     Briefly, the rules are as follows:
   \begin{enumerate}
      \item The tracing method of a collectable \T{Foo} has the following
         general form:
          \begin{verbatim}
             void Foo::trace(GC * gc) 
             {
                ...
             }
          \end{verbatim} 
      \item If class \T{Foo} has a member that is a pointer \verb|p|
            to a collectable object of type \T{Bar}, then add:
           \begin{verbatim}
               p = (Bar)gc->trace(p);
           \end{verbatim}
            to the body of \verb|Foo::trace|. 
     \item  If class \T{Foo} has a member object
            \verb|x| that is a subclass of \T{GCObject}, also add:
           \begin{verbatim}
               gc->trace(x);
           \end{verbatim}
            to the body of \verb|Foo::trace|.
     \item  If class \T{Foo} is derived from a class \verb|Bar| that
            is a subclass of \T{GCObject}, add:
           \begin{verbatim}
               Bar::trace(gc);
           \end{verbatim}
            to the body of \verb|Foo::trace|.
   \end{enumerate} 
    Notice that these methods can be arranged in any order.
\end{itemize}

This protocol can be used by both copying and non-copying collectors.
In addition, the class \CLASSDEF{GCVerifier} also uses this protocol to 
walk the heap in order to verify the integrity of a garbage collected 
data structure.

\subsection{Messages and Statistics}
    All garbage collectors use the following protocols for status
reporting and statistics gathering.  
  
\begin{verbatim}
   class GC {
   public:
      enum GCNotify {
         gc_no_notify,             
         gc_notify_minor_collection,
         gc_notify_major_collection,
         gc_notify_weak_pointer_collection,
         gc_print_collection_time,
         gc_print_debugging_info
      }
      virtual int      verbosity() const;
      virtual void     set_verbosity(int);
      virtual ostream& get_console() const;
      virtual void     set_console(ostream&);
   };
\end{verbatim}

\begin{verbatim}
   class GC {
   public:
      struct Statistics {
         const char *   algorithm;
         const char *   version;
         size_t         bytes_used;
         size_t         bytes_managed;
         size_t         bytes_free;
         struct timeval gc_user_time;
         struct timeval gc_system_time;
         struct timeval total_gc_user_time;
         struct timeval total_gc_system_time;
      }
      virtual Statistics statistics();
   };
\end{verbatim}

    Each collector has a current verbosity level, which can be set
and reset using the methods \verb|set_verbosity(int)| and
\verb|verbosity() const|.  The verbosity level is actually a bit
set containing flags of the type \verb|GC::GCNotify|.  A combination
of these options can be used.
\begin{itemize}
  \item \verb|gc_no_notify| --- no messages at all.
  \item \verb|gc_notify_minor_collection| --- all minor collections
    will be notified by printing a message on the current console.
  \item \verb|gc_notify_major_collection| --- similarly for all major
    collections.
  \item \verb|gc_notify_weak_pointer_collection| --- notify the user
  when weak pointers are being collected.
  \item \verb|gc_print_collection_time| --- this option will let the
    collector to print the time spent during collection (and finalization).
  \item \verb|gc_print_debugging_info| --- this option will print additional
    information such as stack and heap addresses during garbage collection.
\end{itemize}

   The current console of a collector is defaulted to the \Cpp{} stream
\verb|cerr|.  It can be set and inspected with the methods
\verb|set_console(ostream&)| and \verb|ostream& get_console()| respectively.
For example, the user can redirect all garbage collection messages
to a log file \verb|"gc.log"| by executing the following during initialization:

\begin{verbatim}
   ofstream gc_log("gc.log");
   GC::get_default_gc().set_console(gc_log); 
\end{verbatim}

\subsection{The Bartlett style mostly copying collector}

   Currently a Bartlett-style mostly copying collector has been implemented.
The current version is non-generational but we expect that a generational
version will be available in the near future.

   The Bartlett-style collector is implemented as the concrete class
\CLASSDEF{BGC}.  Aside from all the services provided by class \CLASSDEF{GC},
\CLASSDEF{BGC} also provides the following method:

\begin{verbatim}
   class BGC : public CGC {
   public:
      virtual void set_gc_ratio(int);
      virtual void set_initial_heap_size (size_t);
      virtual void set_min_heap_growth   (size_t);
   }
\end{verbatim}

  The gc ratio refers to the ratio of used heap space versus
total heap space.  Class \CLASSDEF{BGC} will invoke the garbage collection
if this is exceeded.  The default gc ratio for \T{BGC} is 75\%.

  The initial size of the heap and the minimal number of bytes to 
request from the system during a heap expansion can be adjusted using 
the methods \verb|set_initial_heap_size| and \verb|set_min_heap_growth|
respectively.  These methods are declared in the \T{GC}
protocol.   By default, the initial heap size for this collector
is 128K and each heap expansion will increase the heap by 512K. 

If an application uses a lot garbage collected storage it is a good
idea to set the heap size to larger capacity during initialization.
Otherwise, the collector will have to perform more garbage collection
and heap expansions to reach a stable state. 

\subsection{The Mark Sweep collector}

   An alternative to the copying collector is the the mark sweep collector.
Like the previous collector, this collector also uses conservative scanning 
to locate roots.  Unlikely the Boehm collector, type accurate marking
is used through the user supplied tracing method.

   Since the default collector is of the \T{BGC} variety, the user must
create an instance of \T{MarkSweepGC} if the mark sweep collector
is to be used.  

\begin{verbatim}
   class MarkSweepGC : public CGC {
   public:
      virtual void set_gc_ratio(int);
      virtual void set_initial_heap_size (size_t);
      virtual void set_min_heap_growth   (size_t);
   }
\end{verbatim}

   The gc ratio in class \T{MarkSweepGC} determines whether heap
expansion is to be performed after a collection.  The default gc ratio
is 50\%, thus if after garbage collection the ratio of used versus total
heap space exceeds one half, the heap will be expanded. 

   For most purposes the two collectors are interchangeable.  Since
all types of garbage collectors under our framework use the same protocol, 
applications can be written without fixing a specific garbage collection
algorithm before hand.  

\subsection{Finalization} 

   One common \Cpp{} idiom uses constructor and destructor for
resource allocation: resources (including memory but may include others,
such as file handles, graphical widgets, etc) that are acquired
in a class constructor are released(finalized) in the class'es destructor.

   There are, however, two opposing views in how finalization should be handled
in a garbage collector.  The first assumes that garbage collection
simulates an infinite amount of memory and so automatic finalization is
inapproriate.   The second takes a more pragmatic view, and assumes that
automatic finalization should be provided since otherwise explicit resource
tracking must be provided by the user for collectable datatypes, making garbage 
collection much less useful than it can be.

   We will refrain from participating in any religious and philosophical
wars on which dogma is the ``One True Way''. 
nstead, both types of collectors are provided.

   By default, all garbage collection classes do 
not perform finalization on garbage 
collected objects.  If object finalization is desired then the user
can do one of two things:
\begin{itemize}
   \item Enable the finalization of the default heap by putting
\begin{verbatim}
   GC::get_default_gc().set_finalization(true);
\end{verbatim}
   in the initialization code, or
   \item Instantiate a different instance of the garbage collector
    and allocate objects needing finalization from this collector.
    This may be a better method since not all objects need finalization
    and those that do not can still be allocated from the default collector.
    This way we only have to pay for the performance penalty (if any)
    proportional to the usage of finalization.
\end{itemize}

   For example, if a collectable class named \T{Foo} should be properly
finalized we can declare it in the following manner:

\begin{verbatim}
   extern BGC bgc_f;  // somewhere an instance is defined

   class Foo : public GCObject {
      Resource r;
   public:
      Foo() { r.allocate(); /* allocate resource */ }
     ~Foo() { r.release(); /* release all resource */ }

      // Make object allocate from bgc_f instead
      // of the default non-collectable gc.
      void * operator new (size_t n) { return bgc_f.m_alloc(n); }
   };
\end{verbatim}

   When an instance of class \verb|Foo| is identified as garbage, its
destructor will be called.  [Aside: {\it currently issue such as 
finalization synchronization is not handled directly by the collector.
So if there is synchronization constraints during finalization it must
be handled by the object destructor.  Future versions of \Prop{} will
provide subclasses with more sophisticated finalization mechanisms.} ]

   Notice that the default \verb|delete| operator of all garbage collectable
classes will call the object destructor explicitly.  It is a good idea to use
explicit \verb|delete| if finalization is a rare need since this service
involves an explicit scan of the garbage memory, which may degrade performace
with excessive swapping: without finalization the garbage pages will not 
have to be referenced with a copying collector.

   It should also be noted that since the collector is conservative, there
is no guarantee that garbage will be identified as such:
there is no guarantee that all garbage resources will be released.
In general, however, the efficacy of the collector is quite good and
so non-critical or plentiful resources can be safely finalized with
this collector.

\Subsubsection{Weak Pointers}

   It is frequently very useful to be able to keep track of garbage
collectable objects through the use of ``{\bf weak pointers}.''
Collectors ignore the presence of weak pointers to garbage collected
objects; objects with only referencing weak pointers will still be
collected.  Weak pointers to objects that become garbage will
be automatically reset to null.
 
   Weak pointers are provided through a smart pointer template 
\CLASSDEF{WeakPointer}, whose definition is shown below: 

\begin{verbatim}
template <class T> class WeakPointer {
   public:
      inline WeakPointer();
      inline WeakPointer(const WeakPointer<T>& wp); 
      inline WeakPointer(T * ptr); 
      inline WeakPointer<T>& operator = (const WeakPointer<T>& wp);
      inline WeakPointer<T>& operator = (T * ptr);
      inline bool is_null() const;
      inline operator const T * () const;
      inline operator       T * ();
      inline const T * operator -> () const;
      inline       T * operator -> ();
      inline const T&  operator *  () const;
      inline       T&  operator *  ();
   };
\end{verbatim}

A weakpointer to a garbage collectable 
class \verb|A| can be defined as \verb|WeakPointer<A>|.
If a \Prop{} datatype \verb|T| has been defined, a weakpointer to type
\verb|T| can defined using the
\TDEF{classof} keyword.  For example:

\begin{verbatim}
   datatype Wff :: collectable = True | False | And(...)  | ...;

   WeakPointer<classof Wff> a = And(True,False);
\end{verbatim}

\Subsubsection{The Heap Walker}

   There is a simple class called \CLASSDEF{GCVerify} that can be used
to verify the integrity of a complex allocated data structure.
This is frequently useful during the development of a new collector,
or a complex class derive class of \T{GCObject}.

   The interface of this class is shown below.

\begin{verbatim}
   class GCVerifier : protected GC {
   public:
      virtual bool is_valid_pointer          (GCObject *);
      virtual bool is_valid_interior_pointer (GCObject *);
      virtual bool is_valid_structure        (GCObject *);
      virtual bool is_valid_pointer          (GCObject *, GC *);
      virtual bool is_valid_interior_pointer (GCObject *, GC *);
      virtual bool is_valid_structure        (GCObject *, GC *);
      size_t number_of_nodes() const;
   };
\end{verbatim}

   Class \CLASSDEF{GCVerify} is derived from class \CLASSDEF{GC} so that the
same object tracing protocol can be used.  Briefly, three different methods 
are provided for verification:
\begin{itemize}
   \item Call to \verb|is_valid_pointer(p,gc)| verifies that
    \verb|p| is a valid pointer to an object within the collector \verb|gc|. 
   \item Call to \verb|is_valid_pointer(p,gc)| verifies that
    \verb|p| is a valid interior pointer to an object 
     within the collector \verb|gc|. 
   \item Call to \verb|is_valid_structure(p,gc)| verifies the entire
     structure reachable by \verb|p| is valid.  This method uses
     \verb|GCObject::trace| to locate the correct sub-structures.
\end{itemize}

   There are alternative versions of the same methods that assumes
the default collector is used.   Furthermore
\begin{itemize}
   \item Method \verb|number_of_nodes()| returns the node count of the
last structure traced using \verb|is_valid_structure|, and 
   \item If \verb|set_verbosity(GC::gc_print_debugging_info)| is used
then error messages will be printed during structure tracing.
\end{itemize}

\bibliography{refman}

%%%%%%%%%%%%%%%%%%%%%%%%%%%%%%%%%%%%%%%%%%%%%%%%%%%%%%%%%%%%%%%%%%%%%%%%%%%%%%%
%
%  Here are some abbreviations used within this document
%
%%%%%%%%%%%%%%%%%%%%%%%%%%%%%%%%%%%%%%%%%%%%%%%%%%%%%%%%%%%%%%%%%%%%%%%%%%%%%%%
\newfont{\bsf}{cmssbx10}
\newfont{\largebsf}{cmssbx12}
%\newfont{\nontermfont}{cmssbx10}
\newcommand{\nontermfont}{\it}
\newcommand{\codefont}{\tt}
\newcommand{\symfont}{\bf}

\newcommand{\Prop}{{\bsf Prop}}               % typeset in sans serif
\newcommand{\LargeProp}{{\largebsf Prop}}
\newcommand{\plusplus}{{\tt++}}             % ++
\newcommand{\Cpp}{{C{\tt++}}}               % C++ logo
\newcommand{\Gpp}{{G{\tt++}}}               % G++ logo
\newcommand{\email}{{\tt leunga@cs.nyu.edu}}  % my email address
\newcommand{\web}{{\tt http://valis.cs.nyu.edu:8888/\~{}leunga}} % my web page
\newcommand{\propweb}{\web{\tt/prop.html}}
\newcommand{\Version}{2.3.4}                  % Prop version
\newcommand{\comment}[1]{}

\newcommand{\sectionfont}{}
%\newcommand{\Section}[1]{\section{\sectionfont #1}}
%\newcommand{\Subsection}[1]{\subsection{\sectionfont #1}}
%\newcommand{\Subsubsection}[1]{\subsubsection{\sectionfont #1}}

\newcommand{\Section}[1]{\pagebreak\section{\sectionfont #1}%
   \markright{\usebox{\LINE}\large \thesection. \sc #1}
   }
\newcommand{\Subsection}[1]{\subsection{\sectionfont #1}}
\newcommand{\Subsubsection}[1]{\subsubsection{\sectionfont #1}}

\newcommand{\N}[1]{\mbox{{\nontermfont #1}}}
\newcommand{\T}[1]{{\codefont #1}}
\newcommand{\MORE}[2]{#1\T{#2} \mbox{$\ldots$} \T{#2} #1}
\newcommand{\ALT}[2]{#1\ \T{#2} \mbox{$\ldots$} \T{#2} #1}
\newcommand{\LIST}[1]{\MORE{#1}{,}}
\newcommand{\SEQ}[1]{\MORE{#1}{;}}
\newcommand{\OPT}[1]{{\it [ #1 ]}}
\newcommand{\C}[1]{\mbox{\rm #1}}
\newcommand{\IS}{{\symfont ::=}}
\newcommand{\OR}{\ \mbox{$\symfont \mid$}\ }
\newcommand{\Id}{\N{Id}\ }
\newcommand{\Pat}{\N{Pat}\ }
\newcommand{\Exp}{\N{Exp}\ }
\newcommand{\TypeExp}{\N{Type\_Exp}\ }
\newenvironment{syntax}{\begin{quotation}\noindent\begin{tabular}{lcll}} %
   {\end{tabular}\end{quotation}}

\newcommand{\CLASSDEF}[1]{\T{#1}\index{Classes!{\codefont #1}}}
\newcommand{\INDEX}[1]{\index{#1}}
\newcommand{\NDEF}[1]{\N{#1}\index{Syntax!Non--terminals!{\nontermfont #1}}}
\newcommand{\OPTIONDEF}[1]{{\tt #1}\index{Command line options!{\tt #1}}}
\newcommand{\TDEF}[1]{\T{#1}\index{Syntax!Keywords!{\codefont #1}}}
\newcommand{\INCOMPLETE}{{\em This section is incomplete.}}
\newenvironment{Error}{\begin{center}\begin{tabular}{p{3in}p{3in}}
  \bf Error & \bf Explanation}{\\\hline\end{tabular}\end{center}}
\newcommand{\errormsg}[1]{\\\hline {\tt #1}&}
\newenvironment{Tips}{\noindent {\bf Tips:} \quad}{}

\newcommand{\support}{
   {\small\noindent This work is supported in part by
    an award from Hewlett-Packard Corporation, from IBM corporation, and
    an NYU Research Challenge Grant.
   }}
\newcommand{\warning}{
   {\small\noindent The author will neither assume responsibility for any
   damages caused by the use of this product, nor accept warranty or update
   claims.  This product is distributed in the hope that it will be useful,
   but WITHOUT ANY WARRANTY; without even the implied warranty of
   MERCHANTABILITY or FITNESS FOR A PARTICULAR PURPOSE.

   \small\noindent This product is in the public domain,
   and may be freely distributed.

   \small\noindent {\sf Prop} is a research prototype. 
    The information contained in this document is subject
    to change without notice.
   }}

   \makeindex

%%%%%%%%%%%%%%%%%%%%%%%%%%%%%%%%%%%%%%%%%%%%%%%%%%%%%%%%%%%%%%%%%%%%%%%%%%%%%%
%
%   Beginning of the document
%
%%%%%%%%%%%%%%%%%%%%%%%%%%%%%%%%%%%%%%%%%%%%%%%%%%%%%%%%%%%%%%%%%%%%%%%%%%%%%%
\documentclass{article}
\usepackage{html}

   \pagestyle{myheadings}

   \title{{\Huge \bf Prop \\ 
           Language Reference Manual} \\ 
          {\large \bf (for version \Version)}
         } 
   \author{{\Large Allen Leung} \\ 
           E-mail: \email \\
           \htmladdnormallink{\web}{http://valis.cs.nyu.edu:8888/~leunga}
	   \\
           Courant Institute of Mathematical Sciences \\
           251 Mercer Street \\
           New York, NY 10012 \\
          }

   \setlength{\textwidth}{6.6in}
   \setlength{\evensidemargin}{0in}
   \setlength{\oddsidemargin}{0in}
   \setlength{\textheight}{8.75in}
   \setlength{\topmargin}{-0.5in}
   \setlength{\parskip}{1mm}
   \setlength{\parindent}{3ex}
   \setlength{\itemsep}{0.5ex}

   \newsavebox{\LINE}
   \savebox{\LINE}[0.0in][l]{\rule[-1ex]{\textwidth}{.01in}}
\begin{document}

   \maketitle
   \vspace{\fill}
   %\support

   \begin{abstract}
    This reference manual provides an introduction to
    \Prop, release \Version.  \Prop{} is a multiparadigm extension of \Cpp,  
    and is designed for building high performance compiler 
    and language transformation systems, using pattern matching and rewriting.  
    This guide describes the syntax and semantics of
    the \Prop{} language and describes how to develop programs
    using the \Prop{} to \Cpp{} translator.

   \end{abstract}
   \warning
   \titlepage
   \tableofcontents
   \bibliographystyle{alpha}

\Section{Introduction} \label{sec:intro}
 
   This is the language reference manual for \Prop, a compiler generator.
\Prop{} can be used to generate lexers, parsers, analysis and 
transformation tools using pattern matching and rewriting.  
The \Prop{} language is a {\em multi-paradigm} 
extension of \Cpp\cite{C++} and
\Cpp{} programs generated from \Prop{} specifications can be compiled
into efficient code.

   Version \Version{} of \Prop{} includes the following major features:
\begin{enumerate}
   \item Algebraic datatypes and pattern matching, and automatic
mapping of algebraic datatypes into \Cpp{} class hierarchies.
   \item A LALR(1) parser generator in the tradition of {\it yacc} and
{\it bison}.  Each grammar is encapsulated into a parser class and 
it is possible to have multiple instances of a parser.
   \item Regular expression string matching and lexer generation.
   \item Transformation of algebraic datatypes using conditional
   rewriting.
   \item In addition, using the view mechanism, 
it is possible to treat externally defined C or \Cpp{}
data structures as if they were \Prop{} defined datatypes.  
The full range of pattern matching and rewriting capability can
be used on these externally defined data..
\end{enumerate}

   In addition, we intend to implement/fix-up the following features in the 
near future:
\begin{enumerate}
   \item Persistence.
   \item Pretty printing/reading.
   \item Visualization of data structures using {\em vcg}\cite{vcg,vcg-manual}.
   \item Automatic mapping of graph structure specifications into
\Cpp{} classes.
   \item Graph rewriting system generation.
\end{enumerate}

\Subsection{Availability}

   The latest release of \Prop{} \Version{} is available from
\begin{quotation}
\htmladdnormallink{\propweb}{http://valis.cs.nyu.edu:8888/~leunga/prop.html}
\end{quotation}
New updates of the source code and documentation will be available
from this site.

\Subsection{Organization of this Reference Manual}
   This reference manual is organized as follows.  Section~\ref{sec:general}
overviews the general features of \Prop.
Section~\ref{sec:parser}
describes the lexer and parser specification formalisms.
Section~\ref{sec:datatype} describes the algebraic datatypes and 
pattern matching features.  Section~\ref{sec:tree-rewriting} describes
the tree rewriting mechanism.  Finally, section~\ref{sec:usage} describes
the process of running the \Prop{} translator.

   We'll use the following extended BNF notation to specify the syntax
of the language.  Terminals are typeset in {\tt typewriter} font,
while non-terminals are typeset in {\nontermfont itallics} font.
Given a syntactic class \N{S}, we use \LIST{\N{S}} to denote one
or more occurrences of \N{S} separated by commas.  Similarly,
\SEQ{\N{S}} denotes one or more occurrences of \N{S} separated by
semi-colons.  Slanted brackets are used to denote an optional occurrence
of a syntactic class.   We'll combined the two forms when we
want to denote zero or more occurrences of some term.
For example, \OPT{\LIST{\N{S}}} denotes
zero or more occurrences of \N{S}.

\Section{General Concepts} \label{sec:general}

\Subsection{Syntactic Conventions}

The syntax of \Prop{} is an extension of \Cpp.  Most new constructs
of \Prop{} starts with new keywords, listed below.  
\begin{center}
\tt
\begin{tabular}{lllllll}
applicative & arb & as & bagof & begin & bitfield & card \\
category & classof & collectable & constraint & dataflow & datatype & declare \\
dequeof & dom & domain & edges: & elsif & end & equiv: \\
exists & expect: & feature & finalizable & forall & fun & function \\
functor & graphrewrite & graphtype & implies: & inference & inherited & inline \\
instantiate & is & law & left: & less & lexeme & listof \\
loop & mapof & match & matchall & matchscan & module & multimapof \\
mutable & nodes: & of & persistent & prec: & printable & priqueueof \\
procedure & queueof & ran & refine & relation & rewrite & right: \\
setof & sharing & signature & space: & syntax & synthesized & then \\
time: & topdown & traced & treeparser & tupleof & unifiable & unique \\
view & where & with & xor: & bottomup: & topdown: & before: \\
preorder: & postorder: & cutrewrite & failrewrite & index: \\
\end{tabular}
\end{center}

   In addition, the following new multi-character symbols are added:
\begin{verbatim}
   <->  <=>  :=  :-  .(   .[
   #[   #(   #{  [|  |]   (|  |)  {|  |} 
\end{verbatim}

\Subsection{Basic syntactic classes}


We'll use \NDEF{Integer}, \NDEF{Char}, \NDEF{Real}, \NDEF{String} 
and \NDEF{Id} to denote the syntactic classes of 
integers, characters, real numbers, strings and identifiers, respectively.
As in \Cpp, characters are quoted with the single quotes \verb|'|, while
strings are quoted with double quotes \verb|"|.  An identifier is
any sequence of alphanumeric of characters that begins with a letter.
In addition, we'll use the syntactic class
\NDEF{Stmt} to denote any valid combination of \Cpp{} and \Prop{}
statements.

\Subsection{Literals}

  In addition to the usual \Cpp{} literals, \Prop{} introduces two new type
of literals: {\em quark literals} and {\em regular expression literals}.
Quark literals are of type \T{Quark} and are written as strings
prefixed by the \verb|#| symbol.
\begin{syntax}
  \NDEF{Quark} & \IS & \T{\#}\N{String} \\
\end{syntax}   

Quarks act like atoms in Lisp, in which name equality implies
pointer equality.  In contrast, in \Cpp{} two string literals that are
identical in value do not necessarily reside in the same address.
Given strings \verb|s1| and \verb|s2|, \verb|strcmp(s1,s2) == 0| does
not imply \verb|s1 == s2|.  Quarks are defined in the library include file
\verb|<AD/strings/quark.h>|.

Regular expression literals are similar to string literals except
that they are quoted by slashes \verb|/|.   We'll discuss
the regular expression in section~\ref{sec:regular-expressions}.

\Subsection{The \Prop{} Language}

The basic \Prop{} language is the same as \Cpp{} with the following
syntactic extensions:
\begin{quotation}
\begin{tabular}{lll}
         & \bf Keywords              & \bf Functions \\
  \bsf 1 & \T{datatype} \T{refine} \T{instantiate}
                                     & Algebraic datatype definitions \\
  \bsf 2 & \T{match} \T{matchall}    & Pattern matching \\
  \bsf 3 & \T{rewrite}               & Rewriting \\
  \bsf 4 & \T{lexeme}                & Lexical category definition \\
  \bsf 5 & \T{matchscan}             & Lexical scanning \\
  \bsf 6 & \T{syntax}                & Parser specification \\
  \bsf 7 & \T{graphtype}             & Graph data structure specification \\
  \bsf 8 & \T{graphrewrite}          & Graph rewriting \\
\end{tabular}
\end{quotation}

Like \Cpp, a \Prop{} is typically divided into the {\em specification},
which defines the data structure and its interface, and the
{\em implementation} parts.   \Prop{} specifications should
be placed in a file with a \verb|.ph| suffix, while an implementation
should be placed in a file with a \verb|.pC|\footnote{The suffixes {\tt .pcc},
{\tt .pCpp} etc. can also be used.}.  The translator will convert each
\verb|.p|$xxx$ into a file with the suffix \verb|.|$xxx$.  The translated
output can then be fed to the \Cpp{} compiler for compilation.
\vspace{\fill}

\Section{Lexer and Parser Generation} \label{sec:parser}

   In this section we'll describe the lexer and parser generation
mechanism of \Prop.  Note that the use of these features is not
mandatory: the user can substitute any other lexer/parser generation tools 
or use hand written lexers and parsers.  
On the other hand, higher level of integration is possible when using
\Prop's lexer and parser mechanisms.

\Subsection{Lexer Specification}
   Lexical analyzers are specified using the \verb|matchscan|
statement.  This construct is responsible for generating the actual string
matching DFA.  The actual buffering mechanisms are provided by the
classes \verb|LexerBuffer|, \verb|IOLexerBuffer| and \verb|IOLexerStack|. 
These classes are part of the support library distributed with \Prop.

   We assume that the user is familiar with lexer generator tools like
{\em lex}\cite{lex} or {\em flex}\cite{flex}.

\Subsubsection{Regular expressions} \label{sec:regular-expressions}
   Figure~\ref{fig:regexp} describes the set of current
supported regular expressions in \Prop.  The syntax is similar
to what is found in {\em lex}, or {\em egrep}.

\begin{figure}[htbp]
\begin{center}
  \begin{tabular}{|l|l|} \hline
    $c$                & matches character $c$ if it is not a meta character \\
    $e_1e_2$           & matches $e_1$ then $e_2$ \\
    \verb|.|           & matches any character except \verb|\n| \\
    $\verb|\|c$        & matches escape sequence $c$ \\
    \verb|^|$e$        & matches $e$ at the start of the line \\
    $e_1\verb.|.e_2$   & matches $e_1$ or $e_2$ \\
    $e\verb|*|$        & matches zero or more $e$ \\
    $e\verb|+|$        & matches one or more $e$ \\
    $e\verb|?|$        & matches zero or one $e$ \\
    \verb|(|$e$\verb|)| & grouping \\
    \verb|<<|$C_1, C_2 \ldots C_n$\verb|>>|$e$ & matches $e$ only
         if we are in a context from one of $C_1, C_2, \ldots C_n$ \\
    \verb|{|{\em lexeme}\verb|}| & matches lexeme \\
  \hline
  \end{tabular}
\end{center}
\caption{\label{fig:regexp} Regular expressions.}
\end{figure}

The symbols \verb%\ [ ] ( ) { } << >> * + . - ? |% are meta characters and are
interpreted non-literally.  The escape character \verb|\| 
can be prepended to the meta characters if they occur as literals in context.  

Precedence-wise, meta characters \verb|*|, \verb|+| and \verb|?| bind tighter
than juxtaposition.  Thus the regular expression
\verb|ab*| means \verb|a(b*)|.  Parenthesis can be used to override
the default precedence.

Character classes are of the form as found in {\em lex}: 
(i) $c_1$\verb|-|$c_2$ denotes the range of characters from
$c_1$ to $c_2$; (ii) single (non-meta) characters denote themselves; 
(iii) the meta character \verb|^| can be used to negate the set.
For example, the regular expression \verb|[a-zA-Z][a-zA-Z0-9]*| 
specifies an alphanumeric identifier that must starts with a letter.
Similarly, the regular expression \verb|[^ \t\n]| matches any character
except a space, a tab or a newline.

{\em Lexemes} are simply abbreviated names given to a regular expression
pattern.  They act like macros in {\em lex}.

While a lexer is scanning, it may be in one of many {\em contexts}.
Contexts can be used to group a set of related lexical rules; such rules
are only applicable when the contexts are active.  This makes the
lexer behave like a set of DFAs, with the ability to switch between
DFAs under programmer control.  


\Subsubsection{Lexeme}

Lexemes are defined using the \T{lexeme} declaration in \Prop. 
Its syntax is
\begin{syntax}
  \NDEF{Lexeme\_Decl} & \IS & \TDEF{lexeme} \ALT{\N{Lexeme\_Eq}}{|} \T{;} \\
  \NDEF{Lexeme\_Eq}   & \IS & \Id \T{=} \N{Regexp} \\
\end{syntax}

For example, the following lexeme definition is used in the \Prop{}
translator to define the lexical structure of common lexical items.

\begin{verbatim}
lexeme digits      = /[0-9]+/
     | sign        = /[\+\-]/
     | integer     = /{digits}/
     | exponent    = /[eE]{sign}?{digits}/
     | mantissa    = /({digits}\.{digits}?|\.{digits})/
     | real        = /{mantissa}{exponent}?/
     | string      = /"([^\\"\n]|\\.)*"/ 
     | character   = /'([^\\'\n]|\\.[0-9a-f]*)'/
     | regexp      = /\/([^\/\n*]|\\.)([^\/\n]|\\.)*\//
     ;
\end{verbatim}

Note that regular expression literals are written between
slashes: \verb|/|$re$\verb|/|.  

\Subsubsection{Lexeme class}

Often we wish to group a set of lexemes together into {\em lexeme classes} if 
they logically behave in some uniform manner; for example, if they
act uniformly in a lexical rule.  By grouping related lexemes together
into a class we can refer to them succinctly by their class name.

The syntax of a lexeme class declaration is

\begin{syntax}
  \NDEF{Lexeme\_Class\_Decl} & \IS & \TDEF{lexeme class} 
       \ALT{\N{Lexeme\_Class\_Eq}}{and} \T{;} \\
  \NDEF{Lexeme\_Class\_Eq} & \IS & \Id \T{=} \ALT{\N{Lexeme\_Spec}}{|} \\
  \NDEF{Lexeme\_Spec} & \IS & \N{String} \\
                   & \OR & \Id \N{Regexp} \\
\end{syntax}

The following example,
the lexeme classes \verb|MainKeywords|, \verb|Symbols|, and \verb|Literals| 
are defined.  
\begin{verbatim}
lexeme class MainKeywords =
    "rewrite" | "inference" | "match" | "matchall" | "matchscan"
  | "refine"  | "classof" | "type" | "datatype" | "instantiate"
  | "lexeme"  | "bitfield" | "begin" | "syntax"
  | "dataflow" | "module" | "signature" | "constraint" | "declare"
  | "procedure" | "fun" | "function" | "domain" 
  | "graphtype" | "graphrewrite"

and  Symbols =
    ".." | "..." | "<->" | "::" | "&&" | "||" | "++" | "--" | "->" 
  | "<<" | ">>" | ">=" | "<=" | "+=" | "-=" | "*=" | "/=" | "%=" | "==" 
  | "!=" | "<<=" | ">>=" | "&=" | "|=" | "^=" | "=>" | "<-" | "<=>"
  | ":=" | ":-" | LONG_BAR /\-\-\-\-\-+/

and Literals =
    INT_TOK     /{integer}/
  | REAL_TOK    /{real}/
  | CHAR_TOK    /{character}/
  | STRING_TOK  /{string}/
  ;
\end{verbatim}

\Subsubsection{Tokens}

Lexical tokens are defined using the \verb|datatype| declaration. 
The syntax is as follows.
\begin{syntax}
  \NDEF{Tokens\_Decl} & \IS & \TDEF{datatype} \Id \T{::} \TDEF{lexeme} \T{=} \\
      & & \quad \ALT{\N{Token\_Spec}}{|} \T{;} \\
  \NDEF{Token\_Spec} & \IS & \TDEF{lexeme class} \Id & 
       \C{Include a lexeme class} \\
                  & \OR & \Id \OPT{\N{Regexp}}  &
       \C{Single token spec} \\
\end{syntax}

A token datatype is defined by including one of more lexeme classes,
and by defining stand-alone tokens.  If a lexeme class is included,
all the tokens defined within the lexeme class are included.  A \Cpp{}
\T{enum} type of the same name as the token datatype is generated.
If a token is given an identifier name, then the same name is used
as the \T{enum} literal.  On the other hand, if a string is used
to denote a token, then it can be referred to by prefixing the
string with a dot \verb|.|  For example, the token \verb|"=>"| can
be referenced as \verb|."=>"| within a program.

As an example, the following token datatype definition is used within
the \Prop{} translator.  Here, the keywords are first partitioned into
6 different lexeme classes.  In additional, the tokens \T{ID\_TOK},
\T{REGEXP\_TOK}, etc. are defined.

\begin{verbatim}
datatype PropToken :: lexeme =
    lexeme class MainKeywords
  | lexeme class Keywords
  | lexeme class SepKeywords
  | lexeme class Symbols
  | lexeme class Special
  | lexeme class Literals
  | ID_TOK       /{patvar}/
  | REGEXP_TOK   /{regexp}/
  | QUARK_TOK    /#{string}/
  | BIGINT_TOK   /#{sign}{integer}/
  | PUNCTUATIONS /[\<\>\,\.\;\&\|\^\!\~\+\-\*\/\%\?\=\:\\]/
  ;
\end{verbatim}

\Subsubsection{The \T{matchscan} statement}\label{sec:matchscan}

The \T{matchscan} statement is used to perform tokenization.  The user can
specify a set of string pattern matching rules within a \T{matchscan}
construct. Given a object of class \verb|LexerBuffer|, the \T{matchscan}
statement looks for the rule that matches the longest prefix from
the input stream and executes the action associated with the rule.
Ties are broken by the lexical ordering of the rules.  

The general syntax is as follows:
\begin{syntax}
\NDEF{Matchscan} 
      & \IS & \N{Matchscan\_Mode} \OPT{\N{Context\_Spec}} \T{(} \N{Exp} \T{)} 
             & variant 1 \\
      & &  \T{\{} \ALT{\T{case} \N{Matchscan\_Rule}}{} \T{\}} \\
      & \OR & \N{Matchscan\_Mode} \OPT{\N{Context\_Spec}} 
            \T{(} \N{Exp} \T{)} & variant 2 \\
      & &   \T{\{} \ALT{\N{Matchscan\_Rule}}{|} \T{\}} \\
      & \OR & \N{Matchscan\_Mode} \OPT{\N{Context\_Spec}} \T{(} \N{Exp} \T{)} 
            \T{of} & variant 3 \\
      & & \quad \ALT{\N{Matchscan\_Rule}}{|} \\
      & & \T{end} \T{matchscan} \T{;}\\ 
\NDEF{Matchscan\_Mode} & \IS & 
           \TDEF{matchscan} \OPT{\T{while}} & case sensitive\\
   & \OR & \TDEF{matchscan*} \OPT{\T{while}}& case insensitive\\
\NDEF{Context\_Spec} & \IS & \T{[} \LIST{\N{Id}} \T{]} \\
\NDEF{Matchscan\_Rule} 
      & \IS & \OPT{\T{<<} \ALT{\N{Context}}{,} \T{>>}} \\
      &     & \quad \T{lexeme} \T{class} \N{Id} \T{:} \N{Matchscan\_Action} \\
      & \OR & \OPT{\T{<<} \ALT{\N{Context}}{,} \T{>>}} \\
      &    &  \quad \N{Regexp} \T{:} \N{Matchscan\_Action} \\
\NDEF{Matchscan\_Action} & \IS & \T{\{} \N{Code} \T{\}} \\
          & \OR & \N{Code} & for variant 1 only \\
\end{syntax}

The two different modes of operation are \T{matchscan} and \T{matchscan*},
which respectively match strings case sensitively and insensitively.
The modifier \T{where} may optionally specify that the matching process
should repeat until no rules apply, or the end of stream condition
is reached.  

By default, if no rules apply and if the input stream is non-empty,
then an error has occurred.  The \T{matchscan} statement will
invoke the method \verb|error()| of the \T{LexerBuffer} object by default.

For example, the following is part of the \Prop{} lexer specification.

\begin{verbatim}
   datatype LexicalContext = NONE | C | PROP | COMMENT | ...;

int PropParser::get_token()
{
   matchscan[LexicalContext] while (lexbuf)
   {  
      ...
   |  <<C>> /[ \t\\\014]/:            { emit(); }
   |  <<C>> /(\/\/.*)?\n/:            { emit(); line++; }
   |  <<C>> /^#.*/:                   { emit(); }
   |  <<PROP>> lexeme class MainKeywords: { return ?lexeme; }
   |  <<PROP>> lexeme class SepKeywords: { return ?lexeme; }
   |  <<PROP>> QUARK_TOK:             { return QUARK_TOK; }
   |  <<PROP>> BIGINT_TOK:            { return BIGINT_TOK; }
   |  <<PROP>> REGEXP_TOK:            { return REGEXP_TOK; }
   |  <<PROP>> PUNCTUATIONS:          { return lexbuf[0]; }
   |  <<PROP>> /[ \t\014]/:           { /* skip */ }
   |  <<PROP>> /(\/\/.*)?\n/:         { line++; }
   |  /\/\*/:                         { emit(); set_context(COMMENT); }
   |  <<COMMENT>> /\*\//:             { emit(); set_context(PROP); }
   |  <<COMMENT>> /\n/:               { emit(); line++; }
   |  <<COMMENT>> /./:                { emit(); }
   |  /./: { error("%Lillegal character %c\n", lexbuf[0]); }
   }
}
\end{verbatim}

Here, the lexer is partitioned in multiple lexical contexts: context
\verb|C| deals with \Cpp{} code while context \verb|PROP| deals with
\Prop{} extensions.  The special context \verb|COMMENT| is used
to parse \verb|/* */| delimited comments.  Contexts are changed
using the \verb|set_context| method defined in class \verb|LexerBuffer|.

The special variable \verb|?lexeme| can be used within a rule 
that matches a lexeme class.  For example, within the rule
\begin{verbatim}
   |  <<PROP>> lexeme class MainKeywords:    { return ?lexeme; }
\end{verbatim}
the variable \verb|?lexeme| is bound to 
the token \verb|."rewrite"| if the string ``rewrite'' is matched;
it is bound to the token \verb|."inference"| if the string
``inference'' is matched and so on.

\Subsubsection{Class \T{LexerBuffer}}
We'll next describe the class \T{LexerBuffer} and its subclasses.

Class \CLASSDEF{LexerBuffer} 
is the base class in the lexical buffer hierarchy.
It is defined in the library include file \verb|<AD/automata/lexerbuf.h>|.
This class is responsible for implementing a string buffer for use
during lexical analysis.   

As it stands, it can be used directly if the lexer input is directly
from a string.  Memory management of the buffer is assumed to be handled
by the user. 

The class \T{LexerBuffer} has three constructors.  The default
constructor initializes the string buffer to NULL.  The two
other constructors initialize the string buffer to a string given
by the user.  In the case when the length is not supplied, the buffer
is assumed to be \verb|'\0'|-terminated.  The two \verb|set_buffer| methods
can be used to set the current string buffer.  Notice that
all lexical analysis operations are done in place.  The user should
not alter the string buffer directly, but should use the interface
provided by this class instead.

\begin{verbatim}
class LexerBuffer {
public:
   LexerBuffer();    
   LexerBuffer(char *);    
   LexerBuffer(char *, size_t);    
   virtual ~LexerBuffer();
   virtual void set_buffer (char *, size_t);
           void set_buffer (char *);
};
\end{verbatim}

The following methods are used access the string buffer.
Method \T{capacity} returns the size of the buffer.  Method
\T{length} returns the length of the current matched token.
Methods \T{text} can be used to obtain a point to location
of the current matched token.  The string returned is guaranteed
to be \verb|'\0'|-terminated.  Methods \T{operator []} return
the $i$th character of the token.  Finally, method \T{lookahead}
returns the character code of the next character to be matched.

\begin{verbatim}
   int capacity () const;
   int length   () const;
   const char * text () const;
         char * text ();
   char  operator [] (int i) const;
   char& operator [] (int i);
   int lookahead () const;
   void push_back (int n) 
\end{verbatim}

In addition to the string buffer, the class \T{LexerBuffer} keeps
track of two additional types of information: the current context
of the DFA, and whether the next token starts at the
beginning of the line, or in our terminology, whether it is
{\em anchored}.  These are manipulated with the following methods:

\begin{verbatim}
   int context    () const;
   void set_context (int c = 0);
   Bool is_anchored() const;
   void set_anchored(Bool a = true);
\end{verbatim}

Finally, the following methods should be redefined by subclasses
to alter the behavior of this class.  By default, the class \T{LexerBuffer}
calls \T{fill\_buffer()} when it reaches the end of the string; subclasses
can use this method to refill the buffer and return the number of
characters read.  Currently, \T{fill\_buffer} is defined to do nothing 
and return 0.  When it reaches the end of the file 
(i.e. when \T{fill\_buffer()} fails to refill the buffer and the 
scanning process finishes), method \T{end\_of\_file} is called.
Currently, this is a no-op.  Finally, the error handling routine
\T{error()} is called with the position of the beginning and the end
of the buffer in which the error occurs.  By default, this routine
prints out the buffer. 

\begin{verbatim}
protected:           
   virtual size_t fill_buffer();
   virtual void   end_of_file();
   virtual void   error(const char * start, const char * stop);
\end{verbatim}

\Subsubsection{Class \T{IOLexerBuffer}}

  Class \CLASSDEF{IOLexerBuffer} 
is a derived class of \T{LexerBuffer}.  It 
automatically manages an internal buffer and receives input from
an \T{istream}.  It is defined in the file \verb|<AD/automata/iolexerbuf.h>|

This class provides the following additional features:
Constructor \verb|IOLexerBuffer(istream&)| ties the input to a
stream.  If the default constructor is used, then it automatically
ties the input to the stream \verb|cin|.  The method \verb|set_stream|
can be used to set the stream to some other input. 

\begin{verbatim}
class IOLexerBuffer : public LexerBuffer {
   size_t    buffer_size;  // size of the buffer
   istream * input;        // input stream
public:
   IOLexerBuffer();    
   IOLexerBuffer(istream&);    
   virtual ~IOLexerBuffer();
   virtual void set_stream (istream&);
};
\end{verbatim}

By default, class \verb|IOLexerBuffer| reads new data from the input
stream using a line buffering discipline.  This mechanism is suitable
for interactive, but for other applications it may be more efficient
to use block buffered input.  The protected method 
\verb|read_buffer| controls this reading behavior; it
is called whenever the data in the string buffer has been
consumed and new input is needed from the stream.  It is
passed the position of the buffer and its remaining capacity,
and it returns the number of new characters that are read.  Subclasses
can redefine this method. 

\begin{verbatim}
protected:
   virtual size_t read_buffer(char *, size_t);
\end{verbatim}

\Subsubsection{Class \T{IOLexerStack}}

Class \CLASSDEF{IOLexerStack}
is a derived class of \T{LexerStack}.  It provides
a mechanism of reading from a stack of \verb|istream|'s.  Streams
can be pushed and popped from the stack.  The next token is 
obtained from the stream from the top of the stack.  The class allows
allows easy implementation of 
constructs such as the \verb|#include| file mechanism of the C preprocessor.

The interface of this class is listed below:
\begin{verbatim}
class IOLexerStack : public IOLexerBuffer {
public:
   IOLexerStack();    
   IOLexerStack(istream&);    
   virtual ~IOLexerStack();
   
   virtual void     push_stream (istream&);
   virtual istream& pop_stream  ();
};
\end{verbatim}

\Subsection{Parser Specification}

Parsers are specified as a two phase process:
\begin{enumerate}
   \item First a {\em syntax class} is defined.  A syntax class
declaration is like a normal \Cpp{} class declaration and has a similar
syntax, except that \Prop{} will also generate the interface of the parser. 
All parsers that \Prop{} generates are encapsulated within a class.  This
makes it easy for the programmer to add additional data for parsing
use, and to have multiple instances of a parser. 
   \item Secondly, the grammar of the language is defined in a
{\em syntax} declaration as a set of productions, 
in a syntax similar to that of {\em yacc}.
\end{enumerate}

We'll describe these two phases.

\Subsubsection{Syntax class}

A syntax class definition specifies an object class that encapsulates
a parser.  Its syntax is the same as the usual \Cpp{} class definition,
except that the prefix \T{syntax} is used: 

\begin{syntax}
\NDEF{Syntax\_Class\_Decl} & \IS & \TDEF{syntax class} \Id 
   \OPT{\T{:} \N{Inherit\_List}} \T{\{} \N{Class\_Body} \T{\}} \T{;} \\
\end{syntax}

This specification automatically generate a class with the following interface
(assuming that the parser class in question is called \T{ParserClass})
\begin{verbatim}
public:  
   ParserClass();  // constructor
public:
   virtual void parse();
\end{verbatim}

The constructor and the method \verb|parse()| will be automatically
generated later. 

A parser class expects the following method to be defined:
\begin{verbatim}
   int get_token();
\end{verbatim}
\noindent The function of \verb|get_token| is to return the
next token from the lexer stream whenever it is called.
It should return \verb|EOF| if the stream is empty.

\Subsubsection{Syntax declaration}

The grammar of a language is specified in the \T{syntax} declaration.
Its syntax is listed as follows:

\begin{syntax}
\NDEF{Syntax\_Decl} & \IS & \TDEF{syntax} \Id \T{\{} \\
    & & \quad \OPT{\MORE{\N{Precedence\_Decl}}{}} \\
    & & \quad \OPT{\N{Expect\_Decl}} \\
    & & \quad \OPT{\MORE{\N{Production\_Rule}}{}} \\
    & & \T{\}} \\
\end{syntax}

The name of a \T{syntax} declaration should match that
of a \T{syntax} \T{class}.  A \T{syntax} declaration is divided
into three parts: (i) the operator precedence section, (ii) the
\T{expect:} $n$ section, and (iii) the grammar section.

\Subsubsection{Precedence}

Operator precedences are specified using precedence declarations.
The syntax is: 

\begin{syntax}
\NDEF{Precedence\_Decl} 
    & \IS & \TDEF{left:} \N{Integer} \ALT{\N{Operator}}{} \T{;} 
          & left associative operators \\
    & \OR & \TDEF{right:} \N{Integer} \ALT{\N{Operator}}{} \T{;} 
    & right associative operators \\
\NDEF{Operator} & \IS & \N{Char}   & single character operator \\
             & \OR & \N{String} & multi-character operator \\
             & \OR & \N{Cons}   & constructor \\
\end{syntax}

The integer associated with a precedence declaration is the precedence
level of an operator.  The higher the level of an operator, the {\em lower} 
its precedence.

For example, the following set of precedences are used in the \Prop{}
language parser:
\begin{verbatim}
left: 23 "as";
left: 22 "::";
left: 21 "||";
left: 20 "equiv:";
left: 19 "xor:";
left: 18 "implies:";
left: 17 "&&" "and";
right: 16 "|=" "&=" "^=" "<<=" ">>=";
right: 15 '=' ":=" "+=" "-=" "*=" "/=" "%=";
left: 14 '|';
left: 13 ':';
left: 12 ';';
left: 11 '^';
left: 10 '&';
left: 9 "==" "!=";
left: 8 '<' '>' ">=" "<=";
left: 7 "<<" ">>";
left: 6 '+' '-' "with" "less";
left: 5 '*' '/' '%';
left: 4 "++" "--";
left: 3 '!' '~' "arb" "card" "dom" "ran";
left: 2 '[' ']' '{' '}' '.' "->" ;
\end{verbatim}

\Subsubsection{{\tt expect:} $n$}

The declaration \T{expect} $n$ specifies that there should be $n$
shift/reduce errors in the grammar.  If the actual grammar deviates
from this number, a warning will be generated.  The alternative
form \T{expect:} \T{\_} suppresses all warnings.  The syntax is:

\begin{syntax}
\NDEF{Expect\_Decl}    & \IS & \TDEF{expect:} \N{Integer} \T{;}
    & expects $n$ shift/reduce errors \\
                     & \OR & \T{expect:} \T{\_} \T{;}
    & expects many shift/reduce errors \\
\end{syntax}

This declaration is optional.  If omitted, warnings will be printed
if parser conflicts are found during parser generation.

\Subsubsection{Production rules}


The syntax of the production rules is as follows: 

\begin{syntax}
\NDEF{Production\_Rule} & \IS & 
  \Id \OPT{\TypeExp} \T{:} \\
  & & \quad \OPT{\ALT{\N{One\_Alt}}{|}} \T{;} \\
\NDEF{One\_Alt}  & \IS & \OPT{\ALT{\N{Symbol}}{}} & one alternative \\
\NDEF{Symbol} 
    & \IS & \Id        & non-terminal \\
    & \OR & \N{Cons}   & terminal (a datatype constructor) \\
    & \OR & \N{String} & terminal (a datatype constructor) \\
    & \OR & \N{Char}   & terminal (from the ASCII character set) \\
    & \OR & \T{?}      & the error terminal \\
    & \OR & \T{\$}     & the end of file terminal \\
    & \OR & \T{\{} \N{Code} \T{\}} & embedded actions \\
\end{syntax}

Each rule specifies a set of productions of the form 
\[ \begin{array}{lcl}
    lhs &  :  & A_{11} A_{12} \ldots A_{1n_1} \\
        & \OR & A_{21} A_{22} \ldots A_{2n_2} \\
        & \vdots & \\
        & \OR & A_{m1} A_{m2} \ldots A_{mn_m} \\
   \end{array}
\]
where $lhs$ is the non-terminal and $A_{ij}$'s are a mixture of 
terminals, non-terminals and embedded actions.  If synthesized
attributes are used, then the type of the s-attribute should be
annotated next to the lhs non-terminal.

Terminals can be specified in a few ways: (i) a character literal
is a predefined terminal of the same name, (ii) a string literal
is matched with a \T{datatype} or a \T{lexeme} \T{class} constructor
of the same name, (iii) an identifier is a terminal if there is a 
\T{datatype} constructor of the same name.  Any other identifiers
are assumed to be non-terminals.

For example, the following are two sets of production rules used in the 
\Prop{} translator itself:
\begin{verbatim}
ty Ty:  simple_ty               { $$ = $1; }
|       ty '=' exp              { $$ = DEFVALty($1,$3); }
|       ty '*'                  { $$ = mkptrty($1); }
|       ty '&'                  { $$ = mkrefty($1); }
|       ty '[' ']'              { $$ = mkptrty($1); }
|       ty '[' exp ']'          { $$ = mkarrayty($1,$3); }
;
exp Exp:
        app_exp                 { $$ = $1; }
|       exp '+' exp             { $$ = BINOPexp("+",$1,$3); }
|       exp '-' exp             { $$ = BINOPexp("-",$1,$3); }
|       exp '*' exp             { $$ = BINOPexp("*",$1,$3); }
|       exp '/' exp             { $$ = BINOPexp("/",$1,$3); }
|       exp '%' exp             { $$ = BINOPexp("%",$1,$3); }
|       exp '^' exp             { $$ = BINOPexp("^",$1,$3); }
|       exp "+=" exp            { $$ = BINOPexp("+=",$1,$3); }
|       exp "-=" exp            { $$ = BINOPexp("-=",$1,$3); }
|       exp "*=" exp            { $$ = BINOPexp("*=",$1,$3); }
|       exp "/=" exp            { $$ = BINOPexp("/=",$1,$3); }
|       exp "%=" exp            { $$ = BINOPexp("%=",$1,$3); }
\end{verbatim}
\noindent {\em etc $\ldots$}

Here, \T{ty} is the non-terminal of this production.  The type of
the synthesized attribute of this rule is \T{Ty}, and it is
written next to the non-terminal.  Like {\em yacc}, the synthesized
attribute of a rule can be referenced using the \verb.$. variables: 
variable \TDEF{\$\$} denote the synthesized attribute of the current rule,
while \TDEF{\$$n$} where $n$ is 1, 2, 3, \ldots denote the synthesized
attribute of the $n$th rhs symbol of the current rule.

\Subsubsection{Parser report}

If the command line option \OPTIONDEF{-r} is used, then
in addition to generating the code for the grammar, the translator
will also generate a comprehensive report of the grammar and
the resulting LALR(1) automaton.  The amount of details in this
report can be varied using the the verbose option
\OPTIONDEF{-v}:

\begin{description}
   \item[{\tt -v1}]  Print full information for all LR(0) states. 
   \item[{\tt -v2}]  Also print out the lookahead set of a reduction rule.
This may be useful for tracking down shift/reduce and reduce/reduce conflicts.
   \item[{\tt -v3}]  Do both \verb|-v1| and \verb|-v2|.
\end{description}

\Subsubsection{Interfacing with the generated lexer}

The easiest way of interfacing with a lexer is to embed
an object of class \T{LexerBuffer}, \T{IOLexerBuffer}, or \T{IOLexerStack}
within a syntax class\footnote{Alternatively, inheritance may be used.}.
We can then define the tokenization method
\verb|get_token| using the \T{matchscan} statement described
in section~\ref{sec:matchscan}.

\Subsection{Debugging Parsers}

Currently, only the most rudimentary debugging facilities have been
implemented for parser.
The user can define the macro \verb|DEBUG_|$C$, where $C$ is the
name of a syntax class before the \T{syntax} definition.  This will activate
the parser debugging code.  During parsing, the reductions that are taken
will be printed to \verb|cerr|.

\Section{Algebraic Datatypes and Pattern Matching} \label{sec:datatype}

    \Prop{} implements algebraic datatypes and pattern matching in the
style of Standard ML\cite{SML}.  Tree, DAG and even graph structures can be
specified as a set of datatype equations in algebraic datatype specifications.
\Prop{} then proceeds to translate these into concrete \Cpp{} classes.  
This makes it very easy to implement complex data structures.

   Algebraic datatypes can be manipulated and transformed
using \Prop{} pattern matching constructs, which decompose a datatype
value into its constituents.  Complex patterns involving multiple
objects can be specified.  These are translated into efficient 
testing and selection code in \Cpp.

In the following we shall give a brief overview of the pattern matching
features of \Prop.  For most users of modern declarative languages many
of these features are already familiar constructs.
 
\Subsection{A quick tour of pattern matching}
 
   Algebraic datatypes are specified using \verb|datatype| definitions,
which define the inductive structure of one of more types using
a tree-grammar like syntax.  
%In addition, pretty printers,
%lexical scanners, parsers, persistence I/O methods and
%garbage collection inferences can also be specified with
%additional options in the same construct.  
When a datatype is declared,
the following operations are implicitly defined by the datatype compiler:
(1) the constructors for all the variants of a type;
(2) the identity test operator \verb|==|, and the assignment
    operator \verb|=| for this type; and
(3) the member functions needed to decompose a datatype value during
    pattern matching.
 
   We'll select the internals of a compiler for a simplified imperative
language as the running example in this section.
Suppose that in this language an expression is composed of
identifiers, integer constants and the four arithmetic operators.
Then the structure of the abstract syntax tree can be specified as follows:
 
\begin{verbatim}
   datatype Exp = INT (int)
                | ID  (const char *)
                | ADD (Exp, Exp)
                | SUB (Exp, Exp)
                | MUL (Exp, Exp)
                | DIV (Exp, Exp)
                ;
\end{verbatim}

Abstract syntax of an expression such as {\em a * b - 17}
can be constructed directly in a prefix syntax, directly mirroring
that of the definition.  The \Prop{} datatype
compiler will automatically generate a \Cpp{} class hierarchy to represent
the variants of type \verb|Exp|.  Datatype constructor
functions(not to be mistaken with \Cpp's class constructors) will
also be automatically generated using the same names as the variants.
 
\begin{verbatim}
   Exp formula = ADD(MUL(ID("a"),ID("b")),INT(17));
\end{verbatim}
 
Datatype values can be decomposed using the \verb|match| statement, which
can be seen as a generalization of \C's \verb|switch| construct.  Pattern
matching is a combination of conditional branching and value
binding.  For example, a typical evaluation function for the type \verb|Exp|
can be written as in the following example.  Notice that each arm of
a \verb|case| is in fact a pattern(with optional variables)
mirroring the syntax of a datatype.  The pattern variables(written
with the prefix \verb|?| in the sequel) of a matching
arm is {\em bound} to the value of the matching value, which can be
subsequently referenced and modified in the action of an arm.
 
\begin{verbatim}
int eval (Exp e, const map<const char *, int>& env)
{  match (e)
   {  case INT ?i:        return ?i;
      case ID  ?id:       return env[?id];
      case ADD (?e1,?e2): return eval(?e1,env) + eval(?e2,env);
      case SUB (?e1,?e2): return eval(?e1,env) - eval(?e2,env);
      case MUL (?e1,?e2): return eval(?e1,env) * eval(?e2,env);
      case DIV (?e1,?e2): return eval(?e1,env) / eval(?e2,env);
   }
}
\end{verbatim}
 
\Subsubsection{Pattern matching versus object-oriented style}
 
  Although a comparable evaluation function can be written
in object oriented style using late binding, as in below, in general pattern
matching is much more powerful than late binding in \Cpp, which only
allows dispatching based on the type of one receiver.
 
\begin{verbatim}
  // Class definitions
  class Exp {
  public:
     virtual int eval(const map<const char *, int>& env) const = 0;
  };
  class INT : Exp {
     int i;
  public:
     int eval(const map<const char *, int>& env);
  };
  class ID : Exp {
     const char * id
  public:
     int eval(const map<const char *, int>& env);
  };
  ...
 
  // Member functions
  int INT::eval(const map<const char *, int>& env) const { return i; }
  int ID ::eval(const map<const char *, int>& env) const { return id; }
  int ADD::eval(const map<const char *, int>& env) const
     { return e1->eval(env) + e2->eval(env); }
  int SUB::eval(const map<const char *, int>& env) const
     { return e1->eval(env) - e2->eval(env); }
  int MUL::eval(const map<const char *, int>& env) const
     { return e1->eval(env) * e2->eval(env); }
  int DIV::eval(const map<const char *, int>& env) const
     { return e1->eval(env) / e2->eval(env); }
\end{verbatim}

For example, in the following function we use nested patterns,
non-linear patterns (i.e. patterns with multiple occurrences of
a pattern variable), and guards to perform algebraic simplification
of an expression.  Although the patterns are relatively simple in
this example, in general arbitrarily complex patterns may be used.
 
\begin{verbatim}
Exp simplify (Exp redex)
{  // recursive traversal code omitted ...
 
   // match while repeats the matching process
   // until no more matches are found.
   match while (redex)
   {  ADD(INT 0,  ?x):     { redex = ?x; }
   |  ADD(INT ?x, INT ?y): { redex = INT(?x+?y); }
   |  ADD(?x,     INT 0)   { redex = ?x; }
   |  SUB(?x,     INT 0):  { redex = ?x; }
   |  SUB(?x,     ?x):     { redex = INT(0); }
   |  SUB(INT ?x, INT ?y): { redex = INT(?x-?y); }
   |  MUL(INT 0,  ?x):     { redex = INT(0); }
   |  MUL(?x,     INT 0):  { redex = INT(0); }
   |  DIV(?x,     ?x):     { redex = INT(1); }
        // don't divide by zero.
   |  DIV(INT ?x, INT ?y) | ?y != 0: { redex = INT(?x/?y); }
   |  ...
   }
   return redex;
}
\end{verbatim}
 
Pattern matching in \Prop{} is not restricted to one datatype at
a time.  In the following example, we use matching on multiple values to
define equality on expressions inductively.  For variety, we'll
use the \verb|fun| variant of \verb|match|, which defines a function
in rule form.  Notice that the last case
of the match set uses {\em wild cards} \verb|_| to catch all the other
non-equal combinations.  Since \Cpp{} does not provide multiple dispatching,
implementing binary (or $n$-ary) matching operations on variant datatypes are
in general cumbersome and verbose in object-oriented style. In contrast,
using an applicative pattern matching style many manipulations and
transformations on variant datatypes with tree-like or graph-like structure
can be expressed succinctly.
 
\begin{verbatim}
fun equal INT ?i,     INT ?j: bool: { return ?i == ?j; }
  | equal ID  ?a,     ID  ?b:       { return strcmp(a,b) == 0; }
  | equal ADD(?a,?b), ADD(?c,?d):   { return equal(?a,?c) && equal(?b,?d); }
  | equal SUB(?a,?b), SUB(?c,?d):   { return equal(?a,?c) && equal(?b,?d); }
  | equal MUL(?a,?b), MUL(?c,?d):   { return equal(?a,?c) && equal(?b,?d); }
  | equal DIV(?a,?b), DIV(?c,?d):   { return equal(?a,?c) && equal(?b,?d); }
  | equal _,          _:            { return false; }
  ;
\end{verbatim}

\Subsubsection{More examples} \label{sec:Wff}
 
  As another example, we can specify the term structure of
{\em well-formed formulas} in proposition calculus as follows.  Notice
that the constructors \verb|F| and \verb|T| are nullary.
 
\begin{verbatim}
   datatype Wff = F
                | T
                | Var     (const char *)
                | And     (Wff, Wff)
                | Or      (Wff, Wff)
                | Not     (Wff)
                | Implies (Wff, Wff)
                ;
\end{verbatim}
 
Datatypes that are parametrically polymorphic, such as lists and trees, can
be defined by parameterizing them with respect to one of more types.
For example, both lists and tree below are parametric on one type argument
\verb|T|.
 
\begin{verbatim}
   datatype List<T> = nil
                    | cons(T, List<T>);
   datatype Tree<T> = empty
                    | leaf(T)
                    | node(Tree<T>, T, Tree<T>);
 
   List<int> primes = cons(2,cons(3,cons(5,cons(7,nil))));
   List<int> more_primes = cons(11,cons(13,primes));
   Tree<char *> names = node(leaf("Church"),"Godel",empty);
\end{verbatim}
 
As a programming convenience, \Prop{} has a set of built-in list-like
syntactic forms.  Unlike languages such as ML, however,
these forms are not predefined to be any specific list types.  Instead, it is
possible for the user to use these forms on any datatypes with
a natural binary {\em cons} operator and a nullary {\em nil} constructor.
For instance, the previous list datatype can be redefined as follows:
 
\begin{verbatim}
   datatype List<T> = #[] | #[ T ... List<T> ];
   List<int> primes = #[ 2, 3, 5, 7 ];
   List<int> more_primes = #[ 11, 13 ... primes ];
   List<char *> names = #[ "Church", "Godel", "Turing", "Curry" ];
 
   template <class T>
      List<T> append (List<T> a, List<T> b)
      {  match (a)
         {  case #[]:          return b;
            case #[hd ... tl]: return #[hd ... append(tl,b)];
         }
      }
\end{verbatim}
The empty list is written as \verb|#[]|, while {\em cons(a,b)}
is written as \verb|#[ a ... b ]|.   An expression of the special form
\verb|#[a, b, c]|, for instance, is simple syntactic sugar for
repeated application of the {\em cons} operator, i.e.
 
\begin{verbatim}
    #[a, b, c] == #[a ... #[ b ... #[ c ... #[] ] ] ].
    #[a, b, c ... d ] == #[a ... #[ b ... #[ c ... d ] ] ].
\end{verbatim}
 
List-like special forms are not limited to datatypes with only two variants.
For example, we can define a datatype similar in structure to S-expressions
in {\em Lisp} or {\em scheme}.  Here's how such as datatype may be
defined\footnote{for simplicity, we'll use a string representation for atoms
instead of a more efficient method.}:
 
\begin{verbatim}
   datatype Sexpr = INT    (int)
                  | REAL   (double)
                  | STRING (char *)
                  | ATOM   (const char *)
                  | #()
                  | #( Sexpr ... Sexpr )
   where type Atom = Sexpr  // synonym for Sexpr
   ;
\end{verbatim}
 
   With this datatype specification in place, we can construct values
of type \verb|Sexpr| in a syntax close to that of {\em Lisp}.  For example,
we can define lambda expressions corresponding to the combinators $I$, $K$
and $S$ as follows:
 
\begin{verbatim}
   Atom LAMBDA = ATOM("LAMBDA");
   Atom f      = ATOM("f");
   Atom x      = ATOM("x");
   Atom y      = ATOM("y");
   Atom NIL    = #();
 
   Sexpr I = #(LAMBDA, #(x), x);
   Sepxr K = #(LAMBDA, #(x), #(LAMBDA, #(y), x));
   Sepxr S = #(LAMBDA, #(f),
                #(LAMBDA, #(x),
                   #(LAMBDA, #(y), #(#(f,x), #(g,x)))));
\end{verbatim}
 
Similar to list-like forms, vector-like forms are also available.
This addresses one of the flaws of the \Cpp{} language, which lacks
first class array constructors.  
Vectors are simply homogeneous arrays whose sizes
are fixed and are determined at creation time\footnote{For efficiency,
bounds checking is {\em not} performed.}.   Random access within
vectors can be done in constant time.  Unlike lists, however, the
prepending operation is not supported.   Vectors literals are delimited
with the composite brackets \verb!(| ... |)!, \verb![| ... |]!, or
\verb!{| ... |}!.  In the following example the datatype \verb|Exp|
uses vectors to represent the cooefficients of the polynomials:
 
\begin{verbatim}
   datatype Vector<T> = (| T |);
   datatype Exp = Polynomial (Var, Vector<int>)
                | Functor (Exp, Vector<Exp>)
                | Atom (Var)
                | ...
   where type Var = const char *;
   Exp formula = Polynomial("X", (| 1, 2, 3 |));
\end{verbatim}
 
Commonly used patterns can be given synonyms so that they can be readily
reused without undue repetition.  This can be accomplished by defining pseudo
datatype constructors to stand for common patterns using
{\em datatype law} definitions.  For example, the following set of
laws define some commonly used special forms for a {\em Lisp}-like language
using the previously defined \verb|Sexpr| datatype.
 
\begin{verbatim}
   datatype law inline Lambda(x,e)  = #(ATOM "LAMBDA", #(x), e)
              | inline Quote(x)     = #(ATOM "QUOTE", x)
              | inline If(a,b,c)    = #(ATOM "IF", a, b, c)
              | inline Nil          = #()
              | inline ProgN(exprs) = #(ATOM "PROGN" ... exprs)
              | SpecialForm  = #(ATOM ("LAMBDA" || "IF" ||
                                       "PROGN" || "QUOTE") ... _)
              | Call(f,args) = ! SpecialForm && #(f ... args)
              ;
\end{verbatim}
 
Note that the pattern \verb|SpecialForm| is meant to match all
special forms in our toy language: i.e. {\em lambdas}, {\em ifs}, {\em progn}'s
and {\em quotes}.  The pattern disjunction connective \verb.||. is used
to link these forms together.  Since we'd like the \verb|Call| pattern
to match only if the S-expression is not a special form, we use
the pattern negation and conjunction operators, \verb|!| and \verb|&&|
are used to screen out special forms.  


With these definitions in place,
an interpreter for our language can be written thus:
 
\begin{verbatim}
   Sexpr eval (Sexpr e)
   {  match (e)
      {  Call(?f,?args):     { /* perform function call */ }
      |  Lambda(?x,?e):      { /* construct closure */ }
      |  If(?e,?then,?else): { /* branching */ }
      |  Quote(?x):          { return ?x; }
      |  ...:                { /* others */ }
      }
   }
\end{verbatim}


As an interesting note, the special form pattern can also be rewritten
using regular expression string matching, as in the following:
 
\begin{verbatim}
 datatype law SpecialForm = #(ATOM /LAMBDA|IF|PROGN|QUOTE/ ... _)
\end{verbatim}
 
In addition, since the law constructors \verb|Lambda|, \verb|Quote|,
\verb|If|, \verb|Nil| and \verb|ProgN| are defined with the \verb|inline|
keyword, these constructors can be used to within expressions as abbreviates
for their rhs definitions.  For example, writing
\verb|Lambda(x,x)| is the same writing \verb|#(ATOM("Lambda"),#(x),x)|.

\Subsubsection{Variants of match}
 
Aside from the usual plain pattern matching constructs, a few variants of the
\verb|match| construct are offered.  We'll briefly enumerate a few of these:
\begin{itemize}
  \item The \verb|matchall| construct is a variant of \verb|match| that
   executes all matching rules(instead of just the first one) in sequence.
  \item Each rule of a \verb|match| statement can have associated cost
    expressions.  Instead of selecting the first matching rule to
    execute as in the default, all matching rules are considered
    and the rule with the least cost is executed.  Ties are broken by
    choosing the rule that comes first lexically.  For example:
 
   \begin{verbatim}
   match (ir)
   {  ADD(LOAD(?r0),?r1) \ e1: { ... }
   |  ADD(?r0,LOAD(?r0)) \ e2: { ... }
   |  ADD(?r0, ?r1)      \ e3: { ... }
   |  ...
   }
   \end{verbatim}
  \item A \verb|match| construct can be modified with the \verb|while|
     modifier, as in the following example.  A match modified thus is
     repeatedly matched until none of the patterns are applicable.
     For example, the following routine uses a \verb|match while| statement
     to traverse to the leftmost leaf.
 
\begin{verbatim}
   template <class T>
   Tree<T> left_most_leaf(Tree<T> tree)
   {  match while (tree)
      {  case node(t,_,_):  tree = t;
      }
      return tree;
   }
\end{verbatim}
  \item Finally, the \verb|matchscan| variant of match can be used to
     perform string matching on a stream.  For example, a simple lexical
     scanner can be written as follows.
 
     \begin{verbatim}
   int lexer (LexerBuffer& in)
   {  matchscan while (in)
      {  /if/:                     { return IF; }
      |  /else/:                   { return ELSE; }
      |  /while/:                  { return WHILE; }
      |  /for/:                    { return FOR; }
      |  /break/:                  { return BREAK; }
      |  /[0-9]+/:                 { return INTEGER; }
      |  /[a-zA-Z_][a-zA-Z_0-9]*/: { return IDENTIFIER; }
      |  /[ \t]+/:                 { /* skip spaces */ }
      |  ... // other rules
      }
   }
     \end{verbatim}
\end{itemize}

\Subsection{Algebraic Datatypes} \label{sec:algebraic-datatypes}
   Algebraic datatypes are defined using the \T{datatype}
construct.  Its syntax is as follows.

\begin{syntax}
\NDEF{Datatype\_Decl} 
   & \IS & \TDEF{datatype} \\
   &     & \quad \OPT{\ALT{\N{Datatype\_Spec}}{and}} \\
   &     & \quad \OPT{\TDEF{where type} \ALT{\N{Type\_Spec}}{and}} \\
   &     & \quad \OPT{\TDEF{law} \ALT{\N{Law\_Spec}}{and}} \\
   &     & \T{;} \\
\end{syntax}

Each \T{datatype} declaration specifies a set of 
(potentially mutually recursive) types, a set of type abbreviations,
and a set of pattern matching abbreviations called {\em laws}.

\begin{syntax}
\NDEF{Datatype\_Spec} 
   & \IS & \Id \OPT{\T{<} \LIST{\N{Id}} \T{>}} \\
   &     & \quad \OPT{\T{:} \N{Inherit\_List}} \\
   &     & \quad \OPT{\T{::} \N{Datatype\_Qualifiers}} \\
   &     & \quad \OPT{\T{=} \N{Cons\_Specs}} \\
\NDEF{Cons\_Specs} 
   & \IS & \ALT{\N{Cons\_Spec}}{|} \\
   &     & \quad \OPT{\T{public:} \N{Datatype\_Body}} \\
\NDEF{Cons\_Spec} 
   & \IS & \N{Simple\_Cons\_Spec}  \\
   &     &  \quad \OPT{\T{with} \T{\{} \N{Class\_Decl} \T{\}}} \\
\NDEF{Simple\_Cons\_Spec} 
   & \IS & \Id \OPT{\N{String}} & unit constructor \\
   & \OR & \N{String}              & unit constructor \\
   & \OR & \Id \OPT{\T{of}} \TypeExp  & constructor with arguments \\
   & \OR & \T{\#[} \T{]}                 & unit list constructor \\
   & \OR & \T{\#\{} \T{\}}               & unit list constructor \\
   & \OR & \T{\#(} \T{)}                 & unit list constructor \\
   & \OR & \T{\#[} \TypeExp \T{...} \TypeExp \T{]}   
      & list constructor \\
   & \OR & \T{\#\{} \TypeExp \T{...} \TypeExp \T{\}} 
      & list constructor \\
   & \OR & \T{\#(} \TypeExp \T{...} \TypeExp \T{)}   
      & list constructor \\
   & \OR & \T{[|} \TypeExp \T{|]} & vector constructor \\
   & \OR & \T{(|} \TypeExp \T{|)} & vector constructor \\
   & \OR & \T{\{|} \TypeExp \T{|\}} & vector constructor \\
\end{syntax} 


Datatypes can be annotated with {\em qualifiers}.  They tell the \Prop{}
translator to generate additional code for each datatype to provide
additional functionality. 

\begin{syntax}
\NDEF{Datatype\_Qualifiers} 
   & \IS & \MORE{\N{Datatype\_Qualifier}}{} \\
\NDEF{Datatype\_Qualifier} 
   & \IS & \TDEF{collectable} & garbage collectable \\
   & \OR & \TDEF{rewrite}     & optimized for tree rewriting \\
   & \OR & \TDEF{persistent}  & generate persistence interface \\
   & \OR & \TDEF{lexeme}      & used for parser/lexer tokens  \\
   & \OR & \TDEF{inline}      & use inlined implementation \\
   & \OR & \TDEF{extern}      & use non-inlined implementation \\
\end{syntax}

Type specifications assign abbreviations to commonly used type expressions. 
They act like \T{typedef}'s in \Cpp, except that \Prop{} can make use
of the type information provided by these specifications.

\begin{syntax}
\NDEF{Type\_Spec} 
   & \IS & \Id \T{=} \TypeExp \\
\end{syntax}

Law specifications are used to define abbreviations to datatype 
patterns\footnote{Patterns are discussed in 
section~\ref{sec:pattern-matching}.}.
They behave like macros in some sense, except that unlike macros, they are
properly type checked by \Prop.  In addition, if the keyword \T{inline} is 
used, then \Prop{} will treat the lhs as an expression and generate 
a function of the same name that can be used to construct the datatype term.

\begin{syntax}
\NDEF{Law\_Spec} 
   & \IS & \OPT{\T{inline}} \Id \OPT{\N{Law\_Arg}} \T{=} \Pat \\
\NDEF{Law\_Arg} 
   & \IS & \Id \\
   & \OR & \T{(} \LIST{\N{Id}} \T{)} \\
\end{syntax}

\Prop{} recognizes the following set of type expressions.  
\begin{syntax}
\NDEF{Type\_Exp}
   & \IS & \N{Type\_Id}                & type name \\
   & \OR & \TypeExp \T{*}              & pointer type \\
   & \OR & \TypeExp \T{\&}             & reference type \\
   & \OR & \N{Type\_Qualifier} \N{Type\_Id}      & qualified type \\
   & \OR & \N{Datatype\_Qualifier} \TypeExp       & annotation  \\
   & \OR & \T{(} \TypeExp \T{)}          & grouping \\
   & \OR & \T{(} \TypeExp \T{,} 
      \LIST{\N{TypeExp}} \T{)} & tuple type \\
   & \OR & \T{(} \TypeExp \T{,} 
      \LIST{\N{TypeExp}} \T{)} & tuple class \\
   & \OR & \T{\{} \LIST{\N{Lab\_Type\_Exp}} \T{\}} & record type \\
   & \OR & \TypeExp \T{=} \N{Exp} & default value \\
\NDEF{Lab\_Type\_Exp} & \IS &
   \Id \T{:} \TypeExp \\
\NDEF{Type\_Qualifier} 
   & \IS & \T{class}       & a class type \\
   & \OR & \T{const}       & constant type \\
   & \OR & \T{rewrite}     & rewritable \\
   & \OR & \T{collectable} & garbage collected \\
   & \OR & \T{persistent}  & persistent object  \\
\end{syntax}

In general, four different types of datatype constructors can be defined:
\begin{description}
 \item[nullary constructors]  These are constructors without an argument.
 \Prop{} maps these into small integers and no heap space will be consumed.
 \item[tuple argument constructors] These are constructors with one
 or more unamed arguments.  
 \item[record argument constructors] These are constructors with one
 or more named arguments written within braces.  For example, in the
 following datatype definition a record argument constructor called
 \verb|Add| is defined:
 \begin{verbatim}
  datatype Exp = Add { left : Exp, right : Exp }
	       | ...
 \end{verbatim}
  To construct an \verb|Add| expression, we can use any one of the following
  forms:
  \begin{verbatim}
     Exp e1 = Add(x,y);
     Exp e2 = Add'{ left = x, right = y };
     Exp e3 = Add'{ right = y, left = x };
  \end{verbatim}
  Note that we are allowed to rearrange the labels in any order if
  the record expression form is used.
 \item[empty constructors] These are constructors written with
 an empty argument \verb|()|.  For example, in the following definition
 an empty constructor called \verb|NONE| is defined:
 \begin{verbatim}
    datatype Exp : public AST
	= NONE ()
	| NONE2
        | ...
 \end{verbatim}
 Here, the constructor \verb|NONE| will inherit from the class \verb|AST|,
 while the nullay constructor \verb|NONE2| will not.
\end{description}

\Subsubsection{Instantiating a datatype}

Each declared datatype should be {\em instantiated} in some implementation
file using the \TDEF{instantiate datatype} statement.  The translator
will generate additional code to implement the datatype.  The general
syntax is:
\begin{syntax}
 \NDEF{Instantiate\_Decl} 
    & \IS &
       \TDEF{instantiate datatype} \LIST{\N{Type\_Spec}} \\
    & \OR & 
       \TDEF{instantiate} \T{extern} \T{datatype} \LIST{\N{Type\_Spec}} \\
 \NDEF{Type\_Spec} & \IS & \Id \OPT{ \T{<} \LIST{\N{TypeExp}} \T{>}} \\
\end{syntax}

\begin{Tips}
   When a datatype is defined with the \TDEF{inline} qualifier,
\Prop{} will try to generate inline methods whenever possible.
On the other hand, if the \TDEF{extern} qualifier is used, \Prop{}
will try to generate an non-inline implementation, and only one
copy are generated at where the \T{instantiate datatype} statement occurs.

Since most \Prop{} datatypes are light-weight objects, \T{inline} is taken
as the default.  The user can override this default using
\OPTIONDEF{-save\_space} option of the translator 
(see page~\pageref{option:save-space}).
\end{Tips}

\begin{Tips}
   When the option \OPTIONDEF{-fno-implicit-templates} is used
under \Gpp, the user should \verb|#define| the symbol
\INDEX{{\tt PROP\_EXPLICIT\_TEMPLATE\_INSTANTIATION}} before
an \T{instantiate datatype} statement.   
This will enable the code for explicit template instantiation.
\end{Tips}

\Subsection{Pattern Matching} \label{sec:pattern-matching}
   The basic pattern matching construct is the \T{match} statement.    
A match statement can be seen as a generalization of \Cpp's \T{switch}
statement, in which the expression that follows a \T{case} can be
a general complex pattern.  

   Its basic syntax is as follows:
\begin{syntax}
\NDEF{Match} & \IS & 
   \N{Match\_Mode} \ALT{\T{(} \N{Exp} \T{)}}{and}  \\
        & & \T{\{} \MORE{\T{case} \N{Match\_Rule}}{} \T{\}} & variant 1 \\
          & \OR & 
   \N{Match\_Mode} \ALT{\T{(} \N{Exp} \T{)}}{and} \\
        & & \T{\{} \ALT{\N{Match\_Rule}}{|} \T{\}} & variant 2 \\
          & \OR &
   \N{Match\_Mode} \ALT{\T{(} \N{Exp} \T{)}}{and} \T{of} & variant 3 \\
        & & \quad \ALT{\N{Match\_Rule}}{|} \\
        & & \T{end} \T{match} \T{;} \\
\NDEF{Match\_Mode} & \IS & \TDEF{match} \OPT{\T{while}} \\
                & \OR & \TDEF{matchall} \OPT{\T{while}} \\
\NDEF{Match\_Rule} & \IS & \Pat \OPT{\N{Guard}} \OPT{\N{Cost}}
        \T{:} \N{Match\_Action} \\
\NDEF{Guard}       & \IS & \T{if} \N{Exp} \\
                & \OR & \T{where} \N{Exp} \\
                & \OR & \T{|} \N{Exp} \\
\NDEF{Cost} & \IS & \verb|\| \Exp & cost expression \\
\NDEF{Match\_Action} & \IS & \T{\{} \N{Code} \T{\}} \\
                  & \OR & \N{Code} & for variant 1 only \\
\end{syntax}

The 3 syntactic variants of the match statement are equivalent in meaning.
A \T{match} statement takes a list of match objects and a
set of pattern matching rules.  The first rule that matches the pattern
completely will be executed.  Notice that guards can be placed on each rule,
and a rule is only applicable if the guards evaluate to true. 

\paragraph{Cost minimization}
In addition, if cost expressions are specified, then the matching process
will proceed as follows: 
\begin{enumerate}
   \item Locate all applicable rules.  If no such rule exists then exit.
   \item Compute the cost expressions of all applicable rules.
   \item Choose the action that has the lowest cost to execute.  If
a tie occurs, resolve by the textual order of the rules. 
\end{enumerate}

\paragraph{Match all rules}
The \T{matchall} statement is a variant of \T{match}.  In the matchall
mode, the actions of {\em all} applicable rules are executed in the
textual order.  Caution: since the matching variable binding 
process occurs {\em before} the rules are executed, the rhs actions
should not alter the matched value or the pattern variables may
be bound to the wrong value.

\paragraph{Repetition} 
In addition, both \T{match} and \T{matchall} can be annotated with the
\TDEF{while} modifier.  It this case the semantics of the matching
construct is to repeat until no more rule is applicable.  For example,
the following code fragment returns the last element of a list
\begin{verbatim}
datatype List = #[] | #[ int ... List ];

int last (List l)
{  match while (l)
   { case #[]:         assert(0); // an error
     case #[i]:        return i;
     case #[h ... t]:  l = t;     
   }
}
\end{verbatim}

\paragraph{Pattern syntax}
The basic form of a pattern is a literal, which
that matches only a single integer, real number, string, etc.  Complex
patterns can be constructed inductively
using datatype constructors, tuples, lists, and 
{\em logical pattern connectives}.

The syntax of the patterns is as follows:
\begin{syntax}
\NDEF{Pat} & 
   \IS & \N{Integer} & matches integer literal \\
 & \OR & \N{Real}    & matches real literal \\
 & \OR & \N{Char}    & matches character literal \\
 & \OR & \N{Quark}   & matches quark literal \\
 & \OR & \N{String}  & matches string literal \\
 & \OR & \N{Regexp}  & matches any string that matches {\em regexp}\\
 & \OR & \N{Pat\_Var}  & a pattern variable; matches anything \\
 & \OR & \T{\_}      & wild card; matches anything \\
 & \OR & \N{Cons}     & matches a datatype constructor  \\
 & \OR & \N{Cons} \N{PatArg}& a constructor application pattern \\
 & \OR & \Id \T{as} \Pat & binds a variable to the sub-pattern \\
 & \OR & \Pat \T{:} \TypeExp & typed pattern \\
 & \OR & \Pat \T{||} \Pat  & matches either pattern \\
 & \OR & \Pat \T{\&\&} \Pat  & matches both patterns \\
 & \OR & \T{!} \Pat             & matches anything but the pattern \\
 & \OR & \T{(} \Pat \T{)}       & Grouping \\
 & \OR & \T{\#[} \LIST{\N{Pat}} \T{]} & list pattern; exact length match \\
 & \OR & \T{\#[} \LIST{\N{Pat}} \T{...} \T{]} & list pattern;   
     non-exact length match \\
 & \OR & \T{[|} \LIST{\N{Pat}} \T{|]} & vector pattern; exact length match \\
 & \OR & \T{[|} \LIST{\N{Pat}} \T{...} \T{|]} 
    & vector pattern; matches to the left \\
 & \OR & \T{[|} \T{...} \LIST{\N{Pat}} \T{|]} 
    & vector pattern; matches to the right \\
\NDEF{PatArg} & \IS & \Pat \\
           & \OR & \T{(} \LIST{\N{Pat}} \T{)} \\
           & \OR & \T{\{} \LIST{\N{Lab\_Pat}} \OPT{\T{...}} \T{\}} \\ 
\NDEF{Lab\_Pat} & \IS & \Id \T{=} \Pat \\
\NDEF{Pat\_Var} & \IS & \Id       \\
            & \OR & \T{?}\Id  \\
\end{syntax}

\Subsection{Refining a datatype}

   A datatype definition can be refined in a few ways:

\begin{enumerate}
  \item{\bf inheritance} A datatype can be declared to be inherited from
one or more classes $C_1, C_2, \ldots, C_n$. 
The effect is that all variants of a datatype will be inherited from the 
classes $C_1, C_2, \ldots C_2$.  

  \item{\bf member functions}  Member functions and addition attributes
can be attached to the datatype variants using the keyword \TDEF{with}.
\end{enumerate}

For example, the following datatype \T{Exp} is inherited from some user
defined classes \verb|AST| and \verb|Object|.   Furthermore, a virtual
member function called print is defined.  

\begin{verbatim}
datatype Exp : public AST, virtual public Object
   = INT int        
         with { ostream& print(ostream&); }
   | PLUS  Exp, Exp 
         with { ostream& print(ostream&); }
   | MINUS Exp, Exp 
         with { ostream& print(ostream&); }
   | MULT  Exp, Exp 
         with { ostream& print(ostream&); }
   | DIV   Exp, Exp 
         with { ostream& print(ostream&); }
   | VAR { name : const char *, typ : Type }
         with { ostream& print(ostream&); }
   public:
   {  virtual ostream& print (ostream&) = 0;
      void * operator new (size_t);
   }
;
\end{verbatim}

The allocator and printing methods can be defined as follows:
\begin{verbatim}
void * classof Exp::operator new(size_t) { ... }
ostream& classof INT::print(ostream& s)   { return s << INT; }
ostream& classof PLUS::print(ostream& s)  
   { return s << '(' << this->#1 << '+' << this->#2 <<')'; }
ostream& classof MINUS::print(ostream& s) 
   { return s << '(' << this->#1 << '-' << this->#2 <<')'; }
ostream& classof MULT::print(ostream& s)  
   { return s << '(' << this->#1 << '*' << this->#2 <<')'; }
ostream& classof DIV::print(ostream& s)  
   { return s << '(' << this->#1 << '/' << this->#2 <<')'; }
ostream& classof VAR::print(ostream& s)   { return s << name; }
\end{verbatim}

The special type form \TDEF{classof} {\em con} is used to reference
the type of a constructor.  Arguments of a constructor uses the following
naming convention: (i) if the constructor $C$ takes only one argument,
then the name of the argument is $C$; (ii) if $C$ takes two or more
unnamed arguments, then these are named \verb|#1|, \verb|#2|, etc;
(iii) finally, labeled arguments are given the same name as the label.

For readability reasons, it is often useful to separate a datatype definition
from the member functions and inheritance definitions.   The \TDEF{refine}
declaration can be used in this situation.  For example, the above
example can be restated more succinctly as follows:

\begin{verbatim}
//
// Datatype definition section
//
datatype Exp = INT int        
             | PLUS  Exp, Exp 
             | MINUS Exp, Exp 
             | MULT  Exp, Exp 
             | DIV   Exp, Exp 
             | VAR { name : const char *, typ : Type }
;

//
// Refinement section
//
refine Exp : public AST, virtual public Object
       { virtual ostream& print (ostream&) = 0;
         void * operator new (size_t);
       }
and    INT, PLUS, MINUS, MULT, DIV, VAR
       { ostream& print(ostream&); 
       }
;
\end{verbatim}

The general syntax of the \TDEF{refine} declaration is as follows:
\begin{syntax}
\NDEF{Refine\_Decl} 
   & \IS & \T{refine} \ALT{\N{Refine\_Spec}}{\T{and}} \T{;} \\
\NDEF{Refine\_Spec} 
   & \IS & \LIST{\N{Id}} \\
   &     & \quad \OPT{\T{:} \N{Inherit\_List}} \\
   &     & \quad \OPT{\T{::} \N{Datatype\_Qualifiers}} \\
   &     & \quad \OPT{\T{\{} \N{Datatype\_Body} \T{\}}} \\
\end{syntax}
Here, \N{Id} refers to either a constructor name or a datatype name.

\Subsection{Memory management}

   By default, a datatype constructor calls the \verb|operator new| to
allocated memory.  There are a few ways to change this behavior: (i)
redefine the \verb|operator new| within the datatype using refinement; or
(ii) inherit from a class that redefines this method.

   If a {\em placement} \verb|operator new| is defined in datatype,
the placement constructor syntax can be used to invoke this
operator:
\begin{syntax}
   \NDEF{Placement\_Cons} 
   & \IS & \N{Cons} \T{'(} \LIST{\N{Exp}} \T{)} \T{(} 
   \OPT{\LIST{\N{Exp}}} \T{)} \\
   & \OR & \N{Cons} \T{'(} \LIST{\N{Exp}} \T{)} \T{\{} 
   \OPT{\LIST{{\N{Id} = \N{Exp}}}}
     \T{\}} \\
\end{syntax}

For example, in the following the extra parameter \verb|mem| will
be passed to placement \verb|new| operator.
\begin{verbatim}
  Exp e1 = INT'(mem)(1);
  Exp e2 = VAR'(mem){ name = "foo", typ = t };
\end{verbatim}

\Subsubsection{Garbage collection}

  The support library contains an implementation\footnote{The implementation
has only been tested on Linux, SunOS and Solaris.  Please contact the
author for details
if you'd like to port the collector to your particular platform.} of a 
conservative garbage collector using two algorithms: one using
a mostly copying scheme\cite{Mostly-copying,Gen-mostly-copying} 
and one using mark-sweep.

  A datatype can be defined to be garbage collectable with the
qualifier \TDEF{collectable}.  In the following example, the datatype
\T{Exp} is will be allocated from the a garbage collected heap instead
of the default heap.  Since it references another 
user defined garbage collectable object
\verb|SymbolTable|, the pointer to that symbol table is also marked
as \T{collectable}.
\begin{verbatim}
   // SymbolTable is collectable
   class SymbolTable: public GCObject
   {
   public:
      class Id : public GCObject { ... };
      ...
   };

   datatype Exp :: collectable
      = INT int
      | VAR (collectable SymbolTable::Id, collectable SymbolTable *)
      | ADD Exp, Exp
      | SUB Exp, Exp
      | MUL Exp, Exp
      | DIV Exp, Exp
      ;
\end{verbatim}

The corresponding \TDEF{instantiate datatype} statement for \T{Exp}
will generate the appropriate tracing methods for garbage collection.

Please see appendix~\ref{appendix:gc} for more details concerning
garbage collection.

\Subsubsection{Persistence}

  Datatypes can be interfaced with the persistent object library
supplied with the translator using the \TDEF{persistent} qualifier.
A persistent object defines the protocol for serializing its representation
into an architecture independent format on a persistence stream.  

For example, the following datatype declaration defines a persistent 
type \verb|Exp|.  Note that a pointer to the symbol table
is annotated with the \T{persisent} qualifier to signify that the object
associated with the pointer should be traversed during the serialization
phase. 
\begin{verbatim}
   // SymbolTable is persistent
   class SymbolTable: public PObject
   {
   public:
      class Id : public PObject { ... };
      ...
   };

   datatype Exp :: persistent
      = INT int
      | VAR (persistent SymbolTable::Id, persistent SymbolTable *)
      | ADD Exp, Exp
      | SUB Exp, Exp
      | MUL Exp, Exp
      | DIV Exp, Exp
      ;
\end{verbatim}

A {persistent object id} {\em must} be provided with each distinct
datatype.  Object ids are used to identify the classes
of persistent objects.  Thus they must be unique within an application.
Each persistent type tag is simply a string, and can be defined using
the \TDEF{refine persistent} statement.  

For example, the following statement declares the persistent object
id for type \verb|Exp| above to be \verb|"Simple expressions"|.
\begin{verbatim}
   refine persistent Exp => "Simple expressions";
\end{verbatim}

The corresponding \TDEF{instantiate datatype} statement will generate
the appropriate methods to communicate with a persistent stream.
\begin{verbatim}
  instantiate datatype Exp;
\end{verbatim}

To write an object $e$ to the datafile \verb|"foo"|, we can say:
\begin{verbatim}
   #include <AD/persist/postream.h>
   ofstream out("foo");
   Postream pout(out);
   pout << e;
   out.close();
\end{verbatim}

To read the object back from the file, we can say:
\begin{verbatim} 
   #include <AD/persist/pistream.h>
   EXP e;
   ifstream in("foo");
   Pistream pin(in);
   e = (Exp)read_object(pin);
   in.close();
\end{verbatim}

For more details concerning persistent streams and persist objects, 
please see directory \verb|<AD/persist>| for details.

\Section{Inference} \label{sec:inference}

   Semantic processing
in compilers and other language processors, such as data flow analysis,
can frequently be specified as in a rule-based,
logical deductive style.  In \Prop, deductive inference using forward
chaining\footnote{
The current
implementation of {\sf Prop} translates inference into very naive code.
This feature will be phased out and replaced by
the more general and much faster graph rewriting mechanism.  
}
is provided as a built-in mechanism, orthogonal
to pattern matching and rewriting, for manipulating user-defined
algebraic datatypes.
 
Similar to syntax classes, {\bf inference classes} may be used
for data encapsulation.  An inference class is a combination of a \Cpp{}
class, a database of inference relations, and a collection of inference rules
of the form {\em lhs \verb|->| rhs}.  The lhs of an inference rule is a set
of patterns in conjunctive form.  During the inference process, a rule is
fired when its lhs condition is satisfied.  A fired rule then executes
the corresponding rhs action, which may assert or retract additional relations
from the database.   Using multiple inheritance, it is possible to combine
a rewriting class with an inference class such that the rewriting process
generates new relations to drive the inference process, or vice versa.

\Subsection{An Example} 
Datatype relations are not a distinct kind of data structure but are in fact
simply algebraic datatypes declared to have a \TDEF{relation} 
attribute.  For example, in
the following definition three relation types \verb|Person|, \verb|Parent|
and \verb|Gen| are defined.
 
\begin{verbatim}
   datatype Person :: relation = person (const char *)
       and  Parent :: relation = parent (const char *, const char *)
       and  Gen    :: relation = same_generation (const char *, const char *);
 
   instantiate datatype Person, Parent, Gen;
\end{verbatim}
 
Using these relations we can define an inference class that computes whether
two persons are in the same generation.   Nine axioms are defined (i.e.
those whose lhs are empty) in the following.  The two inference rules
state that (1) a person is in the same generation as him/herself, 
and (2) two persons are in the same generation if 
their parents are in the same generation.
 
\begin{verbatim}
   inference class SameGeneration {};
 
   inference SameGeneration
   {  -> person("p1") and person("p2") and
         person("p3") and person("p4") and
         person("p5");
 
      -> parent("p1", "p2") and
         parent("p1", "p3") and
         parent("p2", "p4") and
         parent("p3", "P5");
 
      person ?p -> same_generation (?p, ?p);
 
      parent (?x, ?y) and parent (?z, ?w) and same_generation (?x, ?z)
      -> same_generation(?y, ?w);
   };
\end{verbatim}
 
In general, datatypes qualified as \verb|relation|s will inherit
from the base class \verb|Fact|, while a rewrite class definition
implicitly defines two member functions used to assert and retract facts
in the internal database.  For example, in the above example, the following
protocol will be automatically generated by the inference compiler.
 
\begin{verbatim}
   class SameGeneration : ...
   {
   public:
       virtual Rete&    infer       ();       // start the inference process
       virtual ReteNet& operator += (Fact *); // assert fact
       virtual ReteNet& operator -= (Fact *); // retract fact
   };
\end{verbatim}
 

 
Using these methods, an application can insert or remove relations
from an inference class.  This will in turn trigger any attached inference
rules of the class.
 
\Subsubsection{Another example}
 
Consider the following example, which is used to compute Pythagorean
triangles.  Only one axiom and two rules are used.  The axiom and the
first rule are used to assert the relations \verb|num(1)| to \verb|num(n)|
into the database, where \verb|n| is limited by the term \verb|limit(n)|.
The second inference rule is responsible for printing out only
the appropriate combinations of numbers.
 
\begin{verbatim}
   datatype Number :: relation = num int | limit int;
 
   inference class Triangle {};
 
   inference Triangle
   {  ->  num 1;
 
          num m
      and limit n | n > m
      ->  num (m+1);
 
          num a
      and num b
      and num c | a < b && b < c && a*a + b*b == c*c
      ->  { cout << a << " * " << a << " + "
                 << b << " * " << b << " = "
                 << c << " * " << c << "\n";
          };
   };
\end{verbatim}
 
Now, to print all the triangle identities lying in range of $1$ to $100$,
we only have to create an instance of the inference class, insert the
limit, and start the inference process, as in below:
 
\begin{verbatim}
   Triangle triangle;
   triangle += limit(100);
   triangle.infer();
\end{verbatim}

\Subsection{Inference Class}
\INCOMPLETE

\Subsection{Inference Rules}
\INCOMPLETE
 
\Section{Tree Rewriting} \label{sec:tree-rewriting} 

\Prop's tree rewriting mechanism let us transform a tree in
algebraic datatype form into another tree according to a set of 
{\em rewriting rules}.  Unlike plain pattern matching described
in section~\ref{sec:pattern-matching}, which only apply to the
root of a tree, rewriting rules are applicable to all parts of
the tree.  This allows the user to concentrate on developing the
set of transformations; the actual traversal of the tree object
is handled implicitly by \Prop.  When used properly, this
mechanism has an advantage over plain pattern matching since
rewriting rules remain valid even when a new variant to a datatype
is introduced.

In \Prop, a rewriting system is developed in a similar manner as a parser.
The first phase requires the user to defines a {\em rewrite class}
to encapsulate the rewriting rules.  

Frequently, the rewriting mechanism is used to collect information about
a tree structure; furthermore, more transformation rules are {\em conditional}:
i.e. we want them to be applicable only if certain conditions are satisfied. 
We can accomplish both of these tasks by defining data members and methods 
in the rewriting class.  These are accessible during the rewriting
process.  Information collected during rewriting can be stored within
the data members.

In the rewriting formalism, equational
rules of the form {\em lhs} \verb|->| {\em rhs} are specified by the user.
During processing, each instance of the lhs in a complex tree is replaced
by an instance of the rhs, until no such replacement is possible.
Equational rules can often be used to specify semantics based simplification
(e.g. constant folding and simplification based on simple algebraic
identities) or transformation(e.g. code selection in a compiler
backend\cite{codegen-with-trees}).
 
Unlike plain pattern matching, however, the structural traversal process
in rewriting is implicitly inferred from the type structure of an
algebraic datatype, as specified in its definition. 
 
There are two main forms of rewriting modes available:
\begin{itemize}
   \item The first is {\bf normalization} mode: a given tree is reduced
using the matching rules until no more redexes are available.  There
are two modes of operations available:
       \begin{itemize}
          \item in {\bf replacement} mode, the redex of a tree will be
physically overwritten.
          \item in {\bf applicative} mode, on the other hand, a new
tree corresponding to the replacement value will be constructed.
       \end{itemize}
     Replacement mode is used as the default since it is usually
the more efficient of the two.
   \item The second form is {\bf reduction} and {\bf transformation}.
In this mode a tree parse of the input term is computed.  If cost functions
are attached to the rules, then they will also be used to determine a
minimal cost reduction sequence.  During this process attached actions of
a rule may be invoked to synthesize new data.
\end{itemize}
 
   Each independent set of rewriting rules in \Prop{} is encapsulated
in its own {\bf rewrite class}.  A rewrite class is basically a normal \Cpp{}
class with a set of rewriting rules attached.   During rewriting, the
data members and the member functions are visible according to the normal
\Cpp{} scoping rules.  This makes it is easy to encapsulate additional data
computed as a side effect during the rewriting process.
 
\Subsection{A rewriting example}
 
   Consider an abbreviated simplifier for the well-formed formula datatype
\verb|Wff| defined in the section~\ref{sec:Wff}.  The rewrite class for this
can be defined as follows.  Since there is no encapsulated data in this
example, only the default constructor for the class needs to be defined.
A rewrite class definition requires the traversal list to be
specified.  This is simply a list of datatypes 
involved in the rewriting traversal process.  
In this instance only \verb|Wff| is needed.
 
\begin{verbatim}
   rewrite class Simplify (Wff)
   {
   public:
      Simplify() {}
   };
\end{verbatim}
 
   The rewrite rules for the simplifier can then be specified succinctly as
follows.  Like the \verb|match| statement, in general the rhs
of a rewrite rule can be any statement.  A special statement
\verb|rewrite(e)| can be used to rewrite the current redex into another form.
If the rhs is of the form \verb|rewrite(e)|, then it can be abbreviated
to \verb|e|, as in below:
\begin{verbatim}
   rewrite Simplify
   {  And(F,  _):      F
   |  And(_,  F):      F
   |  And(T,  ?X):     ?X
   |  And(?X, T):      ?X
   |  Or (T,  _):      T
   |  Or (_,  T):      T
   |  Or (F,  ?X):     ?X
   |  Or (?X, F):      ?X
   |  Not(Not(?X)):    ?X
   |  Not(And(?X,?Y)): Or(Not(?X), Not(?Y))
   |  Not(Or(?X,?Y)):  And(Not(?X), Not(?Y))
   |  Implies(?X,?Y):  Or(Not(?X), ?Y)
   |  And (?X, ?X):     ?X
   |  Or (?X, ?X):      ?X
   // etc ...
   };
\end{verbatim}
 
The rewrite class definition creates a new class of the same name.  This new
class defines an implicit \verb|operator ()| with the protocol below.  This
member function can be invoked to perform the rewriting in a functional
syntax.
 
\begin{verbatim}
   class Simplify : ... {
   {  ...
   public:
      void operator () (Wff);
      // Wff operator () (Wff); // if rewrite class is applicative
   };
 
   Wff wff = ...;
   Simplify simplify; // create a new instance of the rewrite class
   simplify(wff);    // rewrite the term wff
\end{verbatim}

 
\Subsubsection{Conditional rewriting and actions}
  
   Rewriting rules may be guarded with predicates to limit their
applicability.  In addition, the {\em rhs} of a rewrite rule is not limited
to only a replacement expression: in general, any arbitrarily complex sequence
of code may be used.  For example, in the following set of rewriting
rules we use guards to prevent undesirable replacements to be made during
expression constant folding:
 
\begin{verbatim}
   rewrite ConstantFolding
   {  ADD (INT a, INT b):  INT(a+b)
   |  SUB (INT a, INT b):  INT(a-b)
   |  MUL (INT a, INT b):
      {  int c = a * b;                      // silent overflow
         if (a == 0 || b == 0 || c / b == a) // no overflow?
         {  rewrite(INT(c)); }
         else
         {  cerr << "Overflow in multiply\n"; }
      }
   |  DIV (INT a, INT b) | b == 0:  { cerr << "Division by zero\n"; }
   |  DIV (INT a, INT b):  INT(a/b)
   |  // etc...
   };
\end{verbatim}

\Subsection{Rewrite class} \label{sec:rewrite-class}

The syntax of a rewrite class declaration is as follows:

\begin{syntax}
\NDEF{Rewrite\_Class\_Decl} & \IS & \TDEF{rewrite class} \Id 
            \T{(} \LIST{\N{TypeExp}} \T{)} \\
   & & \quad \OPT{\T{:} \N{Inherit\_List}} 
        \OPT{\T{::} \MORE{\N{Rewrite\_Mode}}{}} \\
   & & \quad \T{\{} \N{Class\_Body} \T{\}} \T{;} \\
\NDEF{Rewrite\_Mode} & \IS & \TDEF{treeparser} \\
                  & \OR & \TDEF{applicative} \\
                  & \OR & \TDEF{topdown} \\
\end{syntax}

This is simply the usual \Cpp{} class declaration syntax
extended to the following information:
\begin{description}
   \item[a traversal list] enclosed in parenthesis.  The traversal
list of a rewrite class defines the set of types that rewriting
should traverse.  If a datatype object contains an argument of a type
not listed in the traversal list, its value will not be altered.
   \item[rewrite mode] this defines the rewriting mode of the
rewrite class.
\end{description}

\Subsection{Rewriting rules}

Rewriting rules are specified in a rewrite declaration.
The syntax of a rewriting specification is as follows:

\begin{syntax}
\NDEF{Rewrite\_Decl} 
   & \IS & \TDEF{rewrite} \Id & variant 1 \\
   &    & \T{\{} \OPT{\N{Index\_Decl}} 
        \ALT{\T{case} \N{Rewrite\_Rules}}{} \T{\}} \\
   & \OR & \T{rewrite} \Id & variant 2 \\
   &    & \T{\{} \OPT{\N{Index\_Decl}} \ALT{\N{Rewrite\_Rules}}{|} \T{\}} \\
\NDEF{Rewrite\_Rule}
  & \IS & \OPT{\N{Rewrite\_Modifier}} \OPT{\N{Id} \T{->}} \\
  & & \quad \Pat \OPT{\N{Guard}} \OPT{\N{Cost}} \T{:} \N{Rewrite\_Action} \\
\NDEF{Rewrite\_Modifier}
  & \IS & \T{bottomup:} & bottom up mode \\
  & \OR & \T{topdown:} & top down mode \\
  & \OR & \T{before:} & before actions \\
  & \OR & \T{preorder:} & preorder actions \\
  & \OR & \T{postorder:} & postorder actions \\
\NDEF{Rewrite\_Action} & \IS & \T{\{} \N{Code} \T{\}} \\
                  & \OR & \Exp \\
\end{syntax}

The name of a rewrite declaration should match the name of
a rewrite class.  
The two syntactic forms of \T{rewrite} have equivalent semantics.

The special statement \T{rewrite}($e$) may be used
inside the rhs action to rewrite the current redex into $e$.
Note that \T{rewrite}($e$) is a branching statement; statements
after \T{rewrite} are not reachable.

\Subsection{Rewriting modifiers}
In version 2.3.3 onward, rewrite rules can be modified by the 
keywords: \T{bottomup:}, \T{topdown:}, \T{before:}, \T{preorder:}
and \T{postorder:}.   These modifiers 
alter the order in which a set of rewriting rules is applied. 

These modifiers act like delimiters, similar
to the way scoping keywords like \T{public:}, \T{protected:} and 
\T{private:} are used to delimit declarations in C++.
For instance, a set of rules prefixed by the modifier \T{topdown:}
will utilize the topdown strategy for reduction.  If no modifier are
specified, then bottom-up will be the default.

Note that all five modes of rules can be used together in a rewriting system 
under tree rewriting mode (see \ref{sec:rewriting-modes})
and their executions are intermixed together.  

The semantics of these modifiers are as follows:
\begin{description}
\item[\TDEF{bottomup:}] This specifies the modified rewriting rules should
be applied in the default bottom-up mode.  In this mode, the 
innermost/leftmost redex is chosen as the next redex.  So reduction
occurs in a bottomup manner.  This means that if a term $t$ is a redex
and if the term $t$ contains a subterm $t'$ which is also a redex, 
$t'$ will be reduced before $t$. 
\item[\TDEF{topdown:}]  This specifies the modified rewriting rules should be
applied in a topdown mode.  In this mode, the rewriter
will first try to locate the outermost/leftmost redex.  Reduction
will occur in a (mostly) topdown manner.
\item[\TDEF{before:}]  This specifies that the modified rewriting rules
should be tried before the topdown phase.   
In addition, unlike topdown mode, if state-caching if used (see
\ref{sec:state-caching}), the rules are only tried once. 
Otherwise, these act just like the preorder modifier, described below. 
\item[\TDEF{preorder:}]  This specifies that
the modified rewriting rules should be
tried only at the preorder traversal of a term.   
\item[\TDEF{postorder:}]  This specifies the rewriting rules should be
tried only at the postorder traversal of a term. 
\end{description}

\Subsubsection{Rewriting modifier example} \label{sec:rewriting-modifier}

We'll use the following example, extracted from a query optimizer for
a subset of the relational calculus, to demonstrate the use
of the rewriting modifiers.

The query optimizer transforms the following abstract syntax, defined
as a set of \Prop's datatypes.  
\begin{verbatim}
datatype List<T> = #[] | #[ T ... List<T> ];
datatype Literal = INT int
                 | STRING const char *
                 | BOOL Bool

and      Exp     = OP Id, Exps
                 | APP Id, Exps
                 | LIT Literal
                 | ID Id
                 | TUPLE Exps                  // [ E1, E2, ... En ]
                 | FORALL Id, Exp, Exp         // forall x in E1 . E2
                 | EXISTS Id, Exp, Exp         // exists x in E1 . E2
                 | GUARD (Exp, Exp)            //
                 | GENERATOR (Ids, Exps, Exp)  // [x_1,...,x_n] in X_1*...*X_n
                 | LET  (Id, Exp, Exp)

law inline Int  i    = LIT(INT i)
|   inline Boolean b = LIT(BOOL b)
|          True      = Boolean true
|          False     = Boolean false
|   inline And a,b   = OP("and",#[a,b])
|   inline Or  a,b   = OP("or", #[a,b])
|   inline Not a     = OP("not",#[a])
|   inline Eq a,b    = OP("=",  #[a,b])
|   inline Ne a,b    = OP("/=", #[a,b])
|   inline Gt a,b    = OP(">",  #[a,b])
|   inline Ge a,b    = OP(">=", #[a,b])
|   inline Lt a,b    = OP("<",  #[a,b])
|   inline Le a,b    = OP("<=", #[a,b])
|   inline In a,b    = OP("in", #[a,b])
|   inline Union a,b = OP("union", #[a,b])
|   inline Count a   = OP("#", #[a])

where type Id         = const char *
and        Ids        = List<Id>
and        Exps       = List<Exp>
;
\end{verbatim}

Note that existential and 
universal quantifiers are represented by the constructors \T{FORALL}
and \T{EXISTS} respectively.  For example, $\exists x \in A.p$ is
presented as the term $\T{EXISTS}(x,A,p)$.  Here $x$ is a {\em binding}
occurrence for $x$; note that the variable $x$ may occur within 
the predicate $p$.  In addition, a list comprehension
expression such as 
\[ \{ e : [x_1,\ldots,x_n] \in X_1 
\times \ldots \times X_n\ |\ p \} \] 
is represented as the compound term 
\[
  \T{GENERATOR}(\#[x_1,\ldots,x_n],\#[X_1,\ldots,X_n],\T{GUARD}(p,e))
\]

Suppose we'll like to rename all variables in a query $Q$ 
so that no two binding occurrences are given the same name.  
This can be easily accomplished using a combination
of \T{preorder:} and \T{postorder:} rules in the following manner: 
\begin{enumerate}
\item[(i)] Use preorder rules to check for new variable bindings,
for example in $\T{EXISTS}(x,A,p)$.
For each new binding occurrence for $x$, create
a new name for $x$.  
These will be performed before the subterms $A$ and $p$ are traversed.
\item[(ii)] Whenever a variable $x$ is found during the sub-traversal
of $A$ and $p$, rename the variable by looking up its new name.
\item[(iii)] Finally, use postorder rules to remove the current 
variable substitution whenever we're exiting a binding occurrence.
\end{enumerate}

For example, the following set of rewriting rules accomplish this renaming
task for our query language using exactly this method.  
\begin{verbatim}
rewrite RemoveDuplicateNames
{
   // We use before and after rules to remove duplicates from variable names.
   // As each new binding occurrence is found, we enter a new binding
   // into the current environment.  Identifiers found within
   // the binding occurrence are then renamed.

preorder: // insert new bindings
   EXISTS(x,_,_):     { new_binding(x); }
|  FORALL(x,_,_):     { new_binding(x); }
|  GENERATOR(xs,_,_): { new_binding(xs); }
|  LET(x,_,_):        { new_binding(x); }

         // rename variables
|  ID x:             { rename(x); }

postorder: // removes the binding
|  EXISTS(x,A,_):      { old_binding(x); }
|  FORALL(x,A,_):      { old_binding(x); }
|  GENERATOR(xs,As,_): { old_binding(xs); }
|  LET(x,_,_):         { old_binding(x); }
};
\end{verbatim}

Note that we have accomplished a depth first traversal using rewriting
without writing any traversal code!  As a side benefit, 
since the traversal is automatically determined by 
the structure of the datatypes, we do not have to rewrite this
renaming code even if the abstract syntax of the query language is
extended, as long as no additional binding operators are added.

\Subsection{The \T{rewrite} statement}
 
   While the rewrite class construct provides a very general abstraction
for rewriting, in general its full power is unneeded.  It is often
convenient to be able to perform rewriting on a term without having
to make a new name for a class just for the occasion, especially if member
functions and member data are unneeded.  To accommodate these situations,
the \verb|rewrite| statement is provided
to perform a set rewriting transformations on a term without having
to define a temporary rewrite class.   It is simply syntactic sugar
for the more general rewrite class and rewrite
rules specifications.
For example, a simplify routine for type \verb|Exp| defined above can be
specified as follows:
 
\begin{verbatim}
   Exp simplify (Exp e)
   {  // transformations on e before
      rewrite (e) type (Exp)
      {  ADD (INT a, INT b):  INT(a+b)
      |  SUB (INT a, INT b):  INT(a-b)
      |  MUL (INT a, INT b):  INT(a*b)
      |  ...
      }
      // transformations on e after
      return e;
   }
\end{verbatim}
 
The \verb|rewrite| normally performs the replacement in-place.
An applicative version of the same can be written as follows\footnote{The
variable {\tt e} is assigned the new copy.}:
 
\begin{verbatim}
   Exp simplify (Exp e)
   {  rewrite (e) => e type (Exp)
      {  ADD (INT a, INT b):  INT(a+b)
      |  SUB (INT a, INT b):  INT(a+b)
      |  MUL (INT a, INT b):  INT(a*b)
      |  ...
      }
      return e;
   }
\end{verbatim}

The syntax of the rewrite statement is as follows.  The
traversal list of the set of rewrite rule is listed next to
the keyword \T{type}.
 
\begin{syntax}
\NDEF{Rewrite\_Stmt} 
   & \IS & \TDEF{rewrite} \T{(} \Exp \T{)} \OPT{\T{=>} \N{Exp}} \\
   &     & \quad \T{type} \T{(} \LIST{\N{TypeExp}} \T{)} \\
   &     & \T{\{} \OPT{\N{Index\_Decl}} \N{Rewrite\_Rules} \T{\}} \\
   & \OR & \TDEF{rewrite} \T{(} \N{Exp} \T{)} \OPT{\T{=>} \N{Exp}} \\
   &    & \quad \TDEF{type} \T{(} \LIST{\N{TypeExp}} \T{)} \T{of} \\
   &    & \quad \OPT{\N{Index\_Decl}} \N{Rewrite\_Rules} \\
   &    & \T{end} \T{rewrite} \T{;} \\ 
\end{syntax}

\Subsubsection{Rewriting modes} \label{sec:rewriting-modes}
   There are two basic modes of operations in a rewrite class: (i) 
{\em tree rewriting} and (ii) {\em tree parsing}.   
Tree rewriting repeatedly looks for redexes in a tree and transforms it into
another tree.   There are two sub-modes of operations: (a) {\em applicative}
and (b) {\em in-place}.  Applicative mode is specified using the
\T{applicative} mode specifier when declaring a rewrite class. 
In this mode, a new tree is built from a 
bottom-up manner, and the subject tree is left unaltered.  The default
mode is ``in-place.''  In this mode, the subject tree is overwritten
with the transformed expression.  

   On the other hand, in tree parsing mode, the left hand side
of a rewrite rules specification is treated as a tree grammar.
The tree parser will look for a derivation of given tree object from
the start non-terminal.
A guarded rule will only be used if the guard evaluates to true.

   If rules are annotated with cost expressions, then the
tree parser will try to locate a derivation chain with a minimal total
cost.  After a derivation tree is found, the tree can be repeated
traversed.  The rhs actions will be invoked during this traversal process.  

\Subsubsection{State caching optimization} \label{sec:state-caching}

   The pattern matching process proceeds in a bottom up manner: 
for each subtree of the form $l(t_1,t_2,\ldots,t_n)$, $n \ge 0$, 
a state number is computed from the current tree label $l$ and the
state numbers of its subtrees $t_1, t_2, \ldots, t_n$.  The set of matched
rules can be directly computed from the current state number.
This means that in the absence of redex replacements, pattern matching
takes time proportional only $O(|t| + r)$, where $|t|$ is the size 
of the input tree $t$ and $r$ is time it takes to execute the rhs actions.  
That is, the runtime is insensitive to the number of rewrite rules or the 
complexity of the patterns.

   However, if the input is a DAG rather than a tree\footnote{Rewriting
of a generic graph structure is not recommended with \T{rewrite}.},
or if replacements are performed, then the above analysis may not hold.
Replacements during rewriting often require state information of the
replacement term to be recomputed to further the matching process.  
Since computation of state can involve a complete traversal of a term, 
replacement can become expensive if the replacement term is large.  
For instance, consider the following replacement rule, which replaces 
all expressions of the form {\em 2*x} into {\em x+x}:
 
\begin{verbatim}
   rewrite class StrengthReduction
   {
      MUL (INT 2, ?x):  ADD(?x, ?x)
      ...
   }
\end{verbatim}
 
\noindent
Since the subterm \verb|?x| could be arbitrarily large, recomputing the
state encoding for \verb|ADD(?x,?x)| naively takes time in proportion to the
size of \verb|?x|.  However, since the subtree \verb|?x| occurs as part
of the pattern, its state has already been computed, and it would be
wasteful to recompute its state from scratch.  

A similar observation can be made to the case when the input is a DAG,
where many subcomponents are shared between nodes.  In order to speed
up the rewriting process, the state of each shared copy of the nodes 
should be computed only once.

In order to speedup the state computation process in these situations,
{\em \INDEX{state caching}} can be enabled by
specifying a {\em rewriting index}.
This can be accomplished in two ways: (i) use the qualifier
\TDEF{rewrite} when defining a datatype.  This specifies the
{\em primary index}.  Or alternatively, (ii) use index declarations
to specify one or more {\em secondary indices}.

We'll first describe the first method. 
A \TDEF{rewrite} qualifier can be used 
in the definition of a datatype.  For instance:
 
\begin{verbatim}
   datatype Exp :: rewrite
      = INT (int)
      | ID  (const char *)
      | ADD (Exp, Exp)
      | SUB (Exp, Exp)
      | MUL (Exp, Exp)
      | DIV (Exp, Exp)
      ;
\end{verbatim}

The \T{rewrite} qualifier tells the translator to generate an extra
state field in the implementation of \T{Exp}.  This state field is implicitly
updated during the rewriting process.  When the rewriter encounters a
node with a cached state it can quickly short-circuit all unnecessary
state computation.

{\bf A word of caution:} since each rewriting system computes
its own encoding, rewriting systems should not be mixed if they shard the 
same index, i.e. actions of a 
rewriting system should not invoke another rewriting
system on a term that is participating in the current rewriting system,
if both systems use the same index.  This limitation can be overcome
by using secondary indices, which we'll discuss next.

\Subsubsection{Specifying secondary indices}

From release 2.3.0 onward, it is possible to specify secondary indices
within a rewriting system.  This makes it possible to invoke other
rewriting systems within the execution of another, while retaining the
state caching optimizations throughout.

There are two forms of secondary index: {\em internal index} and 
{\em external index}.  Internal indices are stored directly within
a rewrite term and require the user to pre-allocate space for each term.
In contrast, external indices are stored in a separate data structure
such as a hash table. 

The syntax of secondary index declarations is as follows:
\begin{syntax}
\NDEF{Index\_Decl} & \IS & \TDEF{index:} \LIST{\N{Index\_Spec}} \T{;} \\
\NDEF{Index\_Spec} & \IS & 
   \TypeExp \T{=} \T{none} & \C{Disable indexing} \\
   & \OR & \TypeExp \T{=} \N{Id} & \C{Internal index} \\
   & \OR & \TypeExp \T{=} \TDEF{extern} \N{Id} & \C{External index} \\
\end{syntax}


\Subsubsection{Using an internal index}

In order to make use of an index, certain member functions have to defined
by the user.  Given an internal index named {\em foo}, the rewriter
invokes two functions named
\begin{quotation}
\noindent
   {\tt int get\_{\em foo}\_state() const; } \\
   {\tt void set\_{\em foo}\_state(int); }
\end{quotation}
\noindent within the generated code.  The user should provide the 
implementations for these functions within the datatypes.  

For example, reusing the well-formed formulas example (see 
section~\ref{sec:Wff}),
an internal index on the datatype \T{Wff} can be implemented as follows:
\begin{verbatim}
#include <AD/rewrite/burs.h>
class WffIndex {
   int state;
public:
   WffIndex() : state(BURS::undefined_state) {}
   int get_wff_state() const { return state; }
   void set_wff_state(int s) { state = s; }
};

datatype Wff : public WffIndex
   = F
   | T
   | Var     (const char *)
   | And     (Wff, Wff)
   | Or      (Wff, Wff)
   | Not     (Wff)
   | Implies (Wff, Wff)
   ;

...

rewrite Simplify {
   index: Wff = wff;
   ...
};
\end{verbatim}

Here, the class \T{WffIndex} provides the internal index interface
functions \verb|get_wff_state| and \verb|set_wff_state| expected by
the rewriter.  Note that each term must be initialized with the
state \verb|BURS::undefined_state|.  This constant is defined within
the library include file \verb|<AD/rewrite/burs.h>|.

To enable the index, in the rewriter 
we specify that datatype \T{Wff} should make use of the index named \T{wff},
using the \T{index:} declaration.  

\Subsubsection{Using an external index}

External indices are specified in a similar manner.  Given a datatype
{\em Foo} and an external index named {\em bar}, the rewriter invokes
calls to the following functions:
\begin{quotation}
\noindent
   {\tt int get\_{\em bar}\_state({\em Foo}) const; } \\
   {\tt void set\_{\em bar}\_state({\em Foo}, int); }
\end{quotation}
Typically, these are implemented as member functions in the rewrite class.

The rewriter uses the function {\tt get\_{\em bar}\_state} to lookup
previously cached information set by the function {\tt set\_{\em bar}\_state}.
Typically, the implementation of the two functions can be implemented
as hash tables, using the addresses of the terms as hash functions and
pointer equality comparisons.   Note that caching can be {\em lossy}; i.e.
it is perfectly alright for the cache to eliminate cached information to
keep its size under control.  If no cache information is found,
the function {\tt get\_{\em bar}\_state} should return 
\verb|BURS::undefined_state|.

\paragraph{Class \CLASSDEF{RewriteCache}}
   To make it easy for the users to implement their own external indices,
two sample external index implementation are provided.  The first
is the class \verb|RewriteCache| defined in the library file 
\verb|<AD/rewrite/cache.h>|.  

   The class \verb|RewriteCache| implements a simple fixed capacity
   cache with the following member functions.
\begin{verbatim}
   RewriteCache();
   RewriteCache(int capacity);
  ~RewriteCache();
 
   void clear();                   // after rewriting
   void initialize();              // before rewriting
   void initialize(int capacity);  // before rewriting/also set capacity
   
   int capacity() const; 
               
   void set_state(const void * t, int s);
   int get_state(const void * t) const;

   void invalidate(const void * t); 
\end{verbatim}

The user should call the function \verb|initialize| before rewriting
in order to setup the cache and/or clear out previously cached result.
The functions \verb|set_state| and \verb|get_state| can be used to
implement the necessary state caching functions expected by the rewriter.

Returning to our \T{Wff} example, we have the following possible implementation
of the external index:
\begin{verbatim}
   rewrite class Simplify (Wff)
   {  RewriteCache cache;
   public:
      int get_wff_state(Wff t) const { return cache.get_state(t); }
      void set_wff_state(Wff t, int s) { cache.set_state(t,s); }  
   };

   rewrite Simplify
   {  index: Wff = extern wff;
      ...
   };
\end{verbatim}

\begin{Tips}  
If manual memory management is used, 
a term should not be deallocated from memory if a cache
contains a handle to it.  The method \verb|invalidate| can be used to
make sure the entry associated with a term is removed from the cache.
\end{Tips}

\paragraph{Class \CLASSDEF{GCRewriteCache}}

{\em This class has not be completely tested.}

Since a garbage collectable object is not reclaimed by the collector
whenever there exists a reference to it, using the class
\verb|RewriteCache| may create problems since objects are not reclaimed
if it is referenced by the cache.  The template class \verb|GCRewriteCache|
solves this problem by using {\em weak pointers} in its internal 
implementation.  

The template class \verb|GCRewriteCache<T>| expects \verb|T| to be
a class derived from \verb|GCObject|.   
This class implements the following functions:
\begin{verbatim}
   GCRewriteCache();
   GCRewriteCache(int capacity);
  ~GCRewriteCache();
 
   void clear();                   // after rewriting
   void initialize();              // before rewriting
   void initialize(int capacity);  // before rewriting/also set capacity
   
   int capacity() const;
               
   void set_state(T * t, int s);
   int get_state(T * t) const;

   void invalidate(T * t); 
\end{verbatim}

The usage of these functions are identical to that of class 
\verb|RewriteCache|.

\Subsection{Short circuiting rewrite paths with \TDEF{cutrewrite}}

   From version 2.3.0 onward, the \T{cutrewrite}($e$) 
statement may be used wherever a \T{rewrite}($e)$ statement is expected.
The \T{cutrewrite} statement can be used to selectively ignore certain
subterms during the rewriting process.

The semantics of \T{cutrewrite}($e$) is to replace the current redex
with $e$ but do not bother with looking for other redexes in the replacement
term.  This means that the replacement term will not be re-traversed
immediately.  Furthermore, the replaced term will not match any other
left hand side sub-patterns except pattern variables and wildcards.

   This feature is very useful for preventing the rewriter from
looking inside certain terms. 

   To demonstrate, consider the simple following example:
\begin{verbatim}
   datatype Exp = NUM of int 
                | ADD of Exp, Exp
                | SUB of Exp, Exp
                | MUL of Exp, Exp
                | DIV of Exp, Exp
                | FINAL of Exp
                ;
   Exp e = ADD(NUM(1),MUL(NUM(2),NUM(3)));
\end{verbatim}

Suppose we want to find all numbers within 
the expression \T{e} and increment their values by 1.  
The straight forward way of writing:
\begin{verbatim}
   rewrite (e) type (Exp) {
      NUM x: NUM(x+1) 
   }
\end{verbatim}
\noindent will not work, since the replacement term is a new redex,
and the rewriting system will not terminate.

We can reimplement the rewriting system as a two stage process.
First, we mark all terms that we do not want to be changed; whenever we
find a marked term, we execute a \T{cutrewrite} to prevent the term
from being changed.  Then we unmark the terms.
\begin{verbatim}
   rewrite (e) type (Exp) {
      NUM x:   FINAL(NUM(x+1));
   preorder:
      FINAL _: cutrewrite;
   }
   rewrite (e) type (Exp) {
      FINAL x: x
   }
\end{verbatim}

In the first rewrite statement, replacement terms that we 
want to stay fixed are encapsulated
within the auxiliary \T{FINAL} constructor.  
Recall that preorder rules are tried before the subterms of a redex are 
reduced.  Thus we can make sure all that
all terms that are arguments to a \T{FINAL} term is left unaltered
by adding a \T{preorder} rule that executes a \T{cutrewrite}\footnote{
If an expression argument is not given to a \T{cutrewrite} statement, then
the redex is left unchanged.}
Finally, after the first rewrite statement finishes, we use 
the second rewrite statement to remove all \T{FINAL} terms generated
in the first.

The semantics of \T{cutrewrite} needs further explanation.  Consider
the following rewrite system:
\begin{verbatim}
   rewrite (e) type (Exp) {
      NUM x:          cutrewrite(NUM(x+1))
   |  MUL(X, NUM 1):  X
   |  MUL(NUM 1,X):   X
   }
\end{verbatim}
Given the term \verb|MUL(NUM(0),NUM(2))|, the subterms \verb|NUM(0)|
and \verb|NUM(1)|
will be rewritten by the first rule
into \verb|NUM(1)| and \verb|NUM(3)| respectively.
Furthermore, the replacement term \verb|NUM(1)| will {\em not} match
the left subpattern in \verb|MUL(NUM 1,X)|.  Thus the result of the rewrite
will be \verb|MUL(NUM(1),NUM(3))|.

\begin{Tips}
Since the semantics of a rewriting system with \T{cutrewrite} can
be tricky to analyze, its use should be avoided as much as possible. 
\end{Tips}

\Subsection{Conditional failure with \TDEF{failrewrite}}

   From version 2.3.0 onward, the \T{failrewrite}
statement may be used wherever a \T{rewrite} statement is expected.
The \T{failrewrite} statement is an alternate way of writing
conditional rewriting rules, and it is often useful when the condition
to be checked is too complex to be written as a guard.

When a \T{failrewrite} statement is executed within the rhs action of a
rewrite rule, the current pattern is rejected and the rewriter
will proceed to try other matches.   

For instance, suppose the current redex is 
\verb|DIV(INT 0,INT 0)| and the following rules are specified within
a rewriting system:
\begin{verbatim}
   DIV(INT X,INT Y): { if (Y == 0) failrewrite; else rewrite(NUM(X/Y); }
|  DIV(INT 0,Y):     NUM(0)
\end{verbatim} 

The execution of these rules will proceed as follows:  both patterns
match the redex, but the first rule will be tried first.  Since
\verb|Y| is zero, the \T{failrewrite} statement is executed.  The rewriter
will proceed to try the
second rule and the redex will be replaced with \verb|NUM(0)|. 
Note that if we omit the \T{failrewrite} statement
in the first rule, as in:
\begin{verbatim}
   DIV(INT X,INT Y): { if (Y != 0) rewrite(NUM(X/Y); }
|  DIV(INT 0,Y):     NUM(0)
\end{verbatim}
the rewriter will commit on the first rule
and no action will be performed on the redex.

Like \T{rewrite} and \T{cutrwerite}, \T{failrewrite}
acts as a branching statement and all statements
immediatedly following \T{failrewrite} are dead code.

\Subsection{Confluence and termination}

   Currently no automatic checking is available to make sure that
a set of rewriting rules is confluent and will terminate.  This is currently
the user's responsibility.  Some verification features will be developed
in the future.

\Subsection{Debugging Tree Rewriting}

   By default, \Prop{} generates a macro \verb|DEBUG_|$C$ for each
rewrite class named $C$.  The user can redefine this macro in order to
trace the replacements that are performed during the execution of a 
set of rewriting rules.  The default definition of \verb|DEBUG_|$C$
is as follows:
\[
\verb|#define DEBUG_|C({\it R,redex,filename,linenumber,ruletext})\ {\it R}
\]
i.e. it simply returns the value of {\it R}.

The arguments to the macro are as follows: 
\begin{itemize} 
\item {\it R} is the new replacement term.
\item {\it redex} is the current redex, before the replacement has been
performed.
\item {\it filename} is the file name of the current rewriting rule.
\item {\it linenumber}  is the location of the current rewriting rule.
\item {\it ruletext} is a string that describes the current rule.
\end{itemize}

In order to activate tracing, the user can redefine the macro
\verb|DEBUG_FOO| (for rewrite class \verb|FOO|) as follows:
\begin{verbatim}
#define DEBUG_FOO(a,b,c,d,e) print_rule(a,b,c,d,e)

template <class T>
  T print_rule(T replacement, T redex, 
               const char * file_name, 
               int line, const char * rule_text)
  { cerr << file_name << ':' << line << ": " << rule_text 
         << " redex = " << redex << " replacement = " << replacement
         << endl;
    return replacement; 
  }
\end{verbatim}
Note that these definitions must appear
before the \verb|rewrite FOO| declaration.

\Subsection{Optimizing Tree Rewriting} \label{sec:optimize-rewriting}

   An experimental rewriting optimizer based on partial evaluation
techniques has been implemented in release 2.3.4 of \Prop.  The optimizer
can be enabled with the option \OPTIONDEF{-Orewriting}.  

Informally speaking, the rewriting optimizer will try to locate 
``longer'' reduction sequence and add these to your rewriting rules.  
The result is that during execution these bigger reduction steps are
taken whenever possible.

   For example, consider the \verb|Wff| datatype defined
in section~\ref{sec:rewriting-modifier}.  Suppose we have the following
two rewriting rules in our rule set:
\begin{verbatim}
   Not(Not X): X
|  FORALL(X,A,P): Not(EXISTS(X,A,Not(P)))
\end{verbatim}

Given an input term of the form \verb|FORALL(X,A,Not(P))|, the term will
first be reduced to \verb|Not(EXISTS(X,A,Not(Not(P))))| via the second rule,
then to \verb|Not(EXISTS(X,A,P))| via the first rule.  The optimizer
can discover that this two step reduction can be compressed into one
step if the rewriting system is augmented with the following rule:

\begin{verbatim}
   FORALL(X,A,Not P): Not(EXISTS(X,A,P))
\end{verbatim}

Note that this augmented rule is a {\em specialization} of the two rules 
given by the user.  In general, the rewriting optimizer will repeatedly
look for useful specializations and add them to the user rule set\footnote{
To ensure termination of the optimization process, 
augmented rules will not be further specialized.}.  
These specialized rules will perform multiple reductions in one single step.

To make use of the optimizer, the rewriting system must have the following
properties:
\begin{enumerate}
   \item  It must be strongly normalizing, i.e. it will terminate under all
   inputs.  
   \item  It must be confluent, or else confluence is not necessary. 
   Notice that the optimizer can and will alternate the reduction order
   of a rule set.  Thus if a rewriting system is non-confluent, an
   optimized rewriting system may give a different result.
\end{enumerate}

In addition, the following things should hold to maximize the effectiveness
of the optimizer.
\begin{enumerate}
   \item  Indices (i.e. state caching) should be enabled.   Otherwise
   the optimizer will not be able to optimize a rule.
   \item  Specializable rules must be simple, i.e. the right hand side
   must be written as an expression consist of only \Prop{} datatypes.
   If the right hand side is a complex statement or if it involves
   external function calls, the optimizer will fail to analyze it properly.
   \item  Only bottom-up rules may be present.
   \item  Avoid conditional rules whenever possible.  
\end{enumerate}

Finally, it should be warned that an optimized rewriting system  
may generate a lot more code than the un-optimized one.  The user should
view the generated report and check that excessive specializations have not 
been performed.

\Section{User defined datatypes: Views} \label{sec:views}

   For many applications it is necessary to interface \Prop{} to 
data structures predefined in libraries, or generated by other tools.
In order to make it possible to work with these external data structures,
\Prop{}, from version 2.3.3. onward, provides a {\em view} mechanism to
interface with these data structures.  Using views, an externally defined
type can be used within \Prop's pattern matching and rewriting
mechanisms transparently, just as if it were defined by \Prop. 
The \Prop{} translator will call the
appropriate interface functions specified by the user to manipulate
these external datatypes.

\Subsection{A first example} \label{sec:tagged-union}.
To illustrate the use of views, we'll begin with a simple example.

Suppose in our application we use the following tagged C union to
represent our expression datatype.  An expression is represented
simply as a pointer to this structure.  An null expression is
represented simply as the NULL pointer.  

\begin{verbatim}
enum exp_tag { Int = 1, Ident, Add, Sub, Mul, Div };
struct exp {
   enum exp_tag tag;
   union {
      int          number;
      const char * ident;
      struct { struct exp * l, * r } children;
   } u;
};
typedef struct exp * Exp;
/* User defined constructor functions */
Exp NONE = NULL;
extern Exp INT(int);
extern Exp IDENT(const char *);
extern Exp ADD(Exp, Exp);
extern Exp SUB(Exp, Exp);
extern Exp MUL(Exp, Exp);
extern Exp DIV(Exp, Exp);
\end{verbatim}

Suppose this data structure is already used extensively in other 
modules, and we'd like to \Prop{} pattern matching mechanisms while keeping 
the old code intact.  We can accomplish this by declaring a
view to this data structure, using a \TDEF{datatype view}
definition:
\begin{verbatim}
dataype Exp :: view =
   match (this ? this->tag : 0)
     view 0     => NONE
   | view Int   => INT(int = this->u.number)
   | view Ident => ID (const char * = this->u.ident)
   | view Add   => ADD(Exp = this->u.children.l, Exp = this->u.children.r)
   | view Sub   => SUB(Exp = this->u.children.l, Exp = this->u.children.r)
   | view Mul   => MUL(Exp = this->u.children.l, Exp = this->u.children.r)
   | view Div   => DIV(Exp = this->u.children.l, Exp = this->u.children.r)
   ;
\end{verbatim}

In the above, a view named \verb|Exp| is defined.   Let's describe
what the definition means: in general, 
a view definition has three main parts, the {\em view selector},
a set of {\em view transformation rules}, and a set {\em of view accessor
expressions}.  These parts perform the following tasks:

\begin{description}
\item[View selector]  The {\em view selector} is specified in the
the \TDEF{match} part of the definition.   In this example, 
the expression associated with \T{match} is 
\verb|this ? 0 : this->tag|.  This returns a variant discrimination integer
from \verb|0| to \verb|Div|.  
Note that the pseudo variable \TDEF{this} 
refers to an object of type \verb|Exp|.

\item[View transformation rules]
Each of the clauses 
\[ \TDEF{view} {\it exp} \verb|=>| {\it term} \]
specifies
a {\em view transformation rule}.   These rules are used to specify
how to transform a computed variant tag into a pattern.    In general,
expression {\it exp} must be a constant expression usable in a 
\verb|case| statement.

\item[View accessors]
In each of the constructor arguments, an {\em view accessor} expression
can be defined to access the logical components of the constructor.
In the rule:
\begin{verbatim}
   | view Add   => ADD(Exp = this->u.children.l, Exp = this->u.children.r)
\end{verbatim}

The expressions \verb|this->u.children.l| and \verb|this->u.children.r|
tell \Prop{} how to access the left and right child of the an addition
expression node.

Note that since the variant \verb|NONE| has no arguments, it is represented
as a nullary constructor, and no accessors expressions are defined.
\end{description}

A pretty printing function for \verb|struct exp| can be implemented
in \Prop{} using pattern matching as follows:
\begin{verbatim}
ostream& operator << (ostream& s, Exp e)
{  match (e)
   {  NONE:     { s << "none"; }
   |  INT i:    { s << i; }
   |  ID id:    { s << id; }
   |  ADD(a,b): { s << '(' << a << " + " << b << '); }
   |  SUB(a,b): { s << '(' << a << " - " << b << '); }
   |  MUL(a,b): { s << '(' << a << " * " << b << '); }
   |  DIV(a,b): { s << '(' << a << " / " << b << '); }
   }
   return s;
}
\end{verbatim}

Note that this is exactly the same code we would've written if
\verb|Exp| is defined as a datatype rather than a view.  This means
that the user can reimplement the expression data structure
as a \Prop's datatype at a latter time, without altering the \Prop{}
source code.  In effect, the {\em logical} view of a datatype, and its
{\em physical} implementation can separately.
With this {\em representation transparency} capability,
programs written in \Prop's pattern matching formalisms can evolve 
in an incrementally and pain-free manner.

The generated code for the above function will resemble the following
output.  Note that all the view selectors and view accessors are
inlined into the code, and in general, views incur no overhead to
their use.
\begin{verbatim}
ostream& operator << (ostream& s, Exp e)
{  switch (e ? e->tag : 0)
   {  case 0:      { s << "none"; }
      case Int:    { s << e->u.number; }
      case Ident:  { s << e->u.ident; }
      case Add:    { s << '(' << e->u.children.l 
                       << " + " << e->children.r << '); }
      case Sub:    { s << '(' << e->u.children.l 
                       << " - " << e->children.r << '); }
      case Mul:    { s << '(' << e->u.children.l 
                       << " * " << e->children.r << '); }
      case Div:    { s << '(' << e->u.children.l 
                       << " / " << e->children.r << '); }
   }
   return s;
}
\end{verbatim}

\Subsection{Another view example}

Suppose we use the following alternative implementation, a
\verb|Expression| class hierarchy to represent
an expression in our AST.  
Note that \verb|Exp| is an abstract base class inherited
by the classes \verb|Number|, \verb|Operator|, \verb|Identifer|, etc.

\begin{verbatim}
class Expression
{
public:

   // This class uses a ``wide'' interface
   virtual int  number() const         = 0;  // returns a number
   virtual const char * ident() const  = 0;  // returns an identifier

   // returns the ith subchild
   virtual const Expression * child(int) const = 0; 
   // mutable version of above
   virtual Expression *& child(int) = 0;     
};

typedef Expression * Exp;

class NullExp  : public Expression { ... };
class Number   : public Expression { ... };
class Operator : public Expression { ... };
class Identifier : public Expression { ... };
\end{verbatim}

In order to provide a suitable interface to \Prop's view mechanism,
we introduce an additional member function to return a variant tag:

\begin{verbatim}
class Expression
{
public:
   enum Variant { None, Int, Ident, Add, Sub, Mul, Div };
   virtual Variant get_variant() const = 0;
}
\end{verbatim}

In addition, we assume the following constructor functions have been
implemented:

\begin{verbatim}
extern Exp INT(int);
extern Exp IDENT(const char *);
extern Exp ADD(Exp, Exp);
extern Exp SUB(Exp, Exp);
extern Exp MUL(Exp, Exp);
extern Exp DIV(Exp, Exp);
\end{verbatim}

Now, with these definitions in place, we can define a datatype view of the
above \verb|Expression| hierarchy using the following definition:

\begin{verbatim}
datatype Exp :: view
   = match (this->get_variant())
     view Expression::None  => NONE
   | view Expression::Int   => INT (int = this->number()) 
   | view Expression::Ident => ID  (const char * = this->ident())
   | view Expression::Add   => ADD (Exp = this->child(0), 
                                    Exp = this->child(1))
   | view Expression::Sub   => SUB (Exp = this->child(0), 
                                    Exp = this->child(1))
   | view Expression::Mul   => MUL (Exp = this->child(0), 
                                    Exp = this->child(1))
   | view Expression::Div   => DIV (Exp = this->child(0), 
                                    Exp = this->child(1))
   ;
\end{verbatim}

An evaluation function can be written
using \Prop's pattern matching construct:

\begin{verbatim}
int eval(Exp e, const Env& env)
{  match (e) 
   {  NONE:          { return 0; }
   |  INT i:         { return i; }
   |  ID id:         { return env(id); }
   |  ADD (x,y):     { return eval(x,env) + eval(y,env); }
   |  SUB (x,y):     { return eval(x,env) - eval(y,env); }
   |  MUL (x,y):     { return eval(x,env) * eval(y,env); }
   |  DIV (x,INT 0): { cerr << "Division by zero"; return 0; }
   |  DIV (x,y):     { return eval(x,env) / eval(y,env); }
   }
}
\end{verbatim}
Note that this code works equally well with the tagged union representation
defined in section~\ref{sec:tagged-union}. 

We can also use rewriting to transform an expression tree as follows

\begin{verbatim}
void simplify(Exp& e)
{
   rewrite (e) type (Exp)
   {  ADD(INT i, INT j): INT(i+j)
   |  SUB(INT i, INT j): INT(i-j)
   |  MUL(INT i, INT j): INT(i*j)
   |  DIV(INT i, INT j): INT(i/j)
   }
}
\end{verbatim}

\Subsection{Syntax of view definitions}

The general syntax of a view definition resembles that of a
\T{datatype} definition:

\begin{syntax}
\NDEF{Datatype\_Spec} 
   & \IS & \N{Datatype\_View\_Spec} \\
\NDEF{Datatype\_View\_Spec} 
   & \IS & \Id \T{::} \T{view} \T{=} \\ 
   &     & \quad \T{match} \T{(} \N{Exp} \T{)} \\
   &     & \quad \N{View\_Cons\_Specs} \\
\NDEF{View\_Cons\_Specs} 
   & \IS & \ALT{\TDEF{view} \N{Exp} \T{=>} \N{Cons\_Spec}}{|} \\
\end{syntax}

Default arguments in a \N{Cons\_Spec} are interpreted as view accessors.

There are a few general rules to observe when defining a view:
\begin{enumerate}
 \item A \T{datatype view} definition only defines the interface
 to an external data structure and no code is generated.  In particular,
 the defined view name is not an actual \Cpp{} type.
 Thus the user is responsible for defining all appropriate type definitions.
 \item All other qualifiers other than \T{view} are ignored.   Thus 
 automatic generation of pretty printers, garbage collection interface
 routines, etc. are disabled for views.
 \item In general, the view accessors must be usable as lvalues
 if the user wants to modify a view datatype.  This includes the
 situation when in-place rewriting is used.
 \item Finally, the user are responsible for generating the
 constructor functions.  In the two examples given above, the user
 should implement the functions \T{INT}, \T{ID}, etc.  
\end{enumerate}

\Section{Graph Types and Graph Rewriting} \label{sec:graph}

A typical compiler uses many types of graphs to represent internal program 
information.  For instance, control flow graphs, 
data dependence graphs, call graphs, def/use chains are
some of the most common examples.  In order to automate the process
of implementing and manipulating these graph structures, \Prop{} provides
a generic formalism-- {\em graph types}--
for specifying multisorted, attributed, hierarchical directed graphs in 
very high level manner.  The data structure mapping process of translating
high level graph specifications into concrete \Cpp{} classes is completely
automated.  Thus the user can concentrate on specifying the semantic
relationships between the elements of the domain, instead of the
implementation details.

The graph structure mapping process uses real-time set machine
simulation\cite{real-time-SETL} and subtype analysis\cite{subtype-SETL} 
to convert associative accesses involving sets, maps, and multimaps
into hash free, worst-case constant time pointer manipulations.
This optimization is performed transparently by the translator.

Manipulation of graphs structures are done in three ways: 
\begin{description} 
\item[object-oriented] 
 Using a standard interface generated from \Prop{} the user can 
manipulate a graph in the usual procedural/object-oriented manner. 
\item[set formalism]  Using an embedded SETL\cite{SETL}-like sublanguage, 
the user can manipulate the graphs using high level set operations 
such as set comprehension.  
\item[graph rewriting]  At the highest level, analysis and transformation 
can be carried out using the graph rewriting formalism.  
\end{description}

In this section we'll describe each of these topics in detail.

\Subsection{Graph Types} \label{sec:graph-types}

A \INDEX{graph type} declaration is used to specify a
graph structure 
with multiple sorts of labeled nodes and labeled directed edges.  
Attributes can be attached to both nodes and edges.  
The syntax of \T{graphtype} declarations is as follows:
\begin{syntax}
\NDEF{Graph\_Type} 
      & \IS & \TDEF{graphtype} \Id \\
      & & \quad \OPT{\T{:} \N{Inherit\_List}} \\
      & & \quad \OPT{\T{::} \MORE{\N{Graph\_Mode}}{}} \\
      & & \T{declare} \\
      & & \quad \TDEF{node:} \ALT{\N{Node\_Def}}{|} \\
      & & \quad \TDEF{edge:} \ALT{\N{Edge\_Def}}{|} \\
      & & \T{begin} \\
      & & \quad \N{Code} \\
      & & \T{end} \T{graphtype} \T{;} \\
\NDEF{Graph\_Mode} & \IS & \\
\NDEF{Node\_Def} & \IS & \Id \OPT{\OPT{\T{of}} \N{TypeExp}} \\
\NDEF{Edge\_Def} 
    & \IS & \Id \OPT{\T{of}} \TypeExp \T{->} \TypeExp \\
    & \OR & \Id \OPT{\T{of}} \TypeExp \T{<->} \TypeExp \\
    & \OR & \Id \OPT{\T{of}} \TypeExp \T{<=>} \TypeExp \\
    & \OR & \Id \OPT{\T{of}} \TypeExp \T{<=>*} \TypeExp \\
\end{syntax}

\Subsection{The Graph Interface} \label{sec:graph-interface}
\INCOMPLETE

\Subsection{Set Formalisms} \label{sec:set-formalisms}
\INCOMPLETE

\Subsection{Graph Rewriting} \label{sec:graph-rewriting}
\INCOMPLETE

\Section{Running the Translator} \label{sec:usage} 

The \Prop{} translator is a program called \verb|prop|.  The translator
uses the following command line syntax:
\begin{syntax}
  \NDEF{Running\_Prop} & ::= &
   \T{prop} \OPT{{\em prop\_options}} \ALT{{\em file}}{} \\
\end{syntax}

Here, \N{File} is a file with suffix \verb|.p|$*$.  By default, the output
file will have the same name with the \verb|p| extension removed.
For example, a file named ``foo.pC'' will be translated into
the file ``foo.C''  Similarly, a file named ``bar.ph'' will be translated 
into ``bar.h''  

\Subsection{Options}
The available options to the translator are as follows:
\begin{description}
\item[\OPTIONDEF{-G -GNU}]      Use GNU style error messages.
\item[\OPTIONDEF{-I{\em path}}] Use the search path {\em path}.
\Prop{} will search {\em path} to look for \Prop{} files with 
the suffix \verb|.ph| inside a \verb|#include| directive.  Multiple
paths can be specified by using multiple \verb|-I| options.
\item[\OPTIONDEF{-l -no\_line\_directives}]   By default, \Prop{} will try
to generate \verb|#line| directives to correlate the source and
the translated output.  This option suppresses this feature.
\item[\OPTIONDEF{-M -make\_depends}]  Similarly to the \verb|-M| option
in \verb|cc| and \verb|cpp|.  This option generates a suitable dependency
list for Makefiles.
\item[\OPTIONDEF{-memory\_usage}]  Print memory use during translation.
\item[\OPTIONDEF{-n -no\_codegen}]  
Don't output any code; perform semantic checking 
only.
\item[\OPTIONDEF{-N -non\_linear}]  
Use non-linear patterns during pattern matching.
A non-linear pattern is one in which a variable occurs more than
once.   By default, \Prop{} consider this an error.
\item[\OPTIONDEF{-o{\em outfile}}]  Send output to {\em outfile} instead of 
the default.
\item[\OPTIONDEF{-Ofast\_string\_match}]  
Generate (potentially) faster string matching
code by expanding string literal tests into character tests.  By default, 
\Prop{} generates string tests using \verb|strcmp| and binary search.
\item[\OPTIONDEF{-Oadaptive\_matching}]  Use the adaptive pattern matching 
algorithm\footnote{The implementation may not be effective.}.
\item[\OPTIONDEF{-Oinline\_casts}]  
Generate inline type casts instead of function
calls that do the same.  This may be faster for compilation.
\item[\OPTIONDEF{-Orewriting}]  Enable the rewriting optimizer. 
See also section~\ref{sec:optimize-rewriting} for details.
\item[\OPTIONDEF{-Otagged\_pointer}]  
Use a tagged pointer representation instead
of representing variant tags explicitly in a structure field.  For
example, if there are two variants called \verb|Foo| and
\verb|Bar|  in a datatype.  Then a pointer to \verb|Foo| can be tagged
with a low order bit 0 while a pointer to \verb|Bar| can be tagged with 
a low order bit 1.  Accessing the tag thus becomes just a simple bit
test.  This should save space and may be faster\footnote{However, this feature
has not been fully tested.}
\item[\OPTIONDEF{-r -report}]  
Generate a report whenever necessary.  Parser, string
matching, and rewriting constructs produce reports.
\item[\OPTIONDEF{-s -strict}]  Use strict semantic checking mode.  
All warnings are considered to be errors.
\item[\OPTIONDEF{-S -save\_space}]\label{option:save-space} 
Use space saving mode.  
Try not to use inline functions if code space can be saved.  
Datatype constructors will not be inlined
in this mode.  Instead, datatype constructors will be generated at
where the corresponding \TDEF{instantiate datatype} declaration occurs.
\item[\OPTIONDEF{-t -stdout}]   Send translated program to the standard output.
\item[\OPTIONDEF{-use\_global\_pool}]  Use global memory pool for allocation.
\item[\OPTIONDEF{-v{\em num}}]  Use verbose mode in report generation.
This will provide more detailed information.
\end{description}

\Subsection{Error Messages}

The following is a canonical list of all error messages generated by
the \Prop{} translator.  Each message is prefixed by the
file name and line number in which the error occurs.    Most of these
errors messages are self explanatory; more detailed explanations are provided
below.

\begin{Error}
\errormsg{unknown option {\em option}}  The translator does not recognize 
the command line {\em option}.   Type {\em prop} to see a list of options.
\errormsg{Error in command: {\em command}}  The translator fails when trying
to execution {\em command}.  When multiple files are specified in a command
line, {\em prop} invokes itself on each of the file. 
\errormsg{{\em pat} with type {\em type} is not unifiable}  Pattern has not
been declared to be a unifiable type.   
\errormsg{Sorry: pattern not supported in rewriting: {\em pat}}  Pattern
{\em pat} is not supported.  Currently, logical pattern connectives
are not supported in rewriting.  You'll have to rewrite them into
non-logical form.
\errormsg{Unknown complexity operator: {\em op}}  
\errormsg{accessor is undefined for view pattern: {\em pat}} An accessor
function has not been declared for a constructor.  When using datatype views
\errormsg{arity mismatch (expecting $n$) in pattern: {\em pat}}  Pattern
{\em pat} is expected to have arity $n$.   The arity is the number of
expressions that you're matching concurrently.  If you are matching $n$
objects then there must be $n$ patterns for each rule.
\errormsg{arity mismatch in logical pattern: {\em pat}}  Logical patterns
do not match the arity of a pattern.
\errormsg{bad constructor type {\em type}}
\errormsg{bad view constructor pattern: {\em pat}}
\errormsg{can't find include file {\em file}}  Additional search paths
can be specified with the \verb|-I| option
\errormsg{component \#$i$ not found in constructor {\em con}}  
Datatype constructor {\em con} does not have a component named \#$i$.
This can happen when you write a record constructor expression and misspelt
one of the labels.
\errormsg{component $l$ not found in constructor {\em con}}  Datatype
constructor {\em con} does not have a labeled component $l$.
\errormsg{constructor {\em con} already has print formats}
\errormsg{constructor {\em con} is not a class}
\errormsg{constructor {\em con} is not a lexeme}  A constructor is used
as a terminal in the parser but it has not been declared to be a lexeme.
\errormsg{constructor {\em con} doesn't take labeled arguments}  You're trying
to use record constructor syntax on a constructor take does not take
record arguments.
\errormsg{constructor {\em con} is undefined}  
\errormsg{cyclic type definition in type {\em id} = {\em type}}
Type abbreviations cannot be cyclic.  For recursive types, use \verb|datatype|
definitions.
\errormsg{datatype {\em T} is not a lexeme type}  {\em T} is used as a
terminal within syntax class when it has not been declared as a lexeme type.
All non-terminals must be a defined using the \verb|lexeme| qualifier.
\errormsg{determinism {\em det} not recognized}
\end{Error}

\begin{Error}
\errormsg{duplicated definition of pattern constructor '{\em con}'}
  The constructor {\em con} has already been in another datatype.  {\em Prop}
  does not allow overloading of constructor names.
\errormsg{duplicated label '$l$' in expression: {\em exp}}
  A record type has one of its labels duplicated.
\errormsg{duplicated label '$l$' in type {\em type}}
\errormsg{duplicated pattern variable '{\em id}'.  Use option -N}
   By default,
   pattern variables may not be duplicated within a pattern.  Non-linear
   patterns are allowed only when the option \verb|-N| is invoked.
\errormsg{edge $e$ is not defined in graphtype {\em id} }
\errormsg{empty type list in rewrite (...) ... }
\errormsg{expecting $c_1$ ... $c_2$ (from line $l$) but found $c_3$ ... $c_4$ 
   instead}  Quotation symbols are not properly balanced. 
\errormsg{expecting non-terminal {\em nt} to have type $t_1$ but found $t_2$}
   Non-terminal {\em nt} has already been defined to have a
   synthesized attribute type of $t_1$ but $t_2$ is found.   This can happen
   if you have rules with the same lhs non-terminal previously defined.  
\errormsg{expecting type $t_1$ but found $t_2$ in pattern variable '{\em v}'}
{\em Prop} performs type inference on all the lhs patterns to determine
the types of all variables.  Type errors can occur if the patterns are
miswritten.
\errormsg{expecting type $t_1$ but found $t_2$}
\errormsg{flexible vector pattern currently not supported in 
rewriting: {\em pat}}
\errormsg{format \#$i$ used on constructor {\em con}}
\errormsg{format \#$l$ used on non-record constructor {\em con}}
\errormsg{function name mismatch: expecting $f$ ...}
\errormsg{illegal character $c$}
\errormsg{illegal context type: {\em type}}  A context type in
a \verb|matchscan| statement must be previously defined to be datatype.
The unit constructors of the datatype can be used as context values.
\errormsg{illegal context(s) in pattern {\em pat}}
\errormsg{illegal format '\_' on constructor {\em con}}
\errormsg{illegal incomplete record pattern: {\em pat}}
\errormsg{illegal print format '$c$' in constructor {\em con}}
\errormsg{illegal record label '$l$' in expression: {\em exp}}
\errormsg{illegal width in bitfield '{\em id} ($n$)'}
\errormsg{inherited attribute '{\em type}' can only be used in treeparser 
mode in rewrite class {\em id}}  
\end{Error}

\begin{Error}
\errormsg{law {\em id}(...) = {\em pat} is not invertible}   Pattern {\em pat}
cannot be treated as an expression.  It may involve logical patterns and
wild cards.
\errormsg{law '{\em id}': bound variable '$v$' is absent in body {\em exp}}
A parameter variable $v$ must be present within the body of a law.

\errormsg{law '{\em id}': type {\em type} cannot be used in parameter {\em id}}

\errormsg{lexeme {\em id} is undefined}

\errormsg{lexeme class {\em id} is undefined}

\errormsg{lexeme pattern is undefined for constructor {\em con}}

\errormsg{lexeme \{{\em id}\} already defined as {\em regexp}}

\errormsg{lexeme \{{\em id}\} is undefined in {\em regexp}}

\errormsg{missing '\}' in regular expression {\em regexp}}

\errormsg{missing label '$l$' in expression: {\em con} {\em exp}}

\errormsg{missing non-terminal in tree grammar rule: {\em nt}}
\errormsg{missing type {\em type} in the traversal list of rewrite class {\em id}} 
Within the rewriting rules, you have used a pattern that involve a constructor
of type {\em  type} directly but no such types are defined in the 
rewrite class definition.  You should add the type to the traversal list.

\errormsg{missing type info in expression {\em exp} : {\em type}}
   Sometimes {\em prop} fails to infer the type of an expression within
a {\em match} statement.  It such
cases it is necessary to help out the translator by adding additional
type information in the patterns.  For example, rewrite some pattern $p$
as $(p : \tau)$ where $\tau$ is the type of $p$.

\errormsg{missing type info in function {\em f} {\em type}}  Similar to above
\errormsg{missing type info in rule: {\em f} {\em pat} : {\em type}}
   Similar to above.
\errormsg{missing view selector for pattern: {\em pat}}
\errormsg{multiple mixed polarity pattern variable '{\em v}'}
\errormsg{multiple unit constructors in polymorphic type {\em id} {\em arg} 
is not supported}
\errormsg{negative cost {\em cost} is illegal}
\errormsg{node {\em id} is not defined in graphtype {\em id}}
\errormsg{non lexeme type {\em type} in pattern {\em pat}}
\errormsg{non-relation type {\em type} in pattern: {\em pat}}
\errormsg{pattern is undefined for lexeme {\em l}}
\errormsg{pattern scope stack overflows}
\errormsg{pattern scope stack underflows}
\end{Error}

\begin{Error}
\errormsg{pattern variable '$v$' has no binding at this point}
\errormsg{persist object id is undefined for {\em con}}
\errormsg{persistence pid {\em name} already allocated for type {\em type}}
\errormsg{persistence redefined for type {\em type}}
\errormsg{precedence symbol must be terminal: {\em term}}
\errormsg{predicate {\em p}: expecting type $t_1$ but found $t_2$}
\errormsg{redefinition of bitfield '{\em field}'.}
\errormsg{redefinition of constructor '{\em con}'}
\errormsg{redefinition of datatype {\em id}}
\errormsg{redefinition of lexeme class {\em id}}
\errormsg{replacement not in rewrite class: rewrite {\em exp}}
\errormsg{rewrite class {\em id} has already been defined}
\errormsg{rewrite class {\em id} is undefined}
\errormsg{rule $r$ has incomplete type {\em type}}
\errormsg{rule $r$ is of a non datatype: {\em type}}
\errormsg{syntax class {\em id} has already been defined}
\errormsg{syntax class {\em id} is undefined}
\errormsg{synthesized attribute '{\em type}' 
can only be used in treeparser mode in rewrite class {\em id}}
\errormsg{the class representation of constructor {\em id} has been elided}
\errormsg{this rule is never selected: $r$}
\errormsg{this rule is never used: $r$}
\errormsg{too few arguments {\em arg} in instantiation of type scheme {\em type}}
\errormsg{too many arguments {\em arg} in instantiation of type scheme {\em type}}
\errormsg{type {\em type} is not a datatype}
\errormsg{type {\em type} is undefined}
\end{Error}

\begin{Error}
\errormsg{type {\em id} = {\em type} is not a datatype}
\errormsg{type {\em id} has already been defined as {\em type}}
\errormsg{type {\em id} has unknown class: {\em C}}
\errormsg{type '{\em type}' is not rewritable in treeparser mode rewrite class 
{\em id}}
    All datatypes used within the treeparser mode of rewriting must be defined
    with the \verb|rewrite| qualifier.
\errormsg{type error in pattern {\em pat}: {\em type}}
\errormsg{type mismatch between nonterminal {\em nt}(type $t_1$) 
and rule $r$(type $t_2$)}
\errormsg{type mismatch in pattern: {\em pat}}
\errormsg{type mismatch in rule $r$}
\errormsg{type mismatch in rule $r$ {\em pat}}
\errormsg{type or constructor {\em con} is undefined}
\errormsg{unable to apply pattern scheme {\em pat}}
\errormsg{unbalanced $c_1$ ... $c_2$ at end of file}
\errormsg{undefined non-terminal {\em nt}}
\errormsg{unification fails occurs check with $t_1$ and $t_2$}
\errormsg{unmatched ending quote $c$}
\errormsg{unrecognized quoted expression `{\em exp}`}
\errormsg{unrecognized quoted pattern `{\em pat}`}
\end{Error}

\pagebreak
\appendix

\Section{Garbage Collection in the {\sf Prop} Library} \label{appendix:gc}
   In this appendix
we describe the design and implementation of a garbage collector
based on the Customisable Memory Management framework(CMM)\cite{CMM} in our
\Prop{} \Cpp{} class library.  Like the previous approach, we have 
implemented a mostly copying conservative collector based on the work
of Bartlett\cite{Mostly-copying}.  
Similar to CMM, our architecture provides a protocol to allow 
multiple garbage collectors using different algorithms to coexist in the 
same memory space.  A few improvements are made to
improve the performance, the flexibility and the functionality of 
our collector: to reduce retention due to false roots identification,
blacklisting\cite{Boehm} is used to identify troublesome heap
addresses; the architecture of the system has been generalized so that it is 
now possible to have multiple instantiations of Bartlett-style heaps; 
finally, object finalization and weak pointer support
are added.  Our collector has been tested on gcc 2.5.8 under Linux,
and SunOS 4.1.x\footnote{An alpha version also works under Solaris. Thanks
to Paul Dietz for the patch.}

\Subsection{Introduction}
   The \Prop{} project involves the design and implementation of 
an environment and an extension language based on \Cpp{} 
for compiler construction and program transformation.  
An outgrowth of this project is the \Prop{} 
\Cpp{} class library, which contains an extensive set of support algorithms 
and data structures.  Since a typical compiler manipulates many complex tree,
dag and graph objects with extended lifetime, manual memory management using
\Cpp's constructors and destructors, reference counting smart pointers, or
some other techniques is frequently complex and error prone.  
Furthermore, with the advent of new algorithms for garbage collection,
manual memory management techniques do not necessarily provide better 
performance or space utilization.  To make it possible to make use of garbage 
collection as a viable alternative for memory management in
\Cpp\cite{Safe-C++-GC}, we have implemented a garbage collection class 
hierarchy as part of the \Prop{} library.  The class library can be used
directly by the users who'd like to program in \Cpp; it can also be
used as part of the runtime system of a highly level language implemented
in \Cpp, as we have done for \Prop.

   We have several good reasons to prefer garbage collection over manual 
memory management.  The \Prop{} language contains many declarative constructs 
and extensions such as algebraic datatypes, pattern matching, rewriting, and
logical inference.  When a user programs in \Prop{} using these constructs, 
an applicative style of programming is the most natural paradigm.  

\Subsection{The Framework}

   We base our design on the work on the Customisable Memory Management(CMM)
system\cite{CMM}.  In this framework, multiple independent heaps(including
the usually non-collectable heap) can coexist with each other.
Bartlett's mostly copying garbage collector is used as the primary collector.
CMM extends the work of Bartlett\cite{Mostly-copying} 
by allowing cross heap pointers and unrestricted interior pointers.

  However, all collectable objects are required to derive from a base class 
and reimplement a tracing method, which identifies the internal pointers of 
an object.  This burden is usually quite small and in fact can be
automated from the type definition\footnote{In \Prop, a special keyword
\T{collectable} is used to identify garbage collected classes and types.
The \Prop{} translator uses this type annotation to derive the appropriate  
tracing methods.}

  One of the major advantages of the CMM framework is that multiple
coorperating collectors can coexist at the same time, which makes it
possible to customize the behavior of each collector with respect to
the lifetime behavior of the objects.  In \cite{CMM}, examples are
given in which the lifetime of certain class of objects exhibit
first-in/first-out behavior.  In this case a special collector can be
written to take full advantage of this property.

\subsection{Our Framework}
   Our framework retains all the benefits and flexibilities of CMM, 
while extending it in several minor but useful ways:  
\begin{itemize}
   \item Like CMM, interior pointers(i.e. pointers to the middle of an object)
and crossheap pointers (i.e. complex data structures linking nodes locating
in multiple logical heaps.) are supported.  Pointers to collectable
objects are non-intrusive: i.e. they are normal \Cpp{} pointers 
and not encapsulated in templates.
   \item Also like CMM, we allow multiple garbage collectors using different
algorithms to coexist.  Unlike CMM, however, we allow multiple 
Bartlett-style collectors to be instantiated.  Most of the services 
involving low level page management and object marking have been relegated
to a separate heap manager.   
   \item We provide support for finalization and weakpointers.
   \item We have implemented blacklisting\cite{Boehm} to reduce the chance
of false roots identification.
\end{itemize}

\subsection{The Implementation}

   Our implementation has been completely written from scratch since
we do not desire to utilize copyrighted code and, more importantly,
we have to make a few important architectural changes:
 All low level memory management services, such as management of
the page table and the object bitmap, is now relegated to the class 
\CLASSDEF{GCHeapManager}.  Collectors now act as clients as of the heap manager
and the Bartlett-style collector no longer has any special status. 

\subsection{Architecture}

   The architecture of the memory management system is partitioned into
a few classes, each responsible for providing distinct services:

\begin{itemize}
   \item \CLASSDEF{GCHeapManager} --- The heap manager.  The heap manager
    manages the heap table, performs page level
    allocation and deallocation, and provides miscellaneous services
    like blacklisting.  It also manages the object bitmaps.
   \item \CLASSDEF{GC} --- The base class for all garbage collectors.
    This base class describes the protocol used by all the collector
    classes\footnote{This base class is also inherited from class
    \CLASSDEF{Mem}, so that it adheres to the \Prop{} memory management
    protocol.}
   \item \CLASSDEF{CGC} --- The base class for conservative collectors.
     This class is inherited from class \CLASSDEF{GC} and implements some
     methods for locating the stack, heap, and static data areas.
   \item \CLASSDEF{BGC} --- Bartlett-style mostly copying collector.
      This class is inherited from class \CLASSDEF{CGC} and implements
      the Bartlett mostly copying algorithm.
   \item \CLASSDEF{MarkSweepGC} --- Mark/sweep style conservative collector.
      This class is inherited from class \CLASSDEF{CGC} and implements
      a mark/sweep collection algorithm.
   \item \CLASSDEF{WeakPointerManager} --- The weakpointer manager.
      This class manages the weak pointer table and provides a few
      weak pointer scavenging services for the collectors.
\end{itemize}
  
\Subsection{The Programmatic Interface}

   The programmatic interface to the garbage collector involves two
base classes, \CLASSDEF{GC} and \CLASSDEF{GCObject}.  The base class 
\CLASSDEF{GC} provides an uniform interface to all types of collectors 
and memory
managers while class {\sf GCObject} provides the interface to all 
collectable classes.  Table~\ref{GC-Classes} contains a listing
of the classes in our hierarchy.

\begin{table}
   \begin{center}
      \begin{tabular}{|l|l|} \hline
        Class           & Purpose \\ \hline \hline
        \sf GCObject    & Collectable object base class \\
        \sf GC          & Garbage collector base class \\
        \sf CGC         & Conservative garbage collector base class \\
        \sf BGC         & Bartlett style mostly copying collector \\
        \sf MarkSweepGC & A mark-sweep collector \\
        \sf UserHeap    & A heap for pointerless object \\
        \sf GCVerifier  & A heap walker that verifies the integrity of a 
                          structure \\
      \hline 
      \end{tabular}
   \end{center}
   \caption{\label{GC-Classes} Garbage Collection Classes.}
\end{table}

Class \CLASSDEF{GCObject} is extremely simple: it redefines the operators
\verb|new| and \verb|delete| to allocate memory from the default collector,
and declares a virtual method ``\verb|trace|'' to be defined by subclasses
(more on this later.)

Memory for a \CLASSDEF{GCObject} is allocated and freed using the usually
\verb|new| and \verb|delete| operators.  Of course, freeing memory explicitly
using \verb|delete| is optional for many subclasses of \CLASSDEF{GC}. 

\begin{verbatim}
   class GCObject {
   public:
      inline void * operator new(size_t n, GC& gc = *GC::default_gc) 
         { return gc.m_alloc(n); }
      inline void * operator new(size_t n, size_t N, GC& gc = *GC::default_gc) 
         { return gc.m_alloc(n > N ? n : N); }
      inline void   operator delete(void *);
      virtual void trace(GC *) = 0;
   };
\end{verbatim}

\subsection{Memory Allocation}

The base class \CLASSDEF{GC} is slightly more complex, as it has to provide
a few different functionalities.  The first service that class \CLASSDEF{GC} 
must 
provide is of course memory allocation and deallocation.  As a time saving
device we can specify what the default collector is using the methods
\verb|get_default_gc| and \verb|set_default_gc|.  Method \verb|GCObject::new| 
will use this collector by default, unless placement syntax is used.
Method \verb|GCObject::delete|, on the other hand, can correctly infer the 
proper collector to use by the address of the object.

The low level methods to allocate and deallocate memory are \verb|m_alloc|
and \verb|free| respectively.  The programmers usually do not have to
use these methods directly. 

The method to invoke a garbage collection of a specific level is 
\verb|collect(int level)|.  This forces an explicit collection.
Method \verb|grow_heap(size_t)| can also be used to explicitly increase
the heap size of a collector.  Depending of the actual behavior
of the subclasses, these methods may have different effects.

\begin{verbatim}
   class GC {
   protected:
      static GC * default_gc;
   public:
      static GC&  get_default_gc()       { return *default_gc; }
      static void set_default_gc(GC& gc) { default_gc = &gc; }
      virtual void * m_alloc   (size_t) = 0;
      virtual void   free      (void *);
      virtual void   collect   (int level = 0) = 0;
      virtual void   grow_heap (size_t) = 0;
      static  void garbage_collect() { default_gc->collect(); }
      virtual void set_gc_ratio(int);
      virtual void set_initial_heap_size (size_t);
      virtual void set_min_heap_growth   (size_t);
   };
\end{verbatim}

\subsection{The GC Protocol}
The collector and collectable objects must cooperate with
each other by abiding to a simple protocol:
\begin{itemize}
   \item All objects that are to be garbage collected must be derived
from \CLASSDEF{GCObject}.  The application programmer must also supply a
``{\bf tracing}'' method.  The purpose of this method is to identify
all internal pointers to other \CLASSDEF{GCObject}'s.   This method is not
used by the application programmer directly but only used internally
by the garbage collectors.
   \item The tracing method of each collectable must in turn call
    the tracing method in the class \CLASSDEF{GC} with each pointer to
    a collectable object that the object owns:
\begin{verbatim}
   class GC {
   public:
      virtual GCObject * trace (GCObject *) = 0;
      inline  void trace (GCObject& obj);
   };
\end{verbatim}
    
     Briefly, the rules are as follows:
   \begin{enumerate}
      \item The tracing method of a collectable \T{Foo} has the following
         general form:
          \begin{verbatim}
             void Foo::trace(GC * gc) 
             {
                ...
             }
          \end{verbatim} 
      \item If class \T{Foo} has a member that is a pointer \verb|p|
            to a collectable object of type \T{Bar}, then add:
           \begin{verbatim}
               p = (Bar)gc->trace(p);
           \end{verbatim}
            to the body of \verb|Foo::trace|. 
     \item  If class \T{Foo} has a member object
            \verb|x| that is a subclass of \T{GCObject}, also add:
           \begin{verbatim}
               gc->trace(x);
           \end{verbatim}
            to the body of \verb|Foo::trace|.
     \item  If class \T{Foo} is derived from a class \verb|Bar| that
            is a subclass of \T{GCObject}, add:
           \begin{verbatim}
               Bar::trace(gc);
           \end{verbatim}
            to the body of \verb|Foo::trace|.
   \end{enumerate} 
    Notice that these methods can be arranged in any order.
\end{itemize}

This protocol can be used by both copying and non-copying collectors.
In addition, the class \CLASSDEF{GCVerifier} also uses this protocol to 
walk the heap in order to verify the integrity of a garbage collected 
data structure.

\subsection{Messages and Statistics}
    All garbage collectors use the following protocols for status
reporting and statistics gathering.  
  
\begin{verbatim}
   class GC {
   public:
      enum GCNotify {
         gc_no_notify,             
         gc_notify_minor_collection,
         gc_notify_major_collection,
         gc_notify_weak_pointer_collection,
         gc_print_collection_time,
         gc_print_debugging_info
      }
      virtual int      verbosity() const;
      virtual void     set_verbosity(int);
      virtual ostream& get_console() const;
      virtual void     set_console(ostream&);
   };
\end{verbatim}

\begin{verbatim}
   class GC {
   public:
      struct Statistics {
         const char *   algorithm;
         const char *   version;
         size_t         bytes_used;
         size_t         bytes_managed;
         size_t         bytes_free;
         struct timeval gc_user_time;
         struct timeval gc_system_time;
         struct timeval total_gc_user_time;
         struct timeval total_gc_system_time;
      }
      virtual Statistics statistics();
   };
\end{verbatim}

    Each collector has a current verbosity level, which can be set
and reset using the methods \verb|set_verbosity(int)| and
\verb|verbosity() const|.  The verbosity level is actually a bit
set containing flags of the type \verb|GC::GCNotify|.  A combination
of these options can be used.
\begin{itemize}
  \item \verb|gc_no_notify| --- no messages at all.
  \item \verb|gc_notify_minor_collection| --- all minor collections
    will be notified by printing a message on the current console.
  \item \verb|gc_notify_major_collection| --- similarly for all major
    collections.
  \item \verb|gc_notify_weak_pointer_collection| --- notify the user
  when weak pointers are being collected.
  \item \verb|gc_print_collection_time| --- this option will let the
    collector to print the time spent during collection (and finalization).
  \item \verb|gc_print_debugging_info| --- this option will print additional
    information such as stack and heap addresses during garbage collection.
\end{itemize}

   The current console of a collector is defaulted to the \Cpp{} stream
\verb|cerr|.  It can be set and inspected with the methods
\verb|set_console(ostream&)| and \verb|ostream& get_console()| respectively.
For example, the user can redirect all garbage collection messages
to a log file \verb|"gc.log"| by executing the following during initialization:

\begin{verbatim}
   ofstream gc_log("gc.log");
   GC::get_default_gc().set_console(gc_log); 
\end{verbatim}

\subsection{The Bartlett style mostly copying collector}

   Currently a Bartlett-style mostly copying collector has been implemented.
The current version is non-generational but we expect that a generational
version will be available in the near future.

   The Bartlett-style collector is implemented as the concrete class
\CLASSDEF{BGC}.  Aside from all the services provided by class \CLASSDEF{GC},
\CLASSDEF{BGC} also provides the following method:

\begin{verbatim}
   class BGC : public CGC {
   public:
      virtual void set_gc_ratio(int);
      virtual void set_initial_heap_size (size_t);
      virtual void set_min_heap_growth   (size_t);
   }
\end{verbatim}

  The gc ratio refers to the ratio of used heap space versus
total heap space.  Class \CLASSDEF{BGC} will invoke the garbage collection
if this is exceeded.  The default gc ratio for \T{BGC} is 75\%.

  The initial size of the heap and the minimal number of bytes to 
request from the system during a heap expansion can be adjusted using 
the methods \verb|set_initial_heap_size| and \verb|set_min_heap_growth|
respectively.  These methods are declared in the \T{GC}
protocol.   By default, the initial heap size for this collector
is 128K and each heap expansion will increase the heap by 512K. 

If an application uses a lot garbage collected storage it is a good
idea to set the heap size to larger capacity during initialization.
Otherwise, the collector will have to perform more garbage collection
and heap expansions to reach a stable state. 

\subsection{The Mark Sweep collector}

   An alternative to the copying collector is the the mark sweep collector.
Like the previous collector, this collector also uses conservative scanning 
to locate roots.  Unlikely the Boehm collector, type accurate marking
is used through the user supplied tracing method.

   Since the default collector is of the \T{BGC} variety, the user must
create an instance of \T{MarkSweepGC} if the mark sweep collector
is to be used.  

\begin{verbatim}
   class MarkSweepGC : public CGC {
   public:
      virtual void set_gc_ratio(int);
      virtual void set_initial_heap_size (size_t);
      virtual void set_min_heap_growth   (size_t);
   }
\end{verbatim}

   The gc ratio in class \T{MarkSweepGC} determines whether heap
expansion is to be performed after a collection.  The default gc ratio
is 50\%, thus if after garbage collection the ratio of used versus total
heap space exceeds one half, the heap will be expanded. 

   For most purposes the two collectors are interchangeable.  Since
all types of garbage collectors under our framework use the same protocol, 
applications can be written without fixing a specific garbage collection
algorithm before hand.  

\subsection{Finalization} 

   One common \Cpp{} idiom uses constructor and destructor for
resource allocation: resources (including memory but may include others,
such as file handles, graphical widgets, etc) that are acquired
in a class constructor are released(finalized) in the class'es destructor.

   There are, however, two opposing views in how finalization should be handled
in a garbage collector.  The first assumes that garbage collection
simulates an infinite amount of memory and so automatic finalization is
inapproriate.   The second takes a more pragmatic view, and assumes that
automatic finalization should be provided since otherwise explicit resource
tracking must be provided by the user for collectable datatypes, making garbage 
collection much less useful than it can be.

   We will refrain from participating in any religious and philosophical
wars on which dogma is the ``One True Way''. 
nstead, both types of collectors are provided.

   By default, all garbage collection classes do 
not perform finalization on garbage 
collected objects.  If object finalization is desired then the user
can do one of two things:
\begin{itemize}
   \item Enable the finalization of the default heap by putting
\begin{verbatim}
   GC::get_default_gc().set_finalization(true);
\end{verbatim}
   in the initialization code, or
   \item Instantiate a different instance of the garbage collector
    and allocate objects needing finalization from this collector.
    This may be a better method since not all objects need finalization
    and those that do not can still be allocated from the default collector.
    This way we only have to pay for the performance penalty (if any)
    proportional to the usage of finalization.
\end{itemize}

   For example, if a collectable class named \T{Foo} should be properly
finalized we can declare it in the following manner:

\begin{verbatim}
   extern BGC bgc_f;  // somewhere an instance is defined

   class Foo : public GCObject {
      Resource r;
   public:
      Foo() { r.allocate(); /* allocate resource */ }
     ~Foo() { r.release(); /* release all resource */ }

      // Make object allocate from bgc_f instead
      // of the default non-collectable gc.
      void * operator new (size_t n) { return bgc_f.m_alloc(n); }
   };
\end{verbatim}

   When an instance of class \verb|Foo| is identified as garbage, its
destructor will be called.  [Aside: {\it currently issue such as 
finalization synchronization is not handled directly by the collector.
So if there is synchronization constraints during finalization it must
be handled by the object destructor.  Future versions of \Prop{} will
provide subclasses with more sophisticated finalization mechanisms.} ]

   Notice that the default \verb|delete| operator of all garbage collectable
classes will call the object destructor explicitly.  It is a good idea to use
explicit \verb|delete| if finalization is a rare need since this service
involves an explicit scan of the garbage memory, which may degrade performace
with excessive swapping: without finalization the garbage pages will not 
have to be referenced with a copying collector.

   It should also be noted that since the collector is conservative, there
is no guarantee that garbage will be identified as such:
there is no guarantee that all garbage resources will be released.
In general, however, the efficacy of the collector is quite good and
so non-critical or plentiful resources can be safely finalized with
this collector.

\Subsubsection{Weak Pointers}

   It is frequently very useful to be able to keep track of garbage
collectable objects through the use of ``{\bf weak pointers}.''
Collectors ignore the presence of weak pointers to garbage collected
objects; objects with only referencing weak pointers will still be
collected.  Weak pointers to objects that become garbage will
be automatically reset to null.
 
   Weak pointers are provided through a smart pointer template 
\CLASSDEF{WeakPointer}, whose definition is shown below: 

\begin{verbatim}
template <class T> class WeakPointer {
   public:
      inline WeakPointer();
      inline WeakPointer(const WeakPointer<T>& wp); 
      inline WeakPointer(T * ptr); 
      inline WeakPointer<T>& operator = (const WeakPointer<T>& wp);
      inline WeakPointer<T>& operator = (T * ptr);
      inline bool is_null() const;
      inline operator const T * () const;
      inline operator       T * ();
      inline const T * operator -> () const;
      inline       T * operator -> ();
      inline const T&  operator *  () const;
      inline       T&  operator *  ();
   };
\end{verbatim}

A weakpointer to a garbage collectable 
class \verb|A| can be defined as \verb|WeakPointer<A>|.
If a \Prop{} datatype \verb|T| has been defined, a weakpointer to type
\verb|T| can defined using the
\TDEF{classof} keyword.  For example:

\begin{verbatim}
   datatype Wff :: collectable = True | False | And(...)  | ...;

   WeakPointer<classof Wff> a = And(True,False);
\end{verbatim}

\Subsubsection{The Heap Walker}

   There is a simple class called \CLASSDEF{GCVerify} that can be used
to verify the integrity of a complex allocated data structure.
This is frequently useful during the development of a new collector,
or a complex class derive class of \T{GCObject}.

   The interface of this class is shown below.

\begin{verbatim}
   class GCVerifier : protected GC {
   public:
      virtual bool is_valid_pointer          (GCObject *);
      virtual bool is_valid_interior_pointer (GCObject *);
      virtual bool is_valid_structure        (GCObject *);
      virtual bool is_valid_pointer          (GCObject *, GC *);
      virtual bool is_valid_interior_pointer (GCObject *, GC *);
      virtual bool is_valid_structure        (GCObject *, GC *);
      size_t number_of_nodes() const;
   };
\end{verbatim}

   Class \CLASSDEF{GCVerify} is derived from class \CLASSDEF{GC} so that the
same object tracing protocol can be used.  Briefly, three different methods 
are provided for verification:
\begin{itemize}
   \item Call to \verb|is_valid_pointer(p,gc)| verifies that
    \verb|p| is a valid pointer to an object within the collector \verb|gc|. 
   \item Call to \verb|is_valid_pointer(p,gc)| verifies that
    \verb|p| is a valid interior pointer to an object 
     within the collector \verb|gc|. 
   \item Call to \verb|is_valid_structure(p,gc)| verifies the entire
     structure reachable by \verb|p| is valid.  This method uses
     \verb|GCObject::trace| to locate the correct sub-structures.
\end{itemize}

   There are alternative versions of the same methods that assumes
the default collector is used.   Furthermore
\begin{itemize}
   \item Method \verb|number_of_nodes()| returns the node count of the
last structure traced using \verb|is_valid_structure|, and 
   \item If \verb|set_verbosity(GC::gc_print_debugging_info)| is used
then error messages will be printed during structure tracing.
\end{itemize}

\bibliography{refman}

%%%%%%%%%%%%%%%%%%%%%%%%%%%%%%%%%%%%%%%%%%%%%%%%%%%%%%%%%%%%%%%%%%%%%%%%%%%%%%%
%
%  Here are some abbreviations used within this document
%
%%%%%%%%%%%%%%%%%%%%%%%%%%%%%%%%%%%%%%%%%%%%%%%%%%%%%%%%%%%%%%%%%%%%%%%%%%%%%%%
\newfont{\bsf}{cmssbx10}
\newfont{\largebsf}{cmssbx12}
%\newfont{\nontermfont}{cmssbx10}
\newcommand{\nontermfont}{\it}
\newcommand{\codefont}{\tt}
\newcommand{\symfont}{\bf}

\newcommand{\Prop}{{\bsf Prop}}               % typeset in sans serif
\newcommand{\LargeProp}{{\largebsf Prop}}
\newcommand{\plusplus}{{\tt++}}             % ++
\newcommand{\Cpp}{{C{\tt++}}}               % C++ logo
\newcommand{\Gpp}{{G{\tt++}}}               % G++ logo
\newcommand{\email}{{\tt leunga@cs.nyu.edu}}  % my email address
\newcommand{\web}{{\tt http://valis.cs.nyu.edu:8888/\~{}leunga}} % my web page
\newcommand{\propweb}{\web{\tt/prop.html}}
\newcommand{\Version}{2.3.4}                  % Prop version
\newcommand{\comment}[1]{}

\newcommand{\sectionfont}{}
%\newcommand{\Section}[1]{\section{\sectionfont #1}}
%\newcommand{\Subsection}[1]{\subsection{\sectionfont #1}}
%\newcommand{\Subsubsection}[1]{\subsubsection{\sectionfont #1}}

\newcommand{\Section}[1]{\pagebreak\section{\sectionfont #1}%
   \markright{\usebox{\LINE}\large \thesection. \sc #1}
   }
\newcommand{\Subsection}[1]{\subsection{\sectionfont #1}}
\newcommand{\Subsubsection}[1]{\subsubsection{\sectionfont #1}}

\newcommand{\N}[1]{\mbox{{\nontermfont #1}}}
\newcommand{\T}[1]{{\codefont #1}}
\newcommand{\MORE}[2]{#1\T{#2} \mbox{$\ldots$} \T{#2} #1}
\newcommand{\ALT}[2]{#1\ \T{#2} \mbox{$\ldots$} \T{#2} #1}
\newcommand{\LIST}[1]{\MORE{#1}{,}}
\newcommand{\SEQ}[1]{\MORE{#1}{;}}
\newcommand{\OPT}[1]{{\it [ #1 ]}}
\newcommand{\C}[1]{\mbox{\rm #1}}
\newcommand{\IS}{{\symfont ::=}}
\newcommand{\OR}{\ \mbox{$\symfont \mid$}\ }
\newcommand{\Id}{\N{Id}\ }
\newcommand{\Pat}{\N{Pat}\ }
\newcommand{\Exp}{\N{Exp}\ }
\newcommand{\TypeExp}{\N{Type\_Exp}\ }
\newenvironment{syntax}{\begin{quotation}\noindent\begin{tabular}{lcll}} %
   {\end{tabular}\end{quotation}}

\newcommand{\CLASSDEF}[1]{\T{#1}\index{Classes!{\codefont #1}}}
\newcommand{\INDEX}[1]{\index{#1}}
\newcommand{\NDEF}[1]{\N{#1}\index{Syntax!Non--terminals!{\nontermfont #1}}}
\newcommand{\OPTIONDEF}[1]{{\tt #1}\index{Command line options!{\tt #1}}}
\newcommand{\TDEF}[1]{\T{#1}\index{Syntax!Keywords!{\codefont #1}}}
\newcommand{\INCOMPLETE}{{\em This section is incomplete.}}
\newenvironment{Error}{\begin{center}\begin{tabular}{p{3in}p{3in}}
  \bf Error & \bf Explanation}{\\\hline\end{tabular}\end{center}}
\newcommand{\errormsg}[1]{\\\hline {\tt #1}&}
\newenvironment{Tips}{\noindent {\bf Tips:} \quad}{}

\newcommand{\support}{
   {\small\noindent This work is supported in part by
    an award from Hewlett-Packard Corporation, from IBM corporation, and
    an NYU Research Challenge Grant.
   }}
\newcommand{\warning}{
   {\small\noindent The author will neither assume responsibility for any
   damages caused by the use of this product, nor accept warranty or update
   claims.  This product is distributed in the hope that it will be useful,
   but WITHOUT ANY WARRANTY; without even the implied warranty of
   MERCHANTABILITY or FITNESS FOR A PARTICULAR PURPOSE.

   \small\noindent This product is in the public domain,
   and may be freely distributed.

   \small\noindent {\sf Prop} is a research prototype. 
    The information contained in this document is subject
    to change without notice.
   }}

   \makeindex

%%%%%%%%%%%%%%%%%%%%%%%%%%%%%%%%%%%%%%%%%%%%%%%%%%%%%%%%%%%%%%%%%%%%%%%%%%%%%%
%
%   Beginning of the document
%
%%%%%%%%%%%%%%%%%%%%%%%%%%%%%%%%%%%%%%%%%%%%%%%%%%%%%%%%%%%%%%%%%%%%%%%%%%%%%%
\documentclass{article}
\usepackage{html}

   \pagestyle{myheadings}

   \title{{\Huge \bf Prop \\ 
           Language Reference Manual} \\ 
          {\large \bf (for version \Version)}
         } 
   \author{{\Large Allen Leung} \\ 
           E-mail: \email \\
           \htmladdnormallink{\web}{http://valis.cs.nyu.edu:8888/~leunga}
	   \\
           Courant Institute of Mathematical Sciences \\
           251 Mercer Street \\
           New York, NY 10012 \\
          }

   \setlength{\textwidth}{6.6in}
   \setlength{\evensidemargin}{0in}
   \setlength{\oddsidemargin}{0in}
   \setlength{\textheight}{8.75in}
   \setlength{\topmargin}{-0.5in}
   \setlength{\parskip}{1mm}
   \setlength{\parindent}{3ex}
   \setlength{\itemsep}{0.5ex}

   \newsavebox{\LINE}
   \savebox{\LINE}[0.0in][l]{\rule[-1ex]{\textwidth}{.01in}}
\begin{document}

   \maketitle
   \vspace{\fill}
   %\support

   \begin{abstract}
    This reference manual provides an introduction to
    \Prop, release \Version.  \Prop{} is a multiparadigm extension of \Cpp,  
    and is designed for building high performance compiler 
    and language transformation systems, using pattern matching and rewriting.  
    This guide describes the syntax and semantics of
    the \Prop{} language and describes how to develop programs
    using the \Prop{} to \Cpp{} translator.

   \end{abstract}
   \warning
   \titlepage
   \tableofcontents
   \bibliographystyle{alpha}

\Section{Introduction} \label{sec:intro}
 
   This is the language reference manual for \Prop, a compiler generator.
\Prop{} can be used to generate lexers, parsers, analysis and 
transformation tools using pattern matching and rewriting.  
The \Prop{} language is a {\em multi-paradigm} 
extension of \Cpp\cite{C++} and
\Cpp{} programs generated from \Prop{} specifications can be compiled
into efficient code.

   Version \Version{} of \Prop{} includes the following major features:
\begin{enumerate}
   \item Algebraic datatypes and pattern matching, and automatic
mapping of algebraic datatypes into \Cpp{} class hierarchies.
   \item A LALR(1) parser generator in the tradition of {\it yacc} and
{\it bison}.  Each grammar is encapsulated into a parser class and 
it is possible to have multiple instances of a parser.
   \item Regular expression string matching and lexer generation.
   \item Transformation of algebraic datatypes using conditional
   rewriting.
   \item In addition, using the view mechanism, 
it is possible to treat externally defined C or \Cpp{}
data structures as if they were \Prop{} defined datatypes.  
The full range of pattern matching and rewriting capability can
be used on these externally defined data..
\end{enumerate}

   In addition, we intend to implement/fix-up the following features in the 
near future:
\begin{enumerate}
   \item Persistence.
   \item Pretty printing/reading.
   \item Visualization of data structures using {\em vcg}\cite{vcg,vcg-manual}.
   \item Automatic mapping of graph structure specifications into
\Cpp{} classes.
   \item Graph rewriting system generation.
\end{enumerate}

\Subsection{Availability}

   The latest release of \Prop{} \Version{} is available from
\begin{quotation}
\htmladdnormallink{\propweb}{http://valis.cs.nyu.edu:8888/~leunga/prop.html}
\end{quotation}
New updates of the source code and documentation will be available
from this site.

\Subsection{Organization of this Reference Manual}
   This reference manual is organized as follows.  Section~\ref{sec:general}
overviews the general features of \Prop.
Section~\ref{sec:parser}
describes the lexer and parser specification formalisms.
Section~\ref{sec:datatype} describes the algebraic datatypes and 
pattern matching features.  Section~\ref{sec:tree-rewriting} describes
the tree rewriting mechanism.  Finally, section~\ref{sec:usage} describes
the process of running the \Prop{} translator.

   We'll use the following extended BNF notation to specify the syntax
of the language.  Terminals are typeset in {\tt typewriter} font,
while non-terminals are typeset in {\nontermfont itallics} font.
Given a syntactic class \N{S}, we use \LIST{\N{S}} to denote one
or more occurrences of \N{S} separated by commas.  Similarly,
\SEQ{\N{S}} denotes one or more occurrences of \N{S} separated by
semi-colons.  Slanted brackets are used to denote an optional occurrence
of a syntactic class.   We'll combined the two forms when we
want to denote zero or more occurrences of some term.
For example, \OPT{\LIST{\N{S}}} denotes
zero or more occurrences of \N{S}.

\Section{General Concepts} \label{sec:general}

\Subsection{Syntactic Conventions}

The syntax of \Prop{} is an extension of \Cpp.  Most new constructs
of \Prop{} starts with new keywords, listed below.  
\begin{center}
\tt
\begin{tabular}{lllllll}
applicative & arb & as & bagof & begin & bitfield & card \\
category & classof & collectable & constraint & dataflow & datatype & declare \\
dequeof & dom & domain & edges: & elsif & end & equiv: \\
exists & expect: & feature & finalizable & forall & fun & function \\
functor & graphrewrite & graphtype & implies: & inference & inherited & inline \\
instantiate & is & law & left: & less & lexeme & listof \\
loop & mapof & match & matchall & matchscan & module & multimapof \\
mutable & nodes: & of & persistent & prec: & printable & priqueueof \\
procedure & queueof & ran & refine & relation & rewrite & right: \\
setof & sharing & signature & space: & syntax & synthesized & then \\
time: & topdown & traced & treeparser & tupleof & unifiable & unique \\
view & where & with & xor: & bottomup: & topdown: & before: \\
preorder: & postorder: & cutrewrite & failrewrite & index: \\
\end{tabular}
\end{center}

   In addition, the following new multi-character symbols are added:
\begin{verbatim}
   <->  <=>  :=  :-  .(   .[
   #[   #(   #{  [|  |]   (|  |)  {|  |} 
\end{verbatim}

\Subsection{Basic syntactic classes}


We'll use \NDEF{Integer}, \NDEF{Char}, \NDEF{Real}, \NDEF{String} 
and \NDEF{Id} to denote the syntactic classes of 
integers, characters, real numbers, strings and identifiers, respectively.
As in \Cpp, characters are quoted with the single quotes \verb|'|, while
strings are quoted with double quotes \verb|"|.  An identifier is
any sequence of alphanumeric of characters that begins with a letter.
In addition, we'll use the syntactic class
\NDEF{Stmt} to denote any valid combination of \Cpp{} and \Prop{}
statements.

\Subsection{Literals}

  In addition to the usual \Cpp{} literals, \Prop{} introduces two new type
of literals: {\em quark literals} and {\em regular expression literals}.
Quark literals are of type \T{Quark} and are written as strings
prefixed by the \verb|#| symbol.
\begin{syntax}
  \NDEF{Quark} & \IS & \T{\#}\N{String} \\
\end{syntax}   

Quarks act like atoms in Lisp, in which name equality implies
pointer equality.  In contrast, in \Cpp{} two string literals that are
identical in value do not necessarily reside in the same address.
Given strings \verb|s1| and \verb|s2|, \verb|strcmp(s1,s2) == 0| does
not imply \verb|s1 == s2|.  Quarks are defined in the library include file
\verb|<AD/strings/quark.h>|.

Regular expression literals are similar to string literals except
that they are quoted by slashes \verb|/|.   We'll discuss
the regular expression in section~\ref{sec:regular-expressions}.

\Subsection{The \Prop{} Language}

The basic \Prop{} language is the same as \Cpp{} with the following
syntactic extensions:
\begin{quotation}
\begin{tabular}{lll}
         & \bf Keywords              & \bf Functions \\
  \bsf 1 & \T{datatype} \T{refine} \T{instantiate}
                                     & Algebraic datatype definitions \\
  \bsf 2 & \T{match} \T{matchall}    & Pattern matching \\
  \bsf 3 & \T{rewrite}               & Rewriting \\
  \bsf 4 & \T{lexeme}                & Lexical category definition \\
  \bsf 5 & \T{matchscan}             & Lexical scanning \\
  \bsf 6 & \T{syntax}                & Parser specification \\
  \bsf 7 & \T{graphtype}             & Graph data structure specification \\
  \bsf 8 & \T{graphrewrite}          & Graph rewriting \\
\end{tabular}
\end{quotation}

Like \Cpp, a \Prop{} is typically divided into the {\em specification},
which defines the data structure and its interface, and the
{\em implementation} parts.   \Prop{} specifications should
be placed in a file with a \verb|.ph| suffix, while an implementation
should be placed in a file with a \verb|.pC|\footnote{The suffixes {\tt .pcc},
{\tt .pCpp} etc. can also be used.}.  The translator will convert each
\verb|.p|$xxx$ into a file with the suffix \verb|.|$xxx$.  The translated
output can then be fed to the \Cpp{} compiler for compilation.
\vspace{\fill}

\Section{Lexer and Parser Generation} \label{sec:parser}

   In this section we'll describe the lexer and parser generation
mechanism of \Prop.  Note that the use of these features is not
mandatory: the user can substitute any other lexer/parser generation tools 
or use hand written lexers and parsers.  
On the other hand, higher level of integration is possible when using
\Prop's lexer and parser mechanisms.

\Subsection{Lexer Specification}
   Lexical analyzers are specified using the \verb|matchscan|
statement.  This construct is responsible for generating the actual string
matching DFA.  The actual buffering mechanisms are provided by the
classes \verb|LexerBuffer|, \verb|IOLexerBuffer| and \verb|IOLexerStack|. 
These classes are part of the support library distributed with \Prop.

   We assume that the user is familiar with lexer generator tools like
{\em lex}\cite{lex} or {\em flex}\cite{flex}.

\Subsubsection{Regular expressions} \label{sec:regular-expressions}
   Figure~\ref{fig:regexp} describes the set of current
supported regular expressions in \Prop.  The syntax is similar
to what is found in {\em lex}, or {\em egrep}.

\begin{figure}[htbp]
\begin{center}
  \begin{tabular}{|l|l|} \hline
    $c$                & matches character $c$ if it is not a meta character \\
    $e_1e_2$           & matches $e_1$ then $e_2$ \\
    \verb|.|           & matches any character except \verb|\n| \\
    $\verb|\|c$        & matches escape sequence $c$ \\
    \verb|^|$e$        & matches $e$ at the start of the line \\
    $e_1\verb.|.e_2$   & matches $e_1$ or $e_2$ \\
    $e\verb|*|$        & matches zero or more $e$ \\
    $e\verb|+|$        & matches one or more $e$ \\
    $e\verb|?|$        & matches zero or one $e$ \\
    \verb|(|$e$\verb|)| & grouping \\
    \verb|<<|$C_1, C_2 \ldots C_n$\verb|>>|$e$ & matches $e$ only
         if we are in a context from one of $C_1, C_2, \ldots C_n$ \\
    \verb|{|{\em lexeme}\verb|}| & matches lexeme \\
  \hline
  \end{tabular}
\end{center}
\caption{\label{fig:regexp} Regular expressions.}
\end{figure}

The symbols \verb%\ [ ] ( ) { } << >> * + . - ? |% are meta characters and are
interpreted non-literally.  The escape character \verb|\| 
can be prepended to the meta characters if they occur as literals in context.  

Precedence-wise, meta characters \verb|*|, \verb|+| and \verb|?| bind tighter
than juxtaposition.  Thus the regular expression
\verb|ab*| means \verb|a(b*)|.  Parenthesis can be used to override
the default precedence.

Character classes are of the form as found in {\em lex}: 
(i) $c_1$\verb|-|$c_2$ denotes the range of characters from
$c_1$ to $c_2$; (ii) single (non-meta) characters denote themselves; 
(iii) the meta character \verb|^| can be used to negate the set.
For example, the regular expression \verb|[a-zA-Z][a-zA-Z0-9]*| 
specifies an alphanumeric identifier that must starts with a letter.
Similarly, the regular expression \verb|[^ \t\n]| matches any character
except a space, a tab or a newline.

{\em Lexemes} are simply abbreviated names given to a regular expression
pattern.  They act like macros in {\em lex}.

While a lexer is scanning, it may be in one of many {\em contexts}.
Contexts can be used to group a set of related lexical rules; such rules
are only applicable when the contexts are active.  This makes the
lexer behave like a set of DFAs, with the ability to switch between
DFAs under programmer control.  


\Subsubsection{Lexeme}

Lexemes are defined using the \T{lexeme} declaration in \Prop. 
Its syntax is
\begin{syntax}
  \NDEF{Lexeme\_Decl} & \IS & \TDEF{lexeme} \ALT{\N{Lexeme\_Eq}}{|} \T{;} \\
  \NDEF{Lexeme\_Eq}   & \IS & \Id \T{=} \N{Regexp} \\
\end{syntax}

For example, the following lexeme definition is used in the \Prop{}
translator to define the lexical structure of common lexical items.

\begin{verbatim}
lexeme digits      = /[0-9]+/
     | sign        = /[\+\-]/
     | integer     = /{digits}/
     | exponent    = /[eE]{sign}?{digits}/
     | mantissa    = /({digits}\.{digits}?|\.{digits})/
     | real        = /{mantissa}{exponent}?/
     | string      = /"([^\\"\n]|\\.)*"/ 
     | character   = /'([^\\'\n]|\\.[0-9a-f]*)'/
     | regexp      = /\/([^\/\n*]|\\.)([^\/\n]|\\.)*\//
     ;
\end{verbatim}

Note that regular expression literals are written between
slashes: \verb|/|$re$\verb|/|.  

\Subsubsection{Lexeme class}

Often we wish to group a set of lexemes together into {\em lexeme classes} if 
they logically behave in some uniform manner; for example, if they
act uniformly in a lexical rule.  By grouping related lexemes together
into a class we can refer to them succinctly by their class name.

The syntax of a lexeme class declaration is

\begin{syntax}
  \NDEF{Lexeme\_Class\_Decl} & \IS & \TDEF{lexeme class} 
       \ALT{\N{Lexeme\_Class\_Eq}}{and} \T{;} \\
  \NDEF{Lexeme\_Class\_Eq} & \IS & \Id \T{=} \ALT{\N{Lexeme\_Spec}}{|} \\
  \NDEF{Lexeme\_Spec} & \IS & \N{String} \\
                   & \OR & \Id \N{Regexp} \\
\end{syntax}

The following example,
the lexeme classes \verb|MainKeywords|, \verb|Symbols|, and \verb|Literals| 
are defined.  
\begin{verbatim}
lexeme class MainKeywords =
    "rewrite" | "inference" | "match" | "matchall" | "matchscan"
  | "refine"  | "classof" | "type" | "datatype" | "instantiate"
  | "lexeme"  | "bitfield" | "begin" | "syntax"
  | "dataflow" | "module" | "signature" | "constraint" | "declare"
  | "procedure" | "fun" | "function" | "domain" 
  | "graphtype" | "graphrewrite"

and  Symbols =
    ".." | "..." | "<->" | "::" | "&&" | "||" | "++" | "--" | "->" 
  | "<<" | ">>" | ">=" | "<=" | "+=" | "-=" | "*=" | "/=" | "%=" | "==" 
  | "!=" | "<<=" | ">>=" | "&=" | "|=" | "^=" | "=>" | "<-" | "<=>"
  | ":=" | ":-" | LONG_BAR /\-\-\-\-\-+/

and Literals =
    INT_TOK     /{integer}/
  | REAL_TOK    /{real}/
  | CHAR_TOK    /{character}/
  | STRING_TOK  /{string}/
  ;
\end{verbatim}

\Subsubsection{Tokens}

Lexical tokens are defined using the \verb|datatype| declaration. 
The syntax is as follows.
\begin{syntax}
  \NDEF{Tokens\_Decl} & \IS & \TDEF{datatype} \Id \T{::} \TDEF{lexeme} \T{=} \\
      & & \quad \ALT{\N{Token\_Spec}}{|} \T{;} \\
  \NDEF{Token\_Spec} & \IS & \TDEF{lexeme class} \Id & 
       \C{Include a lexeme class} \\
                  & \OR & \Id \OPT{\N{Regexp}}  &
       \C{Single token spec} \\
\end{syntax}

A token datatype is defined by including one of more lexeme classes,
and by defining stand-alone tokens.  If a lexeme class is included,
all the tokens defined within the lexeme class are included.  A \Cpp{}
\T{enum} type of the same name as the token datatype is generated.
If a token is given an identifier name, then the same name is used
as the \T{enum} literal.  On the other hand, if a string is used
to denote a token, then it can be referred to by prefixing the
string with a dot \verb|.|  For example, the token \verb|"=>"| can
be referenced as \verb|."=>"| within a program.

As an example, the following token datatype definition is used within
the \Prop{} translator.  Here, the keywords are first partitioned into
6 different lexeme classes.  In additional, the tokens \T{ID\_TOK},
\T{REGEXP\_TOK}, etc. are defined.

\begin{verbatim}
datatype PropToken :: lexeme =
    lexeme class MainKeywords
  | lexeme class Keywords
  | lexeme class SepKeywords
  | lexeme class Symbols
  | lexeme class Special
  | lexeme class Literals
  | ID_TOK       /{patvar}/
  | REGEXP_TOK   /{regexp}/
  | QUARK_TOK    /#{string}/
  | BIGINT_TOK   /#{sign}{integer}/
  | PUNCTUATIONS /[\<\>\,\.\;\&\|\^\!\~\+\-\*\/\%\?\=\:\\]/
  ;
\end{verbatim}

\Subsubsection{The \T{matchscan} statement}\label{sec:matchscan}

The \T{matchscan} statement is used to perform tokenization.  The user can
specify a set of string pattern matching rules within a \T{matchscan}
construct. Given a object of class \verb|LexerBuffer|, the \T{matchscan}
statement looks for the rule that matches the longest prefix from
the input stream and executes the action associated with the rule.
Ties are broken by the lexical ordering of the rules.  

The general syntax is as follows:
\begin{syntax}
\NDEF{Matchscan} 
      & \IS & \N{Matchscan\_Mode} \OPT{\N{Context\_Spec}} \T{(} \N{Exp} \T{)} 
             & variant 1 \\
      & &  \T{\{} \ALT{\T{case} \N{Matchscan\_Rule}}{} \T{\}} \\
      & \OR & \N{Matchscan\_Mode} \OPT{\N{Context\_Spec}} 
            \T{(} \N{Exp} \T{)} & variant 2 \\
      & &   \T{\{} \ALT{\N{Matchscan\_Rule}}{|} \T{\}} \\
      & \OR & \N{Matchscan\_Mode} \OPT{\N{Context\_Spec}} \T{(} \N{Exp} \T{)} 
            \T{of} & variant 3 \\
      & & \quad \ALT{\N{Matchscan\_Rule}}{|} \\
      & & \T{end} \T{matchscan} \T{;}\\ 
\NDEF{Matchscan\_Mode} & \IS & 
           \TDEF{matchscan} \OPT{\T{while}} & case sensitive\\
   & \OR & \TDEF{matchscan*} \OPT{\T{while}}& case insensitive\\
\NDEF{Context\_Spec} & \IS & \T{[} \LIST{\N{Id}} \T{]} \\
\NDEF{Matchscan\_Rule} 
      & \IS & \OPT{\T{<<} \ALT{\N{Context}}{,} \T{>>}} \\
      &     & \quad \T{lexeme} \T{class} \N{Id} \T{:} \N{Matchscan\_Action} \\
      & \OR & \OPT{\T{<<} \ALT{\N{Context}}{,} \T{>>}} \\
      &    &  \quad \N{Regexp} \T{:} \N{Matchscan\_Action} \\
\NDEF{Matchscan\_Action} & \IS & \T{\{} \N{Code} \T{\}} \\
          & \OR & \N{Code} & for variant 1 only \\
\end{syntax}

The two different modes of operation are \T{matchscan} and \T{matchscan*},
which respectively match strings case sensitively and insensitively.
The modifier \T{where} may optionally specify that the matching process
should repeat until no rules apply, or the end of stream condition
is reached.  

By default, if no rules apply and if the input stream is non-empty,
then an error has occurred.  The \T{matchscan} statement will
invoke the method \verb|error()| of the \T{LexerBuffer} object by default.

For example, the following is part of the \Prop{} lexer specification.

\begin{verbatim}
   datatype LexicalContext = NONE | C | PROP | COMMENT | ...;

int PropParser::get_token()
{
   matchscan[LexicalContext] while (lexbuf)
   {  
      ...
   |  <<C>> /[ \t\\\014]/:            { emit(); }
   |  <<C>> /(\/\/.*)?\n/:            { emit(); line++; }
   |  <<C>> /^#.*/:                   { emit(); }
   |  <<PROP>> lexeme class MainKeywords: { return ?lexeme; }
   |  <<PROP>> lexeme class SepKeywords: { return ?lexeme; }
   |  <<PROP>> QUARK_TOK:             { return QUARK_TOK; }
   |  <<PROP>> BIGINT_TOK:            { return BIGINT_TOK; }
   |  <<PROP>> REGEXP_TOK:            { return REGEXP_TOK; }
   |  <<PROP>> PUNCTUATIONS:          { return lexbuf[0]; }
   |  <<PROP>> /[ \t\014]/:           { /* skip */ }
   |  <<PROP>> /(\/\/.*)?\n/:         { line++; }
   |  /\/\*/:                         { emit(); set_context(COMMENT); }
   |  <<COMMENT>> /\*\//:             { emit(); set_context(PROP); }
   |  <<COMMENT>> /\n/:               { emit(); line++; }
   |  <<COMMENT>> /./:                { emit(); }
   |  /./: { error("%Lillegal character %c\n", lexbuf[0]); }
   }
}
\end{verbatim}

Here, the lexer is partitioned in multiple lexical contexts: context
\verb|C| deals with \Cpp{} code while context \verb|PROP| deals with
\Prop{} extensions.  The special context \verb|COMMENT| is used
to parse \verb|/* */| delimited comments.  Contexts are changed
using the \verb|set_context| method defined in class \verb|LexerBuffer|.

The special variable \verb|?lexeme| can be used within a rule 
that matches a lexeme class.  For example, within the rule
\begin{verbatim}
   |  <<PROP>> lexeme class MainKeywords:    { return ?lexeme; }
\end{verbatim}
the variable \verb|?lexeme| is bound to 
the token \verb|."rewrite"| if the string ``rewrite'' is matched;
it is bound to the token \verb|."inference"| if the string
``inference'' is matched and so on.

\Subsubsection{Class \T{LexerBuffer}}
We'll next describe the class \T{LexerBuffer} and its subclasses.

Class \CLASSDEF{LexerBuffer} 
is the base class in the lexical buffer hierarchy.
It is defined in the library include file \verb|<AD/automata/lexerbuf.h>|.
This class is responsible for implementing a string buffer for use
during lexical analysis.   

As it stands, it can be used directly if the lexer input is directly
from a string.  Memory management of the buffer is assumed to be handled
by the user. 

The class \T{LexerBuffer} has three constructors.  The default
constructor initializes the string buffer to NULL.  The two
other constructors initialize the string buffer to a string given
by the user.  In the case when the length is not supplied, the buffer
is assumed to be \verb|'\0'|-terminated.  The two \verb|set_buffer| methods
can be used to set the current string buffer.  Notice that
all lexical analysis operations are done in place.  The user should
not alter the string buffer directly, but should use the interface
provided by this class instead.

\begin{verbatim}
class LexerBuffer {
public:
   LexerBuffer();    
   LexerBuffer(char *);    
   LexerBuffer(char *, size_t);    
   virtual ~LexerBuffer();
   virtual void set_buffer (char *, size_t);
           void set_buffer (char *);
};
\end{verbatim}

The following methods are used access the string buffer.
Method \T{capacity} returns the size of the buffer.  Method
\T{length} returns the length of the current matched token.
Methods \T{text} can be used to obtain a point to location
of the current matched token.  The string returned is guaranteed
to be \verb|'\0'|-terminated.  Methods \T{operator []} return
the $i$th character of the token.  Finally, method \T{lookahead}
returns the character code of the next character to be matched.

\begin{verbatim}
   int capacity () const;
   int length   () const;
   const char * text () const;
         char * text ();
   char  operator [] (int i) const;
   char& operator [] (int i);
   int lookahead () const;
   void push_back (int n) 
\end{verbatim}

In addition to the string buffer, the class \T{LexerBuffer} keeps
track of two additional types of information: the current context
of the DFA, and whether the next token starts at the
beginning of the line, or in our terminology, whether it is
{\em anchored}.  These are manipulated with the following methods:

\begin{verbatim}
   int context    () const;
   void set_context (int c = 0);
   Bool is_anchored() const;
   void set_anchored(Bool a = true);
\end{verbatim}

Finally, the following methods should be redefined by subclasses
to alter the behavior of this class.  By default, the class \T{LexerBuffer}
calls \T{fill\_buffer()} when it reaches the end of the string; subclasses
can use this method to refill the buffer and return the number of
characters read.  Currently, \T{fill\_buffer} is defined to do nothing 
and return 0.  When it reaches the end of the file 
(i.e. when \T{fill\_buffer()} fails to refill the buffer and the 
scanning process finishes), method \T{end\_of\_file} is called.
Currently, this is a no-op.  Finally, the error handling routine
\T{error()} is called with the position of the beginning and the end
of the buffer in which the error occurs.  By default, this routine
prints out the buffer. 

\begin{verbatim}
protected:           
   virtual size_t fill_buffer();
   virtual void   end_of_file();
   virtual void   error(const char * start, const char * stop);
\end{verbatim}

\Subsubsection{Class \T{IOLexerBuffer}}

  Class \CLASSDEF{IOLexerBuffer} 
is a derived class of \T{LexerBuffer}.  It 
automatically manages an internal buffer and receives input from
an \T{istream}.  It is defined in the file \verb|<AD/automata/iolexerbuf.h>|

This class provides the following additional features:
Constructor \verb|IOLexerBuffer(istream&)| ties the input to a
stream.  If the default constructor is used, then it automatically
ties the input to the stream \verb|cin|.  The method \verb|set_stream|
can be used to set the stream to some other input. 

\begin{verbatim}
class IOLexerBuffer : public LexerBuffer {
   size_t    buffer_size;  // size of the buffer
   istream * input;        // input stream
public:
   IOLexerBuffer();    
   IOLexerBuffer(istream&);    
   virtual ~IOLexerBuffer();
   virtual void set_stream (istream&);
};
\end{verbatim}

By default, class \verb|IOLexerBuffer| reads new data from the input
stream using a line buffering discipline.  This mechanism is suitable
for interactive, but for other applications it may be more efficient
to use block buffered input.  The protected method 
\verb|read_buffer| controls this reading behavior; it
is called whenever the data in the string buffer has been
consumed and new input is needed from the stream.  It is
passed the position of the buffer and its remaining capacity,
and it returns the number of new characters that are read.  Subclasses
can redefine this method. 

\begin{verbatim}
protected:
   virtual size_t read_buffer(char *, size_t);
\end{verbatim}

\Subsubsection{Class \T{IOLexerStack}}

Class \CLASSDEF{IOLexerStack}
is a derived class of \T{LexerStack}.  It provides
a mechanism of reading from a stack of \verb|istream|'s.  Streams
can be pushed and popped from the stack.  The next token is 
obtained from the stream from the top of the stack.  The class allows
allows easy implementation of 
constructs such as the \verb|#include| file mechanism of the C preprocessor.

The interface of this class is listed below:
\begin{verbatim}
class IOLexerStack : public IOLexerBuffer {
public:
   IOLexerStack();    
   IOLexerStack(istream&);    
   virtual ~IOLexerStack();
   
   virtual void     push_stream (istream&);
   virtual istream& pop_stream  ();
};
\end{verbatim}

\Subsection{Parser Specification}

Parsers are specified as a two phase process:
\begin{enumerate}
   \item First a {\em syntax class} is defined.  A syntax class
declaration is like a normal \Cpp{} class declaration and has a similar
syntax, except that \Prop{} will also generate the interface of the parser. 
All parsers that \Prop{} generates are encapsulated within a class.  This
makes it easy for the programmer to add additional data for parsing
use, and to have multiple instances of a parser. 
   \item Secondly, the grammar of the language is defined in a
{\em syntax} declaration as a set of productions, 
in a syntax similar to that of {\em yacc}.
\end{enumerate}

We'll describe these two phases.

\Subsubsection{Syntax class}

A syntax class definition specifies an object class that encapsulates
a parser.  Its syntax is the same as the usual \Cpp{} class definition,
except that the prefix \T{syntax} is used: 

\begin{syntax}
\NDEF{Syntax\_Class\_Decl} & \IS & \TDEF{syntax class} \Id 
   \OPT{\T{:} \N{Inherit\_List}} \T{\{} \N{Class\_Body} \T{\}} \T{;} \\
\end{syntax}

This specification automatically generate a class with the following interface
(assuming that the parser class in question is called \T{ParserClass})
\begin{verbatim}
public:  
   ParserClass();  // constructor
public:
   virtual void parse();
\end{verbatim}

The constructor and the method \verb|parse()| will be automatically
generated later. 

A parser class expects the following method to be defined:
\begin{verbatim}
   int get_token();
\end{verbatim}
\noindent The function of \verb|get_token| is to return the
next token from the lexer stream whenever it is called.
It should return \verb|EOF| if the stream is empty.

\Subsubsection{Syntax declaration}

The grammar of a language is specified in the \T{syntax} declaration.
Its syntax is listed as follows:

\begin{syntax}
\NDEF{Syntax\_Decl} & \IS & \TDEF{syntax} \Id \T{\{} \\
    & & \quad \OPT{\MORE{\N{Precedence\_Decl}}{}} \\
    & & \quad \OPT{\N{Expect\_Decl}} \\
    & & \quad \OPT{\MORE{\N{Production\_Rule}}{}} \\
    & & \T{\}} \\
\end{syntax}

The name of a \T{syntax} declaration should match that
of a \T{syntax} \T{class}.  A \T{syntax} declaration is divided
into three parts: (i) the operator precedence section, (ii) the
\T{expect:} $n$ section, and (iii) the grammar section.

\Subsubsection{Precedence}

Operator precedences are specified using precedence declarations.
The syntax is: 

\begin{syntax}
\NDEF{Precedence\_Decl} 
    & \IS & \TDEF{left:} \N{Integer} \ALT{\N{Operator}}{} \T{;} 
          & left associative operators \\
    & \OR & \TDEF{right:} \N{Integer} \ALT{\N{Operator}}{} \T{;} 
    & right associative operators \\
\NDEF{Operator} & \IS & \N{Char}   & single character operator \\
             & \OR & \N{String} & multi-character operator \\
             & \OR & \N{Cons}   & constructor \\
\end{syntax}

The integer associated with a precedence declaration is the precedence
level of an operator.  The higher the level of an operator, the {\em lower} 
its precedence.

For example, the following set of precedences are used in the \Prop{}
language parser:
\begin{verbatim}
left: 23 "as";
left: 22 "::";
left: 21 "||";
left: 20 "equiv:";
left: 19 "xor:";
left: 18 "implies:";
left: 17 "&&" "and";
right: 16 "|=" "&=" "^=" "<<=" ">>=";
right: 15 '=' ":=" "+=" "-=" "*=" "/=" "%=";
left: 14 '|';
left: 13 ':';
left: 12 ';';
left: 11 '^';
left: 10 '&';
left: 9 "==" "!=";
left: 8 '<' '>' ">=" "<=";
left: 7 "<<" ">>";
left: 6 '+' '-' "with" "less";
left: 5 '*' '/' '%';
left: 4 "++" "--";
left: 3 '!' '~' "arb" "card" "dom" "ran";
left: 2 '[' ']' '{' '}' '.' "->" ;
\end{verbatim}

\Subsubsection{{\tt expect:} $n$}

The declaration \T{expect} $n$ specifies that there should be $n$
shift/reduce errors in the grammar.  If the actual grammar deviates
from this number, a warning will be generated.  The alternative
form \T{expect:} \T{\_} suppresses all warnings.  The syntax is:

\begin{syntax}
\NDEF{Expect\_Decl}    & \IS & \TDEF{expect:} \N{Integer} \T{;}
    & expects $n$ shift/reduce errors \\
                     & \OR & \T{expect:} \T{\_} \T{;}
    & expects many shift/reduce errors \\
\end{syntax}

This declaration is optional.  If omitted, warnings will be printed
if parser conflicts are found during parser generation.

\Subsubsection{Production rules}


The syntax of the production rules is as follows: 

\begin{syntax}
\NDEF{Production\_Rule} & \IS & 
  \Id \OPT{\TypeExp} \T{:} \\
  & & \quad \OPT{\ALT{\N{One\_Alt}}{|}} \T{;} \\
\NDEF{One\_Alt}  & \IS & \OPT{\ALT{\N{Symbol}}{}} & one alternative \\
\NDEF{Symbol} 
    & \IS & \Id        & non-terminal \\
    & \OR & \N{Cons}   & terminal (a datatype constructor) \\
    & \OR & \N{String} & terminal (a datatype constructor) \\
    & \OR & \N{Char}   & terminal (from the ASCII character set) \\
    & \OR & \T{?}      & the error terminal \\
    & \OR & \T{\$}     & the end of file terminal \\
    & \OR & \T{\{} \N{Code} \T{\}} & embedded actions \\
\end{syntax}

Each rule specifies a set of productions of the form 
\[ \begin{array}{lcl}
    lhs &  :  & A_{11} A_{12} \ldots A_{1n_1} \\
        & \OR & A_{21} A_{22} \ldots A_{2n_2} \\
        & \vdots & \\
        & \OR & A_{m1} A_{m2} \ldots A_{mn_m} \\
   \end{array}
\]
where $lhs$ is the non-terminal and $A_{ij}$'s are a mixture of 
terminals, non-terminals and embedded actions.  If synthesized
attributes are used, then the type of the s-attribute should be
annotated next to the lhs non-terminal.

Terminals can be specified in a few ways: (i) a character literal
is a predefined terminal of the same name, (ii) a string literal
is matched with a \T{datatype} or a \T{lexeme} \T{class} constructor
of the same name, (iii) an identifier is a terminal if there is a 
\T{datatype} constructor of the same name.  Any other identifiers
are assumed to be non-terminals.

For example, the following are two sets of production rules used in the 
\Prop{} translator itself:
\begin{verbatim}
ty Ty:  simple_ty               { $$ = $1; }
|       ty '=' exp              { $$ = DEFVALty($1,$3); }
|       ty '*'                  { $$ = mkptrty($1); }
|       ty '&'                  { $$ = mkrefty($1); }
|       ty '[' ']'              { $$ = mkptrty($1); }
|       ty '[' exp ']'          { $$ = mkarrayty($1,$3); }
;
exp Exp:
        app_exp                 { $$ = $1; }
|       exp '+' exp             { $$ = BINOPexp("+",$1,$3); }
|       exp '-' exp             { $$ = BINOPexp("-",$1,$3); }
|       exp '*' exp             { $$ = BINOPexp("*",$1,$3); }
|       exp '/' exp             { $$ = BINOPexp("/",$1,$3); }
|       exp '%' exp             { $$ = BINOPexp("%",$1,$3); }
|       exp '^' exp             { $$ = BINOPexp("^",$1,$3); }
|       exp "+=" exp            { $$ = BINOPexp("+=",$1,$3); }
|       exp "-=" exp            { $$ = BINOPexp("-=",$1,$3); }
|       exp "*=" exp            { $$ = BINOPexp("*=",$1,$3); }
|       exp "/=" exp            { $$ = BINOPexp("/=",$1,$3); }
|       exp "%=" exp            { $$ = BINOPexp("%=",$1,$3); }
\end{verbatim}
\noindent {\em etc $\ldots$}

Here, \T{ty} is the non-terminal of this production.  The type of
the synthesized attribute of this rule is \T{Ty}, and it is
written next to the non-terminal.  Like {\em yacc}, the synthesized
attribute of a rule can be referenced using the \verb.$. variables: 
variable \TDEF{\$\$} denote the synthesized attribute of the current rule,
while \TDEF{\$$n$} where $n$ is 1, 2, 3, \ldots denote the synthesized
attribute of the $n$th rhs symbol of the current rule.

\Subsubsection{Parser report}

If the command line option \OPTIONDEF{-r} is used, then
in addition to generating the code for the grammar, the translator
will also generate a comprehensive report of the grammar and
the resulting LALR(1) automaton.  The amount of details in this
report can be varied using the the verbose option
\OPTIONDEF{-v}:

\begin{description}
   \item[{\tt -v1}]  Print full information for all LR(0) states. 
   \item[{\tt -v2}]  Also print out the lookahead set of a reduction rule.
This may be useful for tracking down shift/reduce and reduce/reduce conflicts.
   \item[{\tt -v3}]  Do both \verb|-v1| and \verb|-v2|.
\end{description}

\Subsubsection{Interfacing with the generated lexer}

The easiest way of interfacing with a lexer is to embed
an object of class \T{LexerBuffer}, \T{IOLexerBuffer}, or \T{IOLexerStack}
within a syntax class\footnote{Alternatively, inheritance may be used.}.
We can then define the tokenization method
\verb|get_token| using the \T{matchscan} statement described
in section~\ref{sec:matchscan}.

\Subsection{Debugging Parsers}

Currently, only the most rudimentary debugging facilities have been
implemented for parser.
The user can define the macro \verb|DEBUG_|$C$, where $C$ is the
name of a syntax class before the \T{syntax} definition.  This will activate
the parser debugging code.  During parsing, the reductions that are taken
will be printed to \verb|cerr|.

\Section{Algebraic Datatypes and Pattern Matching} \label{sec:datatype}

    \Prop{} implements algebraic datatypes and pattern matching in the
style of Standard ML\cite{SML}.  Tree, DAG and even graph structures can be
specified as a set of datatype equations in algebraic datatype specifications.
\Prop{} then proceeds to translate these into concrete \Cpp{} classes.  
This makes it very easy to implement complex data structures.

   Algebraic datatypes can be manipulated and transformed
using \Prop{} pattern matching constructs, which decompose a datatype
value into its constituents.  Complex patterns involving multiple
objects can be specified.  These are translated into efficient 
testing and selection code in \Cpp.

In the following we shall give a brief overview of the pattern matching
features of \Prop.  For most users of modern declarative languages many
of these features are already familiar constructs.
 
\Subsection{A quick tour of pattern matching}
 
   Algebraic datatypes are specified using \verb|datatype| definitions,
which define the inductive structure of one of more types using
a tree-grammar like syntax.  
%In addition, pretty printers,
%lexical scanners, parsers, persistence I/O methods and
%garbage collection inferences can also be specified with
%additional options in the same construct.  
When a datatype is declared,
the following operations are implicitly defined by the datatype compiler:
(1) the constructors for all the variants of a type;
(2) the identity test operator \verb|==|, and the assignment
    operator \verb|=| for this type; and
(3) the member functions needed to decompose a datatype value during
    pattern matching.
 
   We'll select the internals of a compiler for a simplified imperative
language as the running example in this section.
Suppose that in this language an expression is composed of
identifiers, integer constants and the four arithmetic operators.
Then the structure of the abstract syntax tree can be specified as follows:
 
\begin{verbatim}
   datatype Exp = INT (int)
                | ID  (const char *)
                | ADD (Exp, Exp)
                | SUB (Exp, Exp)
                | MUL (Exp, Exp)
                | DIV (Exp, Exp)
                ;
\end{verbatim}

Abstract syntax of an expression such as {\em a * b - 17}
can be constructed directly in a prefix syntax, directly mirroring
that of the definition.  The \Prop{} datatype
compiler will automatically generate a \Cpp{} class hierarchy to represent
the variants of type \verb|Exp|.  Datatype constructor
functions(not to be mistaken with \Cpp's class constructors) will
also be automatically generated using the same names as the variants.
 
\begin{verbatim}
   Exp formula = ADD(MUL(ID("a"),ID("b")),INT(17));
\end{verbatim}
 
Datatype values can be decomposed using the \verb|match| statement, which
can be seen as a generalization of \C's \verb|switch| construct.  Pattern
matching is a combination of conditional branching and value
binding.  For example, a typical evaluation function for the type \verb|Exp|
can be written as in the following example.  Notice that each arm of
a \verb|case| is in fact a pattern(with optional variables)
mirroring the syntax of a datatype.  The pattern variables(written
with the prefix \verb|?| in the sequel) of a matching
arm is {\em bound} to the value of the matching value, which can be
subsequently referenced and modified in the action of an arm.
 
\begin{verbatim}
int eval (Exp e, const map<const char *, int>& env)
{  match (e)
   {  case INT ?i:        return ?i;
      case ID  ?id:       return env[?id];
      case ADD (?e1,?e2): return eval(?e1,env) + eval(?e2,env);
      case SUB (?e1,?e2): return eval(?e1,env) - eval(?e2,env);
      case MUL (?e1,?e2): return eval(?e1,env) * eval(?e2,env);
      case DIV (?e1,?e2): return eval(?e1,env) / eval(?e2,env);
   }
}
\end{verbatim}
 
\Subsubsection{Pattern matching versus object-oriented style}
 
  Although a comparable evaluation function can be written
in object oriented style using late binding, as in below, in general pattern
matching is much more powerful than late binding in \Cpp, which only
allows dispatching based on the type of one receiver.
 
\begin{verbatim}
  // Class definitions
  class Exp {
  public:
     virtual int eval(const map<const char *, int>& env) const = 0;
  };
  class INT : Exp {
     int i;
  public:
     int eval(const map<const char *, int>& env);
  };
  class ID : Exp {
     const char * id
  public:
     int eval(const map<const char *, int>& env);
  };
  ...
 
  // Member functions
  int INT::eval(const map<const char *, int>& env) const { return i; }
  int ID ::eval(const map<const char *, int>& env) const { return id; }
  int ADD::eval(const map<const char *, int>& env) const
     { return e1->eval(env) + e2->eval(env); }
  int SUB::eval(const map<const char *, int>& env) const
     { return e1->eval(env) - e2->eval(env); }
  int MUL::eval(const map<const char *, int>& env) const
     { return e1->eval(env) * e2->eval(env); }
  int DIV::eval(const map<const char *, int>& env) const
     { return e1->eval(env) / e2->eval(env); }
\end{verbatim}

For example, in the following function we use nested patterns,
non-linear patterns (i.e. patterns with multiple occurrences of
a pattern variable), and guards to perform algebraic simplification
of an expression.  Although the patterns are relatively simple in
this example, in general arbitrarily complex patterns may be used.
 
\begin{verbatim}
Exp simplify (Exp redex)
{  // recursive traversal code omitted ...
 
   // match while repeats the matching process
   // until no more matches are found.
   match while (redex)
   {  ADD(INT 0,  ?x):     { redex = ?x; }
   |  ADD(INT ?x, INT ?y): { redex = INT(?x+?y); }
   |  ADD(?x,     INT 0)   { redex = ?x; }
   |  SUB(?x,     INT 0):  { redex = ?x; }
   |  SUB(?x,     ?x):     { redex = INT(0); }
   |  SUB(INT ?x, INT ?y): { redex = INT(?x-?y); }
   |  MUL(INT 0,  ?x):     { redex = INT(0); }
   |  MUL(?x,     INT 0):  { redex = INT(0); }
   |  DIV(?x,     ?x):     { redex = INT(1); }
        // don't divide by zero.
   |  DIV(INT ?x, INT ?y) | ?y != 0: { redex = INT(?x/?y); }
   |  ...
   }
   return redex;
}
\end{verbatim}
 
Pattern matching in \Prop{} is not restricted to one datatype at
a time.  In the following example, we use matching on multiple values to
define equality on expressions inductively.  For variety, we'll
use the \verb|fun| variant of \verb|match|, which defines a function
in rule form.  Notice that the last case
of the match set uses {\em wild cards} \verb|_| to catch all the other
non-equal combinations.  Since \Cpp{} does not provide multiple dispatching,
implementing binary (or $n$-ary) matching operations on variant datatypes are
in general cumbersome and verbose in object-oriented style. In contrast,
using an applicative pattern matching style many manipulations and
transformations on variant datatypes with tree-like or graph-like structure
can be expressed succinctly.
 
\begin{verbatim}
fun equal INT ?i,     INT ?j: bool: { return ?i == ?j; }
  | equal ID  ?a,     ID  ?b:       { return strcmp(a,b) == 0; }
  | equal ADD(?a,?b), ADD(?c,?d):   { return equal(?a,?c) && equal(?b,?d); }
  | equal SUB(?a,?b), SUB(?c,?d):   { return equal(?a,?c) && equal(?b,?d); }
  | equal MUL(?a,?b), MUL(?c,?d):   { return equal(?a,?c) && equal(?b,?d); }
  | equal DIV(?a,?b), DIV(?c,?d):   { return equal(?a,?c) && equal(?b,?d); }
  | equal _,          _:            { return false; }
  ;
\end{verbatim}

\Subsubsection{More examples} \label{sec:Wff}
 
  As another example, we can specify the term structure of
{\em well-formed formulas} in proposition calculus as follows.  Notice
that the constructors \verb|F| and \verb|T| are nullary.
 
\begin{verbatim}
   datatype Wff = F
                | T
                | Var     (const char *)
                | And     (Wff, Wff)
                | Or      (Wff, Wff)
                | Not     (Wff)
                | Implies (Wff, Wff)
                ;
\end{verbatim}
 
Datatypes that are parametrically polymorphic, such as lists and trees, can
be defined by parameterizing them with respect to one of more types.
For example, both lists and tree below are parametric on one type argument
\verb|T|.
 
\begin{verbatim}
   datatype List<T> = nil
                    | cons(T, List<T>);
   datatype Tree<T> = empty
                    | leaf(T)
                    | node(Tree<T>, T, Tree<T>);
 
   List<int> primes = cons(2,cons(3,cons(5,cons(7,nil))));
   List<int> more_primes = cons(11,cons(13,primes));
   Tree<char *> names = node(leaf("Church"),"Godel",empty);
\end{verbatim}
 
As a programming convenience, \Prop{} has a set of built-in list-like
syntactic forms.  Unlike languages such as ML, however,
these forms are not predefined to be any specific list types.  Instead, it is
possible for the user to use these forms on any datatypes with
a natural binary {\em cons} operator and a nullary {\em nil} constructor.
For instance, the previous list datatype can be redefined as follows:
 
\begin{verbatim}
   datatype List<T> = #[] | #[ T ... List<T> ];
   List<int> primes = #[ 2, 3, 5, 7 ];
   List<int> more_primes = #[ 11, 13 ... primes ];
   List<char *> names = #[ "Church", "Godel", "Turing", "Curry" ];
 
   template <class T>
      List<T> append (List<T> a, List<T> b)
      {  match (a)
         {  case #[]:          return b;
            case #[hd ... tl]: return #[hd ... append(tl,b)];
         }
      }
\end{verbatim}
The empty list is written as \verb|#[]|, while {\em cons(a,b)}
is written as \verb|#[ a ... b ]|.   An expression of the special form
\verb|#[a, b, c]|, for instance, is simple syntactic sugar for
repeated application of the {\em cons} operator, i.e.
 
\begin{verbatim}
    #[a, b, c] == #[a ... #[ b ... #[ c ... #[] ] ] ].
    #[a, b, c ... d ] == #[a ... #[ b ... #[ c ... d ] ] ].
\end{verbatim}
 
List-like special forms are not limited to datatypes with only two variants.
For example, we can define a datatype similar in structure to S-expressions
in {\em Lisp} or {\em scheme}.  Here's how such as datatype may be
defined\footnote{for simplicity, we'll use a string representation for atoms
instead of a more efficient method.}:
 
\begin{verbatim}
   datatype Sexpr = INT    (int)
                  | REAL   (double)
                  | STRING (char *)
                  | ATOM   (const char *)
                  | #()
                  | #( Sexpr ... Sexpr )
   where type Atom = Sexpr  // synonym for Sexpr
   ;
\end{verbatim}
 
   With this datatype specification in place, we can construct values
of type \verb|Sexpr| in a syntax close to that of {\em Lisp}.  For example,
we can define lambda expressions corresponding to the combinators $I$, $K$
and $S$ as follows:
 
\begin{verbatim}
   Atom LAMBDA = ATOM("LAMBDA");
   Atom f      = ATOM("f");
   Atom x      = ATOM("x");
   Atom y      = ATOM("y");
   Atom NIL    = #();
 
   Sexpr I = #(LAMBDA, #(x), x);
   Sepxr K = #(LAMBDA, #(x), #(LAMBDA, #(y), x));
   Sepxr S = #(LAMBDA, #(f),
                #(LAMBDA, #(x),
                   #(LAMBDA, #(y), #(#(f,x), #(g,x)))));
\end{verbatim}
 
Similar to list-like forms, vector-like forms are also available.
This addresses one of the flaws of the \Cpp{} language, which lacks
first class array constructors.  
Vectors are simply homogeneous arrays whose sizes
are fixed and are determined at creation time\footnote{For efficiency,
bounds checking is {\em not} performed.}.   Random access within
vectors can be done in constant time.  Unlike lists, however, the
prepending operation is not supported.   Vectors literals are delimited
with the composite brackets \verb!(| ... |)!, \verb![| ... |]!, or
\verb!{| ... |}!.  In the following example the datatype \verb|Exp|
uses vectors to represent the cooefficients of the polynomials:
 
\begin{verbatim}
   datatype Vector<T> = (| T |);
   datatype Exp = Polynomial (Var, Vector<int>)
                | Functor (Exp, Vector<Exp>)
                | Atom (Var)
                | ...
   where type Var = const char *;
   Exp formula = Polynomial("X", (| 1, 2, 3 |));
\end{verbatim}
 
Commonly used patterns can be given synonyms so that they can be readily
reused without undue repetition.  This can be accomplished by defining pseudo
datatype constructors to stand for common patterns using
{\em datatype law} definitions.  For example, the following set of
laws define some commonly used special forms for a {\em Lisp}-like language
using the previously defined \verb|Sexpr| datatype.
 
\begin{verbatim}
   datatype law inline Lambda(x,e)  = #(ATOM "LAMBDA", #(x), e)
              | inline Quote(x)     = #(ATOM "QUOTE", x)
              | inline If(a,b,c)    = #(ATOM "IF", a, b, c)
              | inline Nil          = #()
              | inline ProgN(exprs) = #(ATOM "PROGN" ... exprs)
              | SpecialForm  = #(ATOM ("LAMBDA" || "IF" ||
                                       "PROGN" || "QUOTE") ... _)
              | Call(f,args) = ! SpecialForm && #(f ... args)
              ;
\end{verbatim}
 
Note that the pattern \verb|SpecialForm| is meant to match all
special forms in our toy language: i.e. {\em lambdas}, {\em ifs}, {\em progn}'s
and {\em quotes}.  The pattern disjunction connective \verb.||. is used
to link these forms together.  Since we'd like the \verb|Call| pattern
to match only if the S-expression is not a special form, we use
the pattern negation and conjunction operators, \verb|!| and \verb|&&|
are used to screen out special forms.  


With these definitions in place,
an interpreter for our language can be written thus:
 
\begin{verbatim}
   Sexpr eval (Sexpr e)
   {  match (e)
      {  Call(?f,?args):     { /* perform function call */ }
      |  Lambda(?x,?e):      { /* construct closure */ }
      |  If(?e,?then,?else): { /* branching */ }
      |  Quote(?x):          { return ?x; }
      |  ...:                { /* others */ }
      }
   }
\end{verbatim}


As an interesting note, the special form pattern can also be rewritten
using regular expression string matching, as in the following:
 
\begin{verbatim}
 datatype law SpecialForm = #(ATOM /LAMBDA|IF|PROGN|QUOTE/ ... _)
\end{verbatim}
 
In addition, since the law constructors \verb|Lambda|, \verb|Quote|,
\verb|If|, \verb|Nil| and \verb|ProgN| are defined with the \verb|inline|
keyword, these constructors can be used to within expressions as abbreviates
for their rhs definitions.  For example, writing
\verb|Lambda(x,x)| is the same writing \verb|#(ATOM("Lambda"),#(x),x)|.

\Subsubsection{Variants of match}
 
Aside from the usual plain pattern matching constructs, a few variants of the
\verb|match| construct are offered.  We'll briefly enumerate a few of these:
\begin{itemize}
  \item The \verb|matchall| construct is a variant of \verb|match| that
   executes all matching rules(instead of just the first one) in sequence.
  \item Each rule of a \verb|match| statement can have associated cost
    expressions.  Instead of selecting the first matching rule to
    execute as in the default, all matching rules are considered
    and the rule with the least cost is executed.  Ties are broken by
    choosing the rule that comes first lexically.  For example:
 
   \begin{verbatim}
   match (ir)
   {  ADD(LOAD(?r0),?r1) \ e1: { ... }
   |  ADD(?r0,LOAD(?r0)) \ e2: { ... }
   |  ADD(?r0, ?r1)      \ e3: { ... }
   |  ...
   }
   \end{verbatim}
  \item A \verb|match| construct can be modified with the \verb|while|
     modifier, as in the following example.  A match modified thus is
     repeatedly matched until none of the patterns are applicable.
     For example, the following routine uses a \verb|match while| statement
     to traverse to the leftmost leaf.
 
\begin{verbatim}
   template <class T>
   Tree<T> left_most_leaf(Tree<T> tree)
   {  match while (tree)
      {  case node(t,_,_):  tree = t;
      }
      return tree;
   }
\end{verbatim}
  \item Finally, the \verb|matchscan| variant of match can be used to
     perform string matching on a stream.  For example, a simple lexical
     scanner can be written as follows.
 
     \begin{verbatim}
   int lexer (LexerBuffer& in)
   {  matchscan while (in)
      {  /if/:                     { return IF; }
      |  /else/:                   { return ELSE; }
      |  /while/:                  { return WHILE; }
      |  /for/:                    { return FOR; }
      |  /break/:                  { return BREAK; }
      |  /[0-9]+/:                 { return INTEGER; }
      |  /[a-zA-Z_][a-zA-Z_0-9]*/: { return IDENTIFIER; }
      |  /[ \t]+/:                 { /* skip spaces */ }
      |  ... // other rules
      }
   }
     \end{verbatim}
\end{itemize}

\Subsection{Algebraic Datatypes} \label{sec:algebraic-datatypes}
   Algebraic datatypes are defined using the \T{datatype}
construct.  Its syntax is as follows.

\begin{syntax}
\NDEF{Datatype\_Decl} 
   & \IS & \TDEF{datatype} \\
   &     & \quad \OPT{\ALT{\N{Datatype\_Spec}}{and}} \\
   &     & \quad \OPT{\TDEF{where type} \ALT{\N{Type\_Spec}}{and}} \\
   &     & \quad \OPT{\TDEF{law} \ALT{\N{Law\_Spec}}{and}} \\
   &     & \T{;} \\
\end{syntax}

Each \T{datatype} declaration specifies a set of 
(potentially mutually recursive) types, a set of type abbreviations,
and a set of pattern matching abbreviations called {\em laws}.

\begin{syntax}
\NDEF{Datatype\_Spec} 
   & \IS & \Id \OPT{\T{<} \LIST{\N{Id}} \T{>}} \\
   &     & \quad \OPT{\T{:} \N{Inherit\_List}} \\
   &     & \quad \OPT{\T{::} \N{Datatype\_Qualifiers}} \\
   &     & \quad \OPT{\T{=} \N{Cons\_Specs}} \\
\NDEF{Cons\_Specs} 
   & \IS & \ALT{\N{Cons\_Spec}}{|} \\
   &     & \quad \OPT{\T{public:} \N{Datatype\_Body}} \\
\NDEF{Cons\_Spec} 
   & \IS & \N{Simple\_Cons\_Spec}  \\
   &     &  \quad \OPT{\T{with} \T{\{} \N{Class\_Decl} \T{\}}} \\
\NDEF{Simple\_Cons\_Spec} 
   & \IS & \Id \OPT{\N{String}} & unit constructor \\
   & \OR & \N{String}              & unit constructor \\
   & \OR & \Id \OPT{\T{of}} \TypeExp  & constructor with arguments \\
   & \OR & \T{\#[} \T{]}                 & unit list constructor \\
   & \OR & \T{\#\{} \T{\}}               & unit list constructor \\
   & \OR & \T{\#(} \T{)}                 & unit list constructor \\
   & \OR & \T{\#[} \TypeExp \T{...} \TypeExp \T{]}   
      & list constructor \\
   & \OR & \T{\#\{} \TypeExp \T{...} \TypeExp \T{\}} 
      & list constructor \\
   & \OR & \T{\#(} \TypeExp \T{...} \TypeExp \T{)}   
      & list constructor \\
   & \OR & \T{[|} \TypeExp \T{|]} & vector constructor \\
   & \OR & \T{(|} \TypeExp \T{|)} & vector constructor \\
   & \OR & \T{\{|} \TypeExp \T{|\}} & vector constructor \\
\end{syntax} 


Datatypes can be annotated with {\em qualifiers}.  They tell the \Prop{}
translator to generate additional code for each datatype to provide
additional functionality. 

\begin{syntax}
\NDEF{Datatype\_Qualifiers} 
   & \IS & \MORE{\N{Datatype\_Qualifier}}{} \\
\NDEF{Datatype\_Qualifier} 
   & \IS & \TDEF{collectable} & garbage collectable \\
   & \OR & \TDEF{rewrite}     & optimized for tree rewriting \\
   & \OR & \TDEF{persistent}  & generate persistence interface \\
   & \OR & \TDEF{lexeme}      & used for parser/lexer tokens  \\
   & \OR & \TDEF{inline}      & use inlined implementation \\
   & \OR & \TDEF{extern}      & use non-inlined implementation \\
\end{syntax}

Type specifications assign abbreviations to commonly used type expressions. 
They act like \T{typedef}'s in \Cpp, except that \Prop{} can make use
of the type information provided by these specifications.

\begin{syntax}
\NDEF{Type\_Spec} 
   & \IS & \Id \T{=} \TypeExp \\
\end{syntax}

Law specifications are used to define abbreviations to datatype 
patterns\footnote{Patterns are discussed in 
section~\ref{sec:pattern-matching}.}.
They behave like macros in some sense, except that unlike macros, they are
properly type checked by \Prop.  In addition, if the keyword \T{inline} is 
used, then \Prop{} will treat the lhs as an expression and generate 
a function of the same name that can be used to construct the datatype term.

\begin{syntax}
\NDEF{Law\_Spec} 
   & \IS & \OPT{\T{inline}} \Id \OPT{\N{Law\_Arg}} \T{=} \Pat \\
\NDEF{Law\_Arg} 
   & \IS & \Id \\
   & \OR & \T{(} \LIST{\N{Id}} \T{)} \\
\end{syntax}

\Prop{} recognizes the following set of type expressions.  
\begin{syntax}
\NDEF{Type\_Exp}
   & \IS & \N{Type\_Id}                & type name \\
   & \OR & \TypeExp \T{*}              & pointer type \\
   & \OR & \TypeExp \T{\&}             & reference type \\
   & \OR & \N{Type\_Qualifier} \N{Type\_Id}      & qualified type \\
   & \OR & \N{Datatype\_Qualifier} \TypeExp       & annotation  \\
   & \OR & \T{(} \TypeExp \T{)}          & grouping \\
   & \OR & \T{(} \TypeExp \T{,} 
      \LIST{\N{TypeExp}} \T{)} & tuple type \\
   & \OR & \T{(} \TypeExp \T{,} 
      \LIST{\N{TypeExp}} \T{)} & tuple class \\
   & \OR & \T{\{} \LIST{\N{Lab\_Type\_Exp}} \T{\}} & record type \\
   & \OR & \TypeExp \T{=} \N{Exp} & default value \\
\NDEF{Lab\_Type\_Exp} & \IS &
   \Id \T{:} \TypeExp \\
\NDEF{Type\_Qualifier} 
   & \IS & \T{class}       & a class type \\
   & \OR & \T{const}       & constant type \\
   & \OR & \T{rewrite}     & rewritable \\
   & \OR & \T{collectable} & garbage collected \\
   & \OR & \T{persistent}  & persistent object  \\
\end{syntax}

In general, four different types of datatype constructors can be defined:
\begin{description}
 \item[nullary constructors]  These are constructors without an argument.
 \Prop{} maps these into small integers and no heap space will be consumed.
 \item[tuple argument constructors] These are constructors with one
 or more unamed arguments.  
 \item[record argument constructors] These are constructors with one
 or more named arguments written within braces.  For example, in the
 following datatype definition a record argument constructor called
 \verb|Add| is defined:
 \begin{verbatim}
  datatype Exp = Add { left : Exp, right : Exp }
	       | ...
 \end{verbatim}
  To construct an \verb|Add| expression, we can use any one of the following
  forms:
  \begin{verbatim}
     Exp e1 = Add(x,y);
     Exp e2 = Add'{ left = x, right = y };
     Exp e3 = Add'{ right = y, left = x };
  \end{verbatim}
  Note that we are allowed to rearrange the labels in any order if
  the record expression form is used.
 \item[empty constructors] These are constructors written with
 an empty argument \verb|()|.  For example, in the following definition
 an empty constructor called \verb|NONE| is defined:
 \begin{verbatim}
    datatype Exp : public AST
	= NONE ()
	| NONE2
        | ...
 \end{verbatim}
 Here, the constructor \verb|NONE| will inherit from the class \verb|AST|,
 while the nullay constructor \verb|NONE2| will not.
\end{description}

\Subsubsection{Instantiating a datatype}

Each declared datatype should be {\em instantiated} in some implementation
file using the \TDEF{instantiate datatype} statement.  The translator
will generate additional code to implement the datatype.  The general
syntax is:
\begin{syntax}
 \NDEF{Instantiate\_Decl} 
    & \IS &
       \TDEF{instantiate datatype} \LIST{\N{Type\_Spec}} \\
    & \OR & 
       \TDEF{instantiate} \T{extern} \T{datatype} \LIST{\N{Type\_Spec}} \\
 \NDEF{Type\_Spec} & \IS & \Id \OPT{ \T{<} \LIST{\N{TypeExp}} \T{>}} \\
\end{syntax}

\begin{Tips}
   When a datatype is defined with the \TDEF{inline} qualifier,
\Prop{} will try to generate inline methods whenever possible.
On the other hand, if the \TDEF{extern} qualifier is used, \Prop{}
will try to generate an non-inline implementation, and only one
copy are generated at where the \T{instantiate datatype} statement occurs.

Since most \Prop{} datatypes are light-weight objects, \T{inline} is taken
as the default.  The user can override this default using
\OPTIONDEF{-save\_space} option of the translator 
(see page~\pageref{option:save-space}).
\end{Tips}

\begin{Tips}
   When the option \OPTIONDEF{-fno-implicit-templates} is used
under \Gpp, the user should \verb|#define| the symbol
\INDEX{{\tt PROP\_EXPLICIT\_TEMPLATE\_INSTANTIATION}} before
an \T{instantiate datatype} statement.   
This will enable the code for explicit template instantiation.
\end{Tips}

\Subsection{Pattern Matching} \label{sec:pattern-matching}
   The basic pattern matching construct is the \T{match} statement.    
A match statement can be seen as a generalization of \Cpp's \T{switch}
statement, in which the expression that follows a \T{case} can be
a general complex pattern.  

   Its basic syntax is as follows:
\begin{syntax}
\NDEF{Match} & \IS & 
   \N{Match\_Mode} \ALT{\T{(} \N{Exp} \T{)}}{and}  \\
        & & \T{\{} \MORE{\T{case} \N{Match\_Rule}}{} \T{\}} & variant 1 \\
          & \OR & 
   \N{Match\_Mode} \ALT{\T{(} \N{Exp} \T{)}}{and} \\
        & & \T{\{} \ALT{\N{Match\_Rule}}{|} \T{\}} & variant 2 \\
          & \OR &
   \N{Match\_Mode} \ALT{\T{(} \N{Exp} \T{)}}{and} \T{of} & variant 3 \\
        & & \quad \ALT{\N{Match\_Rule}}{|} \\
        & & \T{end} \T{match} \T{;} \\
\NDEF{Match\_Mode} & \IS & \TDEF{match} \OPT{\T{while}} \\
                & \OR & \TDEF{matchall} \OPT{\T{while}} \\
\NDEF{Match\_Rule} & \IS & \Pat \OPT{\N{Guard}} \OPT{\N{Cost}}
        \T{:} \N{Match\_Action} \\
\NDEF{Guard}       & \IS & \T{if} \N{Exp} \\
                & \OR & \T{where} \N{Exp} \\
                & \OR & \T{|} \N{Exp} \\
\NDEF{Cost} & \IS & \verb|\| \Exp & cost expression \\
\NDEF{Match\_Action} & \IS & \T{\{} \N{Code} \T{\}} \\
                  & \OR & \N{Code} & for variant 1 only \\
\end{syntax}

The 3 syntactic variants of the match statement are equivalent in meaning.
A \T{match} statement takes a list of match objects and a
set of pattern matching rules.  The first rule that matches the pattern
completely will be executed.  Notice that guards can be placed on each rule,
and a rule is only applicable if the guards evaluate to true. 

\paragraph{Cost minimization}
In addition, if cost expressions are specified, then the matching process
will proceed as follows: 
\begin{enumerate}
   \item Locate all applicable rules.  If no such rule exists then exit.
   \item Compute the cost expressions of all applicable rules.
   \item Choose the action that has the lowest cost to execute.  If
a tie occurs, resolve by the textual order of the rules. 
\end{enumerate}

\paragraph{Match all rules}
The \T{matchall} statement is a variant of \T{match}.  In the matchall
mode, the actions of {\em all} applicable rules are executed in the
textual order.  Caution: since the matching variable binding 
process occurs {\em before} the rules are executed, the rhs actions
should not alter the matched value or the pattern variables may
be bound to the wrong value.

\paragraph{Repetition} 
In addition, both \T{match} and \T{matchall} can be annotated with the
\TDEF{while} modifier.  It this case the semantics of the matching
construct is to repeat until no more rule is applicable.  For example,
the following code fragment returns the last element of a list
\begin{verbatim}
datatype List = #[] | #[ int ... List ];

int last (List l)
{  match while (l)
   { case #[]:         assert(0); // an error
     case #[i]:        return i;
     case #[h ... t]:  l = t;     
   }
}
\end{verbatim}

\paragraph{Pattern syntax}
The basic form of a pattern is a literal, which
that matches only a single integer, real number, string, etc.  Complex
patterns can be constructed inductively
using datatype constructors, tuples, lists, and 
{\em logical pattern connectives}.

The syntax of the patterns is as follows:
\begin{syntax}
\NDEF{Pat} & 
   \IS & \N{Integer} & matches integer literal \\
 & \OR & \N{Real}    & matches real literal \\
 & \OR & \N{Char}    & matches character literal \\
 & \OR & \N{Quark}   & matches quark literal \\
 & \OR & \N{String}  & matches string literal \\
 & \OR & \N{Regexp}  & matches any string that matches {\em regexp}\\
 & \OR & \N{Pat\_Var}  & a pattern variable; matches anything \\
 & \OR & \T{\_}      & wild card; matches anything \\
 & \OR & \N{Cons}     & matches a datatype constructor  \\
 & \OR & \N{Cons} \N{PatArg}& a constructor application pattern \\
 & \OR & \Id \T{as} \Pat & binds a variable to the sub-pattern \\
 & \OR & \Pat \T{:} \TypeExp & typed pattern \\
 & \OR & \Pat \T{||} \Pat  & matches either pattern \\
 & \OR & \Pat \T{\&\&} \Pat  & matches both patterns \\
 & \OR & \T{!} \Pat             & matches anything but the pattern \\
 & \OR & \T{(} \Pat \T{)}       & Grouping \\
 & \OR & \T{\#[} \LIST{\N{Pat}} \T{]} & list pattern; exact length match \\
 & \OR & \T{\#[} \LIST{\N{Pat}} \T{...} \T{]} & list pattern;   
     non-exact length match \\
 & \OR & \T{[|} \LIST{\N{Pat}} \T{|]} & vector pattern; exact length match \\
 & \OR & \T{[|} \LIST{\N{Pat}} \T{...} \T{|]} 
    & vector pattern; matches to the left \\
 & \OR & \T{[|} \T{...} \LIST{\N{Pat}} \T{|]} 
    & vector pattern; matches to the right \\
\NDEF{PatArg} & \IS & \Pat \\
           & \OR & \T{(} \LIST{\N{Pat}} \T{)} \\
           & \OR & \T{\{} \LIST{\N{Lab\_Pat}} \OPT{\T{...}} \T{\}} \\ 
\NDEF{Lab\_Pat} & \IS & \Id \T{=} \Pat \\
\NDEF{Pat\_Var} & \IS & \Id       \\
            & \OR & \T{?}\Id  \\
\end{syntax}

\Subsection{Refining a datatype}

   A datatype definition can be refined in a few ways:

\begin{enumerate}
  \item{\bf inheritance} A datatype can be declared to be inherited from
one or more classes $C_1, C_2, \ldots, C_n$. 
The effect is that all variants of a datatype will be inherited from the 
classes $C_1, C_2, \ldots C_2$.  

  \item{\bf member functions}  Member functions and addition attributes
can be attached to the datatype variants using the keyword \TDEF{with}.
\end{enumerate}

For example, the following datatype \T{Exp} is inherited from some user
defined classes \verb|AST| and \verb|Object|.   Furthermore, a virtual
member function called print is defined.  

\begin{verbatim}
datatype Exp : public AST, virtual public Object
   = INT int        
         with { ostream& print(ostream&); }
   | PLUS  Exp, Exp 
         with { ostream& print(ostream&); }
   | MINUS Exp, Exp 
         with { ostream& print(ostream&); }
   | MULT  Exp, Exp 
         with { ostream& print(ostream&); }
   | DIV   Exp, Exp 
         with { ostream& print(ostream&); }
   | VAR { name : const char *, typ : Type }
         with { ostream& print(ostream&); }
   public:
   {  virtual ostream& print (ostream&) = 0;
      void * operator new (size_t);
   }
;
\end{verbatim}

The allocator and printing methods can be defined as follows:
\begin{verbatim}
void * classof Exp::operator new(size_t) { ... }
ostream& classof INT::print(ostream& s)   { return s << INT; }
ostream& classof PLUS::print(ostream& s)  
   { return s << '(' << this->#1 << '+' << this->#2 <<')'; }
ostream& classof MINUS::print(ostream& s) 
   { return s << '(' << this->#1 << '-' << this->#2 <<')'; }
ostream& classof MULT::print(ostream& s)  
   { return s << '(' << this->#1 << '*' << this->#2 <<')'; }
ostream& classof DIV::print(ostream& s)  
   { return s << '(' << this->#1 << '/' << this->#2 <<')'; }
ostream& classof VAR::print(ostream& s)   { return s << name; }
\end{verbatim}

The special type form \TDEF{classof} {\em con} is used to reference
the type of a constructor.  Arguments of a constructor uses the following
naming convention: (i) if the constructor $C$ takes only one argument,
then the name of the argument is $C$; (ii) if $C$ takes two or more
unnamed arguments, then these are named \verb|#1|, \verb|#2|, etc;
(iii) finally, labeled arguments are given the same name as the label.

For readability reasons, it is often useful to separate a datatype definition
from the member functions and inheritance definitions.   The \TDEF{refine}
declaration can be used in this situation.  For example, the above
example can be restated more succinctly as follows:

\begin{verbatim}
//
// Datatype definition section
//
datatype Exp = INT int        
             | PLUS  Exp, Exp 
             | MINUS Exp, Exp 
             | MULT  Exp, Exp 
             | DIV   Exp, Exp 
             | VAR { name : const char *, typ : Type }
;

//
// Refinement section
//
refine Exp : public AST, virtual public Object
       { virtual ostream& print (ostream&) = 0;
         void * operator new (size_t);
       }
and    INT, PLUS, MINUS, MULT, DIV, VAR
       { ostream& print(ostream&); 
       }
;
\end{verbatim}

The general syntax of the \TDEF{refine} declaration is as follows:
\begin{syntax}
\NDEF{Refine\_Decl} 
   & \IS & \T{refine} \ALT{\N{Refine\_Spec}}{\T{and}} \T{;} \\
\NDEF{Refine\_Spec} 
   & \IS & \LIST{\N{Id}} \\
   &     & \quad \OPT{\T{:} \N{Inherit\_List}} \\
   &     & \quad \OPT{\T{::} \N{Datatype\_Qualifiers}} \\
   &     & \quad \OPT{\T{\{} \N{Datatype\_Body} \T{\}}} \\
\end{syntax}
Here, \N{Id} refers to either a constructor name or a datatype name.

\Subsection{Memory management}

   By default, a datatype constructor calls the \verb|operator new| to
allocated memory.  There are a few ways to change this behavior: (i)
redefine the \verb|operator new| within the datatype using refinement; or
(ii) inherit from a class that redefines this method.

   If a {\em placement} \verb|operator new| is defined in datatype,
the placement constructor syntax can be used to invoke this
operator:
\begin{syntax}
   \NDEF{Placement\_Cons} 
   & \IS & \N{Cons} \T{'(} \LIST{\N{Exp}} \T{)} \T{(} 
   \OPT{\LIST{\N{Exp}}} \T{)} \\
   & \OR & \N{Cons} \T{'(} \LIST{\N{Exp}} \T{)} \T{\{} 
   \OPT{\LIST{{\N{Id} = \N{Exp}}}}
     \T{\}} \\
\end{syntax}

For example, in the following the extra parameter \verb|mem| will
be passed to placement \verb|new| operator.
\begin{verbatim}
  Exp e1 = INT'(mem)(1);
  Exp e2 = VAR'(mem){ name = "foo", typ = t };
\end{verbatim}

\Subsubsection{Garbage collection}

  The support library contains an implementation\footnote{The implementation
has only been tested on Linux, SunOS and Solaris.  Please contact the
author for details
if you'd like to port the collector to your particular platform.} of a 
conservative garbage collector using two algorithms: one using
a mostly copying scheme\cite{Mostly-copying,Gen-mostly-copying} 
and one using mark-sweep.

  A datatype can be defined to be garbage collectable with the
qualifier \TDEF{collectable}.  In the following example, the datatype
\T{Exp} is will be allocated from the a garbage collected heap instead
of the default heap.  Since it references another 
user defined garbage collectable object
\verb|SymbolTable|, the pointer to that symbol table is also marked
as \T{collectable}.
\begin{verbatim}
   // SymbolTable is collectable
   class SymbolTable: public GCObject
   {
   public:
      class Id : public GCObject { ... };
      ...
   };

   datatype Exp :: collectable
      = INT int
      | VAR (collectable SymbolTable::Id, collectable SymbolTable *)
      | ADD Exp, Exp
      | SUB Exp, Exp
      | MUL Exp, Exp
      | DIV Exp, Exp
      ;
\end{verbatim}

The corresponding \TDEF{instantiate datatype} statement for \T{Exp}
will generate the appropriate tracing methods for garbage collection.

Please see appendix~\ref{appendix:gc} for more details concerning
garbage collection.

\Subsubsection{Persistence}

  Datatypes can be interfaced with the persistent object library
supplied with the translator using the \TDEF{persistent} qualifier.
A persistent object defines the protocol for serializing its representation
into an architecture independent format on a persistence stream.  

For example, the following datatype declaration defines a persistent 
type \verb|Exp|.  Note that a pointer to the symbol table
is annotated with the \T{persisent} qualifier to signify that the object
associated with the pointer should be traversed during the serialization
phase. 
\begin{verbatim}
   // SymbolTable is persistent
   class SymbolTable: public PObject
   {
   public:
      class Id : public PObject { ... };
      ...
   };

   datatype Exp :: persistent
      = INT int
      | VAR (persistent SymbolTable::Id, persistent SymbolTable *)
      | ADD Exp, Exp
      | SUB Exp, Exp
      | MUL Exp, Exp
      | DIV Exp, Exp
      ;
\end{verbatim}

A {persistent object id} {\em must} be provided with each distinct
datatype.  Object ids are used to identify the classes
of persistent objects.  Thus they must be unique within an application.
Each persistent type tag is simply a string, and can be defined using
the \TDEF{refine persistent} statement.  

For example, the following statement declares the persistent object
id for type \verb|Exp| above to be \verb|"Simple expressions"|.
\begin{verbatim}
   refine persistent Exp => "Simple expressions";
\end{verbatim}

The corresponding \TDEF{instantiate datatype} statement will generate
the appropriate methods to communicate with a persistent stream.
\begin{verbatim}
  instantiate datatype Exp;
\end{verbatim}

To write an object $e$ to the datafile \verb|"foo"|, we can say:
\begin{verbatim}
   #include <AD/persist/postream.h>
   ofstream out("foo");
   Postream pout(out);
   pout << e;
   out.close();
\end{verbatim}

To read the object back from the file, we can say:
\begin{verbatim} 
   #include <AD/persist/pistream.h>
   EXP e;
   ifstream in("foo");
   Pistream pin(in);
   e = (Exp)read_object(pin);
   in.close();
\end{verbatim}

For more details concerning persistent streams and persist objects, 
please see directory \verb|<AD/persist>| for details.

\Section{Inference} \label{sec:inference}

   Semantic processing
in compilers and other language processors, such as data flow analysis,
can frequently be specified as in a rule-based,
logical deductive style.  In \Prop, deductive inference using forward
chaining\footnote{
The current
implementation of {\sf Prop} translates inference into very naive code.
This feature will be phased out and replaced by
the more general and much faster graph rewriting mechanism.  
}
is provided as a built-in mechanism, orthogonal
to pattern matching and rewriting, for manipulating user-defined
algebraic datatypes.
 
Similar to syntax classes, {\bf inference classes} may be used
for data encapsulation.  An inference class is a combination of a \Cpp{}
class, a database of inference relations, and a collection of inference rules
of the form {\em lhs \verb|->| rhs}.  The lhs of an inference rule is a set
of patterns in conjunctive form.  During the inference process, a rule is
fired when its lhs condition is satisfied.  A fired rule then executes
the corresponding rhs action, which may assert or retract additional relations
from the database.   Using multiple inheritance, it is possible to combine
a rewriting class with an inference class such that the rewriting process
generates new relations to drive the inference process, or vice versa.

\Subsection{An Example} 
Datatype relations are not a distinct kind of data structure but are in fact
simply algebraic datatypes declared to have a \TDEF{relation} 
attribute.  For example, in
the following definition three relation types \verb|Person|, \verb|Parent|
and \verb|Gen| are defined.
 
\begin{verbatim}
   datatype Person :: relation = person (const char *)
       and  Parent :: relation = parent (const char *, const char *)
       and  Gen    :: relation = same_generation (const char *, const char *);
 
   instantiate datatype Person, Parent, Gen;
\end{verbatim}
 
Using these relations we can define an inference class that computes whether
two persons are in the same generation.   Nine axioms are defined (i.e.
those whose lhs are empty) in the following.  The two inference rules
state that (1) a person is in the same generation as him/herself, 
and (2) two persons are in the same generation if 
their parents are in the same generation.
 
\begin{verbatim}
   inference class SameGeneration {};
 
   inference SameGeneration
   {  -> person("p1") and person("p2") and
         person("p3") and person("p4") and
         person("p5");
 
      -> parent("p1", "p2") and
         parent("p1", "p3") and
         parent("p2", "p4") and
         parent("p3", "P5");
 
      person ?p -> same_generation (?p, ?p);
 
      parent (?x, ?y) and parent (?z, ?w) and same_generation (?x, ?z)
      -> same_generation(?y, ?w);
   };
\end{verbatim}
 
In general, datatypes qualified as \verb|relation|s will inherit
from the base class \verb|Fact|, while a rewrite class definition
implicitly defines two member functions used to assert and retract facts
in the internal database.  For example, in the above example, the following
protocol will be automatically generated by the inference compiler.
 
\begin{verbatim}
   class SameGeneration : ...
   {
   public:
       virtual Rete&    infer       ();       // start the inference process
       virtual ReteNet& operator += (Fact *); // assert fact
       virtual ReteNet& operator -= (Fact *); // retract fact
   };
\end{verbatim}
 

 
Using these methods, an application can insert or remove relations
from an inference class.  This will in turn trigger any attached inference
rules of the class.
 
\Subsubsection{Another example}
 
Consider the following example, which is used to compute Pythagorean
triangles.  Only one axiom and two rules are used.  The axiom and the
first rule are used to assert the relations \verb|num(1)| to \verb|num(n)|
into the database, where \verb|n| is limited by the term \verb|limit(n)|.
The second inference rule is responsible for printing out only
the appropriate combinations of numbers.
 
\begin{verbatim}
   datatype Number :: relation = num int | limit int;
 
   inference class Triangle {};
 
   inference Triangle
   {  ->  num 1;
 
          num m
      and limit n | n > m
      ->  num (m+1);
 
          num a
      and num b
      and num c | a < b && b < c && a*a + b*b == c*c
      ->  { cout << a << " * " << a << " + "
                 << b << " * " << b << " = "
                 << c << " * " << c << "\n";
          };
   };
\end{verbatim}
 
Now, to print all the triangle identities lying in range of $1$ to $100$,
we only have to create an instance of the inference class, insert the
limit, and start the inference process, as in below:
 
\begin{verbatim}
   Triangle triangle;
   triangle += limit(100);
   triangle.infer();
\end{verbatim}

\Subsection{Inference Class}
\INCOMPLETE

\Subsection{Inference Rules}
\INCOMPLETE
 
\Section{Tree Rewriting} \label{sec:tree-rewriting} 

\Prop's tree rewriting mechanism let us transform a tree in
algebraic datatype form into another tree according to a set of 
{\em rewriting rules}.  Unlike plain pattern matching described
in section~\ref{sec:pattern-matching}, which only apply to the
root of a tree, rewriting rules are applicable to all parts of
the tree.  This allows the user to concentrate on developing the
set of transformations; the actual traversal of the tree object
is handled implicitly by \Prop.  When used properly, this
mechanism has an advantage over plain pattern matching since
rewriting rules remain valid even when a new variant to a datatype
is introduced.

In \Prop, a rewriting system is developed in a similar manner as a parser.
The first phase requires the user to defines a {\em rewrite class}
to encapsulate the rewriting rules.  

Frequently, the rewriting mechanism is used to collect information about
a tree structure; furthermore, more transformation rules are {\em conditional}:
i.e. we want them to be applicable only if certain conditions are satisfied. 
We can accomplish both of these tasks by defining data members and methods 
in the rewriting class.  These are accessible during the rewriting
process.  Information collected during rewriting can be stored within
the data members.

In the rewriting formalism, equational
rules of the form {\em lhs} \verb|->| {\em rhs} are specified by the user.
During processing, each instance of the lhs in a complex tree is replaced
by an instance of the rhs, until no such replacement is possible.
Equational rules can often be used to specify semantics based simplification
(e.g. constant folding and simplification based on simple algebraic
identities) or transformation(e.g. code selection in a compiler
backend\cite{codegen-with-trees}).
 
Unlike plain pattern matching, however, the structural traversal process
in rewriting is implicitly inferred from the type structure of an
algebraic datatype, as specified in its definition. 
 
There are two main forms of rewriting modes available:
\begin{itemize}
   \item The first is {\bf normalization} mode: a given tree is reduced
using the matching rules until no more redexes are available.  There
are two modes of operations available:
       \begin{itemize}
          \item in {\bf replacement} mode, the redex of a tree will be
physically overwritten.
          \item in {\bf applicative} mode, on the other hand, a new
tree corresponding to the replacement value will be constructed.
       \end{itemize}
     Replacement mode is used as the default since it is usually
the more efficient of the two.
   \item The second form is {\bf reduction} and {\bf transformation}.
In this mode a tree parse of the input term is computed.  If cost functions
are attached to the rules, then they will also be used to determine a
minimal cost reduction sequence.  During this process attached actions of
a rule may be invoked to synthesize new data.
\end{itemize}
 
   Each independent set of rewriting rules in \Prop{} is encapsulated
in its own {\bf rewrite class}.  A rewrite class is basically a normal \Cpp{}
class with a set of rewriting rules attached.   During rewriting, the
data members and the member functions are visible according to the normal
\Cpp{} scoping rules.  This makes it is easy to encapsulate additional data
computed as a side effect during the rewriting process.
 
\Subsection{A rewriting example}
 
   Consider an abbreviated simplifier for the well-formed formula datatype
\verb|Wff| defined in the section~\ref{sec:Wff}.  The rewrite class for this
can be defined as follows.  Since there is no encapsulated data in this
example, only the default constructor for the class needs to be defined.
A rewrite class definition requires the traversal list to be
specified.  This is simply a list of datatypes 
involved in the rewriting traversal process.  
In this instance only \verb|Wff| is needed.
 
\begin{verbatim}
   rewrite class Simplify (Wff)
   {
   public:
      Simplify() {}
   };
\end{verbatim}
 
   The rewrite rules for the simplifier can then be specified succinctly as
follows.  Like the \verb|match| statement, in general the rhs
of a rewrite rule can be any statement.  A special statement
\verb|rewrite(e)| can be used to rewrite the current redex into another form.
If the rhs is of the form \verb|rewrite(e)|, then it can be abbreviated
to \verb|e|, as in below:
\begin{verbatim}
   rewrite Simplify
   {  And(F,  _):      F
   |  And(_,  F):      F
   |  And(T,  ?X):     ?X
   |  And(?X, T):      ?X
   |  Or (T,  _):      T
   |  Or (_,  T):      T
   |  Or (F,  ?X):     ?X
   |  Or (?X, F):      ?X
   |  Not(Not(?X)):    ?X
   |  Not(And(?X,?Y)): Or(Not(?X), Not(?Y))
   |  Not(Or(?X,?Y)):  And(Not(?X), Not(?Y))
   |  Implies(?X,?Y):  Or(Not(?X), ?Y)
   |  And (?X, ?X):     ?X
   |  Or (?X, ?X):      ?X
   // etc ...
   };
\end{verbatim}
 
The rewrite class definition creates a new class of the same name.  This new
class defines an implicit \verb|operator ()| with the protocol below.  This
member function can be invoked to perform the rewriting in a functional
syntax.
 
\begin{verbatim}
   class Simplify : ... {
   {  ...
   public:
      void operator () (Wff);
      // Wff operator () (Wff); // if rewrite class is applicative
   };
 
   Wff wff = ...;
   Simplify simplify; // create a new instance of the rewrite class
   simplify(wff);    // rewrite the term wff
\end{verbatim}

 
\Subsubsection{Conditional rewriting and actions}
  
   Rewriting rules may be guarded with predicates to limit their
applicability.  In addition, the {\em rhs} of a rewrite rule is not limited
to only a replacement expression: in general, any arbitrarily complex sequence
of code may be used.  For example, in the following set of rewriting
rules we use guards to prevent undesirable replacements to be made during
expression constant folding:
 
\begin{verbatim}
   rewrite ConstantFolding
   {  ADD (INT a, INT b):  INT(a+b)
   |  SUB (INT a, INT b):  INT(a-b)
   |  MUL (INT a, INT b):
      {  int c = a * b;                      // silent overflow
         if (a == 0 || b == 0 || c / b == a) // no overflow?
         {  rewrite(INT(c)); }
         else
         {  cerr << "Overflow in multiply\n"; }
      }
   |  DIV (INT a, INT b) | b == 0:  { cerr << "Division by zero\n"; }
   |  DIV (INT a, INT b):  INT(a/b)
   |  // etc...
   };
\end{verbatim}

\Subsection{Rewrite class} \label{sec:rewrite-class}

The syntax of a rewrite class declaration is as follows:

\begin{syntax}
\NDEF{Rewrite\_Class\_Decl} & \IS & \TDEF{rewrite class} \Id 
            \T{(} \LIST{\N{TypeExp}} \T{)} \\
   & & \quad \OPT{\T{:} \N{Inherit\_List}} 
        \OPT{\T{::} \MORE{\N{Rewrite\_Mode}}{}} \\
   & & \quad \T{\{} \N{Class\_Body} \T{\}} \T{;} \\
\NDEF{Rewrite\_Mode} & \IS & \TDEF{treeparser} \\
                  & \OR & \TDEF{applicative} \\
                  & \OR & \TDEF{topdown} \\
\end{syntax}

This is simply the usual \Cpp{} class declaration syntax
extended to the following information:
\begin{description}
   \item[a traversal list] enclosed in parenthesis.  The traversal
list of a rewrite class defines the set of types that rewriting
should traverse.  If a datatype object contains an argument of a type
not listed in the traversal list, its value will not be altered.
   \item[rewrite mode] this defines the rewriting mode of the
rewrite class.
\end{description}

\Subsection{Rewriting rules}

Rewriting rules are specified in a rewrite declaration.
The syntax of a rewriting specification is as follows:

\begin{syntax}
\NDEF{Rewrite\_Decl} 
   & \IS & \TDEF{rewrite} \Id & variant 1 \\
   &    & \T{\{} \OPT{\N{Index\_Decl}} 
        \ALT{\T{case} \N{Rewrite\_Rules}}{} \T{\}} \\
   & \OR & \T{rewrite} \Id & variant 2 \\
   &    & \T{\{} \OPT{\N{Index\_Decl}} \ALT{\N{Rewrite\_Rules}}{|} \T{\}} \\
\NDEF{Rewrite\_Rule}
  & \IS & \OPT{\N{Rewrite\_Modifier}} \OPT{\N{Id} \T{->}} \\
  & & \quad \Pat \OPT{\N{Guard}} \OPT{\N{Cost}} \T{:} \N{Rewrite\_Action} \\
\NDEF{Rewrite\_Modifier}
  & \IS & \T{bottomup:} & bottom up mode \\
  & \OR & \T{topdown:} & top down mode \\
  & \OR & \T{before:} & before actions \\
  & \OR & \T{preorder:} & preorder actions \\
  & \OR & \T{postorder:} & postorder actions \\
\NDEF{Rewrite\_Action} & \IS & \T{\{} \N{Code} \T{\}} \\
                  & \OR & \Exp \\
\end{syntax}

The name of a rewrite declaration should match the name of
a rewrite class.  
The two syntactic forms of \T{rewrite} have equivalent semantics.

The special statement \T{rewrite}($e$) may be used
inside the rhs action to rewrite the current redex into $e$.
Note that \T{rewrite}($e$) is a branching statement; statements
after \T{rewrite} are not reachable.

\Subsection{Rewriting modifiers}
In version 2.3.3 onward, rewrite rules can be modified by the 
keywords: \T{bottomup:}, \T{topdown:}, \T{before:}, \T{preorder:}
and \T{postorder:}.   These modifiers 
alter the order in which a set of rewriting rules is applied. 

These modifiers act like delimiters, similar
to the way scoping keywords like \T{public:}, \T{protected:} and 
\T{private:} are used to delimit declarations in C++.
For instance, a set of rules prefixed by the modifier \T{topdown:}
will utilize the topdown strategy for reduction.  If no modifier are
specified, then bottom-up will be the default.

Note that all five modes of rules can be used together in a rewriting system 
under tree rewriting mode (see \ref{sec:rewriting-modes})
and their executions are intermixed together.  

The semantics of these modifiers are as follows:
\begin{description}
\item[\TDEF{bottomup:}] This specifies the modified rewriting rules should
be applied in the default bottom-up mode.  In this mode, the 
innermost/leftmost redex is chosen as the next redex.  So reduction
occurs in a bottomup manner.  This means that if a term $t$ is a redex
and if the term $t$ contains a subterm $t'$ which is also a redex, 
$t'$ will be reduced before $t$. 
\item[\TDEF{topdown:}]  This specifies the modified rewriting rules should be
applied in a topdown mode.  In this mode, the rewriter
will first try to locate the outermost/leftmost redex.  Reduction
will occur in a (mostly) topdown manner.
\item[\TDEF{before:}]  This specifies that the modified rewriting rules
should be tried before the topdown phase.   
In addition, unlike topdown mode, if state-caching if used (see
\ref{sec:state-caching}), the rules are only tried once. 
Otherwise, these act just like the preorder modifier, described below. 
\item[\TDEF{preorder:}]  This specifies that
the modified rewriting rules should be
tried only at the preorder traversal of a term.   
\item[\TDEF{postorder:}]  This specifies the rewriting rules should be
tried only at the postorder traversal of a term. 
\end{description}

\Subsubsection{Rewriting modifier example} \label{sec:rewriting-modifier}

We'll use the following example, extracted from a query optimizer for
a subset of the relational calculus, to demonstrate the use
of the rewriting modifiers.

The query optimizer transforms the following abstract syntax, defined
as a set of \Prop's datatypes.  
\begin{verbatim}
datatype List<T> = #[] | #[ T ... List<T> ];
datatype Literal = INT int
                 | STRING const char *
                 | BOOL Bool

and      Exp     = OP Id, Exps
                 | APP Id, Exps
                 | LIT Literal
                 | ID Id
                 | TUPLE Exps                  // [ E1, E2, ... En ]
                 | FORALL Id, Exp, Exp         // forall x in E1 . E2
                 | EXISTS Id, Exp, Exp         // exists x in E1 . E2
                 | GUARD (Exp, Exp)            //
                 | GENERATOR (Ids, Exps, Exp)  // [x_1,...,x_n] in X_1*...*X_n
                 | LET  (Id, Exp, Exp)

law inline Int  i    = LIT(INT i)
|   inline Boolean b = LIT(BOOL b)
|          True      = Boolean true
|          False     = Boolean false
|   inline And a,b   = OP("and",#[a,b])
|   inline Or  a,b   = OP("or", #[a,b])
|   inline Not a     = OP("not",#[a])
|   inline Eq a,b    = OP("=",  #[a,b])
|   inline Ne a,b    = OP("/=", #[a,b])
|   inline Gt a,b    = OP(">",  #[a,b])
|   inline Ge a,b    = OP(">=", #[a,b])
|   inline Lt a,b    = OP("<",  #[a,b])
|   inline Le a,b    = OP("<=", #[a,b])
|   inline In a,b    = OP("in", #[a,b])
|   inline Union a,b = OP("union", #[a,b])
|   inline Count a   = OP("#", #[a])

where type Id         = const char *
and        Ids        = List<Id>
and        Exps       = List<Exp>
;
\end{verbatim}

Note that existential and 
universal quantifiers are represented by the constructors \T{FORALL}
and \T{EXISTS} respectively.  For example, $\exists x \in A.p$ is
presented as the term $\T{EXISTS}(x,A,p)$.  Here $x$ is a {\em binding}
occurrence for $x$; note that the variable $x$ may occur within 
the predicate $p$.  In addition, a list comprehension
expression such as 
\[ \{ e : [x_1,\ldots,x_n] \in X_1 
\times \ldots \times X_n\ |\ p \} \] 
is represented as the compound term 
\[
  \T{GENERATOR}(\#[x_1,\ldots,x_n],\#[X_1,\ldots,X_n],\T{GUARD}(p,e))
\]

Suppose we'll like to rename all variables in a query $Q$ 
so that no two binding occurrences are given the same name.  
This can be easily accomplished using a combination
of \T{preorder:} and \T{postorder:} rules in the following manner: 
\begin{enumerate}
\item[(i)] Use preorder rules to check for new variable bindings,
for example in $\T{EXISTS}(x,A,p)$.
For each new binding occurrence for $x$, create
a new name for $x$.  
These will be performed before the subterms $A$ and $p$ are traversed.
\item[(ii)] Whenever a variable $x$ is found during the sub-traversal
of $A$ and $p$, rename the variable by looking up its new name.
\item[(iii)] Finally, use postorder rules to remove the current 
variable substitution whenever we're exiting a binding occurrence.
\end{enumerate}

For example, the following set of rewriting rules accomplish this renaming
task for our query language using exactly this method.  
\begin{verbatim}
rewrite RemoveDuplicateNames
{
   // We use before and after rules to remove duplicates from variable names.
   // As each new binding occurrence is found, we enter a new binding
   // into the current environment.  Identifiers found within
   // the binding occurrence are then renamed.

preorder: // insert new bindings
   EXISTS(x,_,_):     { new_binding(x); }
|  FORALL(x,_,_):     { new_binding(x); }
|  GENERATOR(xs,_,_): { new_binding(xs); }
|  LET(x,_,_):        { new_binding(x); }

         // rename variables
|  ID x:             { rename(x); }

postorder: // removes the binding
|  EXISTS(x,A,_):      { old_binding(x); }
|  FORALL(x,A,_):      { old_binding(x); }
|  GENERATOR(xs,As,_): { old_binding(xs); }
|  LET(x,_,_):         { old_binding(x); }
};
\end{verbatim}

Note that we have accomplished a depth first traversal using rewriting
without writing any traversal code!  As a side benefit, 
since the traversal is automatically determined by 
the structure of the datatypes, we do not have to rewrite this
renaming code even if the abstract syntax of the query language is
extended, as long as no additional binding operators are added.

\Subsection{The \T{rewrite} statement}
 
   While the rewrite class construct provides a very general abstraction
for rewriting, in general its full power is unneeded.  It is often
convenient to be able to perform rewriting on a term without having
to make a new name for a class just for the occasion, especially if member
functions and member data are unneeded.  To accommodate these situations,
the \verb|rewrite| statement is provided
to perform a set rewriting transformations on a term without having
to define a temporary rewrite class.   It is simply syntactic sugar
for the more general rewrite class and rewrite
rules specifications.
For example, a simplify routine for type \verb|Exp| defined above can be
specified as follows:
 
\begin{verbatim}
   Exp simplify (Exp e)
   {  // transformations on e before
      rewrite (e) type (Exp)
      {  ADD (INT a, INT b):  INT(a+b)
      |  SUB (INT a, INT b):  INT(a-b)
      |  MUL (INT a, INT b):  INT(a*b)
      |  ...
      }
      // transformations on e after
      return e;
   }
\end{verbatim}
 
The \verb|rewrite| normally performs the replacement in-place.
An applicative version of the same can be written as follows\footnote{The
variable {\tt e} is assigned the new copy.}:
 
\begin{verbatim}
   Exp simplify (Exp e)
   {  rewrite (e) => e type (Exp)
      {  ADD (INT a, INT b):  INT(a+b)
      |  SUB (INT a, INT b):  INT(a+b)
      |  MUL (INT a, INT b):  INT(a*b)
      |  ...
      }
      return e;
   }
\end{verbatim}

The syntax of the rewrite statement is as follows.  The
traversal list of the set of rewrite rule is listed next to
the keyword \T{type}.
 
\begin{syntax}
\NDEF{Rewrite\_Stmt} 
   & \IS & \TDEF{rewrite} \T{(} \Exp \T{)} \OPT{\T{=>} \N{Exp}} \\
   &     & \quad \T{type} \T{(} \LIST{\N{TypeExp}} \T{)} \\
   &     & \T{\{} \OPT{\N{Index\_Decl}} \N{Rewrite\_Rules} \T{\}} \\
   & \OR & \TDEF{rewrite} \T{(} \N{Exp} \T{)} \OPT{\T{=>} \N{Exp}} \\
   &    & \quad \TDEF{type} \T{(} \LIST{\N{TypeExp}} \T{)} \T{of} \\
   &    & \quad \OPT{\N{Index\_Decl}} \N{Rewrite\_Rules} \\
   &    & \T{end} \T{rewrite} \T{;} \\ 
\end{syntax}

\Subsubsection{Rewriting modes} \label{sec:rewriting-modes}
   There are two basic modes of operations in a rewrite class: (i) 
{\em tree rewriting} and (ii) {\em tree parsing}.   
Tree rewriting repeatedly looks for redexes in a tree and transforms it into
another tree.   There are two sub-modes of operations: (a) {\em applicative}
and (b) {\em in-place}.  Applicative mode is specified using the
\T{applicative} mode specifier when declaring a rewrite class. 
In this mode, a new tree is built from a 
bottom-up manner, and the subject tree is left unaltered.  The default
mode is ``in-place.''  In this mode, the subject tree is overwritten
with the transformed expression.  

   On the other hand, in tree parsing mode, the left hand side
of a rewrite rules specification is treated as a tree grammar.
The tree parser will look for a derivation of given tree object from
the start non-terminal.
A guarded rule will only be used if the guard evaluates to true.

   If rules are annotated with cost expressions, then the
tree parser will try to locate a derivation chain with a minimal total
cost.  After a derivation tree is found, the tree can be repeated
traversed.  The rhs actions will be invoked during this traversal process.  

\Subsubsection{State caching optimization} \label{sec:state-caching}

   The pattern matching process proceeds in a bottom up manner: 
for each subtree of the form $l(t_1,t_2,\ldots,t_n)$, $n \ge 0$, 
a state number is computed from the current tree label $l$ and the
state numbers of its subtrees $t_1, t_2, \ldots, t_n$.  The set of matched
rules can be directly computed from the current state number.
This means that in the absence of redex replacements, pattern matching
takes time proportional only $O(|t| + r)$, where $|t|$ is the size 
of the input tree $t$ and $r$ is time it takes to execute the rhs actions.  
That is, the runtime is insensitive to the number of rewrite rules or the 
complexity of the patterns.

   However, if the input is a DAG rather than a tree\footnote{Rewriting
of a generic graph structure is not recommended with \T{rewrite}.},
or if replacements are performed, then the above analysis may not hold.
Replacements during rewriting often require state information of the
replacement term to be recomputed to further the matching process.  
Since computation of state can involve a complete traversal of a term, 
replacement can become expensive if the replacement term is large.  
For instance, consider the following replacement rule, which replaces 
all expressions of the form {\em 2*x} into {\em x+x}:
 
\begin{verbatim}
   rewrite class StrengthReduction
   {
      MUL (INT 2, ?x):  ADD(?x, ?x)
      ...
   }
\end{verbatim}
 
\noindent
Since the subterm \verb|?x| could be arbitrarily large, recomputing the
state encoding for \verb|ADD(?x,?x)| naively takes time in proportion to the
size of \verb|?x|.  However, since the subtree \verb|?x| occurs as part
of the pattern, its state has already been computed, and it would be
wasteful to recompute its state from scratch.  

A similar observation can be made to the case when the input is a DAG,
where many subcomponents are shared between nodes.  In order to speed
up the rewriting process, the state of each shared copy of the nodes 
should be computed only once.

In order to speedup the state computation process in these situations,
{\em \INDEX{state caching}} can be enabled by
specifying a {\em rewriting index}.
This can be accomplished in two ways: (i) use the qualifier
\TDEF{rewrite} when defining a datatype.  This specifies the
{\em primary index}.  Or alternatively, (ii) use index declarations
to specify one or more {\em secondary indices}.

We'll first describe the first method. 
A \TDEF{rewrite} qualifier can be used 
in the definition of a datatype.  For instance:
 
\begin{verbatim}
   datatype Exp :: rewrite
      = INT (int)
      | ID  (const char *)
      | ADD (Exp, Exp)
      | SUB (Exp, Exp)
      | MUL (Exp, Exp)
      | DIV (Exp, Exp)
      ;
\end{verbatim}

The \T{rewrite} qualifier tells the translator to generate an extra
state field in the implementation of \T{Exp}.  This state field is implicitly
updated during the rewriting process.  When the rewriter encounters a
node with a cached state it can quickly short-circuit all unnecessary
state computation.

{\bf A word of caution:} since each rewriting system computes
its own encoding, rewriting systems should not be mixed if they shard the 
same index, i.e. actions of a 
rewriting system should not invoke another rewriting
system on a term that is participating in the current rewriting system,
if both systems use the same index.  This limitation can be overcome
by using secondary indices, which we'll discuss next.

\Subsubsection{Specifying secondary indices}

From release 2.3.0 onward, it is possible to specify secondary indices
within a rewriting system.  This makes it possible to invoke other
rewriting systems within the execution of another, while retaining the
state caching optimizations throughout.

There are two forms of secondary index: {\em internal index} and 
{\em external index}.  Internal indices are stored directly within
a rewrite term and require the user to pre-allocate space for each term.
In contrast, external indices are stored in a separate data structure
such as a hash table. 

The syntax of secondary index declarations is as follows:
\begin{syntax}
\NDEF{Index\_Decl} & \IS & \TDEF{index:} \LIST{\N{Index\_Spec}} \T{;} \\
\NDEF{Index\_Spec} & \IS & 
   \TypeExp \T{=} \T{none} & \C{Disable indexing} \\
   & \OR & \TypeExp \T{=} \N{Id} & \C{Internal index} \\
   & \OR & \TypeExp \T{=} \TDEF{extern} \N{Id} & \C{External index} \\
\end{syntax}


\Subsubsection{Using an internal index}

In order to make use of an index, certain member functions have to defined
by the user.  Given an internal index named {\em foo}, the rewriter
invokes two functions named
\begin{quotation}
\noindent
   {\tt int get\_{\em foo}\_state() const; } \\
   {\tt void set\_{\em foo}\_state(int); }
\end{quotation}
\noindent within the generated code.  The user should provide the 
implementations for these functions within the datatypes.  

For example, reusing the well-formed formulas example (see 
section~\ref{sec:Wff}),
an internal index on the datatype \T{Wff} can be implemented as follows:
\begin{verbatim}
#include <AD/rewrite/burs.h>
class WffIndex {
   int state;
public:
   WffIndex() : state(BURS::undefined_state) {}
   int get_wff_state() const { return state; }
   void set_wff_state(int s) { state = s; }
};

datatype Wff : public WffIndex
   = F
   | T
   | Var     (const char *)
   | And     (Wff, Wff)
   | Or      (Wff, Wff)
   | Not     (Wff)
   | Implies (Wff, Wff)
   ;

...

rewrite Simplify {
   index: Wff = wff;
   ...
};
\end{verbatim}

Here, the class \T{WffIndex} provides the internal index interface
functions \verb|get_wff_state| and \verb|set_wff_state| expected by
the rewriter.  Note that each term must be initialized with the
state \verb|BURS::undefined_state|.  This constant is defined within
the library include file \verb|<AD/rewrite/burs.h>|.

To enable the index, in the rewriter 
we specify that datatype \T{Wff} should make use of the index named \T{wff},
using the \T{index:} declaration.  

\Subsubsection{Using an external index}

External indices are specified in a similar manner.  Given a datatype
{\em Foo} and an external index named {\em bar}, the rewriter invokes
calls to the following functions:
\begin{quotation}
\noindent
   {\tt int get\_{\em bar}\_state({\em Foo}) const; } \\
   {\tt void set\_{\em bar}\_state({\em Foo}, int); }
\end{quotation}
Typically, these are implemented as member functions in the rewrite class.

The rewriter uses the function {\tt get\_{\em bar}\_state} to lookup
previously cached information set by the function {\tt set\_{\em bar}\_state}.
Typically, the implementation of the two functions can be implemented
as hash tables, using the addresses of the terms as hash functions and
pointer equality comparisons.   Note that caching can be {\em lossy}; i.e.
it is perfectly alright for the cache to eliminate cached information to
keep its size under control.  If no cache information is found,
the function {\tt get\_{\em bar}\_state} should return 
\verb|BURS::undefined_state|.

\paragraph{Class \CLASSDEF{RewriteCache}}
   To make it easy for the users to implement their own external indices,
two sample external index implementation are provided.  The first
is the class \verb|RewriteCache| defined in the library file 
\verb|<AD/rewrite/cache.h>|.  

   The class \verb|RewriteCache| implements a simple fixed capacity
   cache with the following member functions.
\begin{verbatim}
   RewriteCache();
   RewriteCache(int capacity);
  ~RewriteCache();
 
   void clear();                   // after rewriting
   void initialize();              // before rewriting
   void initialize(int capacity);  // before rewriting/also set capacity
   
   int capacity() const; 
               
   void set_state(const void * t, int s);
   int get_state(const void * t) const;

   void invalidate(const void * t); 
\end{verbatim}

The user should call the function \verb|initialize| before rewriting
in order to setup the cache and/or clear out previously cached result.
The functions \verb|set_state| and \verb|get_state| can be used to
implement the necessary state caching functions expected by the rewriter.

Returning to our \T{Wff} example, we have the following possible implementation
of the external index:
\begin{verbatim}
   rewrite class Simplify (Wff)
   {  RewriteCache cache;
   public:
      int get_wff_state(Wff t) const { return cache.get_state(t); }
      void set_wff_state(Wff t, int s) { cache.set_state(t,s); }  
   };

   rewrite Simplify
   {  index: Wff = extern wff;
      ...
   };
\end{verbatim}

\begin{Tips}  
If manual memory management is used, 
a term should not be deallocated from memory if a cache
contains a handle to it.  The method \verb|invalidate| can be used to
make sure the entry associated with a term is removed from the cache.
\end{Tips}

\paragraph{Class \CLASSDEF{GCRewriteCache}}

{\em This class has not be completely tested.}

Since a garbage collectable object is not reclaimed by the collector
whenever there exists a reference to it, using the class
\verb|RewriteCache| may create problems since objects are not reclaimed
if it is referenced by the cache.  The template class \verb|GCRewriteCache|
solves this problem by using {\em weak pointers} in its internal 
implementation.  

The template class \verb|GCRewriteCache<T>| expects \verb|T| to be
a class derived from \verb|GCObject|.   
This class implements the following functions:
\begin{verbatim}
   GCRewriteCache();
   GCRewriteCache(int capacity);
  ~GCRewriteCache();
 
   void clear();                   // after rewriting
   void initialize();              // before rewriting
   void initialize(int capacity);  // before rewriting/also set capacity
   
   int capacity() const;
               
   void set_state(T * t, int s);
   int get_state(T * t) const;

   void invalidate(T * t); 
\end{verbatim}

The usage of these functions are identical to that of class 
\verb|RewriteCache|.

\Subsection{Short circuiting rewrite paths with \TDEF{cutrewrite}}

   From version 2.3.0 onward, the \T{cutrewrite}($e$) 
statement may be used wherever a \T{rewrite}($e)$ statement is expected.
The \T{cutrewrite} statement can be used to selectively ignore certain
subterms during the rewriting process.

The semantics of \T{cutrewrite}($e$) is to replace the current redex
with $e$ but do not bother with looking for other redexes in the replacement
term.  This means that the replacement term will not be re-traversed
immediately.  Furthermore, the replaced term will not match any other
left hand side sub-patterns except pattern variables and wildcards.

   This feature is very useful for preventing the rewriter from
looking inside certain terms. 

   To demonstrate, consider the simple following example:
\begin{verbatim}
   datatype Exp = NUM of int 
                | ADD of Exp, Exp
                | SUB of Exp, Exp
                | MUL of Exp, Exp
                | DIV of Exp, Exp
                | FINAL of Exp
                ;
   Exp e = ADD(NUM(1),MUL(NUM(2),NUM(3)));
\end{verbatim}

Suppose we want to find all numbers within 
the expression \T{e} and increment their values by 1.  
The straight forward way of writing:
\begin{verbatim}
   rewrite (e) type (Exp) {
      NUM x: NUM(x+1) 
   }
\end{verbatim}
\noindent will not work, since the replacement term is a new redex,
and the rewriting system will not terminate.

We can reimplement the rewriting system as a two stage process.
First, we mark all terms that we do not want to be changed; whenever we
find a marked term, we execute a \T{cutrewrite} to prevent the term
from being changed.  Then we unmark the terms.
\begin{verbatim}
   rewrite (e) type (Exp) {
      NUM x:   FINAL(NUM(x+1));
   preorder:
      FINAL _: cutrewrite;
   }
   rewrite (e) type (Exp) {
      FINAL x: x
   }
\end{verbatim}

In the first rewrite statement, replacement terms that we 
want to stay fixed are encapsulated
within the auxiliary \T{FINAL} constructor.  
Recall that preorder rules are tried before the subterms of a redex are 
reduced.  Thus we can make sure all that
all terms that are arguments to a \T{FINAL} term is left unaltered
by adding a \T{preorder} rule that executes a \T{cutrewrite}\footnote{
If an expression argument is not given to a \T{cutrewrite} statement, then
the redex is left unchanged.}
Finally, after the first rewrite statement finishes, we use 
the second rewrite statement to remove all \T{FINAL} terms generated
in the first.

The semantics of \T{cutrewrite} needs further explanation.  Consider
the following rewrite system:
\begin{verbatim}
   rewrite (e) type (Exp) {
      NUM x:          cutrewrite(NUM(x+1))
   |  MUL(X, NUM 1):  X
   |  MUL(NUM 1,X):   X
   }
\end{verbatim}
Given the term \verb|MUL(NUM(0),NUM(2))|, the subterms \verb|NUM(0)|
and \verb|NUM(1)|
will be rewritten by the first rule
into \verb|NUM(1)| and \verb|NUM(3)| respectively.
Furthermore, the replacement term \verb|NUM(1)| will {\em not} match
the left subpattern in \verb|MUL(NUM 1,X)|.  Thus the result of the rewrite
will be \verb|MUL(NUM(1),NUM(3))|.

\begin{Tips}
Since the semantics of a rewriting system with \T{cutrewrite} can
be tricky to analyze, its use should be avoided as much as possible. 
\end{Tips}

\Subsection{Conditional failure with \TDEF{failrewrite}}

   From version 2.3.0 onward, the \T{failrewrite}
statement may be used wherever a \T{rewrite} statement is expected.
The \T{failrewrite} statement is an alternate way of writing
conditional rewriting rules, and it is often useful when the condition
to be checked is too complex to be written as a guard.

When a \T{failrewrite} statement is executed within the rhs action of a
rewrite rule, the current pattern is rejected and the rewriter
will proceed to try other matches.   

For instance, suppose the current redex is 
\verb|DIV(INT 0,INT 0)| and the following rules are specified within
a rewriting system:
\begin{verbatim}
   DIV(INT X,INT Y): { if (Y == 0) failrewrite; else rewrite(NUM(X/Y); }
|  DIV(INT 0,Y):     NUM(0)
\end{verbatim} 

The execution of these rules will proceed as follows:  both patterns
match the redex, but the first rule will be tried first.  Since
\verb|Y| is zero, the \T{failrewrite} statement is executed.  The rewriter
will proceed to try the
second rule and the redex will be replaced with \verb|NUM(0)|. 
Note that if we omit the \T{failrewrite} statement
in the first rule, as in:
\begin{verbatim}
   DIV(INT X,INT Y): { if (Y != 0) rewrite(NUM(X/Y); }
|  DIV(INT 0,Y):     NUM(0)
\end{verbatim}
the rewriter will commit on the first rule
and no action will be performed on the redex.

Like \T{rewrite} and \T{cutrwerite}, \T{failrewrite}
acts as a branching statement and all statements
immediatedly following \T{failrewrite} are dead code.

\Subsection{Confluence and termination}

   Currently no automatic checking is available to make sure that
a set of rewriting rules is confluent and will terminate.  This is currently
the user's responsibility.  Some verification features will be developed
in the future.

\Subsection{Debugging Tree Rewriting}

   By default, \Prop{} generates a macro \verb|DEBUG_|$C$ for each
rewrite class named $C$.  The user can redefine this macro in order to
trace the replacements that are performed during the execution of a 
set of rewriting rules.  The default definition of \verb|DEBUG_|$C$
is as follows:
\[
\verb|#define DEBUG_|C({\it R,redex,filename,linenumber,ruletext})\ {\it R}
\]
i.e. it simply returns the value of {\it R}.

The arguments to the macro are as follows: 
\begin{itemize} 
\item {\it R} is the new replacement term.
\item {\it redex} is the current redex, before the replacement has been
performed.
\item {\it filename} is the file name of the current rewriting rule.
\item {\it linenumber}  is the location of the current rewriting rule.
\item {\it ruletext} is a string that describes the current rule.
\end{itemize}

In order to activate tracing, the user can redefine the macro
\verb|DEBUG_FOO| (for rewrite class \verb|FOO|) as follows:
\begin{verbatim}
#define DEBUG_FOO(a,b,c,d,e) print_rule(a,b,c,d,e)

template <class T>
  T print_rule(T replacement, T redex, 
               const char * file_name, 
               int line, const char * rule_text)
  { cerr << file_name << ':' << line << ": " << rule_text 
         << " redex = " << redex << " replacement = " << replacement
         << endl;
    return replacement; 
  }
\end{verbatim}
Note that these definitions must appear
before the \verb|rewrite FOO| declaration.

\Subsection{Optimizing Tree Rewriting} \label{sec:optimize-rewriting}

   An experimental rewriting optimizer based on partial evaluation
techniques has been implemented in release 2.3.4 of \Prop.  The optimizer
can be enabled with the option \OPTIONDEF{-Orewriting}.  

Informally speaking, the rewriting optimizer will try to locate 
``longer'' reduction sequence and add these to your rewriting rules.  
The result is that during execution these bigger reduction steps are
taken whenever possible.

   For example, consider the \verb|Wff| datatype defined
in section~\ref{sec:rewriting-modifier}.  Suppose we have the following
two rewriting rules in our rule set:
\begin{verbatim}
   Not(Not X): X
|  FORALL(X,A,P): Not(EXISTS(X,A,Not(P)))
\end{verbatim}

Given an input term of the form \verb|FORALL(X,A,Not(P))|, the term will
first be reduced to \verb|Not(EXISTS(X,A,Not(Not(P))))| via the second rule,
then to \verb|Not(EXISTS(X,A,P))| via the first rule.  The optimizer
can discover that this two step reduction can be compressed into one
step if the rewriting system is augmented with the following rule:

\begin{verbatim}
   FORALL(X,A,Not P): Not(EXISTS(X,A,P))
\end{verbatim}

Note that this augmented rule is a {\em specialization} of the two rules 
given by the user.  In general, the rewriting optimizer will repeatedly
look for useful specializations and add them to the user rule set\footnote{
To ensure termination of the optimization process, 
augmented rules will not be further specialized.}.  
These specialized rules will perform multiple reductions in one single step.

To make use of the optimizer, the rewriting system must have the following
properties:
\begin{enumerate}
   \item  It must be strongly normalizing, i.e. it will terminate under all
   inputs.  
   \item  It must be confluent, or else confluence is not necessary. 
   Notice that the optimizer can and will alternate the reduction order
   of a rule set.  Thus if a rewriting system is non-confluent, an
   optimized rewriting system may give a different result.
\end{enumerate}

In addition, the following things should hold to maximize the effectiveness
of the optimizer.
\begin{enumerate}
   \item  Indices (i.e. state caching) should be enabled.   Otherwise
   the optimizer will not be able to optimize a rule.
   \item  Specializable rules must be simple, i.e. the right hand side
   must be written as an expression consist of only \Prop{} datatypes.
   If the right hand side is a complex statement or if it involves
   external function calls, the optimizer will fail to analyze it properly.
   \item  Only bottom-up rules may be present.
   \item  Avoid conditional rules whenever possible.  
\end{enumerate}

Finally, it should be warned that an optimized rewriting system  
may generate a lot more code than the un-optimized one.  The user should
view the generated report and check that excessive specializations have not 
been performed.

\Section{User defined datatypes: Views} \label{sec:views}

   For many applications it is necessary to interface \Prop{} to 
data structures predefined in libraries, or generated by other tools.
In order to make it possible to work with these external data structures,
\Prop{}, from version 2.3.3. onward, provides a {\em view} mechanism to
interface with these data structures.  Using views, an externally defined
type can be used within \Prop's pattern matching and rewriting
mechanisms transparently, just as if it were defined by \Prop. 
The \Prop{} translator will call the
appropriate interface functions specified by the user to manipulate
these external datatypes.

\Subsection{A first example} \label{sec:tagged-union}.
To illustrate the use of views, we'll begin with a simple example.

Suppose in our application we use the following tagged C union to
represent our expression datatype.  An expression is represented
simply as a pointer to this structure.  An null expression is
represented simply as the NULL pointer.  

\begin{verbatim}
enum exp_tag { Int = 1, Ident, Add, Sub, Mul, Div };
struct exp {
   enum exp_tag tag;
   union {
      int          number;
      const char * ident;
      struct { struct exp * l, * r } children;
   } u;
};
typedef struct exp * Exp;
/* User defined constructor functions */
Exp NONE = NULL;
extern Exp INT(int);
extern Exp IDENT(const char *);
extern Exp ADD(Exp, Exp);
extern Exp SUB(Exp, Exp);
extern Exp MUL(Exp, Exp);
extern Exp DIV(Exp, Exp);
\end{verbatim}

Suppose this data structure is already used extensively in other 
modules, and we'd like to \Prop{} pattern matching mechanisms while keeping 
the old code intact.  We can accomplish this by declaring a
view to this data structure, using a \TDEF{datatype view}
definition:
\begin{verbatim}
dataype Exp :: view =
   match (this ? this->tag : 0)
     view 0     => NONE
   | view Int   => INT(int = this->u.number)
   | view Ident => ID (const char * = this->u.ident)
   | view Add   => ADD(Exp = this->u.children.l, Exp = this->u.children.r)
   | view Sub   => SUB(Exp = this->u.children.l, Exp = this->u.children.r)
   | view Mul   => MUL(Exp = this->u.children.l, Exp = this->u.children.r)
   | view Div   => DIV(Exp = this->u.children.l, Exp = this->u.children.r)
   ;
\end{verbatim}

In the above, a view named \verb|Exp| is defined.   Let's describe
what the definition means: in general, 
a view definition has three main parts, the {\em view selector},
a set of {\em view transformation rules}, and a set {\em of view accessor
expressions}.  These parts perform the following tasks:

\begin{description}
\item[View selector]  The {\em view selector} is specified in the
the \TDEF{match} part of the definition.   In this example, 
the expression associated with \T{match} is 
\verb|this ? 0 : this->tag|.  This returns a variant discrimination integer
from \verb|0| to \verb|Div|.  
Note that the pseudo variable \TDEF{this} 
refers to an object of type \verb|Exp|.

\item[View transformation rules]
Each of the clauses 
\[ \TDEF{view} {\it exp} \verb|=>| {\it term} \]
specifies
a {\em view transformation rule}.   These rules are used to specify
how to transform a computed variant tag into a pattern.    In general,
expression {\it exp} must be a constant expression usable in a 
\verb|case| statement.

\item[View accessors]
In each of the constructor arguments, an {\em view accessor} expression
can be defined to access the logical components of the constructor.
In the rule:
\begin{verbatim}
   | view Add   => ADD(Exp = this->u.children.l, Exp = this->u.children.r)
\end{verbatim}

The expressions \verb|this->u.children.l| and \verb|this->u.children.r|
tell \Prop{} how to access the left and right child of the an addition
expression node.

Note that since the variant \verb|NONE| has no arguments, it is represented
as a nullary constructor, and no accessors expressions are defined.
\end{description}

A pretty printing function for \verb|struct exp| can be implemented
in \Prop{} using pattern matching as follows:
\begin{verbatim}
ostream& operator << (ostream& s, Exp e)
{  match (e)
   {  NONE:     { s << "none"; }
   |  INT i:    { s << i; }
   |  ID id:    { s << id; }
   |  ADD(a,b): { s << '(' << a << " + " << b << '); }
   |  SUB(a,b): { s << '(' << a << " - " << b << '); }
   |  MUL(a,b): { s << '(' << a << " * " << b << '); }
   |  DIV(a,b): { s << '(' << a << " / " << b << '); }
   }
   return s;
}
\end{verbatim}

Note that this is exactly the same code we would've written if
\verb|Exp| is defined as a datatype rather than a view.  This means
that the user can reimplement the expression data structure
as a \Prop's datatype at a latter time, without altering the \Prop{}
source code.  In effect, the {\em logical} view of a datatype, and its
{\em physical} implementation can separately.
With this {\em representation transparency} capability,
programs written in \Prop's pattern matching formalisms can evolve 
in an incrementally and pain-free manner.

The generated code for the above function will resemble the following
output.  Note that all the view selectors and view accessors are
inlined into the code, and in general, views incur no overhead to
their use.
\begin{verbatim}
ostream& operator << (ostream& s, Exp e)
{  switch (e ? e->tag : 0)
   {  case 0:      { s << "none"; }
      case Int:    { s << e->u.number; }
      case Ident:  { s << e->u.ident; }
      case Add:    { s << '(' << e->u.children.l 
                       << " + " << e->children.r << '); }
      case Sub:    { s << '(' << e->u.children.l 
                       << " - " << e->children.r << '); }
      case Mul:    { s << '(' << e->u.children.l 
                       << " * " << e->children.r << '); }
      case Div:    { s << '(' << e->u.children.l 
                       << " / " << e->children.r << '); }
   }
   return s;
}
\end{verbatim}

\Subsection{Another view example}

Suppose we use the following alternative implementation, a
\verb|Expression| class hierarchy to represent
an expression in our AST.  
Note that \verb|Exp| is an abstract base class inherited
by the classes \verb|Number|, \verb|Operator|, \verb|Identifer|, etc.

\begin{verbatim}
class Expression
{
public:

   // This class uses a ``wide'' interface
   virtual int  number() const         = 0;  // returns a number
   virtual const char * ident() const  = 0;  // returns an identifier

   // returns the ith subchild
   virtual const Expression * child(int) const = 0; 
   // mutable version of above
   virtual Expression *& child(int) = 0;     
};

typedef Expression * Exp;

class NullExp  : public Expression { ... };
class Number   : public Expression { ... };
class Operator : public Expression { ... };
class Identifier : public Expression { ... };
\end{verbatim}

In order to provide a suitable interface to \Prop's view mechanism,
we introduce an additional member function to return a variant tag:

\begin{verbatim}
class Expression
{
public:
   enum Variant { None, Int, Ident, Add, Sub, Mul, Div };
   virtual Variant get_variant() const = 0;
}
\end{verbatim}

In addition, we assume the following constructor functions have been
implemented:

\begin{verbatim}
extern Exp INT(int);
extern Exp IDENT(const char *);
extern Exp ADD(Exp, Exp);
extern Exp SUB(Exp, Exp);
extern Exp MUL(Exp, Exp);
extern Exp DIV(Exp, Exp);
\end{verbatim}

Now, with these definitions in place, we can define a datatype view of the
above \verb|Expression| hierarchy using the following definition:

\begin{verbatim}
datatype Exp :: view
   = match (this->get_variant())
     view Expression::None  => NONE
   | view Expression::Int   => INT (int = this->number()) 
   | view Expression::Ident => ID  (const char * = this->ident())
   | view Expression::Add   => ADD (Exp = this->child(0), 
                                    Exp = this->child(1))
   | view Expression::Sub   => SUB (Exp = this->child(0), 
                                    Exp = this->child(1))
   | view Expression::Mul   => MUL (Exp = this->child(0), 
                                    Exp = this->child(1))
   | view Expression::Div   => DIV (Exp = this->child(0), 
                                    Exp = this->child(1))
   ;
\end{verbatim}

An evaluation function can be written
using \Prop's pattern matching construct:

\begin{verbatim}
int eval(Exp e, const Env& env)
{  match (e) 
   {  NONE:          { return 0; }
   |  INT i:         { return i; }
   |  ID id:         { return env(id); }
   |  ADD (x,y):     { return eval(x,env) + eval(y,env); }
   |  SUB (x,y):     { return eval(x,env) - eval(y,env); }
   |  MUL (x,y):     { return eval(x,env) * eval(y,env); }
   |  DIV (x,INT 0): { cerr << "Division by zero"; return 0; }
   |  DIV (x,y):     { return eval(x,env) / eval(y,env); }
   }
}
\end{verbatim}
Note that this code works equally well with the tagged union representation
defined in section~\ref{sec:tagged-union}. 

We can also use rewriting to transform an expression tree as follows

\begin{verbatim}
void simplify(Exp& e)
{
   rewrite (e) type (Exp)
   {  ADD(INT i, INT j): INT(i+j)
   |  SUB(INT i, INT j): INT(i-j)
   |  MUL(INT i, INT j): INT(i*j)
   |  DIV(INT i, INT j): INT(i/j)
   }
}
\end{verbatim}

\Subsection{Syntax of view definitions}

The general syntax of a view definition resembles that of a
\T{datatype} definition:

\begin{syntax}
\NDEF{Datatype\_Spec} 
   & \IS & \N{Datatype\_View\_Spec} \\
\NDEF{Datatype\_View\_Spec} 
   & \IS & \Id \T{::} \T{view} \T{=} \\ 
   &     & \quad \T{match} \T{(} \N{Exp} \T{)} \\
   &     & \quad \N{View\_Cons\_Specs} \\
\NDEF{View\_Cons\_Specs} 
   & \IS & \ALT{\TDEF{view} \N{Exp} \T{=>} \N{Cons\_Spec}}{|} \\
\end{syntax}

Default arguments in a \N{Cons\_Spec} are interpreted as view accessors.

There are a few general rules to observe when defining a view:
\begin{enumerate}
 \item A \T{datatype view} definition only defines the interface
 to an external data structure and no code is generated.  In particular,
 the defined view name is not an actual \Cpp{} type.
 Thus the user is responsible for defining all appropriate type definitions.
 \item All other qualifiers other than \T{view} are ignored.   Thus 
 automatic generation of pretty printers, garbage collection interface
 routines, etc. are disabled for views.
 \item In general, the view accessors must be usable as lvalues
 if the user wants to modify a view datatype.  This includes the
 situation when in-place rewriting is used.
 \item Finally, the user are responsible for generating the
 constructor functions.  In the two examples given above, the user
 should implement the functions \T{INT}, \T{ID}, etc.  
\end{enumerate}

\Section{Graph Types and Graph Rewriting} \label{sec:graph}

A typical compiler uses many types of graphs to represent internal program 
information.  For instance, control flow graphs, 
data dependence graphs, call graphs, def/use chains are
some of the most common examples.  In order to automate the process
of implementing and manipulating these graph structures, \Prop{} provides
a generic formalism-- {\em graph types}--
for specifying multisorted, attributed, hierarchical directed graphs in 
very high level manner.  The data structure mapping process of translating
high level graph specifications into concrete \Cpp{} classes is completely
automated.  Thus the user can concentrate on specifying the semantic
relationships between the elements of the domain, instead of the
implementation details.

The graph structure mapping process uses real-time set machine
simulation\cite{real-time-SETL} and subtype analysis\cite{subtype-SETL} 
to convert associative accesses involving sets, maps, and multimaps
into hash free, worst-case constant time pointer manipulations.
This optimization is performed transparently by the translator.

Manipulation of graphs structures are done in three ways: 
\begin{description} 
\item[object-oriented] 
 Using a standard interface generated from \Prop{} the user can 
manipulate a graph in the usual procedural/object-oriented manner. 
\item[set formalism]  Using an embedded SETL\cite{SETL}-like sublanguage, 
the user can manipulate the graphs using high level set operations 
such as set comprehension.  
\item[graph rewriting]  At the highest level, analysis and transformation 
can be carried out using the graph rewriting formalism.  
\end{description}

In this section we'll describe each of these topics in detail.

\Subsection{Graph Types} \label{sec:graph-types}

A \INDEX{graph type} declaration is used to specify a
graph structure 
with multiple sorts of labeled nodes and labeled directed edges.  
Attributes can be attached to both nodes and edges.  
The syntax of \T{graphtype} declarations is as follows:
\begin{syntax}
\NDEF{Graph\_Type} 
      & \IS & \TDEF{graphtype} \Id \\
      & & \quad \OPT{\T{:} \N{Inherit\_List}} \\
      & & \quad \OPT{\T{::} \MORE{\N{Graph\_Mode}}{}} \\
      & & \T{declare} \\
      & & \quad \TDEF{node:} \ALT{\N{Node\_Def}}{|} \\
      & & \quad \TDEF{edge:} \ALT{\N{Edge\_Def}}{|} \\
      & & \T{begin} \\
      & & \quad \N{Code} \\
      & & \T{end} \T{graphtype} \T{;} \\
\NDEF{Graph\_Mode} & \IS & \\
\NDEF{Node\_Def} & \IS & \Id \OPT{\OPT{\T{of}} \N{TypeExp}} \\
\NDEF{Edge\_Def} 
    & \IS & \Id \OPT{\T{of}} \TypeExp \T{->} \TypeExp \\
    & \OR & \Id \OPT{\T{of}} \TypeExp \T{<->} \TypeExp \\
    & \OR & \Id \OPT{\T{of}} \TypeExp \T{<=>} \TypeExp \\
    & \OR & \Id \OPT{\T{of}} \TypeExp \T{<=>*} \TypeExp \\
\end{syntax}

\Subsection{The Graph Interface} \label{sec:graph-interface}
\INCOMPLETE

\Subsection{Set Formalisms} \label{sec:set-formalisms}
\INCOMPLETE

\Subsection{Graph Rewriting} \label{sec:graph-rewriting}
\INCOMPLETE

\Section{Running the Translator} \label{sec:usage} 

The \Prop{} translator is a program called \verb|prop|.  The translator
uses the following command line syntax:
\begin{syntax}
  \NDEF{Running\_Prop} & ::= &
   \T{prop} \OPT{{\em prop\_options}} \ALT{{\em file}}{} \\
\end{syntax}

Here, \N{File} is a file with suffix \verb|.p|$*$.  By default, the output
file will have the same name with the \verb|p| extension removed.
For example, a file named ``foo.pC'' will be translated into
the file ``foo.C''  Similarly, a file named ``bar.ph'' will be translated 
into ``bar.h''  

\Subsection{Options}
The available options to the translator are as follows:
\begin{description}
\item[\OPTIONDEF{-G -GNU}]      Use GNU style error messages.
\item[\OPTIONDEF{-I{\em path}}] Use the search path {\em path}.
\Prop{} will search {\em path} to look for \Prop{} files with 
the suffix \verb|.ph| inside a \verb|#include| directive.  Multiple
paths can be specified by using multiple \verb|-I| options.
\item[\OPTIONDEF{-l -no\_line\_directives}]   By default, \Prop{} will try
to generate \verb|#line| directives to correlate the source and
the translated output.  This option suppresses this feature.
\item[\OPTIONDEF{-M -make\_depends}]  Similarly to the \verb|-M| option
in \verb|cc| and \verb|cpp|.  This option generates a suitable dependency
list for Makefiles.
\item[\OPTIONDEF{-memory\_usage}]  Print memory use during translation.
\item[\OPTIONDEF{-n -no\_codegen}]  
Don't output any code; perform semantic checking 
only.
\item[\OPTIONDEF{-N -non\_linear}]  
Use non-linear patterns during pattern matching.
A non-linear pattern is one in which a variable occurs more than
once.   By default, \Prop{} consider this an error.
\item[\OPTIONDEF{-o{\em outfile}}]  Send output to {\em outfile} instead of 
the default.
\item[\OPTIONDEF{-Ofast\_string\_match}]  
Generate (potentially) faster string matching
code by expanding string literal tests into character tests.  By default, 
\Prop{} generates string tests using \verb|strcmp| and binary search.
\item[\OPTIONDEF{-Oadaptive\_matching}]  Use the adaptive pattern matching 
algorithm\footnote{The implementation may not be effective.}.
\item[\OPTIONDEF{-Oinline\_casts}]  
Generate inline type casts instead of function
calls that do the same.  This may be faster for compilation.
\item[\OPTIONDEF{-Orewriting}]  Enable the rewriting optimizer. 
See also section~\ref{sec:optimize-rewriting} for details.
\item[\OPTIONDEF{-Otagged\_pointer}]  
Use a tagged pointer representation instead
of representing variant tags explicitly in a structure field.  For
example, if there are two variants called \verb|Foo| and
\verb|Bar|  in a datatype.  Then a pointer to \verb|Foo| can be tagged
with a low order bit 0 while a pointer to \verb|Bar| can be tagged with 
a low order bit 1.  Accessing the tag thus becomes just a simple bit
test.  This should save space and may be faster\footnote{However, this feature
has not been fully tested.}
\item[\OPTIONDEF{-r -report}]  
Generate a report whenever necessary.  Parser, string
matching, and rewriting constructs produce reports.
\item[\OPTIONDEF{-s -strict}]  Use strict semantic checking mode.  
All warnings are considered to be errors.
\item[\OPTIONDEF{-S -save\_space}]\label{option:save-space} 
Use space saving mode.  
Try not to use inline functions if code space can be saved.  
Datatype constructors will not be inlined
in this mode.  Instead, datatype constructors will be generated at
where the corresponding \TDEF{instantiate datatype} declaration occurs.
\item[\OPTIONDEF{-t -stdout}]   Send translated program to the standard output.
\item[\OPTIONDEF{-use\_global\_pool}]  Use global memory pool for allocation.
\item[\OPTIONDEF{-v{\em num}}]  Use verbose mode in report generation.
This will provide more detailed information.
\end{description}

\Subsection{Error Messages}

The following is a canonical list of all error messages generated by
the \Prop{} translator.  Each message is prefixed by the
file name and line number in which the error occurs.    Most of these
errors messages are self explanatory; more detailed explanations are provided
below.

\begin{Error}
\errormsg{unknown option {\em option}}  The translator does not recognize 
the command line {\em option}.   Type {\em prop} to see a list of options.
\errormsg{Error in command: {\em command}}  The translator fails when trying
to execution {\em command}.  When multiple files are specified in a command
line, {\em prop} invokes itself on each of the file. 
\errormsg{{\em pat} with type {\em type} is not unifiable}  Pattern has not
been declared to be a unifiable type.   
\errormsg{Sorry: pattern not supported in rewriting: {\em pat}}  Pattern
{\em pat} is not supported.  Currently, logical pattern connectives
are not supported in rewriting.  You'll have to rewrite them into
non-logical form.
\errormsg{Unknown complexity operator: {\em op}}  
\errormsg{accessor is undefined for view pattern: {\em pat}} An accessor
function has not been declared for a constructor.  When using datatype views
\errormsg{arity mismatch (expecting $n$) in pattern: {\em pat}}  Pattern
{\em pat} is expected to have arity $n$.   The arity is the number of
expressions that you're matching concurrently.  If you are matching $n$
objects then there must be $n$ patterns for each rule.
\errormsg{arity mismatch in logical pattern: {\em pat}}  Logical patterns
do not match the arity of a pattern.
\errormsg{bad constructor type {\em type}}
\errormsg{bad view constructor pattern: {\em pat}}
\errormsg{can't find include file {\em file}}  Additional search paths
can be specified with the \verb|-I| option
\errormsg{component \#$i$ not found in constructor {\em con}}  
Datatype constructor {\em con} does not have a component named \#$i$.
This can happen when you write a record constructor expression and misspelt
one of the labels.
\errormsg{component $l$ not found in constructor {\em con}}  Datatype
constructor {\em con} does not have a labeled component $l$.
\errormsg{constructor {\em con} already has print formats}
\errormsg{constructor {\em con} is not a class}
\errormsg{constructor {\em con} is not a lexeme}  A constructor is used
as a terminal in the parser but it has not been declared to be a lexeme.
\errormsg{constructor {\em con} doesn't take labeled arguments}  You're trying
to use record constructor syntax on a constructor take does not take
record arguments.
\errormsg{constructor {\em con} is undefined}  
\errormsg{cyclic type definition in type {\em id} = {\em type}}
Type abbreviations cannot be cyclic.  For recursive types, use \verb|datatype|
definitions.
\errormsg{datatype {\em T} is not a lexeme type}  {\em T} is used as a
terminal within syntax class when it has not been declared as a lexeme type.
All non-terminals must be a defined using the \verb|lexeme| qualifier.
\errormsg{determinism {\em det} not recognized}
\end{Error}

\begin{Error}
\errormsg{duplicated definition of pattern constructor '{\em con}'}
  The constructor {\em con} has already been in another datatype.  {\em Prop}
  does not allow overloading of constructor names.
\errormsg{duplicated label '$l$' in expression: {\em exp}}
  A record type has one of its labels duplicated.
\errormsg{duplicated label '$l$' in type {\em type}}
\errormsg{duplicated pattern variable '{\em id}'.  Use option -N}
   By default,
   pattern variables may not be duplicated within a pattern.  Non-linear
   patterns are allowed only when the option \verb|-N| is invoked.
\errormsg{edge $e$ is not defined in graphtype {\em id} }
\errormsg{empty type list in rewrite (...) ... }
\errormsg{expecting $c_1$ ... $c_2$ (from line $l$) but found $c_3$ ... $c_4$ 
   instead}  Quotation symbols are not properly balanced. 
\errormsg{expecting non-terminal {\em nt} to have type $t_1$ but found $t_2$}
   Non-terminal {\em nt} has already been defined to have a
   synthesized attribute type of $t_1$ but $t_2$ is found.   This can happen
   if you have rules with the same lhs non-terminal previously defined.  
\errormsg{expecting type $t_1$ but found $t_2$ in pattern variable '{\em v}'}
{\em Prop} performs type inference on all the lhs patterns to determine
the types of all variables.  Type errors can occur if the patterns are
miswritten.
\errormsg{expecting type $t_1$ but found $t_2$}
\errormsg{flexible vector pattern currently not supported in 
rewriting: {\em pat}}
\errormsg{format \#$i$ used on constructor {\em con}}
\errormsg{format \#$l$ used on non-record constructor {\em con}}
\errormsg{function name mismatch: expecting $f$ ...}
\errormsg{illegal character $c$}
\errormsg{illegal context type: {\em type}}  A context type in
a \verb|matchscan| statement must be previously defined to be datatype.
The unit constructors of the datatype can be used as context values.
\errormsg{illegal context(s) in pattern {\em pat}}
\errormsg{illegal format '\_' on constructor {\em con}}
\errormsg{illegal incomplete record pattern: {\em pat}}
\errormsg{illegal print format '$c$' in constructor {\em con}}
\errormsg{illegal record label '$l$' in expression: {\em exp}}
\errormsg{illegal width in bitfield '{\em id} ($n$)'}
\errormsg{inherited attribute '{\em type}' can only be used in treeparser 
mode in rewrite class {\em id}}  
\end{Error}

\begin{Error}
\errormsg{law {\em id}(...) = {\em pat} is not invertible}   Pattern {\em pat}
cannot be treated as an expression.  It may involve logical patterns and
wild cards.
\errormsg{law '{\em id}': bound variable '$v$' is absent in body {\em exp}}
A parameter variable $v$ must be present within the body of a law.

\errormsg{law '{\em id}': type {\em type} cannot be used in parameter {\em id}}

\errormsg{lexeme {\em id} is undefined}

\errormsg{lexeme class {\em id} is undefined}

\errormsg{lexeme pattern is undefined for constructor {\em con}}

\errormsg{lexeme \{{\em id}\} already defined as {\em regexp}}

\errormsg{lexeme \{{\em id}\} is undefined in {\em regexp}}

\errormsg{missing '\}' in regular expression {\em regexp}}

\errormsg{missing label '$l$' in expression: {\em con} {\em exp}}

\errormsg{missing non-terminal in tree grammar rule: {\em nt}}
\errormsg{missing type {\em type} in the traversal list of rewrite class {\em id}} 
Within the rewriting rules, you have used a pattern that involve a constructor
of type {\em  type} directly but no such types are defined in the 
rewrite class definition.  You should add the type to the traversal list.

\errormsg{missing type info in expression {\em exp} : {\em type}}
   Sometimes {\em prop} fails to infer the type of an expression within
a {\em match} statement.  It such
cases it is necessary to help out the translator by adding additional
type information in the patterns.  For example, rewrite some pattern $p$
as $(p : \tau)$ where $\tau$ is the type of $p$.

\errormsg{missing type info in function {\em f} {\em type}}  Similar to above
\errormsg{missing type info in rule: {\em f} {\em pat} : {\em type}}
   Similar to above.
\errormsg{missing view selector for pattern: {\em pat}}
\errormsg{multiple mixed polarity pattern variable '{\em v}'}
\errormsg{multiple unit constructors in polymorphic type {\em id} {\em arg} 
is not supported}
\errormsg{negative cost {\em cost} is illegal}
\errormsg{node {\em id} is not defined in graphtype {\em id}}
\errormsg{non lexeme type {\em type} in pattern {\em pat}}
\errormsg{non-relation type {\em type} in pattern: {\em pat}}
\errormsg{pattern is undefined for lexeme {\em l}}
\errormsg{pattern scope stack overflows}
\errormsg{pattern scope stack underflows}
\end{Error}

\begin{Error}
\errormsg{pattern variable '$v$' has no binding at this point}
\errormsg{persist object id is undefined for {\em con}}
\errormsg{persistence pid {\em name} already allocated for type {\em type}}
\errormsg{persistence redefined for type {\em type}}
\errormsg{precedence symbol must be terminal: {\em term}}
\errormsg{predicate {\em p}: expecting type $t_1$ but found $t_2$}
\errormsg{redefinition of bitfield '{\em field}'.}
\errormsg{redefinition of constructor '{\em con}'}
\errormsg{redefinition of datatype {\em id}}
\errormsg{redefinition of lexeme class {\em id}}
\errormsg{replacement not in rewrite class: rewrite {\em exp}}
\errormsg{rewrite class {\em id} has already been defined}
\errormsg{rewrite class {\em id} is undefined}
\errormsg{rule $r$ has incomplete type {\em type}}
\errormsg{rule $r$ is of a non datatype: {\em type}}
\errormsg{syntax class {\em id} has already been defined}
\errormsg{syntax class {\em id} is undefined}
\errormsg{synthesized attribute '{\em type}' 
can only be used in treeparser mode in rewrite class {\em id}}
\errormsg{the class representation of constructor {\em id} has been elided}
\errormsg{this rule is never selected: $r$}
\errormsg{this rule is never used: $r$}
\errormsg{too few arguments {\em arg} in instantiation of type scheme {\em type}}
\errormsg{too many arguments {\em arg} in instantiation of type scheme {\em type}}
\errormsg{type {\em type} is not a datatype}
\errormsg{type {\em type} is undefined}
\end{Error}

\begin{Error}
\errormsg{type {\em id} = {\em type} is not a datatype}
\errormsg{type {\em id} has already been defined as {\em type}}
\errormsg{type {\em id} has unknown class: {\em C}}
\errormsg{type '{\em type}' is not rewritable in treeparser mode rewrite class 
{\em id}}
    All datatypes used within the treeparser mode of rewriting must be defined
    with the \verb|rewrite| qualifier.
\errormsg{type error in pattern {\em pat}: {\em type}}
\errormsg{type mismatch between nonterminal {\em nt}(type $t_1$) 
and rule $r$(type $t_2$)}
\errormsg{type mismatch in pattern: {\em pat}}
\errormsg{type mismatch in rule $r$}
\errormsg{type mismatch in rule $r$ {\em pat}}
\errormsg{type or constructor {\em con} is undefined}
\errormsg{unable to apply pattern scheme {\em pat}}
\errormsg{unbalanced $c_1$ ... $c_2$ at end of file}
\errormsg{undefined non-terminal {\em nt}}
\errormsg{unification fails occurs check with $t_1$ and $t_2$}
\errormsg{unmatched ending quote $c$}
\errormsg{unrecognized quoted expression `{\em exp}`}
\errormsg{unrecognized quoted pattern `{\em pat}`}
\end{Error}

\pagebreak
\appendix

\Section{Garbage Collection in the {\sf Prop} Library} \label{appendix:gc}
   In this appendix
we describe the design and implementation of a garbage collector
based on the Customisable Memory Management framework(CMM)\cite{CMM} in our
\Prop{} \Cpp{} class library.  Like the previous approach, we have 
implemented a mostly copying conservative collector based on the work
of Bartlett\cite{Mostly-copying}.  
Similar to CMM, our architecture provides a protocol to allow 
multiple garbage collectors using different algorithms to coexist in the 
same memory space.  A few improvements are made to
improve the performance, the flexibility and the functionality of 
our collector: to reduce retention due to false roots identification,
blacklisting\cite{Boehm} is used to identify troublesome heap
addresses; the architecture of the system has been generalized so that it is 
now possible to have multiple instantiations of Bartlett-style heaps; 
finally, object finalization and weak pointer support
are added.  Our collector has been tested on gcc 2.5.8 under Linux,
and SunOS 4.1.x\footnote{An alpha version also works under Solaris. Thanks
to Paul Dietz for the patch.}

\Subsection{Introduction}
   The \Prop{} project involves the design and implementation of 
an environment and an extension language based on \Cpp{} 
for compiler construction and program transformation.  
An outgrowth of this project is the \Prop{} 
\Cpp{} class library, which contains an extensive set of support algorithms 
and data structures.  Since a typical compiler manipulates many complex tree,
dag and graph objects with extended lifetime, manual memory management using
\Cpp's constructors and destructors, reference counting smart pointers, or
some other techniques is frequently complex and error prone.  
Furthermore, with the advent of new algorithms for garbage collection,
manual memory management techniques do not necessarily provide better 
performance or space utilization.  To make it possible to make use of garbage 
collection as a viable alternative for memory management in
\Cpp\cite{Safe-C++-GC}, we have implemented a garbage collection class 
hierarchy as part of the \Prop{} library.  The class library can be used
directly by the users who'd like to program in \Cpp; it can also be
used as part of the runtime system of a highly level language implemented
in \Cpp, as we have done for \Prop.

   We have several good reasons to prefer garbage collection over manual 
memory management.  The \Prop{} language contains many declarative constructs 
and extensions such as algebraic datatypes, pattern matching, rewriting, and
logical inference.  When a user programs in \Prop{} using these constructs, 
an applicative style of programming is the most natural paradigm.  

\Subsection{The Framework}

   We base our design on the work on the Customisable Memory Management(CMM)
system\cite{CMM}.  In this framework, multiple independent heaps(including
the usually non-collectable heap) can coexist with each other.
Bartlett's mostly copying garbage collector is used as the primary collector.
CMM extends the work of Bartlett\cite{Mostly-copying} 
by allowing cross heap pointers and unrestricted interior pointers.

  However, all collectable objects are required to derive from a base class 
and reimplement a tracing method, which identifies the internal pointers of 
an object.  This burden is usually quite small and in fact can be
automated from the type definition\footnote{In \Prop, a special keyword
\T{collectable} is used to identify garbage collected classes and types.
The \Prop{} translator uses this type annotation to derive the appropriate  
tracing methods.}

  One of the major advantages of the CMM framework is that multiple
coorperating collectors can coexist at the same time, which makes it
possible to customize the behavior of each collector with respect to
the lifetime behavior of the objects.  In \cite{CMM}, examples are
given in which the lifetime of certain class of objects exhibit
first-in/first-out behavior.  In this case a special collector can be
written to take full advantage of this property.

\subsection{Our Framework}
   Our framework retains all the benefits and flexibilities of CMM, 
while extending it in several minor but useful ways:  
\begin{itemize}
   \item Like CMM, interior pointers(i.e. pointers to the middle of an object)
and crossheap pointers (i.e. complex data structures linking nodes locating
in multiple logical heaps.) are supported.  Pointers to collectable
objects are non-intrusive: i.e. they are normal \Cpp{} pointers 
and not encapsulated in templates.
   \item Also like CMM, we allow multiple garbage collectors using different
algorithms to coexist.  Unlike CMM, however, we allow multiple 
Bartlett-style collectors to be instantiated.  Most of the services 
involving low level page management and object marking have been relegated
to a separate heap manager.   
   \item We provide support for finalization and weakpointers.
   \item We have implemented blacklisting\cite{Boehm} to reduce the chance
of false roots identification.
\end{itemize}

\subsection{The Implementation}

   Our implementation has been completely written from scratch since
we do not desire to utilize copyrighted code and, more importantly,
we have to make a few important architectural changes:
 All low level memory management services, such as management of
the page table and the object bitmap, is now relegated to the class 
\CLASSDEF{GCHeapManager}.  Collectors now act as clients as of the heap manager
and the Bartlett-style collector no longer has any special status. 

\subsection{Architecture}

   The architecture of the memory management system is partitioned into
a few classes, each responsible for providing distinct services:

\begin{itemize}
   \item \CLASSDEF{GCHeapManager} --- The heap manager.  The heap manager
    manages the heap table, performs page level
    allocation and deallocation, and provides miscellaneous services
    like blacklisting.  It also manages the object bitmaps.
   \item \CLASSDEF{GC} --- The base class for all garbage collectors.
    This base class describes the protocol used by all the collector
    classes\footnote{This base class is also inherited from class
    \CLASSDEF{Mem}, so that it adheres to the \Prop{} memory management
    protocol.}
   \item \CLASSDEF{CGC} --- The base class for conservative collectors.
     This class is inherited from class \CLASSDEF{GC} and implements some
     methods for locating the stack, heap, and static data areas.
   \item \CLASSDEF{BGC} --- Bartlett-style mostly copying collector.
      This class is inherited from class \CLASSDEF{CGC} and implements
      the Bartlett mostly copying algorithm.
   \item \CLASSDEF{MarkSweepGC} --- Mark/sweep style conservative collector.
      This class is inherited from class \CLASSDEF{CGC} and implements
      a mark/sweep collection algorithm.
   \item \CLASSDEF{WeakPointerManager} --- The weakpointer manager.
      This class manages the weak pointer table and provides a few
      weak pointer scavenging services for the collectors.
\end{itemize}
  
\Subsection{The Programmatic Interface}

   The programmatic interface to the garbage collector involves two
base classes, \CLASSDEF{GC} and \CLASSDEF{GCObject}.  The base class 
\CLASSDEF{GC} provides an uniform interface to all types of collectors 
and memory
managers while class {\sf GCObject} provides the interface to all 
collectable classes.  Table~\ref{GC-Classes} contains a listing
of the classes in our hierarchy.

\begin{table}
   \begin{center}
      \begin{tabular}{|l|l|} \hline
        Class           & Purpose \\ \hline \hline
        \sf GCObject    & Collectable object base class \\
        \sf GC          & Garbage collector base class \\
        \sf CGC         & Conservative garbage collector base class \\
        \sf BGC         & Bartlett style mostly copying collector \\
        \sf MarkSweepGC & A mark-sweep collector \\
        \sf UserHeap    & A heap for pointerless object \\
        \sf GCVerifier  & A heap walker that verifies the integrity of a 
                          structure \\
      \hline 
      \end{tabular}
   \end{center}
   \caption{\label{GC-Classes} Garbage Collection Classes.}
\end{table}

Class \CLASSDEF{GCObject} is extremely simple: it redefines the operators
\verb|new| and \verb|delete| to allocate memory from the default collector,
and declares a virtual method ``\verb|trace|'' to be defined by subclasses
(more on this later.)

Memory for a \CLASSDEF{GCObject} is allocated and freed using the usually
\verb|new| and \verb|delete| operators.  Of course, freeing memory explicitly
using \verb|delete| is optional for many subclasses of \CLASSDEF{GC}. 

\begin{verbatim}
   class GCObject {
   public:
      inline void * operator new(size_t n, GC& gc = *GC::default_gc) 
         { return gc.m_alloc(n); }
      inline void * operator new(size_t n, size_t N, GC& gc = *GC::default_gc) 
         { return gc.m_alloc(n > N ? n : N); }
      inline void   operator delete(void *);
      virtual void trace(GC *) = 0;
   };
\end{verbatim}

\subsection{Memory Allocation}

The base class \CLASSDEF{GC} is slightly more complex, as it has to provide
a few different functionalities.  The first service that class \CLASSDEF{GC} 
must 
provide is of course memory allocation and deallocation.  As a time saving
device we can specify what the default collector is using the methods
\verb|get_default_gc| and \verb|set_default_gc|.  Method \verb|GCObject::new| 
will use this collector by default, unless placement syntax is used.
Method \verb|GCObject::delete|, on the other hand, can correctly infer the 
proper collector to use by the address of the object.

The low level methods to allocate and deallocate memory are \verb|m_alloc|
and \verb|free| respectively.  The programmers usually do not have to
use these methods directly. 

The method to invoke a garbage collection of a specific level is 
\verb|collect(int level)|.  This forces an explicit collection.
Method \verb|grow_heap(size_t)| can also be used to explicitly increase
the heap size of a collector.  Depending of the actual behavior
of the subclasses, these methods may have different effects.

\begin{verbatim}
   class GC {
   protected:
      static GC * default_gc;
   public:
      static GC&  get_default_gc()       { return *default_gc; }
      static void set_default_gc(GC& gc) { default_gc = &gc; }
      virtual void * m_alloc   (size_t) = 0;
      virtual void   free      (void *);
      virtual void   collect   (int level = 0) = 0;
      virtual void   grow_heap (size_t) = 0;
      static  void garbage_collect() { default_gc->collect(); }
      virtual void set_gc_ratio(int);
      virtual void set_initial_heap_size (size_t);
      virtual void set_min_heap_growth   (size_t);
   };
\end{verbatim}

\subsection{The GC Protocol}
The collector and collectable objects must cooperate with
each other by abiding to a simple protocol:
\begin{itemize}
   \item All objects that are to be garbage collected must be derived
from \CLASSDEF{GCObject}.  The application programmer must also supply a
``{\bf tracing}'' method.  The purpose of this method is to identify
all internal pointers to other \CLASSDEF{GCObject}'s.   This method is not
used by the application programmer directly but only used internally
by the garbage collectors.
   \item The tracing method of each collectable must in turn call
    the tracing method in the class \CLASSDEF{GC} with each pointer to
    a collectable object that the object owns:
\begin{verbatim}
   class GC {
   public:
      virtual GCObject * trace (GCObject *) = 0;
      inline  void trace (GCObject& obj);
   };
\end{verbatim}
    
     Briefly, the rules are as follows:
   \begin{enumerate}
      \item The tracing method of a collectable \T{Foo} has the following
         general form:
          \begin{verbatim}
             void Foo::trace(GC * gc) 
             {
                ...
             }
          \end{verbatim} 
      \item If class \T{Foo} has a member that is a pointer \verb|p|
            to a collectable object of type \T{Bar}, then add:
           \begin{verbatim}
               p = (Bar)gc->trace(p);
           \end{verbatim}
            to the body of \verb|Foo::trace|. 
     \item  If class \T{Foo} has a member object
            \verb|x| that is a subclass of \T{GCObject}, also add:
           \begin{verbatim}
               gc->trace(x);
           \end{verbatim}
            to the body of \verb|Foo::trace|.
     \item  If class \T{Foo} is derived from a class \verb|Bar| that
            is a subclass of \T{GCObject}, add:
           \begin{verbatim}
               Bar::trace(gc);
           \end{verbatim}
            to the body of \verb|Foo::trace|.
   \end{enumerate} 
    Notice that these methods can be arranged in any order.
\end{itemize}

This protocol can be used by both copying and non-copying collectors.
In addition, the class \CLASSDEF{GCVerifier} also uses this protocol to 
walk the heap in order to verify the integrity of a garbage collected 
data structure.

\subsection{Messages and Statistics}
    All garbage collectors use the following protocols for status
reporting and statistics gathering.  
  
\begin{verbatim}
   class GC {
   public:
      enum GCNotify {
         gc_no_notify,             
         gc_notify_minor_collection,
         gc_notify_major_collection,
         gc_notify_weak_pointer_collection,
         gc_print_collection_time,
         gc_print_debugging_info
      }
      virtual int      verbosity() const;
      virtual void     set_verbosity(int);
      virtual ostream& get_console() const;
      virtual void     set_console(ostream&);
   };
\end{verbatim}

\begin{verbatim}
   class GC {
   public:
      struct Statistics {
         const char *   algorithm;
         const char *   version;
         size_t         bytes_used;
         size_t         bytes_managed;
         size_t         bytes_free;
         struct timeval gc_user_time;
         struct timeval gc_system_time;
         struct timeval total_gc_user_time;
         struct timeval total_gc_system_time;
      }
      virtual Statistics statistics();
   };
\end{verbatim}

    Each collector has a current verbosity level, which can be set
and reset using the methods \verb|set_verbosity(int)| and
\verb|verbosity() const|.  The verbosity level is actually a bit
set containing flags of the type \verb|GC::GCNotify|.  A combination
of these options can be used.
\begin{itemize}
  \item \verb|gc_no_notify| --- no messages at all.
  \item \verb|gc_notify_minor_collection| --- all minor collections
    will be notified by printing a message on the current console.
  \item \verb|gc_notify_major_collection| --- similarly for all major
    collections.
  \item \verb|gc_notify_weak_pointer_collection| --- notify the user
  when weak pointers are being collected.
  \item \verb|gc_print_collection_time| --- this option will let the
    collector to print the time spent during collection (and finalization).
  \item \verb|gc_print_debugging_info| --- this option will print additional
    information such as stack and heap addresses during garbage collection.
\end{itemize}

   The current console of a collector is defaulted to the \Cpp{} stream
\verb|cerr|.  It can be set and inspected with the methods
\verb|set_console(ostream&)| and \verb|ostream& get_console()| respectively.
For example, the user can redirect all garbage collection messages
to a log file \verb|"gc.log"| by executing the following during initialization:

\begin{verbatim}
   ofstream gc_log("gc.log");
   GC::get_default_gc().set_console(gc_log); 
\end{verbatim}

\subsection{The Bartlett style mostly copying collector}

   Currently a Bartlett-style mostly copying collector has been implemented.
The current version is non-generational but we expect that a generational
version will be available in the near future.

   The Bartlett-style collector is implemented as the concrete class
\CLASSDEF{BGC}.  Aside from all the services provided by class \CLASSDEF{GC},
\CLASSDEF{BGC} also provides the following method:

\begin{verbatim}
   class BGC : public CGC {
   public:
      virtual void set_gc_ratio(int);
      virtual void set_initial_heap_size (size_t);
      virtual void set_min_heap_growth   (size_t);
   }
\end{verbatim}

  The gc ratio refers to the ratio of used heap space versus
total heap space.  Class \CLASSDEF{BGC} will invoke the garbage collection
if this is exceeded.  The default gc ratio for \T{BGC} is 75\%.

  The initial size of the heap and the minimal number of bytes to 
request from the system during a heap expansion can be adjusted using 
the methods \verb|set_initial_heap_size| and \verb|set_min_heap_growth|
respectively.  These methods are declared in the \T{GC}
protocol.   By default, the initial heap size for this collector
is 128K and each heap expansion will increase the heap by 512K. 

If an application uses a lot garbage collected storage it is a good
idea to set the heap size to larger capacity during initialization.
Otherwise, the collector will have to perform more garbage collection
and heap expansions to reach a stable state. 

\subsection{The Mark Sweep collector}

   An alternative to the copying collector is the the mark sweep collector.
Like the previous collector, this collector also uses conservative scanning 
to locate roots.  Unlikely the Boehm collector, type accurate marking
is used through the user supplied tracing method.

   Since the default collector is of the \T{BGC} variety, the user must
create an instance of \T{MarkSweepGC} if the mark sweep collector
is to be used.  

\begin{verbatim}
   class MarkSweepGC : public CGC {
   public:
      virtual void set_gc_ratio(int);
      virtual void set_initial_heap_size (size_t);
      virtual void set_min_heap_growth   (size_t);
   }
\end{verbatim}

   The gc ratio in class \T{MarkSweepGC} determines whether heap
expansion is to be performed after a collection.  The default gc ratio
is 50\%, thus if after garbage collection the ratio of used versus total
heap space exceeds one half, the heap will be expanded. 

   For most purposes the two collectors are interchangeable.  Since
all types of garbage collectors under our framework use the same protocol, 
applications can be written without fixing a specific garbage collection
algorithm before hand.  

\subsection{Finalization} 

   One common \Cpp{} idiom uses constructor and destructor for
resource allocation: resources (including memory but may include others,
such as file handles, graphical widgets, etc) that are acquired
in a class constructor are released(finalized) in the class'es destructor.

   There are, however, two opposing views in how finalization should be handled
in a garbage collector.  The first assumes that garbage collection
simulates an infinite amount of memory and so automatic finalization is
inapproriate.   The second takes a more pragmatic view, and assumes that
automatic finalization should be provided since otherwise explicit resource
tracking must be provided by the user for collectable datatypes, making garbage 
collection much less useful than it can be.

   We will refrain from participating in any religious and philosophical
wars on which dogma is the ``One True Way''. 
nstead, both types of collectors are provided.

   By default, all garbage collection classes do 
not perform finalization on garbage 
collected objects.  If object finalization is desired then the user
can do one of two things:
\begin{itemize}
   \item Enable the finalization of the default heap by putting
\begin{verbatim}
   GC::get_default_gc().set_finalization(true);
\end{verbatim}
   in the initialization code, or
   \item Instantiate a different instance of the garbage collector
    and allocate objects needing finalization from this collector.
    This may be a better method since not all objects need finalization
    and those that do not can still be allocated from the default collector.
    This way we only have to pay for the performance penalty (if any)
    proportional to the usage of finalization.
\end{itemize}

   For example, if a collectable class named \T{Foo} should be properly
finalized we can declare it in the following manner:

\begin{verbatim}
   extern BGC bgc_f;  // somewhere an instance is defined

   class Foo : public GCObject {
      Resource r;
   public:
      Foo() { r.allocate(); /* allocate resource */ }
     ~Foo() { r.release(); /* release all resource */ }

      // Make object allocate from bgc_f instead
      // of the default non-collectable gc.
      void * operator new (size_t n) { return bgc_f.m_alloc(n); }
   };
\end{verbatim}

   When an instance of class \verb|Foo| is identified as garbage, its
destructor will be called.  [Aside: {\it currently issue such as 
finalization synchronization is not handled directly by the collector.
So if there is synchronization constraints during finalization it must
be handled by the object destructor.  Future versions of \Prop{} will
provide subclasses with more sophisticated finalization mechanisms.} ]

   Notice that the default \verb|delete| operator of all garbage collectable
classes will call the object destructor explicitly.  It is a good idea to use
explicit \verb|delete| if finalization is a rare need since this service
involves an explicit scan of the garbage memory, which may degrade performace
with excessive swapping: without finalization the garbage pages will not 
have to be referenced with a copying collector.

   It should also be noted that since the collector is conservative, there
is no guarantee that garbage will be identified as such:
there is no guarantee that all garbage resources will be released.
In general, however, the efficacy of the collector is quite good and
so non-critical or plentiful resources can be safely finalized with
this collector.

\Subsubsection{Weak Pointers}

   It is frequently very useful to be able to keep track of garbage
collectable objects through the use of ``{\bf weak pointers}.''
Collectors ignore the presence of weak pointers to garbage collected
objects; objects with only referencing weak pointers will still be
collected.  Weak pointers to objects that become garbage will
be automatically reset to null.
 
   Weak pointers are provided through a smart pointer template 
\CLASSDEF{WeakPointer}, whose definition is shown below: 

\begin{verbatim}
template <class T> class WeakPointer {
   public:
      inline WeakPointer();
      inline WeakPointer(const WeakPointer<T>& wp); 
      inline WeakPointer(T * ptr); 
      inline WeakPointer<T>& operator = (const WeakPointer<T>& wp);
      inline WeakPointer<T>& operator = (T * ptr);
      inline bool is_null() const;
      inline operator const T * () const;
      inline operator       T * ();
      inline const T * operator -> () const;
      inline       T * operator -> ();
      inline const T&  operator *  () const;
      inline       T&  operator *  ();
   };
\end{verbatim}

A weakpointer to a garbage collectable 
class \verb|A| can be defined as \verb|WeakPointer<A>|.
If a \Prop{} datatype \verb|T| has been defined, a weakpointer to type
\verb|T| can defined using the
\TDEF{classof} keyword.  For example:

\begin{verbatim}
   datatype Wff :: collectable = True | False | And(...)  | ...;

   WeakPointer<classof Wff> a = And(True,False);
\end{verbatim}

\Subsubsection{The Heap Walker}

   There is a simple class called \CLASSDEF{GCVerify} that can be used
to verify the integrity of a complex allocated data structure.
This is frequently useful during the development of a new collector,
or a complex class derive class of \T{GCObject}.

   The interface of this class is shown below.

\begin{verbatim}
   class GCVerifier : protected GC {
   public:
      virtual bool is_valid_pointer          (GCObject *);
      virtual bool is_valid_interior_pointer (GCObject *);
      virtual bool is_valid_structure        (GCObject *);
      virtual bool is_valid_pointer          (GCObject *, GC *);
      virtual bool is_valid_interior_pointer (GCObject *, GC *);
      virtual bool is_valid_structure        (GCObject *, GC *);
      size_t number_of_nodes() const;
   };
\end{verbatim}

   Class \CLASSDEF{GCVerify} is derived from class \CLASSDEF{GC} so that the
same object tracing protocol can be used.  Briefly, three different methods 
are provided for verification:
\begin{itemize}
   \item Call to \verb|is_valid_pointer(p,gc)| verifies that
    \verb|p| is a valid pointer to an object within the collector \verb|gc|. 
   \item Call to \verb|is_valid_pointer(p,gc)| verifies that
    \verb|p| is a valid interior pointer to an object 
     within the collector \verb|gc|. 
   \item Call to \verb|is_valid_structure(p,gc)| verifies the entire
     structure reachable by \verb|p| is valid.  This method uses
     \verb|GCObject::trace| to locate the correct sub-structures.
\end{itemize}

   There are alternative versions of the same methods that assumes
the default collector is used.   Furthermore
\begin{itemize}
   \item Method \verb|number_of_nodes()| returns the node count of the
last structure traced using \verb|is_valid_structure|, and 
   \item If \verb|set_verbosity(GC::gc_print_debugging_info)| is used
then error messages will be printed during structure tracing.
\end{itemize}

\bibliography{refman}

\input{refman.ind}
\end{document}

\end{document}

\end{document}

\end{document}
